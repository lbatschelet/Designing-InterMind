% LTeX: language=de-CH
\chapter{Diskussion} \label{sec:diskussion}

Im letzten Kapitel dieser Arbeit diskutiere ich die zentralen Ergebnisse, Entscheidungen und Spannungen, die den Forschungsprozess geprägt haben. Dabei gehe ich sowohl auf methodische Potenziale und Grenzen als auch auf theoretische und infrastrukturelle Fragen ein. Mein Ziel ist es, die Arbeit kritisch zu reflektieren, aufzuzeigen welche Punkte offen bleiben und wie zukünftige Arbeiten darauf aufbauen können.

\section{Ein Forschungsdesign als Schnittstelle}

Ausgangspunkt dieser Arbeit ist die Frage, wie sich der Einfluss räumlicher Umgebungen auf das situierte (Un\nobreakdash-)Wohlbefinden \glslink{intersektionalitaet}{intersektional} positionierter Personen erfassen und analysieren lässt. Eine abschliessende Beantwortung dieser komplexen Leitfrage kann im Rahmen dieser Bachelorarbeit nicht geleistet werden. Stattdessen unternehme ich einen ersten Schritt, ein Forschungsdesign zu entwickeln, das einen Beitrag zu dieser übergeordneten Fragestellung leisten kann.

Im Verlauf der Arbeit zeige ich auf, dass ein geeigneter Erhebungsansatz weit mehr umfasst als die Auswahl methodischer Verfahren. Transparenz, Nachvollziehbarkeit und eine kritische Reflexion der zugrunde liegenden Infrastruktur sind ebenso zentral wie die konkrete Gestaltung der Erhebung. Methodisch erweist sich die Verbindung von \gls{ema}/\gls{gema}-Methoden mit einer intersektionalen Mehrebenenanalyse als vielversprechend, auch wenn Spannungen in der Übersetzung intersektionaler Theorie in quantitative Verfahren bestehen bleiben. Das hier entwickelte Forschungsdesign kann somit als ein möglicher, wenn auch nicht spannungsfreier Weg verstanden werden, situiertes (Un\nobreakdash-)Wohlbefinden in seiner sozialen und räumlichen Einbettung quantitativ zu erfassen.

Mit der Entwicklung und Veröffentlichung einer eigenen \gls{opensource}-Erhebungsplattform versuche ich, Prinzipien wie Nachvollziehbarkeit, Offenheit und Reflexivität praktisch umzusetzen. Auch wenn \gls{intermind} in seiner jetzigen Form nur begrenzt anschlussfähig ist, verdeutlicht das Projekt, dass wissenschaftliche Werkzeuge nicht zwingend an proprietäre Systeme gebunden sein müssen, sondern auch im Sinne einer offenen, gemeinschaftsorientierten Infrastruktur realisierbar sind.

In \cref{sec:pilotstudie} zeige ich, dass die erhobenen Pilotdaten für eine \glslink{intersektionalitaet}{intersektionale} Mehrebenenanalyse nur eingeschränkt geeignet sind. Zwar lassen sich Modellierungen prinzipiell durchführen, die kleine und sehr homogene Stichprobe verhindert jedoch belastbare Ergebnisse. Manche Strata sind so klein, dass eine intrapersonelle Ebene nicht abgebildet werden kann. Hinzu kommt, dass die inhaltlich nicht konsequent theoriegeleitete Entwicklung des Fragebogens, die Datengrundlage zusätzlich schwächt. Entsprechend ist die Überprüfung der Eignung der Analysemethode mit den Daten aus der Pilotstudie nur teilweise möglich.

Ich sehe den Beitrag dieser Arbeit nicht in belastbaren empirischen Resultaten, sondern in der Entwicklung und Erprobung eines durchgängig kritisch ausgerichteten Forschungsdesigns, das Theorie, Methodenwahl, Erhebungsinstrument, Fragebogen und Analyse als zusammenhängendes Ganzes verbindet. Dieser Ansatz macht Spannungen sichtbar -- zwischen Praktikabilität und Anschlussfähigkeit, zwischen theoretischer Schärfe und empirischer Umsetzbarkeit. Gerade dadurch wird deutlich, dass eine \glslink{intersektionalitaet}{intersektionale} Betrachtung von situiertem (Un\nobreakdash-)Wohlbefinden nicht allein auf der Ebene von Items und Skalen beantwortet werden kann, sondern auch grundlegende infrastrukturelle und methodologische Entscheidungen umfasst. In dieser doppelten Hinsicht -- als methodisches Forschungsdesign wie auch als theoretisch-praktische Rahmung -- versteht sich die Arbeit als Beitrag zu einer kritisch-feministischen Geographie, die methodische, digitale und intersektionale Fragen nicht getrennt behandelt, sondern gemeinsam denkt -- auch wenn dies gezwungenermassen Brüche und Unvollständigkeiten mit sich bringt.

Meine Motivation für diese Arbeit ist auch in einer persönlichen Haltung begründet. Ich sehe in der Konzentration digitaler Infrastrukturen bei einigen wenigen Konzernen ein gesamtgesellschaftliches Problem, das Abhängigkeiten und Intransparenzen schafft und Ungleichheiten verstärkt. Gleichzeitig nutze ich selbst täglich Geräte und Dienste dieser Konzerne. Diese Verflechtung von Kritik und Abhängigkeit ist sinnbildlich für meinen Zugang: Sie macht deutlich, wie schwer es ist, konsequent offene Infrastrukturen umzusetzen, und erklärt zugleich, warum mir dieses Anliegen wichtig ist. Offenheit und Nachvollziehbarkeit im Sinne digitaler Souveränität sind nicht nur abstrakte Prinzipien, sondern eine Antwort auf diese Spannung. Diese Spannung zwischen wissenschaftlichem Anspruch und praktischer Machbarkeit zieht sich durch die gesamte Arbeit -- sie erklärt, warum an einigen Stellen Konzepte nur skizziert, Verfahren nicht voll validiert oder Entscheidungen situativ getroffen wurden. Diese Brüche verstehe ich als typische Dynamik eines geographisch-interdisziplinären Arbeitens, bei dem wissenschaftlicher Anspruch und praktische Machbarkeit ständig neu austariert werden müssen.

\section{\textit{InterMind} -- zwischen Prototyp und Plattformabhängigkeit}

Das im Rahmen dieser Arbeit entwickelte Erhebungsplattform \gls{intermind} erweist sich einerseits als funktionales, transparentes und datenschutzfreundliches Werkzeug, und zeigt andererseits auch klare Grenzen auf. Ein zentrales Potenzial liegt in der Modularität und Anpassbarkeit: Das Grundgerüst von \gls{intermind} kann mit überschaubarem Aufwand für andere Fragebögen genutzt und durch zusätzliche Komponenten erweitert werden. Damit entsteht ein flexibles System, das nicht auf ein einzelnes Forschungsszenario beschränkt ist, sondern prinzipiell in ganz unterschiedlichen Kontexten eingesetzt werden kann. Die Entscheidung, den Code offen zu veröffentlichen, stärkt diese Anschlussfähigkeit zusätzlich und macht den Entwicklungsprozess nachvollziehbar.

Gleichzeitig wird deutlich, dass die Plattform für bestimmte Forschungsszenarien noch ergänzt werden müsste. Funktionen wie eine Offline-Nutzung, standortbasierte Trigger oder Echtzeitauswertungen sind bislang nicht umgesetzt. In grösser angelegten Studien wäre es zudem sinnvoll, die Architektur um zusätzliche serverseitige Module zu erweitern, die eine engere Steuerung der Erhebungen erlauben. Solche Erweiterungen sind weniger prinzipielle Grenzen des Systems als vielmehr Ausdruck des Umfangs, der im Rahmen einer Bachelorarbeit realisierbar ist. Gleichzeitig verweisen diese Überlegungen auf grundlegende Spannungen, die jede Weiterentwicklung dieser Plattform mit sich bringt: Mehr Funktionalität bedeutet zunächst, dass Teilnehmende bereit sein müssen, zusätzliche Daten zu teilen -- und damit auch Kontrolle darüber abzugeben, wie diese Daten weiterverarbeitet werden. Gerade hier liegt ein kritischer Punkt: Daten, die im Rahmen einer Wissenschaftlichen Studie erhoben werden, sind immer mit Erwartungen und einem impliziten Vertrauensvorschuss verbunden, der wissenschaftlich legitimiert und geschützt werden muss.

Auch auf der Seite der Forschenden entstehen durch die Erhebung zusätzlicher Daten erhöhte Anforderungen. Mehr Daten bedeuten nicht automatisch mehr Erkenntnis, sondern bergen das Risiko, sensible oder besonders schutzbedürftige Bereiche zu berühren, die über das eigentliche Forschungsinteresse hinausgehen. Die Pilotstudie macht dies deutlich: So habe ich \gls{bspw} präzise Standortdaten erhoben, welche ich in der anschliessenden Analyse nicht verwende. Auch wenn diese Daten technisch sicher gespeichert sind, wurden sie mir damit ohne klaren Erkenntnisgewinn anvertraut. Der in der Befragung implizit suggerierte Nutzen kann so nicht eingelöst werden. Rückblickend wäre es notwendig, gegenüber den Teilnehmenden transparenter zu kommunizieren, dass es sich hier in erster Linie um eine explorative Studie handelt. Dieses Missverhältnis verdeutlicht, wie wichtig es ist, bereits vor Beginn einer Erhebung kritisch zu reflektieren, welche Daten tatsächlich benötigt werden -- und wie eng technische Gestaltung, methodische Entscheidungen und ethische Verantwortung miteinander verflochten sind.

\vspace{1em}

Im Entwicklungsprozess zeigt sich besonders deutlich das Spannungsfeld zwischen der offenen Logik von \gls{opensource}-Software und den geschlossenen Ökosystemen grosser Plattformbetreiber. Zwar steht der Quellcode von \gls{intermind} öffentlich zur Verfügung, die Distribution über App-Stores bleibt jedoch an intransparente und kommerziell geprägte Verfahren gebunden. Die Veröffentlichung im Apple App Store scheitert schliesslich an einer nur schwer nachvollziehbaren und intransparenten Ablehnung, während die Bereitstellung im Google Play Store zusätzliche Gebühren und aufwändige Prüfprozesse erfordert. Hinzu kommt, dass die Entwicklung der App selbst zwar aufwendig ist, aber der zeitliche und organisatorische Aufwand rund um die Veröffentlichung in den App Stores deutlich grösser ausfällt als erwartet -- von Datenschutzrichtlinien und benötigten Webseiten bis hin zu sich ständig verändernden Store-Vorgaben und geforderten Updates. Rückblickend zeigt sich, dass dieser Teil des Projekts wesentlich mehr Ressourcen gebunden hat als die eigentliche Programmierung der App. Während ich den reinen Coding-Aufwand aufgrund meiner bisherigen Erfahrung im Vorfeld relativ gut einschätzen konnte, habe ich die zeitintensiven Prozesse der Distribution massiv unterschätzt.

Besonders deutlich wurde dies im Vorfeld der Pilotstudie: Da ich die einmalige Gelegenheit hatte, die App im Rahmen der Exkursion \enquote{Recht auf Stadt} mit einer ganzen Studierendengruppe zu testen, war der Termin klar vorgegeben und nicht verschiebbar. Damit die Erhebung durchgeführt werden konnte, musste die App rechtzeitig auf beiden Betriebssystemen verfügbar sein. In der unmittelbaren Vorbereitungsphase konzentrierte sich die Arbeit daher unerwartet stark auf die Veröffentlichung in den Stores, was erhebliche zeitliche und organisatorische Ressourcen band. Dieser Aufwand limitierte direkt die Zeit, die für eine sorgfältigere theoretische Fundierung und Ausarbeitung des Fragebogens zur Verfügung stand.

Viele im Prozess getroffene technische Entscheidungen reflektiere ich in dieser Arbeit nicht. Dadurch verlieren sie an Sichtbarkeit und Transparenz, obwohl sie methodisch wie epistemisch bedeutsam sind. Dass solche Entscheidungen im Text unsichtbar bleiben, macht zugleich ein grundlegendes Spannungsfeld sichtbar: Kritisch-sozialwissenschaftliche Ansprüche zielen auf Transparenz und Reflexion, während technische Erfordernisse oft pragmatische und situative Entscheidungen verlangen. Künftig wäre es wichtig, Wege zu finden, auch diese technischen Entscheidungen methodisch sichtbar zu machen, sei es durch begleitende Reflexion oder durch eine engere Verzahnung von Entwicklung und Dokumentation.

\section{Copyleft als Intervention}

Die Offenlegung des Quellcodes schafft eine produktive Ambivalenz. Sie ermöglicht eine einfache Nachnutzung und eröffnet die Möglichkeit, dass andere auf der bestehenden Arbeit aufbauen. Gleichzeitig bedeutet sie, dass ich meinen Code sichtbar mache, obwohl er im Rahmen einer Bachelorarbeit entstanden ist. Üblicherweise gelten Bachelorarbeiten als geschützter Raum, in dem Studierende lernen, experimentieren und Fehler machen können, ohne dass diese unmittelbar für eine breite Öffentlichkeit sichtbar werden. Mit der Veröffentlichung entfällt dieser Schutzraum: Der Code unterliegt keiner formalisierten Qualitätssicherung, sondern steht so zur Verfügung, wie er im Entwicklungsprozess entstanden ist. Damit wird er beurteilbar, kritisierbar und gerade in seiner Prozesshaftigkeit sichtbar. Diese Bereitschaft zur Exponierung markiert zugleich den Kern von \gls{opensource}: Der Wert liegt nicht in einem perfekten Endprodukt, sondern im Teilen von Arbeit im Sinne eines \emph{commons}, das durch kollektives Aufbauen, Weiterentwickeln und gemeinschaftliche Verantwortung lebendig bleibt.

Ein wesentlicher Teil dieser Offenlegung betrifft die Wahl der Lizenz. Sowohl \gls{intermind} als auch die vorliegende Arbeit sind unter \gls{lic:copyleft}-Lizenzen veröffentlicht (\gls{lic:agpl}~3.0 für den Code, \gls{lic:cc-by-sa}~4.0 für den Text). Copyleft-Lizenzen unterscheiden sich von anderen offenen Lizenzen darin, dass sie Offenheit nicht nur erlauben, sondern auch verpflichtend machen: Wer auf dieser Grundlage weiterentwickelt, muss seine eigenen Ableitungen wiederum unter einer offenen Lizenz zugänglich machen. Offenheit wird so nicht nur als Option, sondern als wechselseitiges Prinzip gedacht. 

Gerade in der Research Software Engineering Community sind \gls{lic:copyleft}-Lizenzen umstritten. Viel digitale Wissenschaftsinfrastruktur ist entweder proprietär oder unter permissiven Lizenzen wie MIT oder BSD \parencite{sethiWhyEarthAre2020} verfügbar. Diese gelten als anschlussfähig, weil sie auch die Einbindung in proprietäre Kontexte ermöglichen. Für mich wirft diese Praxis jedoch grundlegende Fragen auf: Warum sollte es ein wissenschaftliches Ziel sein, dass Forschungscode ohne jede wechselseitige Verpflichtung in kommerzielle Produkte integriert werden kann? Offenheit wird hier vor allem als individuelle Grosszügigkeit verstanden, nicht aber als kollektives Prinzip geteilter Verantwortung. Damit bleibt die Debatte verkürzt und blendet zentrale Fragen nach den Bedingungen und Zielen wissenschaftlicher Zusammenarbeit aus.

Die Entscheidung, auch den Text dieser Arbeit unter eine Copyleft-Lizenz zu stellen, ist in diesem Zusammenhang nicht nur eine formale Geste, sondern Ausdruck dieser Haltung. Viele der im Entwicklungsprozess getroffenen technischen Entscheidungen sind nicht neutral, sondern aus den hier dargelegten theoretischen Überlegungen und methodischen Perspektiven hervorgegangen. Der Text begleitet den Code, macht seine Annahmen sichtbar und rahmt seine Nutzung. In diesem Sinne gehört er zur Infrastruktur des Projekts: Ohne die hier formulierten Reflexionen bliebe der Code nur ein technisches Artefakt, während er im Zusammenspiel mit der Arbeit als Teil einer kritisch-reflexiven Forschungsumgebung lesbar wird. 

Gerade darin liegt die Intervention: Indem ich sowohl den Code als auch den Text unter eine \gls{lic:copyleft}-Lizenz stelle, mache ich sichtbar, dass wissenschaftliche Werkzeuge nicht nur technische Instrumente sind, sondern immer auch normative Dimensionen haben. Sie verkörpern Vorstellungen darüber, was als Wissen gilt, wem es gehört und wie es zirkuliert. In diesem Sinn erprobe ich mit dieser Entscheidung einen Gegenentwurf zu bestehenden wissenschaftlichen Infrastrukturen, die meist proprietär oder permissiv lizenziert sind. Diese Entscheidung knüpft an eine feministische Perspektive an, die Offenheit nicht als selbstverständliches oder rein technisches Prinzip versteht, sondern als politisch gestaltete Praxis: eine Praxis, die Machtverhältnisse sichtbar macht, Verantwortung wechselseitig verteilt und gemeinschaftliche Wissensproduktion ermöglicht. Damit nimmt die Arbeit eine Ausnahmeposition ein und verweist zugleich auf eine Lücke: Während Copyleft im Bereich von Code etabliert ist, bleibt es im wissenschaftlichen Publizieren nahezu unsichtbar. Gerade deshalb ist die Entscheidung bewusst als Intervention zu lesen, die eine Diskussion darüber anstossen soll, wie verpflichtendere Formen von Offenheit als kollektives Prinzip in der Wissenschaft verankert werden können.


\section{Was bleibt -- und was weitergeführt werden muss}

Die Verbindung von \gls{ema}/\gls{gema}-Methoden mit intersektionalen Mehrebenenmodellen (\gls{i-maihda}) setze ich als Zugang ein, der situiertes (Un\nobreakdash-)Wohlbefinden im Verhältnis zu sozialen Positionierungen analysierbar macht. Alltagsräume verstehe ich machtkritisch als Gefüge, die soziale Positionierungen und ihre Überschneidungen (Intersektionen) zugleich widerspiegeln \emph{und} mit hervorbringen; sie sind nicht neutrale Kulissen, sondern an der Herstellung und Reproduktion von Machtverhältnissen beteiligt. 
Eine solche Quantifizierung muss aus einer machtkritischen Persektive immer mit qualitativen Verfahren verbunden werden: Die Modelle können verorten, \emph{wo} und \emph{für wen} Unterschiede auftreten, während qualitative Verfahren klären, \emph{wie} und \emph{warum} sie entstehen. Damit das möglich it muss aber während dem ganzen Prozess immer wieder kritisch hinterfragt werden, welche Elemente oder Ebenen tatsächlich relevant sind und welche dazugehörigen qualitativen Aspekte berücksichtigt werden müssen. Ich argumentiere daher explizit für einen Mixed\-Methods\-Ansatz.

Die Umsetzung in dieser Arbeit zeigt zugleich klare Grenzen. Ohne Pretests und mit nur teilweise validierten Skalen bleiben einzelne Items unscharf. Mit einer kleinen und relativ homogenen Stichprobe lassen sich die Potenziale von \gls{i-maihda} nur begrenzt zeigen: Strata sind unterbesetzt, Varianz innerhalb von Personen ist kaum abbildbar, robuste Effektschätzungen sind nicht realistisch. Ich lese diese Punkte als methodische Lehren: eine Studie die diesen hier entwickelten Ansatz umsetzt braucht ausreichend Teilnehmende, eine heterogenere Gesamtstichprobe und inhaltlich besser entwickelte, geprüfte Items, damit der Ansatz sein Potenzial entfaltet.

Perspektivisch ist eine Erweiterung auf \gls{gema} im engeren Sinn anzudenken: die Einbindung externer Kontextdaten (\gls{zb} hochaufgelöster Stadthitzedaten wie im \enquote{Bernometer} \parencite[siehe][]{burgerModellingSpatialPattern2021} oder punktueller Messungen der Umgebungslautstärke). Aus einer feministischen und intersektionalen Perspektive gehen damit aber Fragen der Messpolitik einher: Welche Wirklichkeiten mache ich durch Sensorik überhaupt sichtbar -- und welche blende ich aus? Welche Kategorien schreibe ich damit fest? Wer gewinnt Erkenntnis? Welche Risiken bestehen?

Ein entscheidender Faktor für die Aussagekraft künftiger Studien mit dem hier entwickelten Forschungsdesign liegt nicht allein in der Anzahl der Erhebungen pro Person, sondern in der Diversität und Grösse der Stichprobe insgesamt. Erst wenn genügend unterschiedliche soziale Gruppen erfasst werden und diese zugleich in ausreichender Zahl wiederholt teilnehmen, lassen sich die Potenziale intersektionaler Mehrebenenanalysen ausschöpfen. Damit stellt sich auch die Frage nach der zeitlichen Dimension der Erhebung: Ist es sinnvoll, Teilnehmende über längere Zeiträume hinweg zu begleiten, um Dynamiken sichtbar zu machen -- oder könnten alternative Formate wie einmalige, ortsbasierte Befragungen ebenso produktiv sein? Denkbar wären etwa situative Erhebungen im öffentlichen Raum, die über QR-Codes zugänglich sind und an spezifische Orte gebunden werden.

Unabhängig vom Design bleibt die grösste Herausforderung, eine tatsächlich diverse Datenbasis zu schaffen. Die Rekrutierung von Teilnehmenden ist nicht neutral, sondern strukturiert durch Zugänge, Reichweiten und bestehende Ausschlüsse. Gerade für intersektionale Fragestellungen ist dies entscheidend, da marginalisierte Gruppen, die im Zentrum der Analyse stehen, oft besonders schwer zu erreichen sind. Hinzu kommen Fragen der Barrierefreiheit: Digitale Erhebungen können für ältere Personen eine Hürde darstellen, und bestehende Interfaces sind nicht selbstverständlich für Menschen mit Behinderungen nutzbar. Künftige Arbeiten müssen deshalb Strategien entwickeln, wie diese Gruppen gezielt angesprochen, einbezogen und auch langfristig zur Teilnahme motiviert werden können. Neben einer lokalen Verankerung -- etwa über Quartiere, Schulen oder Vereine -- kann auch die Gestaltung der Rückmeldung an die Teilnehmenden eine Rolle spielen. Anstatt dass die Teilnahme ausschliesslich als Beitrag zu wissenschaftlicher Erkenntnis wahrgenommen wird, könnte sie durch eine aufbereitete, persönliche Auswertung angereichert werden. Solche Rückmeldungen könnten aufzeigen, in welchen Umgebungen sich eine Person wohler oder unwohler fühlt, und damit einen unmittelbaren Nutzen stiften.

Gerade dieser Aspekt markiert eine klare Limitation des vorliegenden Forschungsdesigns: In seiner aktuellen Form ist es nicht partizipativ angelegt und bietet den Teilnehmenden keinen direkten Mehrwert ausser dem Beitrag zu wissenschaftlicher Erkenntnis. Damit reproduziert es ein klassisches und problematisches Verständnis von Forschung, das Daten einfordert, ohne etwas zurückzugeben. Zukünftige Arbeiten sollten deshalb stärker partizipativ ausgerichtet sein und überlegen, wie Erkenntnisse auch für die Teilnehmenden selbst relevant und zugänglich gemacht werden können.

Ich lasse in dieser Arbeit offen, wie sich Ergebnisse visualisieren und kartieren lassen. Die Darstellung von intersektionalem und situiertem (Un-)Wohlbefinden ist methodisch wie epistemisch eine grosse Herausforderung. Komplexe relationale Erfahrungen so abzubilden, dass sie sowohl räumliche Muster sichtbar machen als auch ihrer sozialen Situiertheit gerecht werden, stellt ein offenes Problem dar. In dieser Arbeit werden keine entsprechenden Verfahren entwickelt oder erprobt. Anknüpfungspunkte finden sich aber in bestehenden Ansätzen der \emph{emotional cartographies} \parencite[\gls{bspw}][]{bleischExploratoryGeovisualizationsSupporting2019}.

Zusammenfassend zeige ich in dieser Bachelorarbeit, dass eine intersektionale Erhebung und Analyse von situiertem (Un\nobreakdash-)Wohlbefinden nicht nur eine methodische, sondern auch eine infrastrukturelle und politische Herausforderung darstellt. Mit \textit{InterMind} habe ich einen ersten, prototypischen Ansatz entwickelt, der digitale Offenheit, methodische Reflexivität und kritische Perspektiven miteinander verbindet. Auch wenn die empirische Basis begrenzt bleibt und zahlreiche Weiterentwicklungen notwendig sind, liegt der Beitrag dieser Arbeit darin, ein Forschungsfeld zu skizzieren und ein Werkzeug bereitzustellen, das künftige Arbeiten aufgreifen und weiterführen können. Damit versteht sich die Arbeit nicht als abgeschlossene Antwort, sondern als Einladung, die hier angestossenen Fragen weiterzudenken, kritisch zu vertiefen und in zukünftigen Arbeiten weiterzuentwickeln.

