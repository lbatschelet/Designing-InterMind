% LTeX: language=de-CH
\chapter{Diskussion} \label{sec:diskussion}

Im letzten Kapitel dieser Arbeit diskutiere ich die zentralen Ergebnisse, Entscheidungen und Spannungen, die den Forschungsprozess geprägt haben. Dabei gehe ich sowohl auf methodische Potenziale und Grenzen als auch auf theoretische und infrastrukturelle Fragen ein. Mein Ziel ist es, die Arbeit nicht nur im engeren Forschungskontext zu verorten, sondern auch kritisch zu reflektieren, welche Punkte offen bleiben und wie zukünftige Arbeiten darauf aufbauen können.

\section{Einordnung der Arbeit im Forschungsfeld}

Ausgangspunkt dieser Arbeit ist die Frage, wie sich der Einfluss räumlicher Umgebungen auf das situierte (Un-)Wohlbefinden \glslink{intersektionalitaet}{intersektional} positionierter Personen erfassen und analysieren lässt. Eine abschliessende Beantwortung dieser komplexen Leitfrage kann im Rahmen dieser Bachelorarbeit nicht geleistet werden. Stattdessen unternehme ich einen ersten Schritt, ein Forschungsdesign zu entwickeln, das einen Beitrag zu dieser übergeordneten Fragestellung leisten kann.

Im Verlauf der Arbeit zeige ich auf, dass ein geeigneter Erhebungsansatz weit mehr umfasst als die Auswahl methodischer Verfahren. Transparenz, Nachvollziehbarkeit und eine kritische Reflexion der zugrunde liegenden Infrastruktur sind ebenso zentral wie die konkrete Gestaltung der Erhebung. Methodisch erweist sich die Verbindung von \gls{ema}/\gls{gema}-Methoden mit einer intersektionalen Mehrebenenanalyse als vielversprechend, auch wenn Spannungen in der Übersetzung intersektionaler Theorie in quantitative Verfahren bestehen bleiben. Das hier entwickelte Forschungsdesign kann somit als ein möglicher, wenn auch nicht spannungsfreier Weg verstanden werden, situiertes (Un-)Wohlbefinden in seiner sozialen und räumlichen Einbettung quantitativ zu erfassen.

Mit der Entwicklung und Veröffentlichung einer eigenen \gls{opensource}-Erhebungsplattform versuche ich, Prinzipien wie Nachvollziehbarkeit, Offenheit und Reflexivität praktisch umzusetzen. Auch wenn \gls{intermind} in seiner jetzigen Form nur begrenzt anschlussfähig ist, verdeutlicht das Projekt, dass wissenschaftliche Werkzeuge nicht zwingend an proprietäre Systeme gebunden sein müssen, sondern auch im Sinne einer offenen, gemeinschaftsorientierten Infrastruktur realisierbar sind.

In \cref{sec:pilotstudie} zeige ich, dass die erhobenen Pilotdaten für eine \glslink{intersektionalitaet}{intersektionale} Mehrebenenanalyse nur eingeschränkt geeignet sind. Zwar lassen sich Modellierungen prinzipiell durchführen, die kleine und sehr homogene Stichprobe verhindert jedoch belastbare Ergebnisse. Manche Strata sind so klein, dass eine intrapersonelle Ebene nicht abgebildet werden kann. Hinzu kommt, dass die inhaltlich nicht konsequent theoriegeleitete Entwicklung des Fragebogens, die Datengrundlage zusätzlich schwächt. Entsprechend ist die Überprüfung der Eignung der Analysemethode mit den Daten aus der Pilotstudie nur teilweise möglich.

Ich sehe den Beitrag dieser Arbeit nicht primär in belastbaren empirischen Resultaten, sondern in der Entwicklung und Erprobung eines durchgängig kritisch ausgerichteten Forschungsdesigns, das Theorie, Methodenwahl, Erhebungsinstrument, Fragebogen und Analyse als zusammenhängendes Ganzes verbindet. Dieser Ansatz macht Spannungen sichtbar -- zwischen Transparenz und Praktikabilität, zwischen Offenheit und Anschlussfähigkeit, zwischen theoretischer Schärfe und empirischer Umsetzbarkeit. Gerade dadurch wird deutlich, dass eine \glslink{intersektionalitaet}{intersektionale} Betrachtung von situiertem (Un-)Wohlbefinden nicht allein auf der Ebene von Items und Skalen beantwortet werden kann, sondern auch grundlegende infrastrukturelle und methodologische Entscheidungen umfasst. In dieser doppelten Hinsicht -- als methodisches Forschungsdesign wie auch als theoretisch-praktische Rahmung -- versteht sich die Arbeit als Beitrag zu einer kritisch-feministischen Geographie, die methodische, digitale und intersektionale Fragen nicht getrennt behandelt, sondern gemeinsam denkt -- auch wenn dies gezwungenermassen Brüche und Unvollständigkeiten mit sich bringt.

Meine Motivation für diese Arbeit ist auch in einer persönlichen Haltung begründet. Ich sehe in der Konzentration digitaler Infrastrukturen bei einigen wenigen Konzernen ein gesamtgesellschaftliches Problem, das Abhängigkeiten und Intransparenzen schafft und Ungleichheiten verstärkt. Gleichzeitig nutze ich selbst täglich Geräte und Dienste dieser Konzerne. Diese Verflechtung von Kritik und Abhängigkeit ist sinnbildlich für meinen Zugang: Sie macht deutlich, wie schwer es ist, konsequent offene Infrastrukturen umzusetzen, und erklärt zugleich, warum mir dieses Anliegen wichtig ist. Offenheit und Nachvollziehbarkeit im Sinne digitaler Souveränität sind nicht nur abstrakte Prinzipien, sondern eine Antwort auf diese Spannung. Diese Spannung zwischen wissenschaftlichem Anspruch und praktischer Machbarkeit zieht sich durch die gesamte Arbeit -- sie erklärt, warum an einigen Stellen Konzepte nur skizziert, Verfahren nicht voll validiert oder Entscheidungen situativ getroffen wurden. Diese Brüche verstehe ich als typische Dynamik eines geographisch-interdisziplinären Arbeitens, bei dem wissenschaftlicher Anspruch und praktische Machbarkeit ständig neu austariert werden müssen.

\section{Potenziale und Grenzen von \textit{InterMind}}

Das im Rahmen dieser Arbeit entwickelte Erhebungsplattform \gls{intermind} erweist sich einerseits als funktionales, transparentes und datenschutzfreundliches Werkzeug, und zeigt andererseits auch klare Grenzen auf. Ein zentrales Potenzial liegt in der Modularität und Anpassbarkeit: Das Grundgerüst von \gls{intermind} kann mit überschaubarem Aufwand für andere Fragebögen genutzt und durch zusätzliche Komponenten erweitert werden. Damit entsteht ein flexibles System, das nicht auf ein einzelnes Forschungsszenario beschränkt ist, sondern prinzipiell in ganz unterschiedlichen Kontexten eingesetzt werden kann. Die Entscheidung, den Code offen zu veröffentlichen, stärkt diese Anschlussfähigkeit zusätzlich und macht den Entwicklungsprozess nachvollziehbar.

Gleichzeitig wird deutlich, dass die Plattform für bestimmte Forschungsszenarien noch ergänzt werden müsste. Funktionen wie eine Offline-Nutzung, standortbasierte Trigger oder Echtzeitauswertungen sind bislang nicht umgesetzt. In grösser angelegten Studien wäre es zudem sinnvoll, die Architektur um zusätzliche serverseitige Module zu erweitern, die eine engere Steuerung der Erhebungen erlauben. Solche Erweiterungen sind weniger prinzipielle Grenzen des Systems als vielmehr Ausdruck des Umfangs, der im Rahmen einer Bachelorarbeit realisierbar ist. Gleichzeitig verweisen diese Überlegungen auf grundlegende Spannungen, die jede Weiterentwicklung dieser Plattform mit sich bringt: Mehr Funktionalität bedeutet zunächst, dass Teilnehmende bereit sein müssen, zusätzliche Daten zu teilen -- und damit auch Kontrolle darüber abzugeben, wie diese Daten weiterverarbeitet werden. Gerade hier liegt ein kritischer Punkt: Daten, die im Rahmen einer Wissenschaftlichen Studie erhoben werden, sind immer mit Erwartungen und einem impliziten Vertrauensvorschuss verbunden, der wissenschaftlich legitimiert und geschützt werden muss.

Auch auf der Seite der Forschenden entstehen durch die Erhebung zusätzlicher Daten erhöhte Anforderungen. Mehr Daten bedeuten nicht automatisch mehr Erkenntnis, sondern bergen das Risiko, sensible oder besonders schutzbedürftige Bereiche zu berühren, die über das eigentliche Forschungsinteresse hinausgehen. Die Pilotstudie macht dies deutlich: So habe ich \gls{bspw} präzise Standortdaten erhoben, welche ich in der anschliessenden Analyse nicht verwende. Auch wenn diese Daten technisch sicher gespeichert sind, wurden sie mir damit ohne klaren Erkenntnisgewinn anvertraut. Der in der Befragung implizit suggerierte Nutzen kann so nicht eingelöst werden. Rückblickend wäre es notwendig, gegenüber den Teilnehmenden transparenter zu kommunizieren, dass es sich hier in erster Linie um eine explorative Studie handelt. Dieses Missverhältnis verdeutlicht, wie wichtig es ist, bereits vor Beginn einer Erhebung kritisch zu reflektieren, welche Daten tatsächlich benötigt werden -- und wie eng technische Gestaltung, methodische Entscheidungen und ethische Verantwortung miteinander verflochten sind.

\vspace{1em}

Im Entwicklungsprozess zeigt sich besonders deutlich das Spannungsfeld zwischen der offenen Logik von \gls{opensource}-Software und den geschlossenen Ökosystemen grosser Plattformbetreiber. Zwar steht der Quellcode von \gls{intermind} öffentlich zur Verfügung, die Distribution über App-Stores bleibt jedoch an intransparente und kommerziell geprägte Verfahren gebunden. Die Veröffentlichung im Apple App Store scheitert schliesslich an einer nur schwer nachvollziehbaren und intransparenten Ablehnung, während die Bereitstellung im Google Play Store zusätzliche Gebühren und aufwändige Prüfprozesse erfordert. Hinzu kommt, dass die Entwicklung der App selbst zwar aufwendig ist, aber der zeitliche und organisatorische Aufwand rund um die Veröffentlichung in den App Stores deutlich grösser ausfällt als erwartet – von Datenschutzrichtlinien und benötigten Webseiten bis hin zu sich ständig verändernden Store-Vorgaben und geforderten Updates. Rückblickend zeigt sich, dass dieser Teil des Projekts wesentlich mehr Ressourcen gebunden hat als die eigentliche Programmierung der App. Während ich den reinen Coding-Aufwand aufgrund meiner bisherigen Erfahrung im Vorfeld relativ gut einschätzen konnte, habe ich die zeitintensiven Prozesse der Distribution massiv unterschätzt.

Besonders deutlich wurde dies im Vorfeld der Pilotstudie: Da ich die einmalige Gelegenheit hatte, die App im Rahmen der Exkursion \enquote{Recht auf Stadt} mit einer ganzen Studierendengruppe zu testen, war der Termin klar vorgegeben und nicht verschiebbar. Damit die Erhebung durchgeführt werden konnte, musste die App rechtzeitig auf beiden Betriebssystemen verfügbar sein. In der unmittelbaren Vorbereitungsphase konzentrierte sich die Arbeit daher unerwartet stark auf die Veröffentlichung in den Stores, was erhebliche zeitliche und organisatorische Ressourcen band. Dieser Aufwand limitierte direkt die Zeit, die für eine sorgfältigere theoretische Fundierung und Ausarbeitung des Fragebogens zur Verfügung stand.

Viele im Prozess getroffene technische Entscheidungen reflektiere ich in dieser Arbeit nicht. Dadurch verlieren sie an Sichtbarkeit und Transparenz, obwohl sie methodisch wie epistemisch bedeutsam sind. Dass solche Entscheidungen im Text unsichtbar bleiben, macht zugleich ein grundlegendes Spannungsfeld sichtbar: Kritisch-sozialwissenschaftliche Ansprüche zielen auf Transparenz und Reflexion, während technische Erfordernisse oft pragmatische und situative Entscheidungen verlangen. Künftig wäre es wichtig, Wege zu finden, auch diese technischen Entscheidungen methodisch sichtbar zu machen, sei es durch begleitende Reflexion oder durch eine engere Verzahnung von Entwicklung und Dokumentation.

\section{Offenheit und Copyleft}

Die Veröffentlichung des Quellcodes schafft eine produktive Ambivalenz. Sie ermöglicht eine einfache Nachnutzung und eröffnet die Möglichkeit, dass andere auf der bestehenden Arbeit aufbauen. Gleichzeitig bedeutet sie, sich mit einer Arbeit sichtbar zu machen, die im Rahmen einer Bachelorarbeit entstanden ist und nicht den Anspruch auf Perfektion erheben kann. Anders als in der wissenschaftlichen Publikationspraxis gibt es hier keine formalisierte Qualitätssicherung -- der Code steht so zur Verfügung, wie er entwickelt wurde. Damit wird er auch beurteilbar, kritisierbar und in seiner Prozesshaftigkeit sichtbar. Diese Offenlegung erfordert eine Bereitschaft zur Exponierung, macht aber auch den Kern von \gls{opensource} aus: Der Wert liegt weniger in einem perfekten Endprodukt als in der Transparenz des Entwicklungsprozesses und in der Möglichkeit, dass andere darauf aufbauen.

Ein wesentlicher Teil dieser Offenlegung betrifft die Wahl der Lizenz. Sowohl \gls{intermind} als auch die vorliegende Arbeit sind unter \gls{lic:copyleft}-Lizenzen veröffentlicht (\gls{lic:agpl}~3.0 für den Code, \gls{lic:cc-by-sa}~4.0 für den Text). Copyleft-Lizenzen unterscheiden sich von anderen offenen Lizenzen darin, dass sie Offenheit nicht nur erlauben, sondern auch verpflichtend machen: Wer auf dieser Grundlage weiterentwickelt, muss seine eigenen Ableitungen wiederum unter einer offenen Lizenz zugänglich machen. Damit wird Offenheit als wechselseitiges Prinzip gedacht.

Auffällig ist dabei, dass gerade in der Research Software Engineering Community \gls{lic:copyleft}-Lizenzen besonders umstritten sind. Während sie im weiteren Open-Source-Kontext seit Jahrzehnten etabliert sind, dominieren in der Wissenschaft permissive Lizenzen wie MIT oder BSD \parencite{sethiWhyEarthAre2020}. Diese gelten als anschlussfähig, weil sie auch die Einbindung in proprietäre Kontexte ermöglichen. Die dahinterliegende Logik ist für mich jedoch nur begrenzt nachvollziehbar: Es bleibt unklar, warum es ein wissenschaftliches Ziel sein sollte, dass Forschungscode ohne wechselseitige Verpflichtung in kommerzielle Produkte integriert werden kann. Ich schliesse mich hier \textcite{sethiWhyEarthAre2020} an, der betont, dass diese ablehnende Haltung auf eine verkürzte Debatte verweist: Sie blendet die Frage aus, wie Offenheit nicht bloss als individuelle Grosszügigkeit verstanden werden kann, sondern als wissenschaftlich-kollektives Prinzip geteilter Verantwortung.

Mit dieser Arbeit wird ein Gegenentwurf zu bestehenden wissenschaftlichen Infrastrukturen erprobt: Nebst dem Quellcode ist auch die Arbeit selbst unter einer \gls{lic:copyleft}-Lizenz veröffentlicht. Damit wird zugleich auf eine Leerstelle aufmerksam gemacht: In wissenschaftlichen Publikationen sind \gls{lic:copyleft}-Lizenzen bislang praktisch inexistent. Gerade deshalb ist die Entscheidung bewusst als Intervention zu lesen, die die Spannungen in der aktuellen Praxis sichtbar macht.

\section{Methodische Lehren und Ausblick}

Die Verbindung von \gls{ema}/\gls{gema}-Methoden mit intersektionalen Mehrebenenmodellen (\gls{i-maihda}) setze ich als Zugang ein, der situiertes (Un-)Wohlbefinden im Verhältnis zu sozialen Positionierungen analysierbar macht. Alltagsräume verstehe ich machtkritisch als Gefüge, die soziale Positionierungen und ihre Überschneidungen (Intersektionen) zugleich widerspiegeln \emph{und} mit hervorbringen; sie sind nicht neutrale Kulissen, sondern an der Herstellung und Reproduktion von Machtverhältnissen beteiligt. 
Eine solche Quantifizierung muss aus einer machtkritischen Persektive immer mit qualitativen Verfahren verbunden werden: Die Modelle können verorten, \emph{wo} und \emph{für wen} Unterschiede auftreten, während qualitative Verfahren klären, \emph{wie} und \emph{warum} sie entstehen. Damit das möglich it muss aber während dem ganzen Prozess immer wieder kritisch hinterfragt werden, welche Elemente oder Ebenen tatsächlich relevant sind und welche dazugehörigen qualitativen Aspekte berücksichtigt werden müssen. Ich argumentiere daher explizit für einen Mixed\-Methods\-Ansatz.

Die Umsetzung in dieser Arbeit zeigt zugleich klare Grenzen. Ohne Pretests und mit nur teilweise validierten Skalen bleiben einzelne Items unscharf. Mit einer kleinen und relativ homogenen Stichprobe lassen sich die Potenziale von \gls{i-maihda} nur begrenzt zeigen: Strata sind unterbesetzt, Varianz innerhalb von Personen ist kaum abbildbar, robuste Effektschätzungen sind nicht realistisch. Ich lese diese Punkte als methodische Lehren: eine Studie die diesen hier entwickelten Ansatz umsetzt braucht ausreichend Teilnehmende, eine heterogenere Gesamtstichprobe und inhaltlich besser entwickelte, geprüfte Items, damit der Ansatz sein Potenzial entfaltet.

Perspektivisch ist eine Erweiterung auf \gls{gema} im engeren Sinn anzudenken: die Einbindung externer Kontextdaten (\gls{zb} hochaufgelöster Stadthitzedaten wie im \enquote{Bernometer} \parencite[siehe][]{burgerModellingSpatialPattern2021} oder punktueller Messungen der Umgebungslautstärke). Aus einer feministischen und intersektionalen Perspektive gehen damit aber Fragen der Messpolitik einher: Welche Wirklichkeiten mache ich durch Sensorik überhaupt sichtbar -- und welche blende ich aus? Welche Kategorien schreibe ich damit fest? Wer gewinnt Erkenntnis? Gibt es Risiken?

Entscheidend für eine erfolgreiche Studie mit dem in dieser Arbeit kombinierten Ansatz ist eine hohe Teilnahmestabilität, also eine grosse Anzahl an Erhebungen pro Person. Um dies zu erreichen, sind verschiedene Ansätze denkbar, die ausprobiert und wahrscheinlich auch kombiniert werden müssen. Einerseits ist die Forschungsfrage der durchgeführten Pilotstudie sehr allgemein gehalten. Eine stärkere lokale Einbettung, etwa in einem Quartier, und die Einbindung der Menschen vor Ort könnte die Relevanz für die Teilnehmenden erhöhen und sich dadurch positiv auf die Teilnahmebereitschaft auswirken. Ebenso denkbar ist eine aufbereitete Rückmeldung der eigenen Erhebungen für Teilnehmende, wie sie in \gls{urbanmind} umgesetzt ist.

Zusammenfassend zeige ich in dieser Bachelorarbeit, dass die Erhebung und Analyse von intersektionalem (Un-)Wohlbefinden im Stadtraum nicht nur eine methodische, sondern auch eine infrastrukturelle und politische Herausforderung darstellt. Mit \textit{InterMind} habe ich einen ersten, prototypischen Ansatz entwickelt, der digitale Offenheit, methodische Reflexivität und kritische Perspektiven miteinander verbindet. Auch wenn die empirische Basis begrenzt bleibt und zahlreiche Weiterentwicklungen notwendig sind, liegt der Beitrag dieser Arbeit darin, ein Forschungsfeld zu skizzieren und ein Werkzeug bereitzustellen, das künftige Arbeiten aufgreifen und weiterführen können. Damit versteht sich die Arbeit weniger als abgeschlossene Antwort, sondern als Einladung, die hier angestossenen Fragen weiterzudenken, kritisch zu vertiefen und in zukünftigen Arbeiten weiterzuentwickeln.

% Unbedingt visualisierung diskutieren als limitation dieser arbeit und grosse herausforderung kommender arbeiten. Emotional mapping etc.
