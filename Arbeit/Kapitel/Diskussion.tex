% LTeX: language=de-CH
\chapter{Diskussion} \label{sec:diskussion}

\section{Einordnung der Arbeit im Forschungsfeld}

Ausgangspunkt dieser Arbeit ist die Frage, wie sich der Einfluss räumlicher Umgebungen auf das affektive (Un-)Wohlbefinden \glslink{intersektionalitaet}{intersektional} positionierter Personen erfassen und analysieren lässt. Eine abschliessende Beantwortung dieser komplexen Leitfrage kann im Rahmen dieser Bachelorarbeit nicht geleistet werden. Stattdessen unternehme ich einen ersten Schritt, ein Forschungsdesign zu entwickeln, das einen Beitrag zu dieser übergeordneten Fragestellung leisten kann.

Im Verlauf der Arbeit zeige ich auf, dass ein geeigneter Erhebungsansatz weit mehr umfasst als die Auswahl methodischer Verfahren. Transparenz, Nachvollziehbarkeit und eine kritische Reflexion der zugrunde liegenden Infrastruktur sind ebenso zentral wie die konkrete Gestaltung der Erhebung. Methodisch erweist sich die Verbindung von \gls{ema}/\gls{gema}-Methoden mit einer intersektionalen Mehrebenenanalyse als vielversprechend, auch wenn Spannungen in der Übersetzung intersektionaler Theorie in quantitative Verfahren bestehen bleiben. Das hier entwickelte Forschungsdesign kann somit als ein möglicher, wenn auch nicht spannungsfreier Weg verstanden werden, affektives (Un-)Wohlbefinden in seiner sozialen und räumlichen Einbettung quantitativ zu erfassen.

Mit der Entwicklung und Veröffentlichung einer eigenen \gls{opensource}-Erhebungsplattform versuche ich, Prinzipien wie Nachvollziehbarkeit, Offenheit und Reflexivität praktisch umzusetzen. Auch wenn \gls{intermind} in seiner jetzigen Form nur begrenzt anschlussfähig ist, verdeutlicht das Projekt, dass wissenschaftliche Werkzeuge nicht zwingend an proprietäre Systeme gebunden sein müssen, sondern auch im Sinne einer offenen, gemeinschaftsorientierten Infrastruktur realisierbar sind.

In \cref{sec:pilotstudie} zeige ich, dass die erhobenen Pilotdaten für eine \glslink{intersektionalitaet}{intersektionale} Mehrebenenanalyse nur eingeschränkt geeignet sind. Zwar lassen sich Modellierungen prinzipiell durchführen, die kleine und sehr homogene Stichprobe verhindert jedoch belastbare Ergebnisse. Manche Strata sind so klein, dass eine intrapersonelle Ebene nicht abgebildet werden kann. Hinzu kommt, dass die inhaltlich nicht konsequent theoriegeleitete Entwicklung des Fragebogens, die Datengrundlage zusätzlich schwächt. Entsprechend ist die Überprüfung der Eignung der Analysemethode mit den Daten aus der Pilotstudie nur teilweise möglich.

Ich sehe den Beitrag dieser Arbeit nicht primär in belastbaren empirischen Resultaten, sondern in der Entwicklung und Erprobung eines durchgängig kritisch ausgerichteten Forschungsdesigns, das Theorie, Methodenwahl, Erhebungsinstrument, Fragebogen und Analyse als zusammenhängendes Ganzes verbindet. Dieser Ansatz macht Spannungen sichtbar -- zwischen Transparenz und Praktikabilität, zwischen Offenheit und Anschlussfähigkeit, zwischen theoretischer Schärfe und empirischer Umsetzbarkeit. Gerade dadurch wird deutlich, dass eine \glslink{intersektionalitaet}{intersektionale} Betrachtung von affektivem (Un-)Wohlbefinden nicht allein auf der Ebene von Items und Skalen beantwortet werden kann, sondern auch grundlegende infrastrukturelle und methodologische Entscheidungen umfasst. In dieser doppelten Hinsicht -- als methodisches Forschungsdesign wie auch als theoretisch-praktische Rahmung -- versteht sich die Arbeit als Beitrag zu einer kritisch-feministischen Geographie, die methodische, digitale und intersektionale Fragen nicht getrennt behandelt, sondern gemeinsam denkt -- auch wenn dies gezwungenermassen Brüche und Unvollständigkeiten mit sich bringt.

% TODO: Problemabsatz

Meine Motivation für diese Arbeit ist auch in einer persönlichen Haltung begründet. Ich sehe in der Konzentration digitaler Infrastrukturen bei einigen wenigen gigantischen Konzernen ein gesamtgesellschaftliches Problem, das Abhängigkeiten und Intransparenzen schafft und Ungleichheiten verstärkt. Gleichzeitig nutze ich selbst täglich Geräte und Dienste dieser Konzerne. Diese Verflechtung von Kritik und Abhängigkeit ist sinnbildlich für meinen Zugang: Sie macht deutlich, wie schwer es ist, konsequent offene Infrastrukturen umzusetzen, und erklärt zugleich, warum mir dieses Anliegen wichtig ist. Offenheit und Nachvollziehbarkeit im sinne digitaler Souveränität sind nicht nur abstrakte Prinzipien, sondern eine Antwort auf diese Spannung. Diese Spannung zwischen wissenschaftlichem Anspruch und praktischer Machbarkeit zieht sich durch die gesamte Arbeit -- sie erklärt, warum an einigen Stellen Konzepte nur skizziert, Verfahren nicht voll validiert oder Entscheidungen situativ getroffen wurden. Diese Brüche verstehe ich als typische Dynamik des in der Geographie verbreiteten interdisziplinären Arbeitens, bei dem wissenschaftlicher Anspruch und praktische Machbarkeit ständig neu austariert werden müssen.

\section{Potenziale und Grenzen des entwickelten Instruments}

Das im Rahmen dieser Arbeit entwickelte Instrument erweist sich einerseits als funktionales, transparentes und datenschutzfreundliches Werkzeug, und zeigt andererseits auch klare Grenzen auf. Ein zentrales Potenzial liegt in der Modularität und Anpassbarkeit: Das Grundgerüst von \textit{InterMind} kann mit überschaubarem Aufwand für andere Fragebögen genutzt und durch zusätzliche Komponenten erweitert werden. Damit entsteht ein flexibles System, das nicht auf ein einzelnes Forschungsszenario beschränkt ist, sondern prinzipiell in ganz unterschiedlichen Kontexten eingesetzt werden kann. Die Entscheidung, den Code offen zu veröffentlichen, stärkt diese Anschlussfähigkeit zusätzlich und macht den Entwicklungsprozess nachvollziehbar.

Gleichzeitig wird deutlich, dass die Plattform für bestimmte Forschungsszenarien noch ergänzt werden müsste. Funktionen wie eine Offline-Nutzung, standortbasierte Trigger oder Echtzeitauswertungen sind bislang nicht umgesetzt. In grösser angelegten Studien wäre es zudem sinnvoll, die Architektur um zusätzliche serverseitige Module zu erweitern die eine engere Steuerung der Erhebungen erlauben. Solche Erweiterungen sind weniger prinzipielle Grenzen des Systems als vielmehr Ausdruck des Umfangs, der im Rahmen einer Bachelorarbeit realisierbar ist. Gleichzeitig verweisen diese Überlegungen auf grundlegende Spannungen, die jede Weiterentwicklung mit sich bringt. Mehr Funktionalität bedeutet zunächst, dass Teilnehmende bereit sein müssen, zusätzliche Daten zu teilen -- und damit auch Kontrolle darüber abzugeben, wie diese Daten weiterverarbeitet werden. Gerade hier liegt ein kritischer Punkt: Daten, die im Rahmen einer Wissenschaftlichen Studie erhoben werden, sind immer mit Erwartungen und einem impliziten Vertrauensvorschuss verbunden, der wissenschaftlich legitimiert und geschützt werden muss.

Auch auf der Seite der Forschenden entstehen durch die Erhebung zusätzlicher Daten erhöhte Anforderungen. Mehr Daten bedeuten nicht automatisch mehr Erkenntnis, sondern bergen das Risiko, sensible oder besonders schutzbedürftige Bereiche zu berühren, die über das eigentliche Forschungsinteresse hinausgehen. Die Pilotstudie macht dies deutlich: So habe ich \gls{bspw} präzise Standortdaten erhoben, welche ich in der anschliessenden Analyse nicht verwende. Auch wenn diese Daten technisch sicher gespeichert sind, wurden sie mir damit ohne klaren Erkenntnisgewinn anvertraut. Der in der Befragung implizit suggerierte Nutzen kann so nicht eingelöst werden. Rückblickend wäre es notwendig, gegenüber den Teilnehmenden transparenter zu kommunizieren, dass es sich hier in erster Linie um eine explorative Studie handelt. Dieses Missverhältnis verdeutlicht, wie wichtig es ist, bereits vor Beginn einer Erhebung kritisch zu reflektieren, welche Daten tatsächlich benötigt werden -- und wie eng technische Gestaltung, methodische Entscheidungen und ethische Verantwortung miteinander verflochten sind.

\vspace{1em}

Im Entwicklungsprozess zeigt sich besonders deutlich das Spannungsfeld zwischen der offenen Logik von \gls{opensource}-Software und den geschlossenen Ökosystemen grosser Plattformanbieter. Zwar steht der Quellcode von \gls{intermind} öffentlich zur Verfügung, die Distribution über App-Stores bleibt jedoch an intransparente und kommerziell geprägte Verfahren gebunden. Die Veröffentlichung im Apple App Store scheitert schliesslich an einer undurchsichtigen Ablehnung, während die Bereitstellung im Google Play Store zusätzliche Gebühren und aufwändige Prüfprozesse erfordert. Hinzu kommt, dass die Entwicklung der App selbst zwar aufwendig ist, aber der zeitliche und organisatorische Aufwand rund um die Veröffentlichung in den App Stores deutlich grösser ausfällt als erwartet -- von Datenschutzrichtlinien und benötigten Webseiten bis hin zu sich ständig verändernden Store-Vorgaben und geforderten Updates. Aus diesem Grund habe ich mich entschlossen die App nicht dauerhaft in den Stores verfügbar zu halten: Da aktuell keine Datenerhebungen durchgeführt wird und die fortlaufende Pflege zusätzliche Ressourcen erfordern würde, ist eine Weiterführung in dieser Form nicht sinnvoll. Dies schliesst aber eine Wiederverwendung der App nicht aus.

Die Veröffentlichung des Quellcodes schafft zugleich eine produktive Ambivalenz. Sie ermöglicht eine einfache Nachnutzung und eröffnet die Möglichkeit, dass andere auf der bestehenden Arbeit aufbauen. Gleichzeitig bedeutet sie, sich mit einer Arbeit sichtbar zu machen, die im Rahmen einer Bachelorarbeit entstanden ist und nicht den Anspruch auf Perfektion erheben kann. Anders als in der wissenschaftlichen Publikationspraxis gibt es hier keine formalisierte Qualitätssicherung -- der Code steht so zur Verfügung, wie er entwickelt wurde. Damit wird er auch beurteilbar, kritisierbar und in seiner Prozesshaftigkeit sichtbar. Diese Offenlegung erfordert eine Bereitschaft zur Exponierung, macht aber auch den Kern von \gls{opensource} aus: Der Wert liegt weniger in einem perfekten Endprodukt als in der Transparenz des Entwicklungsprozesses und in der Möglichkeit, dass andere darauf aufbauen.

Obwohl alle mit allen in dieser Arbeit vorgestellten digitalen Wissenschaftsinfrastrukturen Daten gewonnen wurden welche in Open Access Publikationen verwertet wurden, ist keine der Plattformen quelloffen. Hier entsteht ein grundlegendes Spannungsverhältnis: Während wissenschaftliche Publikationen der öffentlichen Nachvollziehbarkeit verpflichtet sind, bleibt die technische Infrastruktur, auf der sie basieren, dieser genauen Prüfung entzogen. Mit \gls{intermind} liegt zumindest ein kleiner Gegenentwurf vor, der Forschungsinfrastruktur selbst transparent, reproduzierbar und offen gestaltet.

\section{Methodische Lehren und Ausblick}

Die Verbindung von \gls{ema}/\gls{gema}-Methoden mit intersektionalen Mehrebenenmodellen (\gls{i-maihda}) setze ich als Zugang ein, der affektives (Un-)Wohlbefinden situativ, räumlich und im Verhältnis zu sozialen Positionierungen analysierbar macht. Alltagsräume verstehe ich machtkritisch als Gefüge, die soziale Positionierungen und ihre Überschneidungen (Intersektionen) zugleich widerspiegeln \emph{und} mit hervorbringen; sie sind nicht neutrale Kulissen, sondern an der Herstellung und Reproduktion von Machtverhältnissen beteiligt. 
Eine solche Quantifizierung muss aus einer machtkritischen Persektive immer mit qualitativen Verfahren verbunden werden: Die Modelle können verorten, \emph{wo} und \emph{für wen} Unterschiede auftreten, während qualitative Verfahren klären, \emph{wie} und \emph{warum} sie entstehen. Damit das möglich it muss aber während dem ganzen Prozess immer wieder kritisch hinterfragt werden, welche Elemente oder Ebenen tatsächlich relevant sind und welche dazugehörigen qualitativen Aspekte berücksichtigt werden müssen. Ich argumentiere daher explizit für einen Mixed\-Methods\-Ansatz.

Die Umsetzung in dieser Arbeit zeigt zugleich klare Grenzen. Ohne Pretests und mit nur teilweise validierten Skalen bleiben einzelne Items unscharf. Mit einer kleinen und relativ homogenen Stichprobe lassen sich die Potenziale von \gls{i-maihda} nur begrenzt zeigen: Strata sind unterbesetzt, Varianz innerhalb von Personen ist kaum abbildbar, robuste Effektschätzungen sind nicht realistisch. Ich lese diese Punkte als methodische Lehren: eine Studie die diesen hier entwickelten Ansatz umsetzt braucht ausreichend Teilnehmende, eine heterogenere Gesamtstichprobe und inhaltlich besser entwickelte, geprüfte Items, damit der Ansatz sein Potenzial entfaltet.

Perspektivisch ist eine Erweiterung auf \gls{gema} im engeren Sinn anzudenken: die Einbindung externer Kontextdaten (\gls{zb} hochaufgelöster Stadthitzedaten wie im \enquote{Bernometer} \parencite[siehe][]{burgerModellingSpatialPattern2021} oder punktueller Messungen der Umgebungslautstärke). Aus einer feministischen und intersektionalen Perspektive gehen damit aber Fragen der Messpolitik einher: Welche Wirklichkeiten mache ich durch Sensorik überhaupt sichtbar -- und welche blende ich aus? Welche Kategorien schreibe ich damit fest? Wer gewinnt Erkenntnis? Gibt es Risiken?

Entscheidend für eine erfolgreiche Studie mit dem in dieser Arbeit kombinierten Ansatz ist eine hohe Teilnahmestabilität, also eine grosse Anzahl an Erhebungen pro Person. Um dies zu erreichen, sind verschiedene Ansätze denkbar, die ausprobiert und wahrscheinlich auch kombiniert werden müssen. Einerseits ist die Forschungsfrage der durchgeführten Pilotstudie sehr allgemein gehalten. Eine stärkere lokale Einbettung, etwa in einem Quartier, und die Einbindung der Menschen vor Ort könnte die Relevanz für die Teilnehmenden erhöhen und sich dadurch positiv auf die Teilnahmebereitschaft auswirken. Ebenso denkbar ist eine aufbereitete Rückmeldung der eigenen Erhebungen für Teilnehmende, wie sie in \gls{urbanmind} umgesetzt ist.

Schliesslich bleibt die Frage der Darstellung der Ergebnisse offen. Wo räumliche und soziale Muster überlagert sind, kommen klassiche kartierungsmethoden an ihre Grenzen. Ich sehe Bedarf an kartographischen und zeitbezogenen Visualisierungen, die Vielschichtigkeit sichtbar machen -- etwa facettierte Karten nach Strata, zeitliche Verläufe pro Person sowie dichte Darstellungen statt isolierter Punkte. Ansätze wie in \textcite{friebelThermischesEmpfindenUnd2024} illustrieren, wie sich mehrschichtige, räumlich verteilte Daten präziser kartieren lassen.

Zusammenfassend zeige ich in dieser Bachelorarbeit, dass die Erhebung und Analyse von intersektionalem (Un-)Wohlbefinden im Stadtraum nicht nur eine methodische, sondern auch eine infrastrukturelle und politische Herausforderung darstellt. Mit \textit{InterMind} habe ich einen ersten, prototypischen Ansatz entwickelt, der digitale Offenheit, methodische Reflexivität und kritische Perspektiven miteinander verbindet. Auch wenn die empirische Basis begrenzt bleibt und zahlreiche Weiterentwicklungen notwendig sind, liegt der Beitrag dieser Arbeit darin, ein Forschungsfeld zu skizzieren und ein Werkzeug bereitzustellen, das künftige Arbeiten aufgreifen und weiterführen können. Damit versteht sich die Arbeit weniger als abgeschlossene Antwort, sondern als Einladung, die hier angestossenen Fragen weiterzudenken, kritisch zu vertiefen und in zukünftigen Arbeiten weiterzuentwickeln.



% Die Entwicklung von \textit{InterMind} war im Rahmen dieser Arbeit nicht nur ein technisches, sondern auch ein methodisches Experiment. Ziel war es, ein Werkzeug zu schaffen, das situative, geolokalisierte Erhebungen zuverlässig durchführen kann -- und dabei die Grundprinzipien von Transparenz, Datenschutz und Anpassungsfähigkeit wahrt. Die in \cref{sec:entwicklung_app} skizzierten technischen Entscheidungen waren dabei stets auch methodische Abwägungen: Sie bestimmten nicht nur, wie die App funktioniert, sondern auch, welche Formen der Datenerhebung und -auswertung überhaupt möglich waren.
% 
% Besonders prägend war die Wahl eines bewusst reduzierten, clientseitig gesteuerten Systemdesigns. Diese Architektur minimierte Abhängigkeiten von externer Infrastruktur, reduzierte potenzielle Datenschutzrisiken und erlaubte eine transparente, vollständig nachvollziehbare Funktionsweise. Gleichzeitig bedeutete sie den Verzicht auf Funktionen, wie sie in komplexeren GEMA-Implementierungen üblich sind -- etwa geofence-basierte Trigger oder serverseitige Kontextlogiken. Dadurch blieb die App methodisch auf feste, vordefinierte Erhebungszeitpunkte beschränkt und konnte nicht adaptiv auf räumliche oder kontextuelle Veränderungen reagieren. Für explorative Pilotstudien wie die vorliegende war dies ausreichend, in längerfristigen oder gross angelegten Projekten wäre jedoch eine dynamischere, kontextsensitivere Architektur wünschenswert.
% 
% Die Entscheidung zur Open-Source-Veröffentlichung stellt einen zentralen Bestandteil des Projekts dar. Sie ermöglicht anderen Forschenden nicht nur die Nachnutzung des Codes, sondern schafft auch die Grundlage für kollaborative Weiterentwicklungen. Gleichzeitig machte die Erfahrung mit den App-Store-Gatekeeping-Prozessen deutlich, dass Offenheit allein keine Garantie für breite Zugänglichkeit ist: Die Distribution über zentrale Plattformen bleibt an kommerzielle und intransparente Strukturen gebunden, die auch nicht-kommerzielle, wissenschaftliche Projekte einschränken können. Hier zeigt sich ein strukturelles Spannungsfeld zwischen der offenen, gemeinschaftsorientierten Logik von Open-Source-Software und den geschlossenen, marktkontrollierten Ökosystemen der grossen Plattformanbieter.
% 
% Im Rückblick wird deutlich, dass die App-Entwicklung in dieser Form einerseits ein funktionierendes, forschungsnahes Werkzeug hervorgebracht hat, andererseits aber auch klare Grenzen aufweist. Diese liegen weniger in der Stabilität oder Bedienbarkeit, sondern vielmehr in der eingeschränkten Kontextanpassung, der fehlenden Echtzeitauswertung und der aufwändigen Anpassbarkeit für andere Forschungssettings. Zukünftige Iterationen könnten hier ansetzen -- etwa durch die Ergänzung serverseitiger Module, die Entwicklung eines webbasierten Dashboards für Monitoring und Feedback, oder die modularisierte Integration zusätzlicher Erhebungsmethoden.
% 
% Damit verdeutlicht \textit{InterMind} sowohl die Chancen als auch die Grenzen einer eigenständigen Entwicklung im Rahmen einer Abschlussarbeit: Sie eröffnet Handlungsspielräume, schafft technologische Unabhängigkeit im Entwicklungsprozess und macht Forschungsinfrastruktur transparent -- bleibt aber eingebettet in grössere, teils restriktive Strukturen, die den Handlungsspielraum letztlich mitbestimmen.
% 
% \section{Reflexion und Weiterentwicklungspotenzial des Fragebogens}
% 
% Der entwickelte Fragebogen erwies sich im Feld als grundsätzlich funktional und gut in den Ablauf der Studie integrierbar. Er erfüllte die Anforderung, situative Erhebungen in kurzer Zeit und mit geringer Belastung für die Teilnehmenden durchführen zu können. Gleichzeitig zeigte sich jedoch, dass diese Stärken teilweise mit methodischen Einbussen erkauft wurden, die den wissenschaftlichen Anspruch der Erhebung begrenzen.
% 
% Besonders deutlich wird dies bei der Auswahl der Items zur Erfassung situativen affektiven Wohlbefindens. Die gewählten Dimensionen -- darunter „generelles Wohlbefinden“, Zufriedenheit, Anspannung, Energie und Zugehörigkeit -- erlaubten zwar eine kompakte Erfassung, entstanden jedoch nicht aus einer stringenten theoretischen Modellierung heraus. Diese pragmatische Herangehensweise erleichterte zwar die Umsetzung im Rahmen einer Mehrfacherhebung, führte aber zu einer geringeren konzeptuellen Schärfe und erschwerte den direkten Vergleich mit bestehenden Studien.
% 
% Auch der Verzicht auf etablierte standardisierte Skalen hatte ambivalente Folgen. Er trug dazu bei, den Fragebogen schlank zu halten und die Akzeptanz bei den Teilnehmenden zu erhöhen, schränkte jedoch die Vergleichbarkeit der Daten und ihre Anschlussfähigkeit an bestehende Forschungsinstrumente ein. Eine gekürzte, modulare Integration validierter Skalen hätte hier einen Ausgleich zwischen Praktikabilität und methodischer Robustheit schaffen können.
% 
% Die mehrsprachige Umsetzung des Instruments war ein wichtiger Schritt in Richtung Zugänglichkeit, blieb jedoch ohne formalisierte Validierung durch muttersprachliche Expert:innen. Dadurch ist nicht auszuschliessen, dass inhaltliche Nuancen, insbesondere bei affektiven Zustandsbeschreibungen, zwischen den Sprachversionen leicht variierten. Diese Unsicherheiten verstärkten sich bei sensiblen Konzepten wie \gls[noindex]{race}, für das im deutschsprachigen Kontext keine etablierten, diskriminierungssensiblen Kategorien verfügbar sind. Die gewählte Operationalisierung über Geburts- und Aufenthaltsland senkte zwar die Erhebungsbarrieren, konnte die Komplexität rassifizierter Erfahrungen jedoch nur unvollständig erfassen.
% 
% Schliesslich war der Entwicklungsprozess des Fragebogens zwar iterativ angelegt und von kontinuierlichem Feedback begleitet, basierte jedoch nicht auf einem formalen Pretest mit einer breiten und divers zusammengesetzten Testgruppe. Dadurch wurden potenzielle Verständnisschwierigkeiten oder kulturelle Unschärfen nur in begrenztem Umfang sichtbar.
% 
% Insgesamt bleibt festzuhalten, dass der Fragebogen in seiner vorliegenden Form eine praktikable, aber methodisch eingeschränkte Lösung darstellt. Für zukünftige Studien bieten sich mehrere Ansatzpunkte zur Weiterentwicklung: eine engere theoretische Anbindung der Items, die gezielte Integration gekürzter validierter Skalen, ein systematischeres Übersetzungs- und Validierungsverfahren sowie umfassendere Pretests. Auf diese Weise liesse sich die inhaltliche Aussagekraft der Erhebung stärken, ohne die für hochfrequente Befragungen notwendige Niedrigschwelligkeit aufzugeben.
% 
% \section{Gedanken für weiterführende Forschung}

% \subsection{Verbesserungsvorschläge zur Erhöhung der Teilnahmequote}

% \subsection{Optimierung der intersektionalen Datenerhebung und Analyse}

% \subsection{Integration qualitativer Verfahren}

% Standorterhebung nicht verwendet. damit eigentlich ema und nicht gema gemacht. da ich das alles auch nicht noch verknüpft habe mit weiteren indikatoren wie bspw temperatur oder so.


% ohne technische details mal hier rüberkopiert:

% \section{Eigenständig, aber nicht unabhängig -- Entwicklung im Plattformzeitalter}

% Die Entwicklung von \gls{intermind} war mein erstes grösseres Projekt in \gls{typescript} und mit \gls{reactnative}. Die Umstellung vom strikt objektorientierten Denken in \gls{java} auf den dynamischeren, komponentenbasierten Ansatz war anspruchsvoll, aber enorm lehrreich. Insbesondere das konsequente Anwenden der \gls{solid}-Prinzipien half dabei, die Struktur der Anwendung nachvollziehbar zu halten -- gerade in einem neuen Ökosystem. Die App funktioniert stabil, sieht gut aus, und hat ihren Zweck erfüllt.

% Trotz einer bewussten Orientierung an Prinzipien wie \gls{solid} und einem grundlegenden Architekturkonzept zeigte sich im Verlauf der Entwicklung, dass eine noch systematischere Auseinandersetzung mit der Softwarearchitektur hilfreich gewesen wäre. Zwar wurde auf eine modulare Struktur geachtet, viele Designentscheidungen wurden jedoch eher situativ getroffen und nicht im Sinne eines übergeordneten Gesamtdesigns immer wieder überprüft. Gerade im weiteren Projektverlauf wäre es sinnvoll gewesen, gezielt zu früheren architektonischen Überlegungen zurückzukehren und diese zu reflektieren oder anzupassen.

% Methoden wie \emph{Test-Driven Development} hätten diesen Prozess zusätzlich stützen können, indem sie klare Schnittstellen und Verantwortlichkeiten frühzeitig erzwingen. Auch der Aufbau automatisierter Tests und eine kontinuierlich integrierte Codeanalyse hätten dazu beigetragen, Fehlerquellen frühzeitig zu identifizieren und die langfristige Wartbarkeit der Anwendung zu verbessern. Viele kleinere Schwächen im Code wurden zwar pragmatisch behoben, ein strukturierteres Qualitätsmanagement hätte jedoch die Notwendigkeit späterer Refactoring-Prozesse deutlich reduziert.

% In diesem Sinne reiht sich die App auch in eine typische Dynamik vieler \gls{opensource}-Projekte ein: Sie wurde aus einem konkreten Forschungsbedarf heraus entwickelt, funktioniert zuverlässig, ist öffentlich dokumentiert -- aber nicht in jedem Teilbereich optimal strukturiert. Durch die Offenlegung des Quellcodes besteht jedoch die Möglichkeit, dass andere Entwickler\genderstern innen auf dieser Grundlage aufbauen, Verbesserungsvorschläge einbringen oder eigene Erweiterungen umsetzen.

% \vspace{1em}

% Trotz stabiler Funktionalität und durchdachter Grundstruktur weist das entwickelte System klare Begrenzungen auf -- insbesondere im Hinblick auf die situative Reaktionsfähigkeit und Kontextanpassung. So verzichtet \gls{intermind} bewusst auf kontinuierliches Geotracking, automatisierte Trigger oder serverseitige Kontextlogiken, wie sie in anderen \glsxtrshort{gema}-Systemen Anwendung finden.

% Ein Beispiel dafür bietet das im Rahmen einer kanadischen Studie zu Nationalparks entwickelte \glsxtrshort{health}-Plattform \parencite{wrayHealthyEnvironmentsActive2025}. Die Dokumentation zu diesem Tool ist erst während der Entstehung dieser Arbeit als Preprint veröffentlicht worden. Die App wird derzeit exklusiv im Rahmen des \textit{ParkSeek}-Projekts\footnote{\href{https://parkseek.ca/}{parkseek.ca}} eingesetzt und ist nicht öffentlich zugänglich. Ihre zugrundeliegende Systemarchitektur erlaubt eine kontinuierliche Standorterfassung und serverseitige Kontextverarbeitung, wodurch komplexe Logiken wie geofence-basierte Trigger umgesetzt werden können. So lassen sich etwa Benachrichtigungen auslösen, wenn sich Teilnehmende über längere Zeit in spezifischen Umwelten aufhalten. Diese technisch anspruchsvolle Lösung erlaubt eine besonders enge Verzahnung zwischen räumlichem Verhalten und situativer Befragung, geht jedoch mit einem hohen Aufwand sowie erheblichen Anforderungen an Datenschutz, Datenmanagement und Infrastruktur einher.

% Im Rahmen eines Bachelorprojekts wäre die Implementierung eines derart umfassenden Systems weder zeitlich noch organisatorisch realistisch gewesen. Stattdessen wurde ein datensparsamer, clientseitig gesteuerter Ansatz gewählt, der mit begrenzten Mitteln eine funktionale, transparente und reflektierte Umsetzung ermöglicht. Die Entscheidung für ein reduziertes Systemdesign war damit nicht nur eine Frage des Aufwands, sondern auch ein bewusster Kompromiss zugunsten von Kontrollierbarkeit und Datenschutz.

% Eine weitere Limitation des aktuellen Systemdesigns liegt im Fehlen eines serverseitigen Dashboards oder einer integrierten Auswertungsoberfläche. Es besteht keine Möglichkeit, Rückmeldungen in Echtzeit zu visualisieren, aggregierte Antworten einzusehen oder Monitoring-Funktionen während der Erhebung zu nutzen. Solche Features wären insbesondere für die Steuerung längerer Erhebungsphasen, die Qualitätssicherung oder für Feedbackschleifen mit den Teilnehmenden von Vorteil gewesen. Ihre Umsetzung hätte jedoch zusätzliche Entwicklungsressourcen sowie eine komplexere Backend-Architektur vorausgesetzt. Gleichwohl bleibt die Möglichkeit bestehen, entsprechende Funktionen in zukünftigen Iterationen oder auf Basis der veröffentlichten Codebasis nachzurüsten.

% \vspace{1em}

% Die Erfahrungen rund um die Veröffentlichung in den App Stores hat zentrale Spannungsfelder digitaler Infrastruktur deutlich gemacht. Obwohl die App funktional einsatzbereit war und über TestFlight bzw. den Play Store zugänglich gemacht werden konnte, blieb die reguläre Veröffentlichung im Apple App Store aufgrund einer intransparenten Ablehnung verwehrt.

% Solche Prozesse offenbaren strukturelle Abhängigkeiten, die weit über dieses Projekt hinausgehen: Zwei gigantische multinationale Tech-Konzerne kontrollieren in weiten Teilen den Zugang zu digitaler Infrastruktur. Dabei wirken sie zugleich als Regelsetzer, Infrastrukturbetreiber und ökonomische Gatekeeper. Diese doppelte Rolle ist nicht demokratisch legitimiert, aber mit erheblicher Lenkungsmacht verbunden. Gerade nicht-kommerzielle, experimentelle oder aktivistische Projekte sind von diesen Kontrollmechanismen besonders betroffen, da sie sich nicht ohne Weiteres den geforderten Verwertungslogiken oder Standardprozessen unterwerfen.

% Auch wenn Open-Source-Prinzipien allein diese strukturellen Hürden nicht auflösen können, war die Entscheidung zur Veröffentlichung des Quellcodes dennoch zentral: Sie schafft Transparenz, ermöglicht Weiterentwicklung und signalisiert ein bewusstes Gegenmodell zu proprietären, intransparenten Systemen. Gleichzeitig offenbart sich hier ein grundlegendes Spannungsfeld: Die offene, zugängliche und gemeinschaftsorientierte Logik von \gls{opensource}-Software steht in einem scharfen Kontrast zu den geschlossenen, marktkontrollierten Strukturen kommerzieller Distributionsplattformen. Wer eine App entwickeln und öffentlich zugänglich machen will, ist faktisch gezwungen, sich diesen Plattformen zu unterwerfen.

% Die Arbeit an der App war dabei nicht nur funktional motiviert, sondern auch von der Erfahrung getragen, ein eigenes digitales Werkzeug gestalten zu können -- mit all seinen Herausforderungen, aber auch mit dem unmittelbaren Lernerfolg und der Freude am konkreten Entstehungsprozess. Gerade vor dem Hintergrund einer zunehmend von privatwirtschaftlichen Plattformen dominierten digitalen Infrastruktur bleibt die Fähigkeit, eigene Werkzeuge zu entwickeln, ein wichtiger Akt technischer Aneignung.

% auch hier nicht nochmal entwickelnd erklären, je nachdem acuh ganz sein lassen

% \section{Klar, verständlich, iterativ -- Der Weg zum finalen Fragebogen}

% Die sprachliche Gestaltung der Fragebogen-Items stellte im Entwicklungsprozess eine zentrale methodische Herausforderung dar. Ziel war es, die Befragung möglichst zugänglich, verständlich und gleichzeitig inhaltlich präzise zu gestalten. Da die Befragung explizit auf eine intersektionale Analyse abzielt, wurde besonderer Wert darauf gelegt, die sprachliche Zugänglichkeit möglichst breit zu gewährleisten. Folglich wurde der Fragebogen bewusst mehrsprachig konzipiert und auf Deutsch, Englisch sowie Französisch umgesetzt. Weitere Sprachversionen wären zwar aus Sicht der intersektionalen Zugänglichkeit wünschenswert gewesen, scheiterten jedoch am hohen Aufwand für qualitativ hochwertige und inhaltlich konsistente Übersetzungen.

% Ein grundsätzliches Anliegen war eine möglichst direkte, adressierende Sprache in der \enquote{Du}-Form, um einen niederschwelligen Zugang zur Befragung zu fördern und hierarchische Distanz zwischen Forschenden und Teilnehmenden zu reduzieren. Gleichzeitig mussten die Formulierungen prägnant, alltagsnah und schnell erfassbar sein, da insbesondere die situativen Erhebungen kurz gehalten werden sollten. Hier ergab sich ein methodischer Balanceakt: Einerseits sollte die Befragung leicht verständlich bleiben, andererseits mussten komplexe Konzepte in zugänglicher Sprache operationalisiert werden. So wurde \gls{bspw} das Konzept der \gls{intersektionalitaet} im Einführungsteil des Fragebogens erläutert, danach jedoch bewusst vermieden, um unnötige Barrieren zu reduzieren. Stattdessen wurden alternative Formulierungen wie \enquote{persönliche Merkmale} verwendet, die jedoch teilweise inhaltliche Unschärfen mit sich brachten. 

% Besonders deutlich wurde diese Herausforderung im Umgang mit dem Konzept \gls[noindex]{race}. Gerade im deutschsprachigen Kontext existieren hier nur schwer geeignete Begrifflichkeiten: Formulierungen wie \enquote{Rasse} oder \enquote{Ethnizität} sind entweder sprachlich ungebräuchlich, problematisch oder stark mit kolonialen und biologistischen Zuschreibungen assoziiert \parencite[\gls{vgl}][]{roigIntersectionalityEuropeDepoliticized2018}. Alternativ verwendete Begriffe wie \enquote{Herkunft} oder \enquote{Aussehen} sind wiederum unpräzise und greifen die Dimension rassifizierter Diskriminierung nur unvollständig auf.

% Die Übersetzung der Items erfolgte nicht wörtlich, sondern sinngemäss, wobei insbesondere bei affektiven Zustandsbeschreibungen semantische Abstimmungen zwischen den Sprachversionen vorgenommen wurden. Dabei wurden auch kulturelle Unterschiede in der Alltagsverwendung bestimmter Begriffe berücksichtigt. Dieser Kompromiss ermöglichte trotz beschränkter Ressourcen eine hinreichend konsistente Mehrsprachigkeit, brachte jedoch gewisse methodische Limitierungen hinsichtlich der Vergleichbarkeit der Sprachversionen mit sich.

% Der beschriebene Sprach- und Übersetzungsprozess war eingebettet in einen breiteren, iterativen Entwicklungsprozess, der sowohl auf der Analyse bestehender Literatur als auch auf kontinuierlichem Feedback basierte. Ausgangspunkt bildeten Studien wie die Urban-Mind-Studie \parencite{bakolisUrbanMindUsing2018}, deren methodische Ansätze zur Erhebung situativen Wohlbefindens und räumlicher Wahrnehmung als Orientierung dienten. Diese Ansätze wurden jedoch um eigene Überlegungen zur intersektionalen Erhebung sozialer Positionierung ergänzt und in mehreren Durchläufen kritisch reflektiert.

% Während der Testphase der App Entwicklung (Siehe \cref{sec:app_entwicklung_feldtest}) sind ebenfalls zahlreiche kleinere Rückmeldungen zu Formulierungen und sprachlichen Feinheiten eingegangen. Diese wurden laufend eingearbeitet. Ebenfalls wurde der Fragebogen mit der betreuenden Dozentin durchgegangen und danach ebenfalls nochmals entsprechend überarbeitet.
% Ein wiederkehrendes methodisches Kriterium bei diesen Diskussionen war stets, die Belastung für Teilnehmende so gering wie möglich zu halten, ohne zentrale Aspekte der Forschungsfrage zu vernachlässigen. Durch den iterativen Ansatz konnte die Perspektive potenzieller Befragter frühzeitig einbezogen werden, was zu einer praxisnahen Optimierung der Items und des Fragebogenaufbaus führte.

% Im Rückblick lassen sich einige Punkte identifizieren, die bei einer erneuten Durchführung anders gestaltet werden könnten. Die Auswahl der Wohlbefindensdimensionen erfolgte nicht vollständig theoriegeleitet; insbesondere das Item „generelles Wohlbefinden“ wirkt im Nachhinein wenig trennscharf. Eine stärkere konzeptuelle Fundierung der Items wäre sinnvoll, um die Aussagekraft einzelner Skalen zu erhöhen.

% Der Verzicht auf standardisierte Skalen ermöglichte eine kompakte Erhebung, schränkt jedoch Vergleichbarkeit und Validität ein. Eine modularisierte Integration validierter Instrumente -- etwa in gekürzter Form -- könnte eine tragfähige Alternative darstellen.

% Die mehrsprachige Umsetzung war aus methodischer Sicht wichtig, konnte jedoch mangels Ressourcen nicht vollständig abgesichert werden. Eine zusätzliche Validierung durch muttersprachliche Expert:innen wäre wünschenswert gewesen, um semantische Konsistenz über Sprachversionen hinweg besser zu gewährleisten.

% Insgesamt zeigen sich an mehreren Stellen Stellschrauben für eine künftige Weiterentwicklung -- etwa durch eine engere theoretische Anbindung, gezielte Pretests oder eine systematischere Überprüfung von Übersetzungen und Antwortformaten. Gleichzeitig hat sich der gewählte Zugang als praktikabel und kontextsensibel erwiesen, insbesondere im Hinblick auf Zugänglichkeit und situative Anschlussfähigkeit.

% Besonders herausfordernd war die Erfassung von \gls[noindex]{race}. Im europäischen Kontext existieren kaum etablierte Kategorien, die rassifizierte Zugehörigkeiten erfassen, ohne problematische koloniale oder biologistische Zuschreibungen zu reproduzieren \parencite[\gls{vgl}][]{roigIntersectionalityEuropeDepoliticized2018}. Anders als in der US-amerikanischen Tradition, in der standardisierte Selbstkategorisierungen weit verbreitet sind, fehlen im hiesigen Kontext praktikable, breit akzeptierte Formate für quantitative Erhebungen. 

% Vor diesem Hintergrund wurde aus pragmatischen Gründen lediglich erfasst, ob Teilnehmende aktuell in einem anderen Land leben als in jenem, in dem sie geboren wurden. Diese Lösung reduzierte die Erhebungsbarrieren, blieb jedoch analytisch begrenzt: Sie kann nur indirekt auf rassifizierte Erfahrungen hinweisen und wird der Komplexität intersektionaler Ungleichheiten nicht gerecht. Rückblickend wäre eine offene, selbstbezeichnungsbasierte Erhebung vorzuziehen gewesen, um dieser sozialen Dimension angemessen Sichtbarkeit zu verleihen.

% Vor diesem Hintergrund wurde ein eigener, stark reduzierter Item-Satz entwickelt, um zentrale Dimensionen des Wohlbefindens situativ abbilden zu können. Ausgewählt wurden fünf Dimensionen: generelles Wohlbefinden, Zufriedenheit, Anspannung, Energie und Zugehörigkeit. Die Antworten wurden über lineare Slider-Skalen erfasst, um eine schnelle und intuitive Bearbeitung zu ermöglichen. 

% Diese Lösung stellt jedoch einen methodischen Kompromiss dar. Die Auswahl der Dimensionen erfolgte nicht auf Basis eines validierten theoretischen Modells, sondern primär pragmatisch und unter der Prämisse minimaler Befragungsdauer. Entsprechend ist die Validität und Vergleichbarkeit der erhobenen Werte eingeschränkt. Rückblickend erweist sich dies als Schwachpunkt der Studie: Die Messung bleibt inhaltlich und konzeptuell weniger trennscharf als wünschenswert, und es fehlt an einer etablierten Referenz, um die Ergebnisse eindeutig einzuordnen. Eine künftige Weiterentwicklung sollte daher auf einer systematischeren Konzeptualisierung beruhen und -- sofern möglich -- gekürzte, validierte Instrumente integrieren, die auf die Anforderungen situativer Mehrfacherhebungen angepasst sind.

% Future forschung:
% RQ-M4 (Kontextintegration ohne „Big GPS“): Wie lassen sich situative Kontextvariablen der Umgebung sinnvoll integrieren, wenn Standortdaten bewusst sparsam genutzt werden (z. B. punktuelle Location, selbstberichtete Kontexte), ohne die intersektionale Logik zu verwässern?
% Was bringt mir überhaupt eine Temperaturmessung bspw im Innenraum? ist das nicht dann auch wieder subjektiv und total abhängig von der einzelnen person. auch hier grosse methodische spannung zwischen qualität und quantität.
% 
% Wie kartiere ich die Daten die ich habe? Wäre das nicht mega spannend? Ref masterarbeit friebelThermischesEmpfindenUnd2024
% 
% 
% Wenn ich code veröffentliche expose ich mich auch. open source hat also auch seine nachteile. es bracuht mut und ein verständnis dass es auf jeden fall nicht perfekt ist. gerade als nicht informatiker finde ich das schon sehr schwer.
% 
% soll ich irgendwo noch meine eigene position reflektieren? finde ich hier irgendwie nicht so wichtig wie wenn ich etwas auswerten würde. was haben meine erfahrungen und meine soziale position damit zu tun wie ich code? das ist halt einigermassen politisch, andere hätten jetzt versucht daraus ein business zu machen. motivation zu code ist halt wirklich aus einer politischen überzeugung heraus entstanden. klar, auch wissenschaftlich argumentiert aber der willen war vor der theorie da.
% 
% Extrem schwer soziale positionierungen in ihrer feinheit quantitativ zu erfassen. wahrscheinlich unmöglich und auch nicht sinnvoll. aber: kann das ganze nicht trotzdem einen sinnvollen beitrag leisten? Sozusagen problemzonen überahupt einmal aufzeigen, ohne einen Anspruch auf vollständigkeit und detailreichtum.
% 
% Ich bennene soziale positionierungen als wandelbar und kontextabhängig, aber forme meine strata aus einmal erfassten demographischen daten. 
% 
% Mache ich kosten sichtbar? ist das ok? Was sind alles Kosten? KI Abos? 
% 
% Verwendete tools für die Arbeit: gehe ich darauf ein, dass ich eben bspw auch nicht word verwende sonder offene tools wie latex?
% 
% Bachelorarbeit als CC-BY-SA. wie exponiere ich mich damit? aber ist das nicht auch die konsequenz aus dem was ich hier schreibe? Gleiches auch mit dem Code. ich bin kein profi coder, ich bin kein profi wissenschaftler. beides früh in der "karriere"
% 
% \begin{quote}
%     \emph{Wie lässt sich der Einfluss räumlicher Umgebungen auf das affektive (Un-)Wohlbefinden intersektional positionierter Personen erfassen und analysieren?}
%     \end{quote}
% 
%     \begin{enumerate}
%         \item Wie muss ein Erhebungsansatz gestaltet sein, um affektives Wohlbefinden intersektional positionierter Personen gemeinsam mit relevanten Kontextmerkmalen wiederholt in situ zu erfassen?
%         \item Welche Anforderungen ergeben sich aus einer kritisch-digitalen Perspektive an eine Infrastruktur, die solche Erhebungen ermöglicht, und wie lassen sie diese praktisch umsetzen?
%         \item Wie geeignet sind die in einer Pilotstudie erhobenen Daten für eine intersektionale Mehrebenenmodellierung?
%     \end{enumerate}
% 
% 
% daten aus der pilotstudie lassen eigentlich nur sehr bedingt eine qualitative beantwortung der dritten frage zu. sehr wenig daten aus einer sehr homogenen gruppe.
% 
% ernsthafte frage: Wie kann es sein, dass in zeiten von open science und open data die wissenschaft es offentichtlich nicht schafft keine open source tools zu entwickeln? Fehlt da irgendwie die affinität? Spielen kommerzielle gedanken eine rolle der entwickelnden unternehmen oder was steckt dahinter?

% offlinefähigkeit der app. abwesenheit deren ist durch das entstanden, dass ich die app und den fragebogen parallel entwickelt habe. ist super doof so. und sollte für die zukunft so umbauen, dass das wieder möglich ist.

% machs nicht. wer auch immer du bist, der das liest. mach nicht sowas als Bachelorarbeit. tus nicht. es ist es nicht wert.

% ich fände eine weiterentwicklung auf jeden fall spannend, müsste mir aber wirklich überlegen wie ich das aufsetzte. untersuche ich konkret ein quartier? wie erreiche ich hohe teilnehmerzahlen?
% 
% Schlechter fragebogen, und dann auch daten aus dem. musste zu schnell gehen, war eigentlich nicht ready resp. habe mir einfach nicht genügend gedanken dazu gemacht wie ich das dann überhaupt auswerten will. 
% 
% behalte ich die app im store? macht eigentlich keinen sinn. nach mir müsste die raus, da ich ja keine daten mehr erhebe. etwas frustriert mit dem ganzen veröffentlichungsprozess. 
% 






