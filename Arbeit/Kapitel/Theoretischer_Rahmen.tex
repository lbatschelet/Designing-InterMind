% LTeX: language=de-CH

\chapter{Verflechtungen verstehen -- Begriffe und Konzepte} \label{sec:theoretischer_rahmen}

In diesem Kapitel führe ich in die zentralen Begriffe und Konzepte ein, die das Erkenntnisinteresse leiten und das methodische Vorgehen rahmen. Ausgangspunkt ist eine intersektionale Perspektive, mit der ich gesellschaftliche Unterschiede nicht isoliert, sondern in ihrer wechselseitigen Verflechtung analysiere. Anschliessend entfalte ich das Konzept des situierten (Un\nobreakdash-)Wohlbefindens als kontextabhängige, räumlich gebundene Erfahrung. Ergänzend nehme ich eine digitale Perspektive ein, die fragt, wie Daten, digitale Infrastrukturen und technologische Gestaltungsprozesse gesellschaftliche Machtverhältnisse widerspiegeln und (re)produzieren. Zusammen stelle ich diese Perspektiven als Grundlage für ein Forschungsdesign vor, das soziale Positionierung, räumliche Kontexte, situative Erfahrungen und digitale Infrastrukturen in Beziehung setzt.

\section{Verwebte Unterschiede -- Intersektionalität als Analyseinstrument}

Gesellschaftliche Wirklichkeiten sind durchzogen von komplexen Ungleichheiten. Menschen erfahren soziale Benachteiligung selten entlang nur einer einzigen Achse -- vielmehr wirken verschiedene Differenzlinien wie \gls{race}, \gls{gender} oder \gls{class} häufig gleichzeitig und verstärken sich wechselseitig. So kann die Erfahrung einer migrantischen Frau auf dem Arbeitsmarkt nicht einfach in \enquote{sexistische} und \enquote{rassistische} Diskriminierung zerlegt werden. Ihre Benachteiligung ergibt sich vielmehr aus der spezifischen Verwobenheit dieser Positionierungen, die durch keine einzelne Kategorie vollständig erfasst wird. In dieser Arbeit beziehe ich mich deshalb auf einen intersektionalen Ansatz, um diese Verflechtungen zu erfassen und einen Rahmen zu produzieren, der Ungleichheitsverhältnisse nicht isoliert betrachtet, sondern ihre Überschneidungen und Wechselwirkungen berücksichtigt.

Geprägt wird der Begriff der \gls{intersektionalitaet} von \textcite{crenshawMappingMarginsIntersectionality1991}, die auf die spezifischen Diskriminierungserfahrungen \emph{\glslink{schwarz}{Schwarzer}}\footnotemark Frauen aufmerksam macht. In ihrer Analyse von Antidiskriminierungsklagen im US-amerikanischen Arbeitsrecht zeigt sie, dass \emph{\glslink{schwarz}{Schwarze}} Frauen häufig keinen Rechtsschutz erhielten. Gerichte verhandelten Diskriminierung entweder als \emph{\gls{gender} discrimination} oder als \emph{\gls{race} discrimination}, jedoch nicht in der Verwobenheit beider Kategorien. Wurde eine Klage als \emph{\gls{gender} discrimination} geprüft, erfolgte der Vergleich mit weissen Frauen; blieben diese unbetroffen, galt die Klage als unbegründet. Wurde sie als \emph{\gls{race} discrimination} geprüft, erfolgte der Vergleich mit \emph{\glslink{schwarz}{Schwarzen}} Männern; auch hier verschwanden die spezifischen Benachteiligungen \emph{\glslink{schwarz}{Schwarzer}} Frauen. Ihre Erfahrungen fielen damit durch die Raster der bestehenden Rechtskategorien und blieben juristisch unsichtbar. Crenshaw argumentiert, dass feministische und antirassistische Theorien in ähnlicher Weise unzureichend sind, um Mehrfachdiskriminierung zu erfassen, und entwickelt \gls{intersektionalitaet} als analytisches Instrument zur Beschreibung solcher überlagerten Ungleichheitsverhältnisse \parencite[\gls{vgl}][]{hancockWhenMultiplicationDoesnt2007}.

\footnotetext{Ich schreibe den Begriff \enquote{\gls{schwarz}} mit grossem Anfangsbuchstaben und verwende ihn als politische Selbstbezeichnung von Menschen, die im Kontext rassistischer Machtverhältnisse positioniert werden. Der Begriff bezeichnet keine biologistische Eigenschaft, sondern eine soziale Positionierung; die Grossschreibung dient der Abgrenzung von äusserlichen Zuschreibungen \parencite{oguntoyeFarbeBekennenAfrodeutsche1986}. }

Ausgangspunkt dieser theoretischen Perspektive ist der Black Feminist Thought, welcher unter anderen in den Arbeiten von \textcite{hooksAintWomanBlack1981}, \textcite{lordeSisterOutsiderEssays1984}, Kimberle~\textcite{crenshawMappingMarginsIntersectionality1991} und \textcite{collinsBlackFeministThought2002} ihren Ausdruck findet. Black Feminist Thought formuliert eine scharfe Kritik an traditionellen feministischen Ansätzen, denen vorgeworfen wird, primär die Erfahrungen weisser, privilegierter Frauen ins Zentrum zu stellen und somit die Lebensrealitäten \emph{\glslink{schwarz}{Schwarzer}} Frauen zu marginalisieren. \textcite{crenshawMappingMarginsIntersectionality1991} entwickelt das Konzept der \gls{intersektionalitaet} explizit als Reaktion auf die Unfähigkeit bestehender theoretischer Ansätze, die spezifischen Diskriminierungserfahrungen \emph{\glslink{schwarz}{Schwarzer}} Frauen adäquat zu erfassen. Dabei verdeutlicht sie, dass Diskriminierung nicht als Summe einzelner, isolierter Erfahrungen verstanden werden kann, sondern als eigenständige Form sozialer Benachteiligung, die sich an der Überschneidung sozialer Kategorien wie \gls{race} und \gls{gender} manifestiert.

\gls{intersektionalitaet} entwickelte sich nicht allein im akademischen Kontext, sondern ist eng mit den politischen Kämpfen sozialer Bewegungen der 1970er- und 1980er-Jahre verbunden, insbesondere im Umfeld feministischer, antirassistischer und antikapitalistischer Strömungen \parencite{collinsBlackFeministThought2002}. Diese Bewegungen machten sichtbar, dass unterschiedliche Formen sozialer Ungleichheit nicht isoliert nebeneinander existieren, sondern in ihrer Verwobenheit erfahrbar werden. Damit legten sie die Grundlage für eine Perspektive, die gesellschaftliche Differenzen nicht als additive Kategorien, sondern als strukturell verknüpfte Machtverhältnisse versteht.

Zentral für die theoretische Fundierung des intersektionalen Ansatzes ist die Einsicht, dass soziale Positionierungen historisch gewachsen und gesellschaftlich konstruiert sind. Kategorien wie \gls{gender}, \gls{race} oder \gls{class} können daher nicht ohne Bezug auf die Macht- und Herrschaftsordnungen verstanden werden, in denen sie entstanden sind. Sie wirken nicht nur beschreibend, sondern ordnen Zugänge zu Ressourcen, Rechten und gesellschaftlicher Teilhabe.

Autorinnen wie Audre Lorde und bell hooks verdeutlichten, dass diese strukturellen Machtverhältnisse nicht abstrakt bleiben, sondern konkrete Auswirkungen auf individuelle Lebensrealitäten haben. Lorde betonte die Bedeutung von Differenz als Quelle von Wissen und Widerstand, während hooks die alltägliche Reproduktion patriarchaler und rassistischer Herrschaftsverhältnisse analysierte \parencite{collinsBlackFeministThought2002, hancockWhenMultiplicationDoesnt2007}. Damit trugen sie entscheidend dazu bei, Intersektionalität als kritisches Instrument zu etablieren, das sowohl strukturelle Dimensionen von Ungleichheit als auch subjektive Erfahrungen in den Blick nimmt.

Von der ursprünglich starken Fokussierung auf \textit{race} und \textit{gender} wird das Konzept in den folgenden Jahrzehnten zunehmend erweitert und schliesst heute oft eine Vielzahl sozialer Positionierungen und Identitäten ein, darunter etwa \emph{Sexualität}, \emph{Alter}, \emph{Behinderung}, \emph{Nationalität} oder \emph{Religion} \parencite{bauerIntersectionalityQuantitativeResearch2021, bowlegInvitedReflectionQuantifying2016}. Diese Erweiterung verdeutlicht die breite theoretische und empirische Anwendbarkeit von \gls{intersektionalitaet} als Analyseinstrument zur kritischen Untersuchung gesellschaftlicher Ungleichheiten und Diskriminierungserfahrungen. \gls{intersektionalitaet} hat sich somit nicht nur als theoretisches Konzept, sondern auch als methodische Grundlage etabliert, welche insbesondere in feministisch und sozialwissenschaftlich orientierten Diskursen verwendet wird, um die komplexen Wechselwirkungen gesellschaftlicher Machtverhältnisse zu analysieren.

\vspace{2em}

Die Anwendung intersektionaler Perspektiven auf räumliche Fragestellungen stellt eine zentrale Weiterentwicklung des ursprünglichen Konzepts der \gls{intersektionalitaet} dar. Seit den 2000er-Jahren etablierte sich eine eigenständige geographische Perspektive, die räumliche Kontextualität und situative Dimensionen sozialer Ungleichheiten explizit in den Mittelpunkt rückt \parencite{valentineTheorizingResearchingIntersectionality2007,rodo-de-zarateIntersectionalityFeministGeographies2018}.

Zentral für diesen Perspektivwechsel ist das Verständnis von Raum als gesellschaftliches Produkt. \textcite{lefebvreProductionLespace1974} betont, dass Raum kein neutrales Behältnis ist, in dem soziale Prozesse einfach stattfinden, sondern ein Produkt sozialer Praktiken, Aushandlungen und Machtbeziehungen. Raum entsteht durch Planung, Nutzung und alltägliche Routinen -- etwa durch Wohnungs- und Stadtbaupolitik, durch Verkehrs- und Infrastrukturen oder durch symbolische Markierungen wie Namen, Grenzen und Symbole. Machtverhältnisse schreiben sich in diese Strukturen ein und wirken dadurch stabilisierend: Wer Zugang zu bestimmten Räumen hat, wer ausgeschlossen bleibt oder wie Räume bewertet werden, reproduziert gesellschaftliche Hierarchien.

\textcite{foucaultEspacesAutres2004} erweitert diese Perspektive mit dem Konzept der Heterotopien. Damit bezeichnet er Räume, die gesellschaftliche Normen zugleich widerspiegeln und infrage stellen. Solche Räume sind ambivalent: Sie können dominante Ordnungen stabilisieren, indem sie Abweichungen räumlich „einschliessen“ (wie etwa Gefängnisse, Kliniken oder Kasernen), oder sie können alternative Formen des Zusammenlebens sichtbar machen (wie etwa Gärten, Festivals oder subkulturelle Treffpunkte). Heterotopien machen deutlich, dass Räume nicht nur materielle Anordnungen sind, sondern gesellschaftliche Ordnungen verkörpern und potenziell auch verschieben können.

Auf dieser theoretischen Grundlage argumentiert \textcite{valentineTheorizingResearchingIntersectionality2007}, dass soziale Kategorien nicht unabhängig vom Raum wirken. Ihre Bedeutung entfaltet sich erst im Zusammenspiel mit konkreten räumlichen Kontexten. Valentine zeigt dies am Beispiel muslimischer Frauen in britischen Städten. Ihre Erfahrungen im öffentlichen Raum sind nicht überall gleich, sondern variieren je nach Ort und sozialer Situation: In bestimmten Strassen oder Nachbarschaften sind sie aufgrund sichtbarer religiöser Zugehörigkeit -- etwa durch das Tragen eines Kopftuchs -- rassistischen und sexistischen Anfeindungen ausgesetzt. Dieselben Frauen erleben in anderen Kontexten, zum Beispiel in Moscheen, Community-Zentren oder stärker divers geprägten Quartieren, Sicherheit, Zugehörigkeit und Anerkennung. Entscheidend ist damit nicht allein die soziale Positionierung, sondern deren situative Übersetzung in räumliche Erfahrungen. Der Raum fungiert nicht als neutraler Hintergrund, sondern als aktiver Vermittler, der Zugehörigkeit ermöglichen oder ausschliessen kann. Ungleichheiten sind somit nicht nur verteilt im Raum, sondern werden durch räumliche Anordnungen hervorgebracht und für unterschiedliche Gruppen in spezifischer Weise erfahrbar gemacht.

\textcite{mccallComplexityIntersectionality2005} unterscheidet drei methodische Zugänge zu \gls{intersektionalitaet}. Der \emph{interkategoriale} Ansatz vergleicht festgelegte soziale Kategorien miteinander, um deren Überschneidungen sichtbar zu machen -- etwa indem Lohnunterschiede zwischen \emph{\glslink{schwarz}{Schwarzen}} Frauen, weissen Frauen, \emph{\glslink{schwarz}{Schwarzen}} Männern und weissen Männern analysiert werden. Der \emph{intrakategoriale} Ansatz richtet den Blick auf Unterschiede innerhalb einer einzelnen Kategorie, insbesondere dort, wo diese intern heterogen ist. So kann etwa untersucht werden, wie sich die Erfahrungen von Frauen unterscheiden, je nachdem ob sie gleichzeitig rassistische oder klassistische Diskriminierung erfahren. Der \emph{antikategoriale} Ansatz schliesslich stellt die Stabilität und analytische Nützlichkeit solcher Kategorien grundsätzlich infrage und fragt, ob festgelegte Identitätsachsen nicht selbst Teil des Problems sind.

Diese Systematisierung hat auch in geographischen Arbeiten Bedeutung erlangt, da sie methodisch begründet, wie sich verschiedene Dimensionen sozialer Differenz in räumlichen Analysen miteinander verknüpfen lassen. McCall betont zudem, dass \gls{gender} nicht isoliert betrachtet werden kann, sondern als interdependente Kategorie zu verstehen ist, deren Wirkung nur im Zusammenspiel mit anderen Differenzachsen entsteht \parencite{mccallSpatialRoutesGender1998}. Diese Wechselwirkungen sind wiederum stets in spezifische räumliche und historische Kontexte eingebettet, die ihre Ausprägung und Bedeutung prägen.

Empirische Arbeiten in der Geographie operationalisieren diese theoretischen Ansätze auf unterschiedliche Weise. Ein frühes Beispiel liefert \textcite{mccallSpatialRoutesGender1998}, die mit multilevel-statistischen Analysen regionale Strukturen und geschlechtsspezifische Lohnunterschiede verknüpft. Auch wenn ihre Arbeit der expliziten intersektionalen Wende in der Geographie noch vorausgeht, verdeutlicht sie, wie sich interkategoriale Zugänge nutzen lassen, um räumliche Muster sozialer Disparitäten sichtbar zu machen. \textcite{fensterRightGenderedCity2005} entwickelt diese Perspektive weiter, indem sie narrative und ethnographische Methoden einsetzt, um alltägliche Erfahrungen von Frauen in städtischen Kontexten zu untersuchen. Sie zeigt, dass das Konzept des „Right to the City“ patriarchale Machtverhältnisse unzureichend berücksichtigt und dass Zugehörigkeit und Teilhabe durch geschlechtsspezifische Ausschlüsse strukturiert sind. Mit einem dezidiert intersektionalen Anspruch führt \textcite{rodo-de-zarateDevelopingGeographiesIntersectionality2014} schliesslich ein Instrument ein, das soziale Positionierungen, emotionale Dimensionen und Orte systematisch miteinander verbindet. Die \emph{Relief Maps} ermöglichen es, subjektive Erfahrungen räumlicher Ungleichheit nicht nur zu erfassen, sondern auch visuell darzustellen und vergleichbar zu machen.

\vspace{2em}

Obwohl intersektionale Forschung historisch in qualitativen und aktivistischen Traditionen verankert ist, gewinnen quantitative Verfahren zunehmend an Relevanz, insbesondere in sozialpolitischen und raumplanerischen Kontexten \parencite{bauerIntersectionalityQuantitativeResearch2021}. Diese Verfahren bieten die Möglichkeit, strukturelle Muster intersektionaler Benachteiligung über grössere Stichproben sichtbar und empirisch überprüfbar zu machen.

Neuere methodische Entwicklungen wie \gls{i-maihda} \parencite[\gls{ua}][]{evansMultilevelApproachModeling2018,bellExtendingIntersectionalMultilevel2023} versuchen, dieser Herausforderung zu begegnen, indem sie intersektionale Positionierungen nicht als feste Gruppenmerkmale behandeln, sondern als dynamische, verschachtelte Konstellationen modellieren. Solche Ansätze zeigen, dass auch quantitative Forschung produktiv an intersektionale Theorien anschliessen kann. Gerade in der Geographie eröffnet dies die Möglichkeit, intersektionale Ungleichheiten nicht nur statistisch nachzuzeichnen, sondern auch in ihrer räumlichen Dimension sichtbar zu machen.

Jedoch ist die Übertragung intersektionaler Theorien in quantitative Methoden mit erheblichen Herausforderungen verbunden. Zentral ist die Kritik, dass traditionelle statistische Verfahren soziale Kategorien oft eindimensional oder additiv behandeln, was der komplexen theoretischen Vorstellung intersektionaler Verschachtelungen nicht gerecht wird \parencite{hancockWhenMultiplicationDoesnt2007, bowlegInvitedReflectionQuantifying2016}. Insbesondere birgt die numerische Operationalisierung sozialer Identitäten die Gefahr, die Fluidität und Kontextabhängigkeit dieser Kategorien zu ignorieren und damit ungewollt jene komplexen Wechselwirkungen zu nivellieren, die intersektionale Ansätze ursprünglich sichtbar machen wollen \parencite{scottIntersectionalityQuantitativeMethods2017}.

Um diesen Herausforderungen zu begegnen, bedarf es einer reflexiven und kontextsensiblen Operationalisierung intersektionaler Kategorien. Dies beinhaltet, soziale Gruppen nicht als statische Entitäten zu behandeln, sondern ihre relationalen und kontextuellen Eigenschaften explizit zu berücksichtigen \parencite{rodo-de-zarateDevelopingGeographiesIntersectionality2014, websterCenteringSocialtechnicalRelations2021}.


\section{Gefühlte Orte -- Situiertes (Un-)Wohlbefinden als räumliche Erfahrung}

(Un\nobreakdash-)Wohlbefinden ist flüchtig und kontextabhängig. In diesem Abschnitt versuche ich zu entwickeln, wie Situationen entstehen, in denen Orte, Praktiken, Atmosphären und Positionierungen (Un\nobreakdash-)Wohlbefinden ermöglichen oder begrenzen.

In den Sozial- und Gesundheitswissenschaften kursieren unterschiedliche Konzepte von Wohlbefinden, die jeweils eigene theoretische Setzungen und politische Implikationen mittragen. Der aus der Psychologie stammende Begriff \enquote{Subjektives Wohlbefinden} wird häufig über standardisierte Skalen erfasst und als individuelle Eigenschaft begriffen. Kritische sozialwissenschaftliche und feministische Perspektiven -- einschliesslich geographischer Arbeiten -- weisen darauf hin, dass dieses Verständnis hochgradig individualisiert ist und zu einer neoliberalen Regierungstechnik werden kann: Es verlagert Verantwortung auf Einzelne und blendet strukturelle Ungleichheiten, zeitliche und räumliche Skalen sowie Relationen aus \parencite{atkinsonToxicEffectsSubjective2021}. Sichtbar wird dies in politischen Wohlbefindens-Indizes wie dem \emph{World Happiness Report} oder nationalen Befragungen, die Lebenszufriedenheit über Einzelfragen messen und politische Steuerung an individuelle Bewertungen koppeln. Auch Corporate-Wellbeing-Programme oder Public-Health-Strategien, die auf Resilienztrainings und Lifestyle-Optimierung setzen, illustrieren diese Tendenz: Sie fördern Anpassung an bestehende Strukturen, anstatt ungleiche Lebensbedingungen oder diskriminierende Machtverhältnisse zu problematisieren \parencite[\gls{vgl}]{atkinsonToxicEffectsSubjective2021}. Diese Kritik motiviert eine stärker relationale, räumlich-kontextualisierte und machtsensible Fassung von Wohlbefinden.

Das Konzept des \enquote{affektiven Wohlbefindens} ist zentral in den \emph{affective geographies} verankert \parencite{hoSocialGeographyIII2024}. Wohlbefinden wird hier nicht als rein innerer Zustand verstanden, sondern als Ergebnis von räumlichen Anordnungen, sozialen Praktiken und Atmosphären. \textcite{ahmedAffectiveEconomies2004} analysiert dies am Beispiel politischer Diskurse in den USA, etwa auf Webseiten der \emph{Aryan Nations}. Sie zeigt, wie Emotionen wie Hass oder Angst nicht einfach \enquote{individuell} entstehen, sondern als \enquote{affective economies} zwischen Körpern, Symbolen und Orten zirkulieren. Indem etwa Migration oder Diversität mit Bedrohung verknüpft wird, \enquote{haften} Emotionen an bestimmten Körpern und Orten -- und schaffen dadurch Zugehörigkeiten für \enquote{weisse} einerseits und Abgrenzungen gegenüber \enquote{Anderen} andererseits.

Zugleich problematisiert \textcite{hemmingsInvokingAffectCultural2005} den sogenannten \emph{affective turn} in den Kultur- und Sozialwissenschaften. Sie kritisiert insbesondere die Tendenz, Affekte als vorsprachliche, universelle Intensitäten zu deuten. Eine solche Lesart, so Hemmings, droht Unterschiede und Machtverhältnisse auszublenden, weil sie affektive Erfahrungen von historischen und sozialen Kontexten ablöst. Für diese Arbeit ist deshalb entscheidend, affektive Dimensionen mitzudenken, ohne sie zu naturalisieren: Affekte werden hier als historisch und sozial situierte Relationen verstanden, die in konkreten räumlichen Konstellationen (Un\nobreakdash-)Wohlbefinden hervorbringen.

\textcite{smithWhichBeingWellbeing2018} entwickeln als Gegenentwurf zu individualisierten Konzepten das Verständnis eines \enquote{intra-aktiven Wohlbefindens}. Sie greifen auf Karen Barads (\citeyear{baradMeetingUniverseHalfway2007}) Theorie des Agentiellen Realismus zurück, die davon ausgeht, dass Entitäten nicht unabhängig voneinander existieren und erst nachträglich in Relation treten, sondern dass sie durch materielle und diskursive \emph{Intra-Aktionen} überhaupt erst entstehen. Diese Perspektive verschiebt den Blick: Wohlbefinden ist nicht länger eine Eigenschaft isolierter Subjekte, sondern ein Effekt von Verflechtungen zwischen Menschen, Atmosphären, Infrastrukturen, Technologien und Dingen. Damit rückt eine \emph{more-than-human}-Lesart ins Zentrum, die die Mitwirkung materieller Umwelten und nicht-menschlicher Akteure ernst nimmt. 

Für die geographische Forschung eröffnet dieses Konzept neue Anschlussmöglichkeiten, etwa indem auch die Gestaltung von Stadträumen, die Materialität von Wohnumgebungen oder die Rolle digitaler Infrastrukturen als konstitutiv für Erfahrungen von (Un\nobreakdash-)Wohlbefinden verstanden werden können. Zugleich bleibt dieser Zugang sprachlich schwer zugänglich und in der Wellbeing-Forschung bislang wenig etabliert, was seine Übertragung in empirische Studien erschwert.

Vor diesem Hintergrund verwende ich in dieser Arbeit den Begriff \emph{situiertes (Un\nobreakdash-)Wohlbefinden}. Er knüpft an \textcite{leeUnderstandingDisruptedParticipation2021} an, die den Begriff in der Analyse von Sport- und Freizeitpraktiken eingeführt haben, sowie an feministische Epistemologien situierten Wissens \parencite{harawaySituatedKnowledgesScience1988}. Mit dieser Begriffswahl möchte ich Wohlbefinden nicht als innere, universelle Eigenschaft oder rein individuelles Affektgeschehen fassen, sondern als relationales, machtsensibles Erleben, das in spezifischen räumlichen und sozialen Konstellationen hervorgebracht wird. 

Die Schreibweise \enquote{(Un\nobreakdash-)} macht sichtbar, dass negative Erfahrungen wie Ausschluss, Angst oder Unsicherheit analytisch gleichwertig zu positiven Momenten von Zugehörigkeit oder Sicherheit sind und nicht lediglich als Abweichungen von einem \enquote{normalen} Zustand von Wohlbefinden verstanden werden dürfen. 

Der Begriff \emph{situiert} verweist in dreifacher Hinsicht auf die theoretische Rahmung dieser Arbeit: Erstens betont er die Verkörperung und Kontextgebundenheit von Erfahrung, die nicht unabhängig von Orten, Atmosphären oder Praktiken gedacht werden kann. Zweitens rückt er die Rolle von Machtverhältnissen und intersektionaler Positionierungen in den Vordergrund, durch die Wohlbefinden ermöglicht oder eingeschränkt wird. Drittens markiert er eine erkenntnistheoretische Haltung, wie sie \textcite{harawaySituatedKnowledgesScience1988} formuliert: Wissen -- und in diesem Fall Erfahrung -- ist stets perspektivisch, partiell und situiert, niemals neutral oder allumfassend. 

Mit dem Konzept des \emph{sitiuierten (Un\nobreakdash-)Wohlbefindens} ziele ich darauf, eine begriffliche Brücke zu schlagen: zwischen affektiven und atmosphärischen Dynamiken, den politischen Dimensionen von Macht und Zugehörigkeit sowie einer epistemologischen Reflexion über die Bedingungen, unter denen Wohlbefinden überhaupt erfahrbar und analysierbar wird.

\vspace{1em}

Aus dieser begrifflichen Herleitung folgt für meine Arbeit: Wenn (Un\nobreakdash-)Wohlbefinden als \emph{situiert} verstanden wird, muss gezeigt werden, \emph{wie} Situationen entstehen. Ich greife dafür das Konzept kollektiver Stimmungen und Atmosphären auf, das eine analytische Ebene eröffnet, über die das Zusammenspiel von Orten, Praktiken und Machtverhältnissen erfahrbar wird und konkrete Verkörperungen von (Un\nobreakdash-)Wohlbefinden sichtbar werden.

\textcite{andersonAffectiveAtmospheres2009} versteht unter \emph{affective atmospheres} kollektive, nicht vollständig repräsentierbare Stimmungen, die im Zusammenspiel räumlicher Faktoren (\gls{zb} Sichtachsen, Beleuchtung, Dichte, Lärm, Überwachung) und sozialer Ordnungen (\gls{zb} Zugangsregime, informelle Normen) entstehen. Atmosphären sind dabei nicht einfach die Summe dieser Elemente, sondern überschreiten sie, indem sie als schwer fassbare Qualitäten wirken, die zugleich materiell gebunden und flüchtig, bestimmt und unbestimmt sind. Sie prägen situativ (Un\nobreakdash-)Wohlbefinden und erklären, warum derselbe Ort für unterschiedliche Gruppen gegensätzlich wirken kann. Ein stark kontrollierter Eingangsbereich oder ein nächtlicher Platz mag für privilegierte Gruppen belebt und angenehm erscheinen, während er für marginalisierte Gruppen als belastend oder bedrohlich erfahrbar ist. Diese Differenzen sind nicht allein individualpsychologisch, sondern in intersektionalen Machtverhältnissen verankert.

Solche Machtverhältnisse lassen sich über Zugehörigkeitsordnungen analytisch fassen. \textcite{antonsichSearchingBelongingAnalytical2010} versteht Zugehörigkeit nicht als stabilen Status, sondern als relationalen, umkämpften Prozess und unterscheidet zwischen \emph{place-belongingness} (Gefühle des Dazugehörens) und \emph{politics of belonging} (Regeln und Grenzziehungen, die Zugehörigkeit herstellen und begrenzen). \textcite{painGlobalizedFearEmotional2009} zeigt in diesem Sinn, wie Sicherheit und Unsicherheit affektiv-geopolitisch produziert werden -- durch Überwachung, Kontrolle, mediale Diskurse oder lokale Praktiken. Solche Prozesse lassen sich als \emph{bordering} fassen \parencite[\gls{vgl}][]{yuval-davisBelongingPoliticsBelonging2006}: Sie materialisieren sich in Blicken, räumlichen Markierungen oder scheinbar neutralen Routinen des Zugangs und wirken damit unmittelbar in Atmosphären hinein. Zugehörigkeit und Ausschluss werden so nicht abstrakt, sondern leiblich-situativ erfahrbar und strukturieren (Un\nobreakdash-)Wohlbefinden.

Besonders prägnant illustriert dies \textcite{ahmedPhenomenologyWhiteness2007} mit einer phänomenologischen Analyse von \emph{Whiteness} im alltäglichen Raum. Sie argumentiert, dass Räume in mehrheitlich weissen Gesellschaften implizit auf weisse Körper ausgerichtet sind: Bewegungen wie das Betreten öffentlicher Gebäude verlaufen für sie \enquote{unmarkiert} und selbstverständlich. Nicht-weisse Körper dagegen stossen in denselben Situationen auf Widerstände -- etwa durch häufigere Polizeikontrollen, Blicke oder subtile Praktiken der Exklusion. Während weisse Subjekte sich im öffentlichen Raum mit einem Gefühl von Selbstverständlichkeit bewegen können, erfahren nicht-weisse Subjekte dieselben Orte als potenziell feindlich oder begrenzend. \emph{Whiteness} erscheint damit nicht nur als soziale Position, sondern als räumliche Orientierung, die alltägliche Handlungsräume strukturiert und die affektive Erfahrung von Vertrautheit oder Bedrohung ungleich verteilt.

\textcite{bissellPassengerMobilitiesAffective2010} zeigt am Beispiel von Zugfahrten, wie Atmosphären nicht nur als diffuse Stimmungen, sondern als leiblich-situative Kräfte wirksam werden. In überfüllten oder besonders lauten Waggons verdichten sich etwa Gereiztheit und Anspannung, sodass Fahrgäste bewusst bestimmte Abteile meiden, länger stehen oder Umwege in Kauf nehmen. Solche Atmosphären entstehen aus der Verwobenheit körperlicher Dispositionen (Müdigkeit, Schmerz, sensorische Empfindlichkeit) mit materiellen Arrangements (Enge, Sitzordnung, Überwachung) und sozialen Erwartungen und prägen dadurch mikro-leibliche Praktiken des Alltags. Damit macht Bissell Atmosphären körperlich und situativ fassbar: Wohlbefinden im Raum lässt sich nicht nur über strukturelle Zugehörigkeitsordnungen begreifen, sondern auch über verkörperte Rhythmen und Affekte, die in Bewegung, Stillstand und alltäglichen Routinen entstehen.

Für die vorliegende Arbeit ziehe ich daraus drei analytische Konsequenzen: (1) (Un\nobreakdash-)Wohlbefinden ist als \emph{kontextabhängiger Effekt} konkreter räumlich-sozialer Konstellationen zu untersuchen, nicht als innere Eigenschaft von Individuen; (2) Unterschiede im Erleben sind \emph{macht- und positionssensibel} zu lesen, das heisst intersektional verortet und durch Zugehörigkeitsordnungen strukturiert; (3) empirische Zugänge müssen \emph{verkörperte, situative Dynamiken} erfassen, ohne die situative Einbettung zu verlieren.

\vspace{1em}

Situiertes (Un\nobreakdash-)Wohlbefinden wird in der Geographie auf unterschiedlichen Massstabsebenen (scales) untersucht. Ein Fokus auf körpernahe, individuelle Erlebnisse erlaubt es, feinste situative Veränderungen des Wohlbefindens zu erfassen -- etwa wie sich Stress oder Entspannung im direkten Kontakt mit spezifischen Orten oder Personen zeigt. Auf einer meso-räumlichen Ebene geraten kollektive Atmosphären in Quartieren, Stadtteilen oder anderen lokalisierten Gemeinschaftsräumen in den Blick, zum Beispiel Nachbarschaften, in denen Sicherheit, Lärm oder soziale Dichte das gemeinsame Erleben prägen. Makro-räumliche Analysen beziehen hingegen nationale oder transnationale Strukturen ein -- etwa migrationspolitische Rahmenbedingungen oder globale Ungleichheitsordnungen --, die emotionale Erfahrungen rahmen und begrenzen \parencite{howittScaleRelationMusical1998,marstonHumanGeographyScale2005}. Dieses skalierende Verständnis macht deutlich, dass situiertes (Un\nobreakdash-)Wohlbefinden weder rein individuell noch vollständig lokal erklärbar ist, sondern immer in ein Geflecht aus Mikroerfahrungen, kollektiven Dynamiken und übergeordneten gesellschaftlich-räumlichen Strukturen eingebettet ist.

Die Geographie nutzt dieses skalierende Verständnis, um Fragen räumlicher Gerechtigkeit und sozialer Teilhabe zu untersuchen. Indem Mikroerfahrungen des Alltags mit kollektiven Dynamiken und übergeordneten gesellschaftlich-räumlichen Strukturen in Beziehung gesetzt werden, lassen sich ungleiche Verteilungen von Möglichkeiten, Sicherheit oder Zugang sichtbar machen. Damit wird situiertes (Un\nobreakdash-)Wohlbefinden zu einem analytischen Zugang, der alltägliche emotionale Erfahrungen mit den Macht- und Ungleichheitsverhältnissen verknüpft, in die sie eingebettet sind.

\section{Digitale Werkzeuge -- Data Feminism, Open Source und digitale Souveränität}
\label{sec:datafeminism}

Digitale Technologien strukturieren zunehmend gesellschaftliche Realitäten -- sie beeinflussen, was sichtbar wird, wie Wissen entsteht und wer daran teilhat. Wer Software verwendet oder entwickelt, Daten sammelt oder Infrastrukturen kontrolliert, gestaltet diese Prozesse aktiv mit. Digitale Technologien sind daher nie neutral, sondern Ausdruck bestehender Machtverhältnisse. Eine kritische Auseinandersetzung mit digitalen Technologien und Infrastrukturen muss deshalb deren soziale und politische Dimension systematisch in den Blick nehmen.

Einen geeigneten theoretischen Rahmen hierfür bietet das Konzept des \textit{Data Feminism} von \textcite{dignazioDataFeminism2020}. Ausgangspunkt ist die Einsicht, dass Daten nie neutral sind, sondern stets Ausdruck gesellschaftlicher Machtverhältnisse \parencite[\gls{vgl}][\gls{s}~53ff.]{dignazioDataFeminism2020}. Sie entstehen nicht in einem Vakuum, sondern in konkreten sozialen, politischen und ökonomischen Kontexten, die prägen, was überhaupt als \enquote{Daten} gilt, wie sie erhoben werden und welche Fragen damit gestellt werden können. Data Feminism fordert deshalb, diese Bedingungen systematisch offenzulegen. Nur so wird sichtbar, dass Daten nicht einfach objektive Abbilder einer Realität darstellen, sondern mitbestimmen, was sichtbar wird und was unsichtbar bleibt. In diesem Sinne haben Daten eine konstitutive Funktion: Sie machen bestimmte Erfahrungen zähl- und vergleichbar, während andere Perspektiven ausgeblendet oder gar unsichtbar gemacht werden.

Damit knüpft Data Feminism an intersektionale Analysen an, die verdeutlichen, dass Ungleichheiten nicht entlang einer einzigen sozialen Kategorie erklärt werden können, sondern sich über mehrere \glslink{identitaetsachse}{Achsen} wie \gls{gender}, \gls{class}, \gls{race} oder \emph{Alter} verschränken \parencite[\gls{vgl}][\gls{s}~131ff.]{dignazioDataFeminism2020}. Genau diese Verwobenheit macht sichtbar, dass auch in Datenpraktiken Ausschlüsse und Hierarchien reproduziert werden: Wer gezählt wird, wer eine eigene Kategorie erhält, wessen Erfahrungen in Zahlen übersetzt werden und wessen nicht -- all das ist Ergebnis sozialer Aushandlungen und Machtbeziehungen. Die zentrale Erkenntnis von Data Feminism liegt darin, dass Daten zweischneidig sind: Sie können zur Stabilisierung bestehender Ungleichheiten beitragen, etwa indem sie dominante Kategorien unhinterfragt fortschreiben \parencite[\gls{vgl}][\gls{s}~27]{dignazioDataFeminism2020}. Gleichzeitig können sie aber auch als Werkzeuge genutzt werden, um Machtasymmetrien offenzulegen, marginalisierte Perspektiven sichtbar zu machen und gerechtere Wissensordnungen zu schaffen.

Für die eigene Arbeit bedeutet das, dass Daten nicht einfach als technische Ressource betrachtet werden dürfen, sondern als soziale Praxis, die kritisch reflektiert werden muss. Data Feminism sensibilisiert dafür, dass auch methodische Entscheidungen -- welche Variablen erhoben werden, welche Kategorisierungen vorgenommen werden, wie Ergebnisse dargestellt werden -- nie rein technisch sind, sondern normative Annahmen transportieren. Diese Perspektive eröffnet die Möglichkeit, Datenerhebung nicht nur kritisch zu hinterfragen, sondern auch aktiv so zu gestalten, dass vielfältige Erfahrungen sichtbar werden und bestehende Ausschlüsse nicht weiter verstärkt werden. Damit liefert das Konzept eine wichtige Grundlage für eine Forschung, die soziale Gerechtigkeit nicht nur thematisiert, sondern auch methodisch einlöst.

\vspace{1em}

Digitale Infrastrukturen sind nicht nur technische Systeme, sondern Ausdruck und Austragungsorte gesellschaftlicher Machtverhältnisse. Im Anschluss an feministische Geographien lassen sie sich als Räume verstehen, in denen Fragen von Sichtbarkeit, Teilhabe und Gerechtigkeit immer wieder neu verhandelt werden \parencite{elwoodFeministDigitalGeographies2018}. Datenpraktiken und digitale Technologien erhalten damit eine politische Dimension: Sie strukturieren, wer Zugang hat, welche Perspektiven sichtbar werden und wie Wissen in Infrastrukturen eingeschrieben ist.

So zeigen \textcite{dignazioGeographiesMissingData2024} anhand einer Untersuchung von 33 zivilgesellschaftlichen Monitoring-Initiativen gegen Feminizide in 15 Ländern, wie digitale Praktiken zur Gegen-Datenproduktion eingesetzt werden können. Diese Initiativen reagieren auf systematische Leerstellen staatlicher und institutioneller Statistiken, in denen Gewalt an Frauen und queeren Personen unsichtbar bleibt. Indem sie eigene Datenbanken aufbauen, einzelne Fälle dokumentieren und in transnationale Netzwerke einspeisen, schaffen sie Räume der Erinnerung und Solidarität. Diese Praktiken machen deutlich, dass digitale Infrastrukturen nicht nur Gewalt sichtbar machen, sondern auch als Mittel widerständiger Raumpolitik fungieren können: Sie eröffnen Orte, an denen alternative Wissensordnungen etabliert werden und gesellschaftliche Machtverhältnisse herausgefordert werden.

Diese Initiativen verdeutlichen, dass digitale Infrastrukturen nicht nur Daten speichern oder bereitstellen, sondern selbst zu Schauplätzen politischer Auseinandersetzungen werden. Genau hier setzt die Debatte um digitale Souveränität an: Sie beschreibt den Anspruch, digitale Technologien nicht einfach als externe Vorgaben hinzunehmen, sondern die Bedingungen ihrer Nutzung und Gestaltung selbst mitzubestimmen. Während politische Diskurse den Begriff oft auf nationale Unabhängigkeit oder technologische Leistungsfähigkeit verengen -- etwa im Aufbau eigener Rechenzentren oder der Regulierung von Datenflüssen --, betonen politisch-geographische Ansätze, dass digitale Souveränität in spezifischen räumlichen Ordnungen hervorgebracht und verhandelt wird \parencite{glaszeContestedSpatialitiesDigital2023}. Sie ist damit weder ein klar umrissener Rechtsbegriff noch eine rein technische Fähigkeit, sondern ein umkämpftes, diskursives Konzept, das unterschiedliche normative Ansprüche bündelt \parencite{pohleDigitalSovereignty2020}. In dieser Perspektive geht es nicht nur um staatliche Kontrolle, sondern ebenso um Fragen des Zugangs, der Teilhabe und der kollektiven Befähigung, digitale Infrastrukturen kritisch zu reflektieren, partizipativ zu gestalten und als Gemeingüter zugänglich zu machen.

So zeigen \textcite{glaszeContestedSpatialitiesDigital2023}, dass digitale Souveränität in unterschiedlichen Kontexten -- etwa in der EU und in Russland -- zwar auf ähnliche Leitbilder verweist, in der praktischen Umsetzung jedoch stark divergiert. Klassische Vorstellungen von Souveränität, Territorialität und staatlicher Kontrolle konfigurieren sich im digitalen Raum neu: Cloud-Computing, Plattformökonomien oder transnationale Datenströme erzeugen räumliche Spannungen, die weder allein durch nationale Gesetze noch durch technische Standards aufgelöst werden können. Damit wird digitale Transformation nicht als neutraler Modernisierungsprozess, sondern als politisch und räumlich situierte Auseinandersetzung sichtbar.

Auch \textcite{zhangBordersBorderingSovereignty2023} heben hervor, dass digitale Souveränität immer mit Prozessen der Grenzziehung verknüpft ist. Sie zeigen, dass diese digitalen Grenzen keineswegs nur an staatlichen Territorien verlaufen, sondern auf unterschiedlichen Massstabsebenen entstehen: von geopolitischen Regulierungen bis hin zu geschlossenen WhatsApp-Gruppen oder plattforminternen Zugangsbarrieren. Solche digitalen Grenzen strukturieren, wer Zugang zu Informationen und Infrastrukturen erhält, wer ausgeschlossen bleibt und wie digitale Räume angeeignet werden können.

Vor diesem Hintergrund lässt sich digitale Souveränität nicht als ortloses Prinzip begreifen, sondern als Ergebnis konkreter räumlicher Praktiken, Infrastrukturen und Machtverhältnisse. Eine kritische Perspektive versteht darunter zugleich die kollektive Fähigkeit, digitale Infrastrukturen nicht nur zu nutzen, sondern sie auch kritisch zu reflektieren, partizipativ zu gestalten und als Gemeingüter zugänglich zu machen \parencite{baackDataficationEmpowermentHow2015}. 

\vspace{1em}

\gls{opensource}-Praktiken können in diesem Zusammenhang als konkrete Werkzeuge einer relational verstandenen digitalen Souveränität gelesen werden. Ursprünglich aus der Softwareentwicklung stammend, bezeichnen sie die Praxis, Quellcode offen zugänglich zu machen, weiterzugeben und gemeinschaftlich weiterzuentwickeln. Damit durchbrechen sie Logiken exklusiven Eigentums und verlagern technische Gestaltung in kollektive Aushandlungsprozesse \parencite{mathewFeministManifestoResistance2021}. In dieser Perspektive wird Wissen nicht als Ware begriffen, sondern als \emph{commons}, also als geteilte Ressource, die durch Gemeinschaften gepflegt und verändert wird. \textcite{mathewFeministManifestoResistance2021} zeigt, dass gerade geistige Eigentumsregime tief in eurozentrische, patriarchale und kapitalistische Machtverhältnisse eingeschrieben sind und Wissen privatisieren. Ein feministischer und dekolonialer Zugang betont demgegenüber die Notwendigkeit, den öffentlichen Raum von Wissen zurückzugewinnen und alternative Formen kollektiver Wissensproduktion zu etablieren.

Vor diesem Hintergrund argumentieren \textcite{gurumurthyDataBodiesNew2022} aus einer feministischen Perspektive, dass Souveränität über Daten nicht allein durch individuelle Kontrolle oder \enquote{data ownership} gewährleistet werden kann. Am Beispiel von Menstruations-Apps verdeutlichen sie, dass solche Ansätze unzureichend sind, um strukturelle Machtasymmetrien im Datenregime zu adressieren. Stattdessen fordern sie, Daten als \emph{social knowledge commons} zu verstehen und kollektive Praktiken der Kontrolle und Gestaltung in den Vordergrund zu stellen. In dieser Logik erscheint \gls{opensource} nicht nur als technisches Modell, sondern als relationales Prinzip, das auf Kooperation, geteilte Verantwortung und eine feministische Ethik der Datenpraktiken verweist.

Einen verwandten Zugang eröffnet \textcite{baackDataficationEmpowermentHow2015}, der am Beispiel der Open-Data-Bewegung aufzeigt, wie Praktiken und Werte aus der \gls{opensource}-Kultur in demokratische Prozesse übersetzt werden. Indem Daten nicht exklusiv von staatlichen Institutionen interpretiert, sondern offen geteilt werden, wird die Möglichkeit geschaffen, neue Öffentlichkeiten und Formen politischer Teilhabe zu etablieren. \gls{opensource} fungiert hier als Infrastruktur demokratischer Ermächtigung, die sowohl Transparenz herstellt als auch neue Rollen für intermediäre Akteure wie Journalist\genderstern innen oder Civic-Tech-Kollektive eröffnet.

\textcite{wilshireTimeRebootFeminism2024} versteht Offenheit nicht als selbstverständlich inklusives Prinzip: Sie muss aktiv gestaltet und kritisch reflektiert werden, weil offene Infrastrukturen ebenso Ausschlüsse produzieren können wie geschlossene Systeme. Entscheidend ist, wer tatsächlich Zugang erhält, wer von Offenheit profitiert und wessen Perspektiven unsichtbar bleiben. Damit verweist die Verbindung von Data Feminism und digitaler Souveränität auf eine doppelte Aufgabe: Offenheit kann nur dann emanzipatorisch wirken, wenn sie gegen hegemoniale Strukturen verteidigt und mit einer politischen Praxis der Teilhabe verbunden wird. Für diese Arbeit bedeutet das, Offenheit nicht nur als technische Eigenschaft zu verstehen, sondern als normative Orientierung: Indem ich auf offene Software und transparente Infrastrukturen setze, verorte ich mein Projekt bewusst in einer Praxis, die Machtasymmetrien nicht reproduzieren, sondern kritisch hinterfragen und gerechtere Zugänge ermöglichen soll.