% LTeX: language=de-CH

% TODO: Sprachlich überarbeiten ich und präsens

\chapter{Verflechtungen verstehen -- Begriffe und Konzepte} \label{sec:theoretischer_rahmen}

In diesem Kapitel führe ich in die zentralen Begriffe und Konzepte ein, die das Erkenntnisinteresse leiten und das methodische Vorgehen rahmen. Ausgangspunkt ist eine intersektionale Perspektive, mit der ich gesellschaftliche Unterschiede nicht isoliert, sondern in ihrer wechselseitigen Verflechtung analysiere. Anschliessend entfalte ich das Konzept des situierten (Un-)Wohlbefindens als kontextabhängige, räumlich gebundene Erfahrung. Ergänzend nehme ich eine digitale Perspektive ein, die fragt, wie Daten, digitale Infrastrukturen und technologische Gestaltungsprozesse gesellschaftliche Machtverhältnisse widerspiegeln und (re)produzieren. Zusammen stelle ich diese Perspektiven als Grundlage für ein Forschungsdesign vor, das soziale Positionierung, räumliche Kontexte, situative Erfahrungen und digitale Infrastrukturen in Beziehung setzt.

\section{Verwebte Unterschiede -- Intersektionalität als Analyseinstrument}

Gesellschaftliche Wirklichkeiten sind durchzogen von komplexen Ungleichheiten. Menschen erfahren soziale Benachteiligung selten entlang nur einer einzigen Achse -- vielmehr wirken verschiedene Differenzlinien wie \gls{race}, \gls{gender} oder \gls{class} häufig gleichzeitig und verstärken sich wechselseitig. In dieser Arbeit beziehe ich mich deshalb auf einen intersektionalen Ansatz, um diese Verflechtungen zu erfassen und einen Rahmen zu produzieren, der Ungleichheitsverhältnisse nicht isoliert betrachtet, sondern ihre Überschneidungen und Wechselwirkungen in den Blick nimmt.

Geprägt wird der Begriff der \gls{intersektionalitaet} von \textcite{crenshawMappingMarginsIntersectionality1991}, die auf die spezifischen Diskriminierungserfahrungen \emph{\glslink{schwarz}{Schwarzer}}\footnotemark Frauen aufmerksam macht. Sie argumentiert, dass bestehende feministische und antirassistische Theorien nicht ausreichen, um Mehrfachdiskriminierung zu erfassen, und entwickelt \gls{intersektionalitaet} als analytisches Instrument zur Beschreibung solcher überlagerten Ungleichheitsverhältnisse \parencite[\gls{vgl}][]{hancockWhenMultiplicationDoesnt2007}.

\footnotetext{Ich schreibe den Begriff \enquote{\gls{schwarz}} mit grossem Anfangsbuchstaben und verwende ihn als politische Selbstbezeichnung von Menschen, die im Kontext rassistischer Machtverhältnisse positioniert werden. Der Begriff bezeichnet keine biologische Eigenschaft, sondern eine soziale Positionierung; die Grossschreibung dient der Abgrenzung von äusserlichen Zuschreibungen \parencite{oguntoyeFarbeBekennenAfrodeutsche1986}.}

Ausgangspunkt dieser theoretischen Perspektive ist der Black Feminist Thought, welcher unter anderen in den Arbeiten von \textcite{hooksAintWomanBlack1981}, \textcite{lordeSisterOutsiderEssays1984}, Kimberle~\textcite{crenshawMappingMarginsIntersectionality1991} und \textcite{collinsBlackFeministThought2002} ihren Ausdruck findet. Black Feminist Thought formuliert eine scharfe Kritik an traditionellen feministischen Ansätzen, denen vorgeworfen wird, primär die Erfahrungen weisser, privilegierter Frauen ins Zentrum zu stellen und somit die Lebensrealitäten \emph{\glslink{schwarz}{Schwarzer}} Frauen zu marginalisieren. \textcite{crenshawMappingMarginsIntersectionality1991} entwickelt das Konzept der \gls{intersektionalitaet} explizit als Reaktion auf die Unfähigkeit bestehender theoretischer Ansätze, die spezifischen Diskriminierungserfahrungen \emph{\glslink{schwarz}{Schwarzer}} Frauen adäquat zu erfassen. Dabei verdeutlicht sie, dass Diskriminierung nicht als Summe einzelner, isolierter Erfahrungen verstanden werden kann, sondern als eigenständige Form sozialer Benachteiligung, die sich an der Überschneidung sozialer Kategorien wie \gls{race} und \gls{gender} manifestiert.

\gls{intersektionalitaet} entwickelte sich somit nicht allein im akademischen Kontext, sondern ist stark verwurzelt in den politischen Kämpfen sozialer Bewegungen, insbesondere im Kontext feministischer, antirassistischer und antikapitalistischer Aktivismen der 1970er- und 1980er-Jahre \parencite{collinsBlackFeministThought2002}. Zentral für die theoretische Grundlage des intersektionalen Ansatzes ist die Anerkennung von Machtverhältnissen und sozialen Ungleichheiten als strukturell verankert und historisch bedingt. Gesellschaftliche Positionierungen werden als sozial konstruierte Kategorien verstanden, die immer in Verbindung mit bestehenden Machtsystemen wie Sexismus, Rassismus oder Klassismus betrachtet werden müssen. Audre Lorde und bell hooks betonten insbesondere die Rolle struktureller Unterdrückung und verdeutlichten, wie sich dominante Gesellschaftsstrukturen auf individueller Ebene reproduzieren und sich somit wechselseitig verstärken \parencite{collinsBlackFeministThought2002, hancockWhenMultiplicationDoesnt2007}.

Von der ursprünglich starken Fokussierung auf \textit{race} und \textit{gender} wird das Konzept in den folgenden Jahrzehnten zunehmend erweitert und schliesst heute oft eine Vielzahl sozialer Positionierungen und Identitäten ein, darunter etwa \emph{Sexualität}, \emph{Alter}, \emph{Behinderung}, \emph{Nationalität} oder \emph{Religion} \parencite{bauerIntersectionalityQuantitativeResearch2021, bowlegInvitedReflectionQuantifying2016}. Diese Erweiterung verdeutlicht die breite theoretische und empirische Anwendbarkeit von \gls{intersektionalitaet} als Analyseinstrument zur kritischen Untersuchung gesellschaftlicher Ungleichheiten und Diskriminierungserfahrungen. \gls{intersektionalitaet} hat sich somit nicht nur als theoretisches Konzept, sondern auch als methodische Grundlage etabliert, welche insbesondere in feministisch und sozialwissenschaftlich orientierten Diskursen verwendet wird, um die komplexen Wechselwirkungen gesellschaftlicher Machtverhältnisse zu analysieren.

\vspace{2em}

Die Anwendung intersektionaler Perspektiven auf räumliche Fragestellungen stellt eine zentrale Weiterentwicklung des ursprünglichen Konzepts der \gls{intersektionalitaet} dar. Seit den 2000er-Jahren etablierte sich eine eigenständige geographische Perspektive, die räumliche Kontextualität und situative Dimensionen sozialer Ungleichheiten explizit in den Mittelpunkt rückt \parencite{valentineTheorizingResearchingIntersectionality2007,rodo-de-zarateIntersectionalityFeministGeographies2018}.

Zentral für diesen Perspektivwechsel ist das Verständnis von Raum als gesellschaftlich produzierter Grösse. \textcite{lefebvreProductionLespace1974} argumentiert, dass Raum kein neutrales Behältnis ist, sondern als Produkt sozialer Praktiken und Beziehungen verstanden werden muss. Machtverhältnisse schreiben sich demnach in Raumstrukturen und Nutzungen ein und reproduzieren sich über diese. \textcite{foucaultEspacesAutres2004} erweitert diese Perspektive mit dem Konzept der Heterotopien: Räume spiegeln gesellschaftliche Normen nicht nur wider, sondern bieten auch die Möglichkeit ihrer Infragestellung und Verschiebung.

Auf dieser theoretischen Grundlage argumentiert \textcite{valentineTheorizingResearchingIntersectionality2007}, dass soziale Kategorien nicht unabhängig vom Raum wirken. Sie entfalten ihre Bedeutung erst im Zusammenspiel mit konkreten räumlichen Kontexten. Ungleichheiten sind somit nicht nur räumlich verteilt, sondern werden durch räumliche Anordnungen hervorgebracht und erfahrbar gemacht. Räume erzeugen je nach sozialer Positionierung unterschiedliche Bedeutungen, Zugänglichkeiten und emotionale Resonanzen -- etwa in Form von \textit{Safe Spaces} oder Zonen der Exklusion \parencite[\gls{vgl}][]{rodo-de-zarateIntersectionalityFeministGeographies2018}.

\textcite{mccallComplexityIntersectionality2005} unterscheidet drei methodische Zugänge zu \gls{intersektionalitaet}: Der \emph{interkategoriale} Ansatz vergleicht festgelegte soziale Kategorien miteinander, um deren Wechselwirkungen zu analysieren. Der \emph{intrakategoriale} Ansatz richtet den Blick auf Erfahrungen innerhalb einer einzelnen Kategorie, insbesondere dort, wo diese intern heterogen ist. Der \emph{antikategoriale} Ansatz hinterfragt die Stabilität und Nützlichkeit solcher Kategorien grundsätzlich. Diese Systematisierung wird auch in geographischen Arbeiten aufgegriffen, um methodisch zu begründen, wie sich unterschiedliche Dimensionen sozialer Differenz in räumlichen Analysen verknüpfen lassen. McCall betont zudem, dass \gls{gender} nicht isoliert betrachtet werden kann, sondern als interdependente Kategorie zu verstehen ist, deren Wirkung nur im Zusammenspiel mit anderen Differenzachsen entsteht \parencite{mccallSpatialRoutesGender1998}. Diese Wechselwirkungen sind zudem stets in spezifische räumliche und historische Kontexte eingebettet, die ihre Ausprägung und Bedeutung mitbestimmen.

Empirische Arbeiten in der Geographie operationalisieren diese theoretischen Ansätze auf unterschiedliche Weise: Beispielhaft nutzt \textcite{fensterRightGenderedCity2005} narrative, qualitativ-ethnographische Zugänge -- etwa Interviews --, um zu untersuchen, wie \gls{gender} und Raum zusammenwirken. \textcite{rodo-de-zarateDevelopingGeographiesIntersectionality2014} bringt partizipative Kartierungen und visuelle Instrumente wie die \emph{Relief Maps} ein; diese verbinden bewusst soziale Positionen, emotionale Dimensionen und Orte und visualisieren subjektive Erfahrungen räumlicher Ungleichheit. \textcite{mccallSpatialRoutesGender1998} verwendet quantitative, multilevel-statistische Analysen, um regionale Strukturen mit geschlechtsspezifischen Lohnunterschieden zu verbinden und räumliche Muster intersektionaler Disparitäten aufzudecken. 

\vspace{2em}

Obwohl intersektionale Forschung historisch in qualitativen und aktivistischen Traditionen verankert ist, gewinnen quantitative Verfahren zunehmend an Relevanz, insbesondere in sozialpolitischen und raumplanerischen Kontexten \parencite{bauerIntersectionalityQuantitativeResearch2021}. Diese Verfahren bieten die Möglichkeit, strukturelle Muster intersektionaler Benachteiligung über grössere Stichproben sichtbar und empirisch überprüfbar zu machen.

Jedoch ist die Übertragung intersektionaler Theorien in quantitative Methoden mit erheblichen Herausforderungen verbunden. Zentral ist die Kritik, dass traditionelle statistische Verfahren soziale Kategorien oft eindimensional oder additiv behandeln, was der komplexen theoretischen Vorstellung intersektionaler Verschachtelungen nicht gerecht wird \parencite{hancockWhenMultiplicationDoesnt2007, bowlegInvitedReflectionQuantifying2016}. Insbesondere birgt die numerische Operationalisierung sozialer Identitäten die Gefahr, die Fluidität und Kontextabhängigkeit dieser Kategorien zu ignorieren und damit ungewollt jene komplexen Wechselwirkungen zu nivellieren, die intersektionale Ansätze ursprünglich sichtbar machen wollen \parencite{scottIntersectionalityQuantitativeMethods2017}.

Um diesen Herausforderungen zu begegnen, bedarf es einer reflexiven und kontextsensiblen Operationalisierung intersektionaler Kategorien. Dies beinhaltet, soziale Gruppen nicht als statische Entitäten zu behandeln, sondern ihre relationalen und kontextuellen Eigenschaften explizit zu berücksichtigen \parencite{rodo-de-zarateDevelopingGeographiesIntersectionality2014, websterCenteringSocialtechnicalRelations2021}.


\section{Gefühlte Orte -- Situiertes (Un-)Wohlbefinden als räumliche Erfahrung}

(Un-)Wohlbefinden ist flüchtig und kontextabhängig. In diesem Abschnitt versuche ich zu entwickeln, wie Situationen entstehen, in denen Orte, Praktiken, Atmosphären und Positionierungen (Un-)Wohlbefinden ermöglichen oder begrenzen.

In den Sozial- und Gesundheitswissenschaften kursieren unterschiedliche Konzepte von Wohlbefinden, die jeweils eigene theoretische Setzungen und politische Implikationen mittragen. Der aus der Psychologie stammende Begriff \enquote{Subjektives Wohlbefinden} wird häufig über standardisierte Skalen erfasst und als individuelle Eigenschaft begriffen. Kritische sozialwissenschaftliche und feministische Perspektiven -- einschliesslich geographischer Arbeiten -- weisen darauf hin, dass dieses Verständnis hochgradig individualisiert ist und zu einer neoliberalen Regierungstechnik werden kann: Es verlagert Verantwortung auf Einzelne und blendet strukturelle Ungleichheiten, zeitliche und räumliche Skalen sowie Relationen aus \parencite{atkinsonToxicEffectsSubjective2021}. Diese Kritik motiviert eine stärker relationale, räumlich-kontextualisierte und machtsensible Fassung von Wohlbefinden.

Das Konzept des \enquote{affektiven Wohlbefindens} wird in den \emph{affective geographies} aufgenommen \parencite{hoSocialGeographyIII2024}. Aus dieser Perspektive ist Wohlbefinden eng mit räumlichen Anordnungen, sozialen Praktiken und Atmosphären verschränkt. \textcite{ahmedAffectiveEconomies2004} zeigt, wie Emotionen\footnotemark als \enquote{affective economies} zirkulieren, an Körper, Dinge und Orte \enquote{haften} und dadurch Zugehörigkeiten, Abgrenzungen und Orientierungen stabilisieren. Aufbauend darauf fasst \textcite{ahmedPhenomenologyWhiteness2007} \emph{Whiteness} als räumliche Orientierung, die für einige Körper Wege öffnet und für andere versperrt. Sie zeigt damit, dass Empfindungen von Vertrautheit, Sicherheit oder Bedrohung machtvoll situiert sind. Zugleich kritisieren Autor\genderstern innen wie \textcite{hemmingsInvokingAffectCultural2005}, dass Teile der \emph{affect studies} -- und damit auch Stränge der \emph{affective geographies} -- Affekte als vorsprachliche, universelle Intensitäten ontologisieren. Eine solche Lesart riskiert, Unterschiede in der Erfahrung zu entpolitisieren. Für diese Arbeit ist daher zentral, affektive Dimensionen einzubeziehen, ohne sie zu naturalisieren.

\footnotetext{Ich verwende den Begriff \emph{Emotion}, um zeitlich begrenzte, sozial und kulturell geformte, benennbare Zustände zu bezeichnen, die in spezifischen räumlichen Kontexten entstehen und wahrgenommen werden können \parencite{bondiIntroductionGeographysEmotional2006}. Der  Begriff \enquote{Gefühl} vermeide ich hingegen, da er unscharf ist und sowohl körperliche Empfindungen, Stimmungen als auch Emotionen meinen kann.}

\textcite{smithWhichBeingWellbeing2018} entwickeln als Gegenentwurf das Konzept des \enquote{intra-aktiven Wohlbefindens}. Inspiriert von \textcite{baradMeetingUniverseHalfway2007} Agentiell-Realismus verstehen sie Wohlbefinden als Effekt von \emph{Intra-Aktionen}: Entitäten (Menschen, Dinge, Atmosphären, Technologien) gehen nicht vor Relationen voraus, sondern entstehen durch materielle-diskursive Verflechtungen; entsprechend rückt eine \emph{more-than-human}-Perspektive ins Zentrum. Dieser Zugang macht die Mitwirkung materieller Umwelten, Infrastrukturen und Artefakte sichtbar und ist theoretisch anschlussfähig aber zugleich sprachlich schwer zugänglich und wenig etabliert.

Vor diesem Hintergrund verwende ich in dieser Arbeit den Begriff \emph{situiertes (Un-)Wohlbefinden}. Anknüpfend an \textcite{leeUnderstandingDisruptedParticipation2021} sowie an feministische Epistemologien situierten Wissens \parencite{harawaySituatedKnowledgesScience1988} bezeichnet er (Un-)Wohlbefinden nicht als innere, individuelle Eigenschaft oder universelles Affektgeschehen, sondern als relationales, machtsensibles Erleben, das in spezifischen räumlichen und sozialen Konstellationen hervorgebracht wird. Mit der Schreibweise \enquote{(Un-)} will ich sichtbar machen, dass negative Erfahrungen (Unsicherheit, Ausschluss, Angst) analytisch gleichwertig sind und nicht als Defizite eines \enquote{positiven} Normalzustands behandelt werden sollen. \emph{Situiert} bündelt drei Stränge: (1) die Sensibilität für affektive Dynamiken und Atmosphären; (2) die Betonung von Machtverhältnissen, Zugehörigkeitsordnungen und intersektionaler Positionierung; (3) die epistemologische Setzung, dass Erfahrung perspektivisch, verkörpert und ortsspezifisch ist.

\vspace{1em}

Aus dieser begrifflichen Herleitung folgt für die weitere Argumentation: Wenn (Un-)Wohlbefinden als \emph{situiert} verstanden wird, muss gezeigt werden, \emph{wie} Situationen hervorgebracht werden. Im Folgenden entwickle ich ein theoretisches Fundament, das diese Hervorbringung analytisch fasst. Dafür verschränke ich drei Perspektiven: erstens \emph{affective atmospheres} als milieuhafte Rahmung von Erfahrung; zweitens Zugehörigkeitsordnungen und Grenzziehungen, die (Un-)Sicherheit und Machtverhältnisse strukturieren ; drittens Verkörperung und (Im-)Mobilitäten als leibliche Einschreibungen räumlich-sozialer Ordnungen.

\textcite{andersonAffectiveAtmospheres2009} versteht unter \emph{affective atmospheres} kollektive, nicht vollständig repräsentierbare Stimmungen, die im Zusammenspiel räumlicher Faktoren (\gls{zb} Sichtachsen, Beleuchtung, Dichte, Lärm, Überwachung) und sozialer Ordnungen (\gls{zb} Zugangsregime, informelle Normen) entstehen. Solche Atmosphären prägen situativ (Un-)Wohlbefinden und erklären, warum derselbe Ort für unterschiedliche Gruppen gegensätzlich wirken kann. Ein stark kontrollierter Eingangsbereich oder ein nächtlicher Platz mag für privilegierte Gruppen belebt und angenehm erscheinen, während er für marginalisierte Gruppen als belastend oder bedrohlich erfahrbar ist. Diese Differenzen sind nicht allein individualpsychologisch, sondern in intersektionalen Machtverhältnissen verankert \parencite{valentineTheorizingResearchingIntersectionality2007,rodo-de-zarateIntersectionalityFeministGeographies2018,rodo-de-zarateIntersectionalitySpatialityEmotions2023}.

\textcite{antonsichSearchingBelongingAnalytical2010} fasst Zugehörigkeit nicht als stabilen Status, sondern als relationalen, umkämpften Prozess: analytisch unterscheidet er zwischen \emph{place-belongingness} (Gefühle des Dazugehörens) und \emph{politics of belonging} (Regeln und Grenzziehungen, die Zugehörigkeit herstellen und begrenzen). Daran anknüpfend zeigt \textcite{painGlobalizedFearEmotional2009}, wie Sicherheit und Unsicherheit im Alltag affektiv-geopolitisch produziert werden -- durch Erfahrungen von Überwachung, Kontrolle, medialen Diskursen und lokalen Praktiken. Solche Zugehörigkeitsordnungen entstehen in alltäglichen Grenzziehungen, etwa Blicken, Regeln, räumlichen Markierungen oder scheinbar neutralen Routinen des Zugangs; präziser lässt sich dies als \emph{bordering} fassen \parencite[\gls{vgl}][]{yuval-davisBelongingPoliticsBelonging2006}. Im Sinn einer phänomenologischen Perspektive auf Rassifizierung beschreibt \textcite{ahmedPhenomenologyWhiteness2007} \emph{Whiteness} als räumliche Orientierung, die Wege für einige Körper öffnet und für andere versperrt; Zugehörigkeit wird so als machtvoll situiertes Möglichkeitsfeld erfahrbar. Vor diesem Hintergrund sind Atmosphären -- im Anschluss an \textcite{andersonAffectiveAtmospheres2009} -- nicht bloss \enquote{Stimmungen}, sondern tragen Zugehörigkeiten: Sie können Vertrautheit verdichten oder Fremdmachung erzeugen und damit (Un-)Wohlbefinden strukturieren.

\textcite{mccormackEngineeringAffectiveAtmospheres2008} zeigt, dass Atmosphären nicht einfach \enquote{um} Körper existieren, sondern durch \emph{Bewegung} hervorgebracht und moduliert werden: durch Tempo, Takt und kleinteilige Bewegungsmuster (Mikro-Kinästhetik), die Haltungen, Gangarten, Blickführungen sowie Sitz- und Stehweisen prägen. Daran anknüpfend zeigt \textcite{bissellPassengerMobilitiesAffective2010} am Beispiel des öffentlichen Verkehrs, wie affektive Atmosphären im Fahrzeuginnenraum aufkommen, sich verdichten und wieder abflauen; sie rahmen Alltagsentscheidungen unterwegs (etwa Umwege wählen oder Bereiche meiden) und strukturieren so Müdigkeit, Anspannung oder Wachsamkeit. Leibliche Dispositionen (Müdigkeit, Schmerz, sensorische Empfindlichkeit) verschränken sich dabei mit materiellen Umwelten (\gls{zb} Möblierung, Dichte, Überwachungstechnologien, Verkehrsführung) und normativen Erwartungen an angemessenes Verhalten; (Un-)Wohlbefinden zeigt sich so als leiblich vermitteltes Verhältnis zur Umgebung. In diesem Sinn sind (Im-)Mobilitäten nicht nur Ausdruck, sondern auch Medium der Herstellung ungleicher Möglichkeiten, sich zu bewegen oder zu verweilen \parencite[\gls{vgl}][]{ahmedPhenomenologyWhiteness2007}.

Für die vorliegende Arbeit ziehe ich daraus drei analytische Konsequenzen: (1) (Un-)Wohlbefinden ist als \emph{kontextabhängiger Effekt} konkreter räumlich-sozialer Konstellationen zu untersuchen, nicht als innere Eigenschaft von Individuen; (2) Unterschiede im Erleben sind \emph{macht- und positionssensibel} zu lesen, das heisst intersektional verortet und durch Zugehörigkeitsordnungen strukturiert; (3) empirische Zugänge müssen \emph{verkörperte, situative Dynamiken} erfassen, ohne die situative Einbettung zu verlieren.

\vspace{1em}

Empirisch wird (Un-)Wohlbefinden in der Geographie vor allem mit feldnahen, in-situ- und sinnesorientierten Verfahren untersucht. \textcite{kusenbachStreetPhenomenologyGoAlong2003} entwickeln \emph{go-along}/walking interviews, die räumliches Erleben im Gehen erfassen und situierte Deutungen unmittelbar mit Bewegungen, Blicken und Mikropraktiken koppeln. Daran anknüpfend systematisiert \textcite{pinkDoingSensoryEthnography2009} eine \emph{sensory ethnography}, die multimodale Sinnesmodalitäten (Sehen, Hören, Tasten, Geruch) als konstitutiv für räumliches Erleben ernst nimmt. \textcite{buscherIntroductionMobileMethods2010} rahmen \emph{mobile methods} als Forschungsdesigns, die Unterwegs-Situationen, Ko-Präsenz und taktile/kinästhetische Abstimmungen methodisch zugänglich machen. Solche Zugänge erlauben es, atmosphärische Feinheiten, materielle Umwelten und intersektionale Positionierungen zusammenzudenken und situierte Konstellationen nicht nur zu beschreiben, sondern \emph{mit} den Praktiken ihrer Hervorbringung zu beobachten. Gleichwohl bleiben vergleichende Analysen zwischen unterschiedlich positionierten Gruppen am selben Ort häufig retrospektiv oder fallbasiert und sind seltener systematisch wiederholend angelegt \textcite{hoSocialGeographyIII2024}.

Die beschriebenen Unterschiede im emotionalen Erleben werden in der Geographie auf unterschiedlichen Massstabsebenen (scales) untersucht. Ein Fokus auf körpernahe, individuelle Erlebnisse erlaubt es, feinste situative Veränderungen des Wohlbefindens zu erfassen und deren Zusammenhang mit unmittelbaren räumlichen und sozialen Kontexten zu analysieren. Auf einer meso-räumlichen Ebene geraten kollektive Atmosphären in Quartieren, Stadtteilen oder anderen lokalisierten Gemeinschaftsräumen in den Blick, während makro-räumliche Analysen nationale oder transnationale Strukturen einbeziehen, die emotionale Erfahrungen rahmen und begrenzen \parencite{howittScaleRelationMusical1998,marstonHumanGeographyScale2005}. Dieses skalierende Verständnis macht deutlich, dass situiertes (Un-)Wohlbefinden weder rein individuell noch vollständig lokal erklärbar ist, sondern immer in ein Geflecht aus Mikroerfahrungen, kollektiven Dynamiken und übergeordneten gesellschaftlich-räumlichen Strukturen eingebettet ist.

Die Geographie nutzt dieses skalierende Verständnis, um Fragen räumlicher Gerechtigkeit und sozialer Teilhabe zu untersuchen. Indem Mikroerfahrungen des Alltags mit kollektiven Dynamiken und übergeordneten gesellschaftlich-räumlichen Strukturen in Beziehung gesetzt werden, lassen sich ungleiche Verteilungen von Möglichkeiten, Sicherheit oder Zugang sichtbar machen. Damit wird situiertes (Un-)Wohlbefinden zu einem analytischen Zugang, der alltägliche emotionale Erfahrungen mit den Macht- und Ungleichheitsverhältnissen verknüpft, in die sie eingebettet sind.

\section{Digitale Werkzeuge -- Data Feminism, Open Source und digitale Souveränität}
\label{sec:datafeminism}

Digitale Technologien strukturieren zunehmend gesellschaftliche Realitäten -- sie beeinflussen, was sichtbar wird, wie Wissen entsteht und wer daran teilhat. Wer Software verwendet oder entwickelt, Daten sammelt oder Infrastrukturen kontrolliert, gestaltet diese Prozesse aktiv mit. Digitale Technologien sind daher nie neutral, sondern Ausdruck bestehender Machtverhältnisse. Eine kritische Auseinandersetzung mit digitalen Technologien und Infrastrukturen muss deshalb deren soziale und politische Dimension systematisch in den Blick nehmen.

Einen geeigneten theoretischen Rahmen hierfür bietet das Konzept des \textit{Data Feminism} von \textcite{dignazioDataFeminism2020}. Data Feminism hinterfragt vermeintliche Objektivität und Neutralität von Daten und Algorithmen, indem es deren Entstehungskontexte, Produktionsbedingungen und zugrunde liegende Machtverhältnisse offenlegt. Aus dieser Perspektive erscheinen Daten nicht als neutrale Fakten, sondern als gesellschaftliche Konstrukte, die Ausschlüsse produzieren, Hierarchien festigen oder marginalisierte Gruppen unsichtbar machen können.

Digitale Infrastrukturen sind Ausdruck und Austragungsorte gesellschaftlicher Machtverhältnisse. Im Sinne feministischer Geographien lassen sich digitale Technologien als Räume verstehen, in denen Fragen von (Un-)Sichtbarkeit, Teilhabe und Gerechtigkeit neu verhandelt werden \parencite{elwoodFeministDigitalGeographies2018}. Aus dieser Perspektive gewinnen datenbezogene Praktiken politische Relevanz, gerade dann, wenn sie hegemoniale Strukturen hinterfragen und eigene Infrastrukturen schaffen. So zeigen \gls{bspw} feministische Initiativen, die im Kontext von Feminiziden Gewalt sichtbar machen, Erinnerungspraktiken etablieren und eigene Datenräume schaffen, wie digitale Praktiken als Mittel widerständiger Raumpolitik fungieren können \parencite{dignazioGeographiesMissingData2024}.

Diese Beispiele zeigen, dass digitale Infrastrukturen nicht nur technische Artefakte, sondern politische Räume sind, in denen Fragen nach Kontrolle, Zugang und Gestaltungsmacht neu verhandelt werden. In wissenschaftlichen und politischen Debatten wird dieser Aushandlungsprozess zunehmend unter dem Begriff der digitalen Souveränität gefasst \parencite{glaszeContestedSpatialitiesDigital2023}.

Während digitale Souveränität in politischen Diskursen oft als nationale Strategie oder technische Fähigkeit verstanden wird, rückt eine geographische Perspektive ihre räumlichen Dimensionen in den Vordergrund. Politisch-geographische Arbeiten betonen, dass digitale Souveränität stets in räumliche Ordnungen eingebettet ist und durch diese hervorgebracht wird \parencite{glaszeContestedSpatialitiesDigital2023,zhangBordersBorderingSovereignty2023}. Digitale Infrastrukturen produzieren und transformieren dabei Grenzen auf unterschiedlichen Massstabsebenen -- von staatlich regulierten Datenflüssen und territorial verankerten Rechenzentren bis hin zu unsichtbaren Abgrenzungen innerhalb digitaler Plattformen oder geschlossener Kommunikationsgruppen. Solche \enquote{digitalen Grenzen} bestimmen nicht nur, wer auf welche Daten und Dienste zugreifen kann und wer davon ausgeschlossen ist, sondern prägen auch, wie digitale Räume genutzt, wahrgenommen und angeeignet werden. Digitale Souveränität ist damit kein ortloses Prinzip, sondern in konkreten räumlichen Praktiken, Infrastrukturen und Machtverhältnissen verankert.

In diesem Verständnis bezeichnet digitale Souveränität auch die kollektive Befähigung, digitale Infrastrukturen kritisch zu reflektieren, partizipativ zu gestalten und als Gemeingüter zugänglich zu machen \parencite{baackDataficationEmpowermentHow2015,glaszeContestedSpatialitiesDigital2023}. Diese Perspektive begreift technologische Gestaltung als sozialen und politischen Aushandlungsprozess, in dem Fragen von (Un-)Sichtbarkeit, Teilhabe und Verantwortung neu verhandelt werden.

\emph{\gls{opensource}}-Praktiken können in diesem Kontext als konkrete Werkzeuge einer relational verstandenen digitalen Souveränität gelesen werden. Sie ermöglichen kollektive Kontrolle über technische Systeme, fördern Transparenz und erlauben es, digitale Infrastrukturen gemeinschaftlich weiterzuentwickeln \parencite{gurumurthyDataBodiesNew2022}. Damit tragen sie zur Demokratisierung technischer Expertise bei \parencite{baackDataficationEmpowermentHow2015, pohleDigitalSovereignty2020}.

Ergänzend verweisen \emph{Open Science} und \emph{Open Data} auf die Öffnung wissenschaftlicher Prozesse und Datenbestände. Während Open Science vor allem Transparenz und Reproduzierbarkeit betont, zielt Open Data auf die Bereitstellung von Forschungs- und Infrastrukturdaten, die in unterschiedlichen Kontexten genutzt und kritisch hinterfragt werden können \parencite{fecherWhatDrivesAcademic2014}. Beide Praktiken erweitern damit die Möglichkeiten für partizipative Forschung und methodische Innovationen, werfen jedoch zugleich Fragen nach Standards, Zugänglichkeit und Ausschlüssen auf.

Eine Entscheidung für Offenheit und digitale Souveränität erfordert deshalb eine kontinuierliche Reflexion über zugrunde liegende Bedingungen, Herausforderungen und mögliche Ausschlüsse. Es gilt stets kritisch zu fragen, wer Zugang zu offenen Infrastrukturen hat, wer von ihnen profitiert und wer ausgeschlossen bleibt. Gerade feministische Perspektiven betonen, dass Offenheit nicht automatisch Gleichheit bedeutet, sondern aktiv gestaltet und gegen hegemoniale Machtverhältnisse verteidigt werden muss \parencite{wilshireTimeRebootFeminism2024}.