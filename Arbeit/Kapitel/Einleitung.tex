% LTeX: language=de-CH
\chapter{Einleitung} \label{sec:einleitung}

In einer datenjournalistischen Auswertung zeigt sich, dass Hitzeinseln in Schweizer Städten ungleich verteilt sind und häufiger ärmere Quartiere betreffen \parencite{albisserUngleichheitStaedtenHitzeinseln2023}. Temperatur ist jedoch nur einer von vielen Faktoren, die alltägliches (Un\nobreakdash-)Wohl\-be\-find\-en prägen; dieses entsteht in komplexen, situativen Konstellationen aus räumlichen Anordnungen, gesellschaftlichen Regeln, Routinen und sozialen Positionierungen. So reduziert die Analyse von Hitzeinseln Ungleichheit letztlich auf die Kopplung einer Exposition (Temperatur) mit einem Einzelmerkmal (sozioökonomischer Status), während viele Differenzen tiefer in soziale und räumliche Ordnungen verwoben sind und sich als flüchtige, kontextabhängige Erfahrungen im Raum zeigen. Vor diesem Hintergrund entwickle und erprobe ich in dieser Arbeit ein Forschungsdesign, das solche situativen Erfahrungen des (Un\nobreakdash-)Wohl\-be\-find\-ens kontextualisiert, wiederholt und ortsnah erfassbar macht und für intersektional sensible Analysen anschlussfähig ist.

Ich beziehe mich dafür auf das Feld der \emph{emotional}, die alltägliche Gefühle und ihre räumlichen Dimensionen untersuchen \parencite{hoSocialGeographyIII2024}. Dieses Forschungsfeld zeigt, dass Erfahrungen von (Un\nobreakdash-)Wohl\-be\-find\-en nicht rein individuell sind, sondern relational entstehen: Sie knüpfen an materielle Umgebungen, soziale Machtstrukturen und an die Positionierung von Körpern im Raum. Damit bieten sie mir ein theoretisches Fundament, um (Un\nobreakdash-)Wohl\-be\-find\-en als situierte Erfahrung zu verstehen, die zugleich flüchtig und strukturiert, persönlich erlebt und gesellschaftlich geformt ist.

Ich arbeite mit einem intersektionalen Verständnis von Ungleichheiten, dem zufolge sich Erfahrungen nicht entlang einzelner sozialer Kategorien -- etwa \emph{Geschlecht}, \emph{Alter} oder \gls[noindex]{class}\footnotemark -- erklären lassen, sondern nur in ihrer Verschränkung. Ich gehe davon aus, dass Diskriminierungen und Privilegien nicht additiv nebeneinanderstehen, sondern sich in ihrem Zusammenspiel gegenseitig verstärken, überlagern oder abschwächen.

\footnotetext{Ich markiere in dieser Arbeit Begriffe wie \gls[noindex]{race}, \emph{Geschlecht}, \gls[noindex]{class}, \emph{Alter}, \emph{Frau} oder \emph{Behinderung} durch Kursivierung, um zu unterstreichen, dass es sich dabei um sozial konstruierte, wandelbare und gesellschaftlich wirkmächtige Kategorien handelt. Die Begriffe \gls[noindex]{race} und \gls[noindex]{class} verbleiben in englischer Sprache, da ihre deutschen Übersetzungen umstritten sind. Ausführlichere Erläuterungen finden sich im \nameref{sec:glossary}.}

Gerade im Zusammenspiel dieser beiden Perspektiven ergibt sich eine forschungspraktische Herausforderung: Zwar gibt es zahlreiche qualitativ ausgerichtete Studien, die emotionale Erfahrungen und intersektionale Positionierungen in alltäglichen Situationen beschreiben, jedoch fehlen bislang methodische Ansätze, mit denen sich solche Dynamiken systematisch und quantitativ erfassen lassen. Nur wenige Versuche existieren,  (Un\nobreakdash-)Wohl\-be\-find\-en in seiner räumlich-situativen und intersektionalen Dimension zugleich systematisch und quantitativ zu erfassen. Mit dieser Arbeit möchte ich deshalb einen Beitrag dazu leisten, intersektionale Ungleichheiten auch in quantitativen Forschungsdesigns sichtbar zu machen -- ein Bereich, der in der Geographie bislang kaum entwickelt ist.

\vspace{1em}

Aus dieser Leerstelle ergibt sich die zentrale Forschungsfrage dieser Arbeit:
\begin{quote}
\emph{Wie lässt sich der Einfluss räumlicher Umgebungen auf das situierte (Un\nobreakdash-)Wohl\-be\-find\-en intersektional positionierter Personen quantifizierbar erfassen und analysieren?}
\end{quote}

Zur Bearbeitung dieser Leitfrage formuliere ich drei spezifische Teilfragen, die deren Beantwortung aus methodischer, infrastruktureller und empirischer Perspektive vorbereiten:

\begin{enumerate}
    \item Wie muss ein Forschungsdesign gestaltet sein, um situiertes (Un\nobreakdash-)Wohl\-be\-find\-en intersektional positionierter Personen gemeinsam mit relevanten Kontextmerkmalen wiederholt zu erfassen?
    \item Welche Anforderungen ergeben sich aus einer feministisch"=digitalen Perspektive an eine Infrastruktur, die solche Erhebungen ermöglicht, und wie lassen sich diese praktisch umsetzen?
    \item Welche Möglichkeiten und Grenzen bieten die erhobenen Daten im Hinblick auf eine intersektionale Mehrebenenmodellierung?
\end{enumerate}

Die Beantwortung dieser Fragen erfordert einen Ansatz, der wiederholte Befragungen im Alltag mit räumlichem Bezug ermöglicht. Als methodische Basis verwende ich dafür die \glsxtrfull{ema}-Methode, die darauf abzielt, Erfahrungen möglichst unmittelbar im jeweiligen Kontext zu erfassen. Erweiterungen im Sinne der \glsxtrfull{gema}-Methode beziehen zusätzlich situative Umgebungsbedingungen ein und ermöglichen so beispielsweise, über Standortdaten auch Einflüsse wie Temperatur, Lärmbelastung oder Grünflächenanteil zu erfassen. In der Geographie werden beide Methoden bislang jedoch nur vereinzelt angewendet, obwohl sie ein hohes Potenzial bieten, den Einfluss räumlicher Kontexte systematisch zu untersuchen.

Für die Durchführung von \gls{ema}- und \gls{gema}-Studien existieren verschiedene digitale Infrastrukturen. Ich zeige in dieser Arbeit auf, dass diese entweder nicht offen zugänglich (proprietär) sind oder ihre Datenverarbeitung nur teilweise nachvollziehbar ist. Für die Erhebung sensibler Daten stellt dies eine zentrale Einschränkung dar: Wenn unklar bleibt, wie Daten gespeichert, verarbeitet oder weitergegeben werden, widerspricht dies den Prinzipien von Transparenz und Nachvollziehbarkeit. Aus einer feministisch"=digitalen Perspektive ist deshalb eine offene Infrastruktur erforderlich, die den gesamten Datenfluss überprüfbar macht und die Kontrolle über die erhobenen Daten sowohl bei den Teilnehmenden als auch bei den Forschenden belässt.

Vor diesem Hintergrund entwickle ich in dieser Arbeit mit der App \gls[noindex]{intermind}\footnotemark eine quelloffene Erhebungsplattform für \gls{ema}- und \gls{gema}-Studien. \gls[noindex]{intermind} ermöglicht wiederholte Befragungen im Alltag, erfasst Standortdaten und setzt dabei auf eine transparente und sichere Datenverarbeitung. Die offene Auslegung schafft die Grundlage für eine langfristig nutzbare, überprüfbare Infrastruktur zur Erhebung kontextualisierter Alltagsdaten.

\footnotetext{Ich habe mich dazu entschieden, nach der im Rahmen dieser Arbeit durchgeführten Datenerhebung, die App wieder aus den App Stores zu entfernen. In \cref{sec:diskussion} begründe ich diese Entscheidung ausführlicher. Der Quellcode der App ist vollständig auf \gls[noindex]{github} (\href{https://github.com/lbatschelet/InterMind}{github.com/lbatschelet/InterMind}) unter einer \gls{lic:agpl}-Lizenz veröffentlicht.}

In einer explorativen Pilotstudie erprobe ich das entwickelte Forschungsdesign aus Infrastruktur, Erhebungsdesign und Auswertungspfad. Dabei prüfe ich, ob die erhobenen Daten die nötige Differenzierung und Qualität für eine intersektionale Mehrebenenanalyse aufweisen und wo die Grenzen des Ansatzes liegen. Ziel ist der methodische Machbarkeitsnachweis, nicht die Generalisierung inhaltlicher Effekte.

Der Aufbau der Arbeit folgt einer Abfolge von theoretischer Rahmung, methodischer Herleitung und empirischer Umsetzung. In \Cref{sec:theoretischer_rahmen} führe ich zentrale Konzepte ein: \gls{intersektionalitaet} als Analyseperspektive, situiertes (Un\nobreakdash-)Wohl\-be\-find\-en als Gegenstand sowie feministisch"=digitale Ansätze als Leitlinie für die Gestaltung der Forschungsinfrastruktur. Darauf aufbauend verorte ich die Arbeit in \Cref{sec:methodik} im Feld wiederholter Alltagsbefragungen und diskutiere, wie sich diese mit intersektionalen Auswertungsansätzen verbinden lassen. 

Im Anschluss wende ich mich der praktischen Umsetzung zu: In \Cref{sec:entwicklung_app} beschreibe ich die Entwicklung der App \gls[noindex]{intermind}, bevor ich in \Cref{sec:fragebogenentwicklung} die Konstruktion des Fragebogens darstelle. Die Durchführung und Auswertung der Pilotstudie präsentiere ich in \Cref{sec:pilotstudie}. Den Abschluss bildet \Cref{sec:diskussion}, in dem ich zentrale Befunde reflektiere, methodische Implikationen diskutiere und Perspektiven für künftige Forschung skizziere.

Mit dieser Arbeit ziele ich darauf, methodische Potenziale einer intersektionalen, kontextnahen Erhebung von situiertem (Un\nobreakdash-)Wohl\-be\-find\-en sichtbar zu machen und eine Grundlage für künftige Anwendungen zu schaffen.
