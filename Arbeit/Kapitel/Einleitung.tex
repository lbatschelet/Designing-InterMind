% LTeX: language=de-CH
\chapter{Einleitung} \label{sec:einleitung}

Eine Parkbank steht am Rande eines kleinen Platzes. Unter den Füssen Beton, ein Baum wirft etwas Schatten, im Hintergrund sind Kinderstimmen zu hören. Doch dieser Ort löst nicht bei allen dasselbe aus. Für manche bedeutet er Ruhe, für andere Anspannung oder Distanz. Solche situativen Emotionen entstehen im Zusammenspiel materieller Eigenschaften (Licht, Geräusche, Gerüche, Temperatur), sozialer Dynamiken und individueller Erfahrungen -- und sie sind durch soziale Positionierungen mitgeprägt. In dieser Arbeit rücke ich dieses situative, kontextgebundene Erleben in den Mittelpunkt.

In der Geographie haben sich seit den 2000er-Jahren \emph{emotional} und \emph{affective geographies} herausgebildet, die alltägliche Gefühle und ihre räumlichen Dimensionen untersuchen \parencite{hoSocialGeographyIII2024}. Während \emph{emotional geographies} stärker auf bewusst artikulierte Gefühle fokussieren, betonen \emph{affective geographies} die körperlich-vorsprachliche Dimension von Empfindungen. Emotionen und Affekte werden dabei als verkörperte, relationale und situierte Phänomene verstanden, die Orte prägen und durch Orte geprägt werden. 

Die Trennung zwischen Affekt und Emotion ist jedoch umstritten. Feministische Autor\genderstern innen kritisieren, dass \emph{affective geographies} häufig zu wenig machtkritisch argumentieren, weil sie Affekte als vor- oder unbewusst behandeln und dadurch soziale Positionierungen und Ungleichheiten in den Hintergrund geraten \parencite{bondiIntroductionGeographysEmotional2006,rodo-de-zarateIntersectionalitySpatialityEmotions2023}. Diese Arbeit greift daher Ansätze auf, die affektives Erleben explizit mit intersektionalen Machtverhältnissen zusammendenken.

Intersektionale Ansätze wiederum zeigen, dass solche Erfahrungen nicht entlang einzelner sozialer Merkmale -- etwa \gls[noindex]{gender}, \emph{Alter} oder \gls[noindex]{class}\footnote{Soziale Kategorien wie \gls[noindex]{race}, \gls[noindex]{gender}, \gls[noindex]{class}, \emph{Alter} oder \emph{Behinderung} werden in dieser Arbeit kursiv gesetzt, um ihre Bedeutung als sozial konstruierte, wandelbare und gesellschaftlich wirkungsmächtige Kategorien hervorzuheben. Die Begriffe \gls[noindex]{race}, \gls[noindex]{gender} und \gls[noindex]{class} werden zudem in englischer Sprache verwendet, da ihre deutschen Übersetzungen in der wissenschaftlichen Debatte umstritten sind. Ausführlichere Erläuterungen finden sich im \nameref{sec:glossary}.} -- verstanden werden können, sondern durch deren Verschränkung und die daraus entstehenden sozialen Positionierungen. Der Begriff der \gls{intersektionalitaet} macht sichtbar, dass Diskriminierungen und Privilegien nicht additiv nebeneinanderstehen, sondern sich in ihrer Verschränkung gegenseitig verstärken, überlagern oder abschwächen. In der Geographie dominieren bislang qualitative Arbeiten, die solche Differenzen und Machtverhältnisse detailliert sichtbar machen, während systematische quantitative Umsetzungen selten sind.

Gerade im Zusammenspiel dieser beiden Perspektiven ergibt sich eine forschungspraktische Herausforderung: Zwar gibt es zahlreiche qualitativ ausgerichtete Studien, die affektive Erfahrungen und intersektionale Positionierungen in alltäglichen Situationen beschreiben, jedoch fehlen bislang methodische Ansätze, mit denen sich solche Dynamiken systematisch und quantitativ erfassen lassen. Nur wenige Versuche existieren, affektives Wohlbefinden in seiner räumlich-situativen und intersektionalen Dimension zugleich empirisch zu analysieren. Genau hier setzt diese Arbeit an: Sie entwickelt ein Verfahren zur Erhebung kontextgebundener Alltagsdaten und prüft ihre Eignung für intersektionale quantitative Auswertungen.

\vspace{1em}

Aus dieser forschungspraktischen Leerstelle ergeben sich die zentrale Forschungsfrage dieser Arbeit:
\begin{quote}
\emph{Wie lässt sich der Einfluss räumlicher Umgebungen auf das affektive Wohlbefinden intersektional positionierter Personen erfassen und analysieren?}
\end{quote}

Zur Bearbeitung dieser Leitfrage werden drei spezifische Teilfragen formuliert, die deren Beantwortung aus methodischer, infrastruktureller und empirischer Perspektive vorbereiten:

\begin{enumerate}
    \item Wie muss ein Erhebungsansatz gestaltet sein, um affektives Wohlbefinden intersektional positionierter Personen gemeinsam mit relevanten Kontextmerkmalen wiederholt in situ zu erfassen?
    \item Welche Anforderungen ergeben sich aus einer kritisch-digitalen Perspektive an eine Infrastruktur, die solche Erhebungen ermöglicht, und wie lassen sie diese praktisch umsetzen?
    \item Wie geeignet sind die in einer Pilotstudie erhobenen Daten für eine intersektionale Mehrebenenmodellierung?
\end{enumerate}

Die Beantwortung dieser Fragen erfordert einen Ansatz, der wiederholte Befragungen im Alltag mit räumlichem Bezug ermöglicht. Methodisch hat sich dafür in anderen Disziplinen die \gls{ema}-Methode etabliert, die darauf abzielt, Erfahrungen möglichst unmittelbar im jeweiligen Kontext zu erfassen. Erweiterungen im Sinne der \gls{gema}-Methode beziehen zusätzlich situative Umgebungsbedingungen ein und ermöglichen so beispielsweise, über Standortdaten auch Einflüsse wie Temperatur sichtbar zu machen. In der Geographie wurden diese Methoden bislang jedoch nur vereinzelt angewendet, obwohl sie ein hohes Potenzial bieten, den Einfluss räumlicher Kontexte systematisch zu untersuchen.

Für die Durchführung von \gls{ema}- und \gls{gema}-Studien existieren verschiedene digitale Infrastrukturen. Ich zeige in dieser Arbeit auf, dass diese entweder proprietär sind oder ihre Datenverarbeitung nur teilweise nachvollziehbar ist. Für die Erhebung sensibler Daten stellt dies eine zentrale Einschränkung dar: Wenn unklar bleibt, wie Daten gespeichert, verarbeitet oder weitergegeben werden, widerspricht dies den Prinzipien von Transparenz, Datensparsamkeit und Nachvollziehbarkeit. Aus einer kritisch-digitalen Perspektive ist deshalb eine offene Infrastruktur erforderlich, die den gesamten Datenfluss überprüfbar macht und die Kontrolle über die erhobenen Daten sowohl bei den Teilnehmenden als auch bei den Forschenden belässt.

Vor diesem Hintergrund habe ich in dieser Arbeit mit der App \gls[noindex]{intermind}\footnote{\href{https://intermind.ch/app}{intermind.ch/app}} eine quelloffene Infrastruktur für \gls{ema}- und \gls{gema}-Studien entwickelt. \gls[noindex]{intermind} ermöglicht wiederholte Befragungen im Alltag, erfasst Standortdaten und speichert Daten anonymisiert. Die offene Auslegung schafft die Grundlage für eine langfristig nutzbare, überprüfbare Infrastruktur zur Erhebung kontextualisierter Alltagsdaten.

In einer explorativen Pilotstudie erprobe ich das Zusammenspiel aus Infrastruktur, Erhebungsdesign und Auswertungspfad. Dabei prüfe ich, ob die erhobenen Daten die nötige Differenzierung und Qualität für eine intersektionale Mehrebenenanalyse aufweisen und wo die Grenzen des Ansatzes liegen. Ziel ist der methodische Machbarkeitsnachweis, nicht die Generalisierung inhaltlicher Effekte.

Der Aufbau der Arbeit folgt einer klaren Abfolge von theoretischer Rahmung, methodischer Herleitung und empirischer Umsetzung. In \Cref{sec:theoretischer_rahmen} führe ich zentrale Konzepte ein: \gls{intersektionalitaet} als Analyseperspektive, affektives Wohlbefinden als Gegenstand sowie kritisch-digitale Ansätze als Leitlinie für die Gestaltung der Forschungsinfrastruktur. Darauf aufbauend verorte ich die Arbeit in \Cref{sec:methodik} im Feld wiederholter Alltagsbefragungen und diskutiere, wie sich diese mit intersektionalen Auswertungsansätzen verbinden lassen. 

Im Anschluss wende ich mich der praktischen Umsetzung zu: In \Cref{sec:entwicklung_app} beschreibe ich die Entwicklung der offenen App \gls[noindex]{intermind}, bevor ich in \Cref{sec:fragebogenentwicklung} die Konstruktion des Fragebogens darstelle. Die Durchführung und Auswertung der Pilotstudie präsentiere ich in \Cref{sec:pilotstudie}. Den Abschluss bildet \Cref{sec:diskussion}, in der ich zentrale Befunde reflektiere, methodische Implikationen diskutiere und Perspektiven für künftige Forschung skizziere.

Mit dieser Arbeit ziele ich darauf, methodische Potenziale einer intersektionalen, kontextnahen Erhebung von Wohlbefinden sichtbar zu machen und eine Grundlage für künftige Anwendungen zu schaffen.
