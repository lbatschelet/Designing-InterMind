Diese Bachelorarbeit entwickelt und dokumentiert ein kritisch-digitales Forschungsdesign zur raumbezogenen Erfassung situierten (Un\nobreakdash-)Wohlbefindens aus intersektionaler Perspektive. Theoretisch knüpft die Arbeit an \emph{emotional geographies} und Intersektionalitätsforschung an und versteht Wohlbefinden als situierte Erfahrung, die durch räumliche und soziale Lagen mitgeprägt wird. Methodisch verbindet sie den \glsxtrshort{ema}/\glsxtrshort{gema}-Ansätze mit einer offenen, nachvollziehbaren Forschungsinfrastruktur: Mit der eigens entwickelten App InterMind werden wiederholte, geolokalisierte Befragungen im Alltag erhoben; Code und Workflows sind quelloffen und auf Transparenz, Nachnutzbarkeit und Datenschutz ausgelegt. Eine explorative Pilotstudie dient der Prüfung von Machbarkeit, Erhebungslogistik, Datenqualität und Nutzererfahrung sowie der Frage, ob die Datenstruktur für intersektionale Mehrebenenmodelle (\glsxtrshort[noindex]{i-maihda}) geeignet ist. Die Pilotstudie verfolgt ausdrücklich keinen inhaltlichen Wirksamkeitsnachweis; sie validiert Prozesse und macht methodische Grenzen sichtbar. Der Beitrag der Arbeit liegt in (1) der Bereitstellung einer quelloffenen  Erhebungsplattform für kontextsensitive Wiederholungsmessungen (\glsxtrshort{ema}/\glsxtrshort{gema}), (2) der Dokumentation eines reproduzierbaren Analysepfads und (3) einer reflektierten methodischen Einordnung, die Anschlussstellen für grössere, heterogenere Stichproben und weiterführende Mixed-Methods-Studien aufzeigt.