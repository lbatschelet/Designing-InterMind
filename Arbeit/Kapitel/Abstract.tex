% LTeX: language=de-CH

\chapter*{Abstract}

In dieser Bachelorarbeit entwickle und dokumentiere ich ein feministisch-digitales Forschungsdesign zur raumbezogenen Erfassung situierten (Un\nobreakdash-)Wohl\-be\-fin\-dens. Theoretisch knüpfe ich an \emph{emotional geographies} und \emph{health geographies} an und arbeite mit einer intersektionalen Perspektive. Ich verstehe Wohlbefinden als situierte Erfahrung, die durch räumliche und soziale Lagen (re\nobreakdash-)pro\-du\-ziert wird. Methodisch verbinde ich die \glsxtrshort{ema}/\glsxtrshort{gema}-Ansätze mit einer offenen, nachvollziehbaren Forschungsinfrastruktur: Mit der eigens entwickelten App \gls[noindex]{intermind} erhebe ich wiederholte, geolokalisierte Befragungen im Alltag; Code und Workflows sind quelloffen und auf Nachnutzbarkeit und Datenschutz ausgelegt. Eine explorative Pilotstudie dient der Prüfung von Machbarkeit, Erhebungslogistik, Datenqualität und Nutzererfahrung sowie der Frage, ob die Datenstruktur für intersektionale Mehrebenenmodelle (\glsxtrshort[noindex]{i-maihda}) geeignet ist. Sie verfolgt ausdrücklich keinen inhaltlichen Wirksamkeitsnachweis, sondern validiert Prozesse und macht methodische Grenzen sichtbar. Der Beitrag meiner Arbeit liegt in (1) der Bereitstellung einer quelloffenen Erhebungsplattform für kontextsensitive Wiederholungsmessungen (\glsxtrshort{ema}/\glsxtrshort{gema}), (2) der Dokumentation eines reproduzierbaren Analysepfads und (3) einer reflektierten methodischen Einordnung, die Anschlussstellen für weiterführende Mixed-Methods-Studien aufzeigt.