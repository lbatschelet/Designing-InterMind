\chapter{Kontextspezifisch und alltagstauglich -- Entwicklung des Fragebogens}
\label{sec:fragebogenentwicklung}

Zentrales methodisches Instrument dieser Arbeit ist ein Fragebogen, der erfasst, \emph{wie räumliche Umgebungen das momentane (Un-)Wohlbefinden intersektional positionierter Personen im Alltag beeinflussen}. Die Entwicklung des Fragebogens war unabhängig von der technischen Umsetzung in der App (\gls[noindex]{vgl} \cref{sec:entwicklung_app}) konzipiert und diente zugleich dazu, deren Flexibilität und Praxistauglichkeit zu prüfen.

Kernherausforderung war es, zwei Aspekte zu verbinden: Zum einen sollten grundlegende Merkmale zur Charakterisierung der Stichprobe erhoben werden (Baseline-Modul), zum anderen das situative, affektive (Un-)Wohlbefinden im unmittelbaren räumlichen und sozialen Kontext (\gls{ema}-Modul). Die Befragung sollte dabei so kurz wie möglich bleiben, um Akzeptanz und Teilnahmebereitschaft zu sichern. Als Zielvorgaben wurden eine maximale Dauer von zehn Minuten für die Baseline und drei Minuten für die wiederholten situativen Erhebungen festgelegt. Ergänzend wurde der Fragebogen mehrsprachig in Deutsch, Englisch und Französisch umgesetzt, um den Zugang für eine breite Teilnehmendengruppe zu ermöglichen.

Die Aufteilung in ein einmaliges Baseline-Modul und wiederholte situative Erhebungen folgt direkt aus den methodischen Anforderungen der Forschungsfrage: Die Baseline dient der Charakterisierung der Stichprobe für differenzierte intersektionale Analysen, während die situativen Fragen den eigentlichen Kern der Datenerhebung bilden, indem sie (Un-)Wohlbefinden in konkreten Alltagskontexten erfassen.

Der vollständige Fragebogen ist in \cref{app:appendix_fragebogen} zu finden.

\section{Kontext schaffen -- Einmalige Eingangsbefragung}

Die einmalige Baseline-Erhebung (siehe \cref{tab:baseline-fragen}) zielte darauf ab, die sozialen Positionierungen der Teilnehmenden möglichst differenziert zu erfassen. Erhoben wurden Merkmale wie Alter, \gls{gender}, sexuelle Orientierung, Behinderung\footnote{Der Begriff \emph{Behinderung} wird in dieser Arbeit bewusst gegenüber abwertenden und stigmatisierenden Bezeichnungen wie \emph{Invalidität} verwendet, entsprechend der inklusiven und aktuellen deutschen Sprachpraxis \parencite{gunkelSchreibenUndDiskutieren2022}.}
sowie \gls{class} \parencite{bauerIntersectionalityQuantitativeResearch2021}.

Die Erfassung von \gls[noindex]{race} erwies sich als methodisch anspruchsvoll. Im europäischen Kontext existieren kaum etablierte Kategorien, die rassifizierte Zugehörigkeiten erfassen, ohne problematische koloniale oder biologistische Zuschreibungen zu reproduzieren \parencite[\gls{vgl}][]{roigIntersectionalityEuropeDepoliticized2018}. Anders als in der US-amerikanischen Tradition, in der standardisierte Selbstkategorisierungen verbreitet sind, fehlen im hiesigen Kontext praktikable, breit akzeptierte Formate für quantitative Erhebungen. Aus diesem Grund wurde im Fragebogen lediglich erfasst, ob Teilnehmende aktuell in einem anderen Land leben als in jenem, in dem sie geboren wurden.

Auch die Erfassung von \gls{class} stellte methodische Anforderungen. Sie erfolgte über eine Kombination mehrerer sozioökonomischer Indikatoren: höchster Bildungsabschluss, aktuelle Beschäftigungssituation, Haushaltseinkommen sowie Anzahl der Haushaltsmitglieder und deren Einkommensbeitrag. Auf klassische Schemata wie \Acrfull{egp} oder \acrfull{esec} wurde verzichtet, da deren Operationalisierung detailliertere Daten zu standardisierten Berufen und sozialstrukturellen Kategorien erfordert hätte, was im Rahmen dieser Erhebung nicht praktikabel war \parencite{bihagenSocialClassEmployment2010}. Stattdessen wurde eine pragmatische, mehrdimensionale Annäherung gewählt, die zentrale Aspekte sozialer Lage abbildet, ohne den Fragebogen unnötig zu verlängern.

Zur Erfassung bereits erlebter Diskriminierung wurde ergänzend eine Multiple-Choice-Frage eingesetzt, die sowohl das Vorhandensein als auch den Kontext der Diskriminierung aus Sicht der Befragten erfasst. Die Antwortoptionen beziehen sich auf gesellschaftlich relevante Diskriminierungsdimensionen und wurden auf Basis einer pragmatischen Abwägung zwischen analytischer Relevanz, praktischer Umsetzbarkeit und der Zielsetzung einer kurzen und zugänglichen Befragung ausgewählt.

\section{Vom Ort zur Emotion -- situativ befragen}

Der situative Teil des Fragebogens (siehe \cref{tab:wiederholte-fragen}) erfasste die unmittelbare räumliche und soziale Umgebung der Befragten, um deren Einfluss auf das momentane affektive Wohlbefinden abzubilden. Zunächst wurde zwischen Innen- und Aussenaufenthalt unterschieden, gefolgt von einer genaueren Ortskategorisierung (\gls[noindex]{zb} Zuhause, Arbeitsplatz, Café, Park, öffentlicher Verkehr). Weitere erfasste Merkmale waren die Geräuschkulisse, Sichtbarkeit von Pflanzen oder Bäumen, Lebhaftigkeit sowie die subjektiv wahrgenommene Qualität des Ortes. Die soziale Umgebung wurde durch Angaben zu anwesenden Personen und deren Beziehung zu den Befragten beschrieben.

Die Gestaltung dieser Items orientierte sich an der Urban-Mind-Studie \parencite{bakolisUrbanMindUsing2018}, wurde jedoch in kompakter Form umgesetzt. Längere standardisierte Skalen zur Umgebungsqualität (\gls[noindex]{zb} \acrfull{peqi} \parencite{bonaiutoPerceivedResidentialEnvironment2015}, \acrfull{news} \parencite{saelensNeighborhoodEnvironmentWalkability2018}) erwiesen sich aufgrund ihrer Länge und Komplexität als ungeeignet für wiederholte Erhebungen. Die kompakte Umsetzung stellt somit einen bewussten methodischen Kompromiss dar.

Nach aktuellem Forschungsstand existiert kein standardisiertes und breit eingesetztes Instrument zur Erfassung \emph{situativen} affektiven Wohlbefindens, das für mehrfache Erhebungen pro Tag konzipiert ist. Die gängigen Skalen -- etwa \gls{panas} \parencite{yountMeasuringMoodComparison2023}, WHO-5 \parencite{toppWHO5WellBeingIndex2015} oder \gls{wemwbs} \parencite{tennantWarwickEdinburghMentalWellbeing2007} -- stammen überwiegend aus der psychologischen Gesundheitsforschung und sind auf mittlere bis längere Zeiträume (\gls[noindex]{zb} die letzten zwei Wochen) ausgelegt. Sie sind in Umfang und Formulierung nicht auf hochfrequente Erhebungen zugeschnitten und würden den zeitlichen Rahmen von wenigen Minuten pro Befragung deutlich überschreiten.

Vor diesem Hintergrund wurde ein eigener, stark reduzierter Item-Satz entwickelt, um zentrale Dimensionen des Wohlbefindens situativ abbilden zu können. Ausgewählt wurden fünf Dimensionen: generelles Wohlbefinden, Zufriedenheit, Anspannung, Energie und Zugehörigkeit. Die Antworten wurden über lineare Slider-Skalen erfasst, um eine schnelle und intuitive Bearbeitung zu ermöglichen. 

Ein zentrales Merkmal des Moduls war die Einbindung intersektionaler Perspektiven auf situativer Ebene. Ziel war es, nicht nur strukturelle Positionierungen (wie im Baseline-Modul), sondern auch deren situative Wechselwirkungen mit Raum und sozialer Wahrnehmung zu erfassen. Zu diesem Zweck wurden zwei Items entwickelt, die abfragten, ob das aktuelle Zugehörigkeits- oder Fremdheitsgefühl am Ort mit der eigenen gesellschaftlichen Positionierung zusammenhängt, sowie in welchen Merkmalen sich die Befragten im Vergleich zu Anwesenden als zugehörig oder nicht zugehörig empfanden.

Die Entwicklung dieser Items orientierte sich inhaltlich an den Überlegungen von \textcite{rodo-de-zarateIntersectionalitySpatialityEmotions2023} zur räumlichen Dimension von Emotionen und deren Rolle bei der (Re-)Produktion intersektionaler Ungleichheiten. Insbesondere die von Rodó-de-Zárate vorgeschlagene Differenzierung von (Un-)Wohlbefinden in Relation zu Machtgeometrien diente als konzeptioneller Ausgangspunkt. Mangels eines standardisierten, auf situative Mehrfacherhebungen zugeschnittenen Instruments erfolgte die konkrete Formulierung jedoch in einem pragmatischen, explorativen Prozess, mit dem Ziel, die Fragen in wenigen Sekunden beantworten zu können.

Ergänzend boten zwei offene Fragen Raum für die Benennung weiterer kontextgebundener Gründe für situatives (Un-)Wohlbefinden. Diese qualitativen Elemente ermöglichen es, affektive und kontextuelle Faktoren sichtbar zu machen, die durch geschlossene Fragen nicht erfasst werden können, und verhindern so eine Reduktion komplexer Ungleichheitsverhältnisse auf rein numerische Merkmale.

\section{Klar, verständlich, iterativ -- Der Weg zum finalen Fragebogen}

Die sprachliche Gestaltung der Fragebogen-Items stellte im Entwicklungsprozess eine zentrale methodische Herausforderung dar. Ziel war es, die Befragung möglichst zugänglich, verständlich und gleichzeitig inhaltlich präzise zu gestalten. Da die Erhebung explizit auf eine intersektionale Analyse abzielt, wurde besonderer Wert darauf gelegt, die sprachliche Zugänglichkeit möglichst breit zu gewährleisten. Der Fragebogen wurde daher mehrsprachig konzipiert und auf Deutsch, Englisch sowie Französisch umgesetzt. Weitere Sprachversionen wären aus Sicht der Zugänglichkeit sinnvoll gewesen, erforderten jedoch einen hohen Übersetzungs- und Abstimmungsaufwand, um inhaltliche Konsistenz zu sichern.

Ein bewusst gewählter Bestandteil der Konzeption war eine direkte, adressierende Sprache in der \enquote{Du}-Form. Sie sollte einen niederschwelligen Zugang fördern und hierarchische Distanz zwischen Forschenden und Teilnehmenden verringern. Gleichzeitig mussten komplexe Konzepte so operationalisiert werden, dass sie in alltagsnaher, schnell erfassbarer Form vermittelt werden konnten. So wurde das Konzept der \gls{intersektionalitaet} im Einführungsteil erläutert, in den eigentlichen Items jedoch vermieden, um unnötige Barrieren zu verhindern. Stattdessen kamen allgemeinere Formulierungen wie \enquote{persönliche Merkmale} zum Einsatz.

Besondere Aufmerksamkeit erforderte die Übersetzung und Anpassung zentraler Begriffe zwischen den Sprachversionen. Im Fall von \gls[noindex]{race} stellte sich insbesondere im deutschsprachigen Kontext die Frage nach geeigneten Begrifflichkeiten, da etablierte Termini entweder ungebräuchlich, problematisch oder unpräzise sind \parencite[\gls{vgl}][]{roigIntersectionalityEuropeDepoliticized2018}. Auch bei affektiven Zustandsbeschreibungen wurden die Formulierungen nicht wörtlich, sondern sinngemäß übertragen und kulturelle Unterschiede in der Wortverwendung berücksichtigt.

Der Übersetzungsprozess war Teil eines iterativen Entwicklungsablaufs, der auf Literaturrecherche, Rückmeldungen aus der Testphase der App (siehe \cref{sec:app_entwicklung_feldtest}) und Abstimmungen mit der betreuenden Dozentin basierte. Mehrere Überarbeitungsrunden führten zu sprachlichen und strukturellen Anpassungen, die sowohl die Verständlichkeit als auch die Anschlussfähigkeit der Items verbesserten. Ein durchgängiges Kriterium war dabei, den zeitlichen und kognitiven Aufwand für Teilnehmende gering zu halten, ohne zentrale Aspekte der Forschungsfrage zu vernachlässigen.

