\chapter{Kontextspezifisch und alltagstauglich -- Entwicklung des Fragebogens}
\label{sec:fragebogenentwicklung}

Zentrales methodisches Instrument dieser Arbeit ist ein Fragebogen, der erfasst, \emph{wie räumliche Umgebungen das momentane (Un\nobreakdash-)Wohl\-be\-find\-en intersektional positionierter Personen im Alltag beeinflussen}. Die Entwicklung des Fragebogens ist unabhängig von der technischen Umsetzung in der App (\cref{sec:entwicklung_app}) konzipiert und dient aber gleichzeitig dazu, deren Flexibilität und Praxistauglichkeit zu prüfen.

Kernherausforderung ist es, zwei Aspekte zu verbinden: Zum einen sollen grundlegende Merkmale zur Charakterisierung der Stichprobe erhoben werden die im Anschluss eine intersektionale Mehrebenenanalyse ermöglichen (Baseline-Modul), zum anderen das situierte (Un\nobreakdash-)Wohl\-be\-find\-en im unmittelbaren räumlichen und sozialen Kontext (\gls{ema}-Modul). Die Befragung soll dabei so kurz wie möglich bleiben, um Akzeptanz und Teilnahmebereitschaft zu sichern. Als Zielvorgaben lege ich dafür eine maximale Dauer von zehn Minuten für die Baseline und drei Minuten für die wiederholten situativen Erhebungen fest. Der Fragebogen ist mehrsprachig in Deutsch, Englisch und Französisch umgesetzt, und orientiert sich an einer Alltagssprache, um den Zugang für eine breite Teilnehmendengruppe zu ermöglichen. 

Die Aufteilung in ein einmaliges Baseline-Modul und wiederholte situative Erhebungen folgt direkt aus den methodischen Anforderungen der Forschungsfrage: Die Baseline dient der Charakterisierung der Stichprobe für eine intersektionale Mehrebenenanalyse, während die situativen Fragen den eigentlichen Kern der Datenerhebung bilden, indem sie (Un\nobreakdash-)Wohl\-be\-find\-en in konkreten Alltagskontexten erfassen.

Der vollständige Fragebogen ist in \cref{app:appendix_fragebogen} zu finden.

\section{Kontext schaffen -- Einmalige Eingangsbefragung}

Die einmalige Baseline-Erhebung (siehe \cref{tab:baseline-fragen}) zielt darauf ab, die sozialen Positionierungen der Teilnehmenden möglichst differenziert zu erfassen. Erhoben werden Merkmale wie \emph{Alter}, \emph{Geschlecht}, \emph{sexuelle Orientierung}, \emph{(Dis-)Ability}
sowie \gls{class} \parencite{bauerIntersectionalityQuantitativeResearch2021}.

Die Erfassung von \gls{race} erweist sich als methodisch anspruchsvoll. Im deutschsprachigen Kontext existieren keine in der Alltagssprache verwendeten und etablierte Kategorien, die rassifizierte Zugehörigkeiten erfassen, ohne problematische koloniale oder biologistische Zuschreibungen zu reproduzieren \parencite[\gls{vgl}][]{roigIntersectionalityEuropeDepoliticized2018}. In der US-amerikanischen Tradition sind standardisierte Selbstkategorisierungen zwar weit verbreitet, jedoch nicht unproblematisch, da sie spezifische historische Kontexte widerspiegeln und Kategorien naturalisieren können. Vor diesem Hintergrund erfasse ich im Fragebogen lediglich, ob die Teilnehmenden aktuell in einem anderen Land leben als jenem, in dem sie geboren wurden.

Auch die Erfassung von \gls{class} stellt methodische Anforderungen. Sie erfolgt über eine Kombination mehrerer sozioökonomischer Indikatoren: höchster Bildungsabschluss, aktuelle Beschäftigungssituation, Haushaltseinkommen sowie Anzahl der Haushaltsmitglieder und deren Einkommensbeitrag. Ich verzichte auf klassische Schemata wie \glsxtrfull{egp} oder \glsxtrfull{esec}, da deren Operationalisierung detailliertere Daten zu standardisierten Berufen und sozialstrukturellen Kategorien erfordert \parencite{bihagenSocialClassEmployment2010}, was im Rahmen dieser Erhebung nicht praktikabel ist. Stattdessen wähle ich eine pragmatische, mehrdimensionale Annäherung, die zentrale Aspekte sozialer Lage abbildet, ohne den Fragebogen unnötig zu verlängern.

Zur Erfassung erlebter Formen der Diskriminierung setzte ich ergänzend eine Multiple-Choice-Frage ein, die sowohl das Vorhandensein als auch den Kontext der Diskriminierung aus Sicht der Befragten erfasst. Die Antwortoptionen beziehen sich auf gesellschaftlich relevante Diskriminierungsdimensionen und sind auf Basis einer pragmatischen Abwägung zwischen analytischer Relevanz, praktischer Umsetzbarkeit und der Zielsetzung einer kurzen und zugänglichen Befragung ausgewählt.

\section{Vom Ort zur Emotion -- situativ befragen}

Der situative Teil des Fragebogens (siehe \cref{tab:wiederholte-fragen}) erfasst die unmittelbare räumliche und soziale Umgebung der Befragten, um deren Einfluss auf das situierte (Un\nobreakdash-)Wohl\-be\-find\-en abzubilden. Zunächst unterscheide ich ob sich die Teilnehmenden Drinnen oder Draussen befinden, gefolgt von einer genaueren Ortskategorisierung (\gls[noindex]{zb} Zuhause, Arbeitsplatz, Café, Park, öffentlicher Verkehr). Weiter erfasse ich Merkmale wie die Geräuschkulisse, Sichtbarkeit von Pflanzen oder Bäumen, Lebhaftigkeit sowie die subjektiv wahrgenommene Qualität des Ortes. Die soziale Umgebung wird durch Angaben zu anwesenden Personen und deren Beziehung zu den Teilnehmenden beschrieben.

Die Gestaltung dieser Items orientiert sich an der Urban Mind-Studie \parencite{bakolisUrbanMindUsing2018}, wird jedoch in veränderter Form umgesetzt. Längere standardisierte Skalen zur Umgebungsqualität (\gls[noindex]{zb} \glsxtrfull{peqi} \parencite{bonaiutoPerceivedResidentialEnvironment2015}, \glsxtrfull{news} \parencite{saelensNeighborhoodEnvironmentWalkability2018}) sind aufgrund ihrer Länge und Komplexität ungeeignet für wiederholte Erhebungen. Die kompakte Umsetzung stellt somit einen bewussten methodischen Kompromiss dar.

Es existiert kein standardisiertes und breit eingesetztes Instrument zur Erfassung situierten Wohlbefindens, das für mehrfache Erhebungen pro Tag konzipiert ist. Die gängigen Skalen -- etwa \gls{panas} \parencite{yountMeasuringMoodComparison2023}, WHO-5 \parencite{toppWHO5WellBeingIndex2015} oder \gls{wemwbs} \parencite{tennantWarwickEdinburghMentalWellbeing2007} -- stammen überwiegend aus der psychologischen Gesundheitsforschung und sind auf mittlere bis längere Zeiträume (\gls[noindex]{zb} die letzten zwei Wochen) ausgelegt. Sie sind in Umfang und Formulierung nicht auf hochfrequente Erhebungen zugeschnitten und würden den zeitlichen Rahmen von wenigen Minuten pro Befragung deutlich überschreiten.

Vor diesem Hintergrund entwickle ich einen eigenen, stark reduzierten Item-Satz, um zentrale Dimensionen des Wohlbefindens situativ abbilden zu können. Dafür wähle ich fünf Dimensionen: generelles Wohlbefinden, Zufriedenheit, Anspannung, Energie und Zugehörigkeit. Die Antworten werden über lineare Slider-Skalen erfasst, um eine schnelle und intuitive Bearbeitung zu ermöglichen. 

Ein zentrales Merkmal des Moduls ist die Einbindung intersektionaler Perspektiven auf situativer Ebene. Ziel ist es, nicht nur strukturelle Positionierungen (wie im Baseline-Modul), sondern auch deren situative Wechselwirkungen mit Raum und sozialer Wahrnehmung zu erfassen. Zu diesem Zweck entwickle ich zwei Items, die abfragen, ob das aktuelle Zugehörigkeits- oder Fremdheitsgefühl am Ort mit der eigenen sozialen Positionierung zusammenhängt, sowie in welchen Merkmalen sich die Befragten im Vergleich zu Anwesenden als zugehörig oder nicht zugehörig empfanden.

Die Entwicklung dieser Items orientiert sich inhaltlich an den Überlegungen von \textcite{rodo-de-zarateIntersectionalitySpatialityEmotions2023} zur räumlichen Dimension von Emotionen und deren Rolle bei der (Re-)Produktion intersektionaler Ungleichheiten. Insbesondere die von Rodó-de-Zárate vorgeschlagene Differenzierung von (Un\nobreakdash-)Wohl\-be\-find\-en in Relation zu Machtgeometrien diente als konzeptioneller Ausgangspunkt. Mangels eines standardisierten, auf situative Mehrfacherhebungen zugeschnittenen Instruments erfolgt die konkrete Formulierung jedoch in einem pragmatischen, explorativen Prozess, mit dem Ziel, die Fragen in wenigen Sekunden beantworten zu können.

Ergänzend bieten zwei offene Fragen Raum für die Benennung weiterer kontextgebundener Gründe für situatives (Un\nobreakdash-)Wohl\-be\-find\-en. Diese qualitativen Elemente ermöglichen es, affektive und kontextuelle Faktoren sichtbar zu machen, die durch geschlossene Fragen nicht erfasst werden können, und verhindern so eine Reduktion komplexer Ungleichheitsverhältnisse auf rein numerische Merkmale.

\section{Klar, verständlich, iterativ -- Der Weg zum finalen Fragebogen}

Die sprachliche Gestaltung der Fragebogen-Items stellt im Entwicklungsprozess eine zentrale methodische Herausforderung dar. Ziel ist es, die Befragung möglichst zugänglich, verständlich und gleichzeitig inhaltlich präzise zu gestalten. Da meine Erhebung explizit auf eine intersektionale Analyse abzielt, lege ich besonderen Wert auf eine möglichst breite sprachliche Zugänglichkeit. Deshalb konzipiere ich den Fragebogen mehrsprachig und setze ihn auf Deutsch, Englisch sowie Französisch um. Weitere Sprachversionen wären aus meiner Sicht sinnvoll, erfordern jedoch einen hohen Übersetzungs- und Abstimmungsaufwand, um inhaltliche Konsistenz zu sichern.

Ein bewusst gewählter Bestandteil der Konzeption ist eine direkte, adressierende Sprache in der \enquote{Du}-Form. Sie sollte einen niederschwelligen Zugang fördern und hierarchische Distanz zwischen mir und den Teilnehmenden verringern. Gleichzeitig musste ich komplexe Konzepte so operationalisieren, dass sie in alltagsnaher, schnell erfassbarer Form vermittelt werden können. So wird \gls{bspw} das Konzept der \gls{intersektionalitaet} im Einführungsteil erläutert, in den eigentlichen Items jedoch vermieden, um unnötige Barrieren zu verhindern. Stattdessen kamen allgemeinere Formulierungen wie \enquote{persönliche Merkmale} zum Einsatz.

Besondere Aufmerksamkeit erfordert die Übersetzung und Anpassung zentraler Begriffe zwischen den Sprachversionen. Im Fall von \gls[noindex]{race} stellt sich insbesondere die Frage nach geeigneten Begrifflichkeiten, da Begriffe entweder ungebräuchlich, problematisch oder unpräzise sind \parencite[\gls{vgl}][]{roigIntersectionalityEuropeDepoliticized2018}. So versuche ich auch bei affektiven Zustandsbeschreibungen die Formulierungen nicht wörtlich, sondern sinngemäss zu übertragen und kulturelle Unterschiede in der Wortverwendung zu berücksichtigen.

Der Übersetzungsprozess ist damit ebenso Teil eines iterativen Entwicklungsablaufs, der auf Literaturrecherche, Rückmeldungen aus der Testphase der App (siehe \cref{sec:app_entwicklung_feldtest}) und Abstimmungen mit den Betreuenden dieser Bachelorarbeit basiert. Mehrere Überarbeitungsrunden führen zu sprachlichen und strukturellen Anpassungen, die sowohl die Verständlichkeit als auch die Anschlussfähigkeit der Items verbessern. Ein durchgängiges Kriterium ist dabei, den zeitlichen und kognitiven Aufwand für Teilnehmende gering zu halten, ohne zentrale Aspekte der Forschungsfrage zu vernachlässigen.

