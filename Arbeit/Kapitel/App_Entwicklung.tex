\chapter{«Build your own tools»: Entwicklung der App Intermind}
\label{sec:entwicklung_app}

Im Zuge dieser Arbeit wurde die App \gls{intermind} entwickelt, die als technische Grundlage für pseudonymisierte \gls[noindex]{ema} und \gls[noindex]{gema}-Befragungen dient. Die App und der in dieser Arbeit eingesetzte Fragenkatalog wurden parallel und iterativ konzipiert. Während dieser Abschnitt die technische Entwicklung der App dokumentiert, wird die inhaltliche Gestaltung des Fragebogens im \cref{sec:fragebogenentwicklung} erläutert. 

Der vollständig dokumentierte Quellcode der App ist auf \gls[noindex]{github}\footnote{\href{https://github.com/lbatschelet/intermind}{https://github.com/lbatschelet/intermind}} unter einer \gls{lic:agpl}-Lizenz veröffentlicht.


\section{From Scratch -- Warum eine eigene App?}
\label{sec:entwicklung_app_begruendung}

Um die Fragestellung dieser Arbeit zu bearbeiten, wurde eine Plattform benötigt, welche wiederholte, geolokalisierte und kontextsensitive Erhebungen im Alltag der Teilnehmenden ermöglicht. Naheliegend wäre der Rückgriff auf bestehende und in Forschung eingesetzte Plattformen wie \gls{urbanmind}. Wie in \cref{sec:vergleich_bestehender_instrumente} beschrieben, ist diese App aber nicht quelloffen und daher weder vollständig nachvollziehbar noch eigenständig anpassbar. Insbesondere bei der Erhebung sensibler Daten zu Wohlbefinden, sozialen Positionierungen und erlebter Diskriminierung ist eine transparente, kontrollierbare und sichere Datenverarbeitung jedoch essenziell.

Auch kommerzielle Lösungen wie die Marktforschungsplattform \textit{Avicenna}\footnote{\href{https://avicennaresearch.com/}{https://avicennaresearch.com/}} kommen nicht infrage -- neben hohen Lizenzkosten bieten auch sie nur eingeschränkte Anpassungs- und Kontrollmöglichkeiten und erfüllen zentrale ethische Anforderungen nicht.

Aus dieser Analyse ergibt sich die Notwendigkeit, ein eigene Plattform zu entwickeln, die diesen Anforderungen gerecht wird. Sie soll mobil und einfach nutzbar sein, Antworten im situativen Alltag der Teilnehmenden ermöglichen und Standortdaten automatisch erfassen. Dabei sollen datenschutzrechtliche und technische Hürden möglichst gering gehalten und die Umsetzung im Rahmen dieser Arbeit realisierbar sein.
Gleichzeitig soll sie so flexibel und nachhaltig gestaltet sein, dass Fragenkataloge, Inhalte und Erhebungslogik für zukünftige Forschungsvorhaben problemlos angepasst werden können.

Die Entscheidung zur Entwicklung einer eigenen Erhebungs-Plattform ist nicht nur technisch motiviert, sondern folgt auch einer forschungsethischen Logik: Wie im \cref{sec:datafeminism} dargelegt, sind digitale Infrastrukturen nie neutral, sondern Ausdruck gesellschaftlicher Machtverhältnisse. Eine transparente und kontrollierbare Datenverarbeitung ist insbesondere dann zentral, wenn -- wie im vorliegenden Projekt -- sensible Informationen zu Wohlbefinden, sozialer Zugehörigkeit und Diskriminierung erhoben werden. Die Entscheidung für eine \gls[noindex]{opensource}-Architektur ist dabei Ausdruck eines bewussten Gestaltungswillens im Sinne digitaler Souveränität: Die gesamte Infrastruktur soll nachvollziehbar, anpassbar und kollektiv weiterentwickelbar bleiben, um technologische Gestaltungsmacht nicht an proprietäre Systeme abzugeben, sondern sie partizipativ zurückzugewinnen.

\section{Konzeption und Anforderungen -- Der Weg zur eigenen Infrastruktur}
\label{sec:app_entwicklung_anforderungen}

Auf Basis der beschriebenen Anforderungen wurde zunächst ein detaillierter Anforderungskatalog entwickelt, der als zentraler Leitfaden für die weiteren Schritte der Entwicklung diente. Dieser Katalog wurde iterativ ergänzt, konkretisiert und während des gesamten Entwicklungsprozesses kontinuierlich an methodische und technische Erkenntnisse angepasst. Die Klassifikation der Anforderungen erfolgt orientiert an der in der Softwareentwicklung üblichen Unterscheidung zwischen funktionalen und nicht-funktionalen Anforderungen.

Funktionale Anforderungen definieren konkret, \textit{was} die App leisten muss, und legen somit die notwendigen Funktionen und Abläufe der Anwendung fest. Für diese Anwendung bedeutet dies insbesondere, dass die App den Teilnehmenden täglich mehrere zufällig verteilte Zeitfenster zur Beantwortung von Fragen ermittelt und jeweils zu Beginn dieser Zeiträume \glspl{pushnotification} sendet. Da gängige Webbrowser keine verlässlichen Push-Benachrichtigungen oder zeitgesteuerten Hintergrundprozesse erlauben, schliesst diese Anforderung eine browserbasierte Erhebung aus und führt zur Entscheidung für eine App-basierte Lösung. Die App erfasst bei jeder Befragung automatisiert den aktuellen \gls{gps}-Standort. Um die Erhebung flexibel und bedarfsgerecht zu gestalten, unterstützt sie verschiedene Fragetypen -- darunter Single-Choice, Multiple-Choice, Skalen-basierte Fragen (Slider) sowie Freitextfelder. Im Sinne der Selbstbestimmung über die eigenen Daten ist es funktional zwingend vorgesehen, dass Teilnehmende sämtliche mit ihrem Gerät verknüpften Daten eigenständig und dauerhaft löschen können. Die Teilnahme erfolgt vollständig pseudonym, ohne dass eine Registrierung oder die Angabe personenbezogener Daten erforderlich ist. Darüber hinaus muss die App auf \gls{android}- und \gls{ios}-Geräten lauffähig sein, in Deutsch, Englisch und Französisch verfügbar sein und die Möglichkeit zur Erweiterung um weitere Sprachen bieten. Eine ursprünglich geplante Offlinefähigkeit wurde im Verlauf der Entwicklung verworfen, da sie zu Inkompatibilitäten bei der Aktualisierung des Fragenkatalogs geführt hätte.

Nicht-funktionale Anforderungen legen fest, \textit{wie} die oben beschriebenen Funktionen umgesetzt werden sollen, und beschreiben qualitative Merkmale wie Sicherheit, Benutzerfreundlichkeit oder technische Nachvollziehbarkeit. Zu den zentralen nicht-funktionalen Anforderungen zählen Datenschutz, Datensicherheit und technische Qualität. Sämtliche Datenverarbeitungsprozesse müssen im Einklang mit dem Schweizer Datenschutzgesetz (\glsxtrshort{dsg}) sowie der Europäischen Datenschutzgrundverordnung (\glsxtrshort{dsgvo}) erfolgen. Darüber hinaus ist sicherzustellen, dass alle Datenübertragungen verschlüsselt erfolgen und keine Dritten Zugriff auf die gespeicherten Daten erhalten. Diese Ausgestaltung folgt nicht nur rechtlichen Vorgaben, sondern knüpft auch an die im \cref{sec:datafeminism} entwickelten Prinzipien einer digitalen Souveränität an, die Transparenz, Kontrolle und Selbstbestimmung in den Mittelpunkt stellt. Eine offene, modulare und nachvollziehbare Codebasis soll gewährleisten, dass Anpassungen und Erweiterungen des Systems durch andere Forschende mit minimalem Aufwand möglich sind. Dies wurde durch die Veröffentlichung der App als \gls{opensource}-Projekt auf \gls{github} umgesetzt.

Zur systematischen Umsetzung der Anforderungen wird ein iterativer Entwicklungsprozess auf Basis von \glspl{githubissue} genutzt, in dem jede funktionale und nicht-funktionale Anforderung als eigenes \glslink{githubissue}{Issue} dokumentiert und mit einem Meilenstein versehen ist, der den geplanten Umsetzungszeitpunkt markiert. Diese Meilensteine orientieren sich an vier Entwicklungsstufen: Als \textit{core \glsxtrfull{mvp}} wird die minimal funktionsfähige Version der App bezeichnet, die alle für die Durchführung der Studie zwingend notwendigen Funktionen enthält, wie etwa die zeitgesteuerte Versendung von \glspl{pushnotification}, die Erfassung des \gls{gps}-Standorts oder die Bereitstellung zentraler Fragetypen. Das \textit{extended \gls{mvp}} umfasst zusätzliche Funktionen, die den Erhebungsprozess verbessern, für die Beantwortung der Forschungsfragen jedoch nicht zwingend erforderlich sind, beispielsweise die Unterstützung mehrerer Sprachen oder zusätzliche Fragetypen. Der Meilenstein \textit{app store release} umfasst alle Aufgaben, die für die Veröffentlichung in App-Stores erforderlich sind, jedoch keinen direkten Einfluss auf die eigentliche Datenerhebung oder Kernfunktionen der App haben. Dazu zählen begleitende Arbeiten wie die Erstellung einer Projektwebsite mit Datenschutzrichtlinie, die Bereitstellung der für die App-Store-Einreichung notwendigen Assets, die Einrichtung einer kontinuierlichen Integrations- und Auslieferungspipeline (\gls{cicd}) sowie die Durchführung des formalen Prüf- und Freigabeprozesses der App-Stores. Unter \textit{future enhancements} werden schliesslich langfristig geplante Erweiterungen verstanden, die den Funktionsumfang der App über die Anforderungen der vorliegenden Arbeit hinaus erweitern. Derzeit stehen hier vor allem eine Offlinefähigkeit der App sowie die Möglichkeit einer direkten Auswertung der erhobenen Daten innerhalb der App auf der Liste, wobei noch offen ist, ob und in welchem Umfang diese Funktionen umgesetzt werden. Die Priorisierung innerhalb dieser Kategorien orientiert sich an den Forschungszielen, den rechtlichen Vorgaben, der technischen Machbarkeit sowie den in \cref{sec:datafeminism} ausgeführten Prinzipien, wobei Änderungen am Funktionsumfang während der Entwicklung fortlaufend in den entsprechenden \glslink{githubissue}{Issues} dokumentiert werden.


\section{Technische Umsetzung -- Prinzipien, Praktiken und Kompromisse}
\label{sec:app_entwicklung_technische_umsetzung}

Die technische Umsetzung folgt etablierten Prinzipien der Softwareentwicklung, insbesondere \textit{Privacy by Design} \parencite{cavoukianPrivacyDesign72009} und den Gestaltungsprinzipien von \gls{solid} \parencite{martinCleanArchitectureCraftsmans2018}. Ziel ist eine modulare, wartbare und erweiterbare Architektur, die funktionale Anforderungen effizient umsetzt und nicht-funktionale Anforderungen -- insbesondere Datenschutz und Sicherheit -- von Beginn an integriert. Dabei wird eine klare Trennung zwischen Anwendungslogik, Datenhaltung und Benutzeroberfläche konsequent umgesetzt, um spätere Anpassungen und Erweiterungen mit minimalem Eingriff in bestehende Komponenten zu ermöglichen.

Für die Entwicklung der mobilen Anwendung wurde \gls{reactnative} in Kombination mit \gls{expo} gewählt. \gls{reactnative} ist ein von \gls{meta} entwickeltes, \gls{opensource} \gls{framework}, das die Entwicklung plattformübergreifender Anwendungen mit einer einzigen Codebasis ermöglicht. Dadurch können \gls{ios}- und \gls{android}-Versionen parallel gepflegt werden, was den Entwicklungs- und Wartungsaufwand erheblich reduziert. Obwohl React Native ursprünglich von einem grossen Technologiekonzern stammt, erfolgt in diesem Projekt keinerlei Datenaustausch mit \gls{meta}, da ausschliesslich das in der Entwicklungsumgebung installierte Framework verwendet wird, das weder auf den Endgeräten der Teilnehmenden noch auf externen Servern von \gls{meta} ausgeführt wird.

\gls{expo} ergänzt React Native um eine ebenfalls \gls{opensource} integrierte Entwicklungsumgebung mit Werkzeugen für Build, Test und Veröffentlichung. Dies erlaubt es, zentrale Infrastrukturaufgaben ohne eigenes \gls{devops}-Team effizient umzusetzen. Insbesondere die Möglichkeit, native Funktionen wie Push-Benachrichtigungen, Kamera- oder Standortzugriff über ein einheitliches API zu nutzen, beschleunigt die Umsetzung und reduziert die Komplexität der Codebasis. 

Als serverseitige Infrastruktur kommt \gls{supabase} zum Einsatz -- ein \gls{opensource} \gls{backend}-as-a-Service auf Basis von \gls{postgresql}, das Authentifizierung, Autorisierung, Datenspeicherung und Schnittstellenbereitstellung integriert. Die Entscheidung für Supabase erfolgte bewusst gegen den Einsatz von \gls{firebase}, das als De-facto-Standard für mobile Anwendungen gilt und in vielen Bereichen eine einfachere Implementierung ermöglicht hätte. Firebase ist jedoch ein proprietärer Dienst von \gls{google}, der zentrale Kontrolle über die Infrastruktur ausübt, den Serverstandort nicht frei wählen lässt und potenziell die Datenhoheit der Forschenden einschränkt. Wie in \cref{sec:datafeminism} ausgeführt, stehen solche zentralistischen Strukturen im Widerspruch zu Prinzipien digitaler Souveränität. Supabase ermöglicht hingegen, den Standort des Servers (hier: Schweiz) festzulegen und bietet die Option eines vollständig selbstverwalteten und gehosteten Betriebs. Neben der offenen Lizenz und der \gls{tech:sql}-basierten Datenstruktur war auch die Möglichkeit eines kostenlosen Hostings für kleine Projekte ausschlaggebend, wodurch der Betrieb ohne zusätzliche Infrastrukturkosten möglich ist. Die Wahl dieser Toolchain stellt damit einen pragmatischen Kompromiss dar: Sie bietet die notwendige technische Leistungsfähigkeit und Flexibilität, ohne die Kontrolle über Daten an externe Plattformanbieter abzugeben.

\begin{figure}[h]
    \centering
    \begin{minipage}[t]{0.38\textwidth}
        \centering
        \includegraphics[width=\textwidth]{Arbeit/Bilder/printscreens/startscreen.jpeg}
        \caption{Startbildschirm der App \gls[noindex]{intermind}}
        \label{fig:startscreen}
    \end{minipage}
    \hspace{0.1\textwidth}
    \begin{minipage}[t]{0.38\textwidth}
        \centering
        \includegraphics[width=\textwidth]{Arbeit/Bilder/printscreens/welcome.jpeg}
        \caption{Begrüssungstext der App \gls[noindex]{intermind}}
        \label{fig:welcome}
    \end{minipage}
\end{figure}

Der Quellcode folgt einer komponentenbasierten Struktur, in der jede Funktion klar abgegrenzte Verantwortlichkeiten besitzt. Diese Struktur erleichtert nicht nur die Wiederverwendung bestehender Module, sondern unterstützt auch die Adaption der Anwendung für andere Forschungsprojekte mit ähnlichem methodischen Aufbau. Die konkreten Fragebögen (\gls[noindex]{vgl} \cref{sec:fragebogenentwicklung}) werden nicht im Quellcode gespeichert, sondern als \gls{json}-Konfigurationsdateien in der \gls{tech:db} hinterlegt. Die App lädt diese Inhalte dynamisch beim Start oder bei Bedarf nach, wodurch Änderungen am Fragenkatalog ohne App-Update möglich sind. Die Entscheidung für serverseitige Speicherung erhöht die Flexibilität, birgt jedoch den Nachteil, dass eine aktive Internetverbindung erforderlich ist. Auf eine vollständige Offlinefähigkeit wird bewusst verzichtet, um Inkonsistenzen zwischen verschiedenen App-Versionen zu vermeiden und stets aktuelle Inhalte bereitzustellen.

Die datenschutzbezogene Umsetzung basiert auf einer strikten Pseudonymisierung. Beim ersten Start generiert die App automatisch eine gerätegebundene \gls{uuid}, die für alle weiteren Interaktionen verwendet wird. Aus Sicht des Systems existieren damit keine individuellen Nutzer\genderstern innen, sondern ausschliesslich Geräte-IDs. Personenbezogene Daten wie Name, Telefonnummer oder E-Mail-Adresse werden nicht erhoben. Standortdaten werden ausschliesslich zum Zeitpunkt einer beantworteten Befragung erfasst. Die Löschung aller mit einer \gls{uuid} verknüpften Datensätze kann jederzeit direkt in der App ausgelöst werden und entfernt sämtliche Einträge aus der \gls{tech:db}.

Der Zugriffsschutz wird durch eine Zugriffskontrolle auf Zeilenebene (\gls{rls}) in der \gls{postgresql}-\gls{tech:db} realisiert. Jede Anfrage an den Server ist an die jeweilige \gls{uuid} gebunden; Abfragen liefern nur Datensätze, die mit dieser ID verknüpft sind. Alle Datenübertragungen zwischen App und Server erfolgen verschlüsselt über authentifizierte Schnittstellen. Die Serverinfrastruktur befindet sich physisch in der Schweiz und unterliegt damit dem Schweizer Datenschutzgesetz (\gls{dsg}); zusätzlich werden die Vorgaben der Europäischen Datenschutzgrundverordnung (\gls{dsgvo}) eingehalten. Die vollständigen Regelungen sind in einer öffentlich zugänglichen Datenschutzrichtlinie dokumentiert, die in der App sowie auf der Projektwebseite\footnote{\href{https://intermind.ch/privacy-policy.html}{https://intermind.ch/privacy-policy.html}} verfügbar ist.

\begin{figure}[h]
    \centering
    \begin{minipage}[t]{0.38\textwidth}
        \centering
        \includegraphics[width=\textwidth]{Arbeit/Bilder/printscreens/beschaeftigung.jpeg}
        \caption{Multiple-Choice-Frage zur aktuellen Beschäftigung}
        \label{fig:beschaeftigung}
    \end{minipage}
    \hspace{0.1\textwidth}
    \begin{minipage}[t]{0.38\textwidth}
        \centering
        \includegraphics[width=\textwidth]{Arbeit/Bilder/printscreens/zugehoerigkeit.jpeg}
        \caption{Slider-Frage zur sozialen Zugehörigkeit}
        \label{fig:zugehoerigkeit}
    \end{minipage}
\end{figure}

Die Befragungslogik berechnet nach der ersten Teilnahme täglich drei zufällige Befragungszeitpunkte, die innerhalb fester Tagesabschnitte (Morgen, Mittag/Nachmittag, Abend) ausgewählt werden. Diese Zeitpunkte werden lokal auf dem Gerät gespeichert. Zwischen zwei Befragungen wird ein Mindestabstand von zwei Stunden eingehalten, gerechnet zwischen dem Ende des vorigen und dem Beginn des nächsten Befragungsfensters, um zu vermeiden, dass Teilnehmende bei kurzfristiger Nichtverfügbarkeit mehrere Erhebungen unmittelbar hintereinander verpassen. Zum Start eines Zeitfensters wird eine Push-Benachrichtigung versendet; der Fragebogen kann innerhalb einer Stunde beantwortet werden, danach verfällt der Slot.

Die Entscheidung für dieses Zeitplanmodell orientiert sich am Design der \textit{Urban Mind}-App \parencite{bakolisUrbanMindUsing2018}, das sich in der Praxis als gut umsetzbar erwiesen hat. Die Kombination aus zufälliger Platzierung der Startzeiten innerhalb fest definierter Tagesfenster und einer begrenzten Bearbeitungsdauer ermöglicht es, Antworten zu unterschiedlichen Zeitpunkten des Tages zu erfassen und damit Variabilität im Tagesablauf der Teilnehmenden abzubilden. Gleichzeitig wird vermieden, dass Befragungen immer zu denselben Uhrzeiten stattfinden, was potenzielle Antwortmuster verzerren könnte.

Die Eckzeiten der drei Hauptzeitfenster sind als Variablen in der Anwendung hinterlegt und können für andere Studien oder Fragebogendesigns angepasst werden. Auf diese Weise lässt sich der Befragungsrhythmus flexibel anpassen, beispielsweise indem Tagesfenster auf Grundlage individueller Angaben zu Aufsteh- und Schlafenszeiten definiert werden. Eine solche Erweiterung würde auch nicht-normative Tagesrhythmen berücksichtigen und könnte die Erreichbarkeit der Teilnehmenden weiter verbessern.

\begin{figure}[h]
    \centering
    \begin{minipage}[t]{0.38\textwidth}
        \centering
        \includegraphics[width=\textwidth]{Arbeit/Bilder/printscreens/fragen_zu_dir.jpeg}
        \caption{Überleitungsbildschirm zu den einmaligen Fragen}
        \label{fig:ueberleitungsbildschirm}
    \end{minipage}
    \hspace{0.1\textwidth}
    \begin{minipage}[t]{0.38\textwidth}
        \centering
        \includegraphics[width=\textwidth]{Arbeit/Bilder/printscreens/offen_unwohl.jpeg}
        \caption{Offene Textfrage zu weiteren Gründen für Unwohlsein an diesem Ort}
        \label{fig:offene_textfrage}
    \end{minipage}
\end{figure}

Die Benutzeroberfläche ist bewusst reduziert und funktional gestaltet, um eine intuitive Bedienung zu ermöglichen und die Fragen möglichst neutral darzustellen \parencite{rogersInteractionDesignHumancomputer2023}. Die App umfasst drei Hauptbereiche: den Startbildschirm (\cref{fig:startscreen}), der standardmässig den nächstmöglichen Befragungszeitpunkt prominent anzeigt und -- sofern aktuell eine Befragung verfügbar ist -- direkt einen „Umfrage starten“-Button einblendet; den Fragebogenbereich (\cref{fig:welcome,fig:beschaeftigung,fig:zugehoerigkeit,fig:ueberleitungsbildschirm,fig:offene_textfrage}), der sowohl einleitende und überleitende Texte als auch die einzelnen Fragen in einem klar strukturierten Layout präsentiert; sowie einen Informations- und Einstellungsbereich mit Hinweisen zum Datenschutz und zur Studie.

Grafiken werden ausschliesslich auf Einleitungs-, Überleitungs- und Informationsbildschirmen eingesetzt, nicht jedoch während der eigentlichen Befragung. Diese bewusste Trennung soll sicherstellen, dass die Beantwortung der Fragen nicht durch Designelemente beeinflusst wird. Für diese visuellen Elemente kommen ausschliesslich \gls{opensource}-Vektorgrafiken von Katerina Limpitsouni\footnote{\href{https://undraw.co/}{undraw.co/}} zum Einsatz, die thematisch passend, aber stilistisch neutral gehalten sind.

Die Farbpalette ist dezent gewählt, um Barrierefreiheit zu fördern und gute Lesbarkeit unter verschiedenen Lichtbedingungen sicherzustellen. Die Navigation ist linear aufgebaut: Nach Abschluss einer Befragung kehren die Nutzenden automatisch zum Startbildschirm zurück, wodurch der Fokus klar auf den nächsten Befragungszeitpunkt gelenkt wird. Komplexe Menüs oder verschachtelte Navigationsebenen werden vermieden, um die Nutzung auch für Personen mit geringer technischer Erfahrung zu erleichtern.

\section{Von der Simulation zum Alltagstest -- Feldtest und Feinschliff}
\label{sec:app_entwicklung_feldtest}

Zur Überprüfung der technischen Funktionsfähigkeit wurde ein zweistufiges Testverfahren durchgeführt: fortlaufende Tests während der Entwicklung sowie ein abschliessender interner Pretest. Auf automatisierte Tests wurde verzichtet, da deren Relevanz zu Beginn des Projekts unterschätzt und eine nachträgliche Integration als zu aufwändig eingeschätzt wurde. Stattdessen kam ein manueller, iterativer Ansatz zum Einsatz, bei dem die App regelmässig in \glspl{emulator} unterschiedlicher Bildschirmgrössen und auf physischen Geräten geprüft wurde. Die modulare Struktur der Codebasis erleichterte dabei die gezielte Überprüfung einzelner Komponenten. Im Mittelpunkt standen die dynamische Verarbeitung des Fragenkatalogs, die Datenübertragung an das \gls{supabase}-\gls{backend}, das Verhalten bei instabiler Internetverbindung sowie die Funktionsweise der lokalen Push-Benachrichtigungen.

Der anschliessende interne Pretest wurde mit vier Personen durchgeführt, die über die offiziellen Plattformen (\gls{testflight} und \gls{googleplayconsole}) Zugang zur App erhielten und diese über zwei Wochen im Alltag nutzten. Ziel war es, zentrale Funktionen unter realen Bedingungen zu überprüfen, insbesondere das Verhalten beim ersten App-Start, die Stabilität der Datenerfassung und die Darstellung auf unterschiedlichen Geräten. Rückmeldungen zur Bedienbarkeit wurden laufend dokumentiert.

Aus den Testergebnissen ergaben sich mehrere Anpassungen. Die Logik zur Planung der Slots und Benachrichtigungen wurde grundlegend überarbeitet: Anstelle von Hintergrundprozessen werden nun sämtliche Befragungszeitpunkte direkt nach Abschluss der ersten Befragung berechnet und lokal gespeichert, wodurch die Abhängigkeit von Betriebssystemprozessen entfällt. Zudem wurden verschiedene Anpassungen an der Benutzeroberfläche umgesetzt, etwa zur optimierten Darstellung auf kleineren Bildschirmen und zur besseren Lesbarkeit von Slider-Beschriftungen. Diese Änderungen verbesserten die visuelle Konsistenz und Zuverlässigkeit der App auf unterschiedlichen Endgeräten.

\section{App-Veröffentlichung -- Prozesse, Plattformen, Abhängigkeiten}

Um die entwickelte App für die Datenerhebung bereitzustellen, war eine Veröffentlichung über die offiziellen Distributionsplattformen von \gls{apple} (\gls{ios}) und \gls{google} (\gls{android}) vorgesehen. Beide Anbieter stellen unterschiedliche technische, administrative und finanzielle Anforderungen, die den Veröffentlichungsprozess beeinflussten.

Für den \gls{apple} App Store war der Erwerb einer kostenpflichtigen Entwicklerlizenz erforderlich (CHF 100 pro Jahr). Bereits das Testen einer App auf einem physischen \gls{ios}-Gerät setzt ein solches Entwicklerkonto voraus; ohne Lizenz ist die Ausführung nur in einem \gls{emulator} möglich. Nach Einrichtung des Kontos wurde die App über das \gls{apple} Developer Portal eingereicht und durchlief den obligatorischen Prüfprozess. Eine Veröffentlichung im regulären App Store wurde zunächst abgelehnt, mit der Begründung, die App biete zu wenig inhaltlichen Mehrwert. Zum Zeitpunkt des Abschlusses dieser Arbeit war der Fall noch nicht abschliessend geklärt. Parallel konnte die App über die \gls{apple}-Plattform \gls{testflight} für öffentliche Beta-Tests bereitgestellt werden, sodass Teilnehmende über einen Einladungslink Zugriff erhielten.

\gls{google} erhebt für die Veröffentlichung im Play Store keine wiederkehrenden Gebühren, verlangt jedoch vor einer offenen Betaversion einen geschlossenen Test mit mindestens 20 Personen über zwei Wochen. Die Verwaltung erfolgt über die \gls{googleplayconsole}. Da diese Anforderung im Projektzeitrahmen nicht durch eigene Rekrutierung erfüllbar war, wurde ein externer Testdienst beauftragt (Kosten: CHF 30). Nach Abschluss des Tests und der formalen Prüfung wurde die App als offene Beta im Play Store veröffentlicht und war damit öffentlich verfügbar.

Beide Plattformen setzen zudem eine öffentlich zugängliche Datenschutzrichtlinie voraus. Hierfür musste eine eigenständige Projektwebseite\footnote{\href{https://intermind.ch/privacy-policy.html?lang=de}{intermind.ch/privacy-policy}} eingerichtet werden, auf der die vollständige Erklärung abrufbar ist. Obwohl inhaltlich bereits eine Datenschutzdokumentation vorlag, erwies sich die formale Umsetzung als zeitaufwändiger als erwartet: Neben der Erstellung einer mobilfreundlichen HTML-Version mussten die Richtlinien in einer klar strukturierten, rechtlich konsistenten Form bereitgestellt und über eine dauerhaft erreichbare URL zugänglich gemacht werden. Die einmaligen Kosten für die Domainregistrierung beliefen sich auf CHF 10; für das Hosting konnte auf bestehende Infrastruktur zurückgegriffen werden.

\section{Struktur, Qualitätssicherung und Optimierungspotenzial}

Die Entwicklung von \gls{intermind} erfolgte in \gls{typescript} unter Verwendung von \gls{reactnative} und \gls{expo}. Der komponentenbasierte Ansatz in Kombination mit den \gls{solid}-Prinzipien ermöglichte eine nachvollziehbare Strukturierung der Anwendung und erleichterte gezielte Anpassungen im Entwicklungsverlauf.

Rückblickend zeigte sich jedoch, dass eine von Beginn an systematischere Auseinandersetzung mit der Softwarearchitektur von Vorteil gewesen wäre. Zwar wurde eine modulare Struktur umgesetzt, viele Designentscheidungen wurden jedoch situativ getroffen und nicht regelmässig im Sinne eines Gesamtkonzepts überprüft. Ein methodisch enger geführter Architekturprozess hätte hier zu klareren Abhängigkeiten und stabileren Schnittstellen geführt.

Die Anwendung von Methoden wie \emph{Test-Driven Development} hätte diesen Prozess zusätzlich unterstützt, indem Schnittstellen und Verantwortlichkeiten bereits in frühen Entwicklungsphasen festgelegt worden wären. Auch automatisierte Tests und eine kontinuierliche Codeanalyse hätten dazu beigetragen, Fehler frühzeitig zu erkennen und die langfristige Wartbarkeit zu erhöhen. Während viele kleinere Schwächen pragmatisch behoben wurden, hätte ein strukturierteres Qualitätsmanagement den späteren \gls{refactoring}-Aufwand verringert.

Ein weiteres Optimierungspotenzial liegt in der Gestaltung des Interfaces zur Datenbank. Derzeit erfolgt der Datenaustausch überwiegend über verschachtelte \gls{json}-Strings, teils aus pragmatischen Gründen, um serverseitige Verarbeitung zu vermeiden. Eine stärkere Modularisierung und Entkopplung dieser Schnittstelle von der restlichen Anwendungslogik würde die Lesbarkeit verbessern, Fehlerquellen reduzieren und künftige Anpassungen -- etwa bei der Erweiterung des Datenmodells -- erleichtern.

In dieser Hinsicht weist das Projekt Parallelen zu vielen \gls{opensource}-Entwicklungen auf: Es wurde aus einem konkreten Bedarf heraus realisiert, ist funktionsfähig und dokumentiert, jedoch nicht in allen Teilen optimal strukturiert. Die Veröffentlichung des Quellcodes eröffnet allerdings auch die Möglichkeit, dass andere Entwickler\genderstern innen auf der bestehenden Basis aufbauen, Verbesserungsvorschläge einbringen oder Erweiterungen umsetzen können.
