\newacronym{esm}{ESM}{Experience Sampling Method}
\newacronym{ema}{EMA}{Ecological Momentary Assessment}
\newacronym{gema}{GEMA}{Geographically Explicit Ecological Momentary Assessment}
\newacronym{drm}{DRM}{Day Reconstruction Method}
\newacronym{panas}{PANAS}{The Positive \& Negative Affect Schedule}
\newacronym{maihda}{MAIHDA}{Multilevel Analysis of Individual Heterogeneity and Discriminatory Accuracy}
\newacronym{cart}{CART}{Classification and Regression Trees}
\newacronym{json}{JSON}{JavaScript Object Notation}
\newacronym{dsg}{DSG}{Schweizer Datenschutzgesetz}
\newacronym{dsgvo}{DSGVO}{EuropäischeDatenschutz-Grundverordnung}


\newglossaryentry{race}{
    name={\textit{race}},
    description={Eine im englischsprachigen Raum etablierte, gesellschaftlich konstruierte Kategorie, die rassifizierende Zugehörigkeiten beschreibt. In dieser Arbeit kursiv gesetzt, um ihre soziokulturelle Bedeutung zu betonen und sie vom biologistischen Begriff „Rasse“ abzugrenzen.},
    sort=race
}

\newglossaryentry{gender}{
    name={\textit{gender}},
    description={Bezeichnet die soziale Konstruktion von Geschlecht. Der Begriff verweist auf gesellschaftlich geprägte Vorstellungen und Erwartungen von Geschlechtsidentität. In dieser Arbeit kursiv gesetzt.},
    sort=gender
}

\newglossaryentry{schwarz}{
    name={Schwarz},
    description={Politische Selbstbezeichnung von Menschen, die im Kontext rassistischer Machtverhältnisse positioniert werden. Grossgeschrieben zur Abgrenzung von farblichen Zuschreibungen.},
    sort=schwarz
}

\newglossaryentry{intersektionalitaet}{
    name={Intersektionalität},
    description={Analytisches Konzept zur Untersuchung sich überschneidender Machtverhältnisse wie Rassismus, Sexismus, Klassismus etc. Ursprünglich von Kimberlé Crenshaw eingeführt.}
}
\newglossaryentry{class}{
    name={\textit{class}},
    description={Sozial konstruierte Kategorie, die ökonomische und symbolische Ungleichheiten beschreibt. In dieser Arbeit kursiv gesetzt.},
    sort=class
}

\newglossaryentry{reactnative}{
  name={React Native},
  description={Ein Framework zur plattformübergreifenden Entwicklung mobiler Apps. Es erlaubt die Programmierung mit \gls{javascript} oder \gls{typescript}, wobei der Code nativ auf Android- und iOS-Geräten ausgeführt wird.}
}

\newglossaryentry{expo}{
  name={Expo},
  description={Ein Toolchain und Dienst, der die Entwicklung mit \gls{reactnative} vereinfacht. Expo stellt Werkzeuge zum Testen, Debuggen und Veröffentlichen von Apps bereit – ohne dass native Programmierkenntnisse erforderlich sind.}
}

\newglossaryentry{supabase}{
  name={Supabase},
  description={Ein Open-Source-Backend, das als Alternative zu Firebase dient. Es basiert auf einer \gls{datenbank} (PostgreSQL) und bietet Funktionen wie Authentifizierung, Datei-Hosting und \gls{rls}.}
}

\newglossaryentry{javascript}{
  name={JavaScript},
  description={Eine weit verbreitete Programmiersprache für Webentwicklung, die auch in mobilen Frameworks wie \gls{reactnative} verwendet wird. Sie ist dynamisch und flexibel, aber nicht typensicher.}
}

\newglossaryentry{typescript}{
  name={TypeScript},
  description={Eine von Microsoft entwickelte Programmiersprache, die auf \gls{javascript} basiert, aber zusätzliche statische Typisierung bietet. Sie erhöht die Wartbarkeit und Fehlervermeidung in grösseren Softwareprojekten.}
}

\newglossaryentry{python}{
  name={Python},
  description={Eine interpretierte, einfach lesbare Programmiersprache, die häufig in Wissenschaft, Datenanalyse und Automatisierung eingesetzt wird. Sie wurde auch zur Auswertung der App-Daten verwendet.}
}

\newglossaryentry{uuid}{
  name={UUID},
  description={Abkürzung für Universally Unique Identifier. Eine \textit{UUID} ist eine zufällig generierte Zeichenkette, die zur eindeutigen Identifikation eines Geräts oder Datensatzes dient, ohne personenbezogene Daten zu erfassen.},
  first={Universally Unique Identifier (\glsentrytext{uuid})}
}

\newglossaryentry{opensource}{
  name={Open-Source},
  description={Bezeichnet Software, deren Quellcode öffentlich einsehbar, veränderbar und frei verwendbar ist. \gls{supabase} und viele Komponenten von \gls{reactnative} und \gls{expo} sind Open-Source.}
}

\newglossaryentry{datenbank}{
  name={Datenbank},
  description={Ein digitales System zur strukturierten Speicherung, Abfrage und Verwaltung von Daten. In der App kommt eine relationale Datenbank zum Einsatz, die durch \gls{supabase} bereitgestellt wird.}
}

\newglossaryentry{rls}{
  name={RLS},
  description={Ein feingranulares Zugriffsmodell in einer \gls{datenbank}, das sicherstellt, dass Nutzer:innen nur jene Datenzeilen sehen oder ändern können, für die sie berechtigt sind. \gls{supabase} unterstützt RLS standardmässig.},
  first={Row-Level Security (\glsentrytext{rls})}
}

\newglossaryentry{github}{
  name={GitHub},
  description={Eine webbasierte Plattform zur Versionsverwaltung und Zusammenarbeit an Softwareprojekten. Sie basiert auf \gls{git} und wird häufig zur Entwicklung und Veröffentlichung von \gls{opensource}-Software verwendet}
}

\newglossaryentry{git}{
  name={Git},
  description={Ein verteiltes Versionskontrollsystem zur Nachverfolgung von Änderungen im Quellcode. Git ermöglicht es, Entwicklungsschritte lokal oder kollaborativ zu verwalten und ist Grundlage vieler Plattformen wie \gls{github}}
}

\newglossaryentry{frontend}{
  name={Frontend},
  description={Der Teil einer Software, der für Nutzer:innen sichtbar und direkt bedienbar ist – etwa die Benutzeroberfläche einer App. Wird meist in Kombination mit dem \gls{backend} verwendet.}
}

\newglossaryentry{backend}{
  name={Backend},
  description={Der Teil einer Software, der im Hintergrund läuft und Daten verarbeitet, speichert oder bereitstellt. In dieser Arbeit wird dafür \gls{supabase} verwendet.}
}

\newglossaryentry{framework}{
  name={Framework},
  description={Ein vorgefertigtes Gerüst für die Softwareentwicklung, das häufig genutzte Funktionen bereitstellt. \gls{reactnative} ist ein Beispiel für ein solches Framework.},
  plural={Frameworks}
}

\newglossaryentry{pushnotification}{
  name={Push-Benachrichtigung},
  description={Eine Mitteilung, die von einer App aktiv an das Gerät gesendet wird – auch wenn die App im Hintergrund läuft. In dieser Studie werden so die Teilnehmenden zur Beantwortung der Fragen aufgefordert.},
  plural={Push-Benachrichtigungen}
}

\newglossaryentry{ui}{
  name={UI},
  description={Die Benutzeroberfläche einer App, die für Nutzer:innen sichtbar und direkt bedienbar ist. In dieser Arbeit wird die Benutzeroberfläche der App als \gls{ui} bezeichnet.},
  first={User Interface (\glsentrytext{ui})}
}

\newglossaryentry{postgresql}{
  name={PostgreSQL},
  description={Eine relationale \gls{datenbank}, die als \gls{backend} für \gls{supabase} verwendet wird.}
}

\newglossaryentry{urbanmind}{
  name={Urban Mind},
  description={Eine mobile App zur Echtzeit-Erhebung subjektiven Wohlbefindens mittels \gls{gema}. Die App erfasst affektive Zustände mehrfach täglich und verknüpft sie mit räumlichen Kontexten wie Naturerleben. \glslink{intersektionalitaet}{Intersektionale} Perspektiven werden nicht systematisch berücksichtigt. Siehe \href{https://www.urbanmind.info/}{urbanmind.info}},
  first={Urban Mind}
}

\newglossaryentry{reliefmaps}{
  name={Relief Maps+},
  description={Ein webbasiertes Tool zur retrospektiven und \glslink{intersektionalitaet}{intersektionalen} Reflexion subjektiver Raumerfahrungen. Nutzer\genderstern innen verorten emotionale Bewertungen entlang von \glspl{identitaetsachse} auf einer Karte. Siehe \href{https://reliefmaps.upf.edu/}{reliefmaps.upf.edu}},
  first={Relief Maps+}
}

\newglossaryentry{intermind}{
  name={InterMind},
  description={Eine in dieser Arbeit entwickelte \gls{opensource}-App zur \gls{gema}-basierten Erhebung situativer Erfahrungen. Die App ist technisch unabhängig vom Fragebogen, der in dieser Studie eine \glslink{intersektionalitaet}{intersektionale} Analyse ermöglichen soll. Siehe \href{https://intermind.ch/}{intermind.ch}},
  first={InterMind}
}

\newglossaryentry{identitaetsachse}{
  name={Identitätsachse},
  plural={Identitätsachsen},
  description={Begriff aus der \glslink{intersektionalitaet}{intersektionalen} Theorie, der eine einzelne soziale Kategorie wie \gls{gender}, \gls{race}, \gls{class}, sexuelle Orientierung, (Dis-)Ability oder Alter bezeichnet. Solche Achsen strukturieren gesellschaftliche Positionierungen und prägen Erfahrungen von Privilegierung oder Diskriminierung.}
}

\newglossaryentry{ios}{
  name={iOS},
  description={Eine mobile Betriebssystem-Plattform, die von Apple entwickelt wird. Sie wird hauptsächlich auf Geräten des iPhone- und iPad-Produktlinien verwendet.}
}

\newglossaryentry{android}{
  name={Android},
  description={Eine mobile Betriebssystem-Plattform, die von Google entwickelt wird. Sie wird hauptsächlich auf Geräten des Android-Produktlinien verwendet.}
}

\newglossaryentry{authentifizierung}{
  name={Authentifizierung},
  description={Vorgang zur Überprüfung der Identität eines Nutzers oder Geräts, typischerweise durch die Eingabe eines Passworts, Tokens oder durch kryptografische Verfahren. In der App erfolgt die Authentifizierung gerätebasiert mittels UUID, ohne Eingabe persönlicher Informationen}
}

\newglossaryentry{autorisierung}{
  name={Autorisierung},
  description={Festlegung von Zugriffsrechten auf bestimmte Ressourcen oder Daten nach erfolgreicher Authentifizierung. In diesem Projekt bedeutet dies, dass nur das eigene Gerät auf die jeweils verknüpften Datensätze zugreifen kann (Row-Level Security)}
}

\newglossaryentry{solid}{
  name={SOLID},
  description={Akronym für fünf grundlegende Prinzipien guter objektorientierter Softwarearchitektur: \textit{Single Responsibility, Open/Closed, Liskov Substitution, Interface Segregation, Dependency Inversion}. Die Prinzipien sollen verständliche, wartbare und erweiterbare Softwaresysteme ermöglichen \parencite{martinCleanArchitecture2018}}
}
