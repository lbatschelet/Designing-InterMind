\newacronym{esm}{ESM}{Experience Sampling Method}
\newacronym{ema}{EMA}{Ecological Momentary Assessment}
\newacronym{gema}{GEMA}{Geographically Explicit Ecological Momentary Assessment}
\newacronym{drm}{DRM}{Day Reconstruction Method}
\newacronym{panas}{PANAS}{The Positive \& Negative Affect Schedule}
\newacronym{maihda}{MAIHDA}{Multilevel Analysis of Individual Heterogeneity and Discriminatory Accuracy}
\newacronym{cart}{CART}{Classification and Regression Trees}
\newacronym{json}{JSON}{JavaScript Object Notation}
\newacronym{dsg}{DSG}{Schweizer Datenschutzgesetz}
\newacronym{dsgvo}{DSGVO}{EuropäischeDatenschutz-Grundverordnung}


\newglossaryentry{race}{
    name={\textit{race}},
    description={Eine im englischsprachigen Raum etablierte, gesellschaftlich konstruierte Kategorie, die rassifizierende Zugehörigkeiten beschreibt. In dieser Arbeit kursiv gesetzt, um ihre soziokulturelle Bedeutung zu betonen und sie vom biologistischen Begriff „Rasse“ abzugrenzen.}
}

\newglossaryentry{gender}{
    name={\textit{gender}},
    description={Bezeichnet die soziale Konstruktion von Geschlecht. Der Begriff verweist auf gesellschaftlich geprägte Vorstellungen und Erwartungen von Geschlechtsidentität. In dieser Arbeit kursiv gesetzt.}
}

\newglossaryentry{schwarz}{
    name={Schwarz},
    description={Politische Selbstbezeichnung von Menschen, die im Kontext rassistischer Machtverhältnisse positioniert werden. Grossgeschrieben zur Abgrenzung von farblichen Zuschreibungen.}
}

\newglossaryentry{intersektionalitaet}{
    name={Intersektionalität},
    description={Analytisches Konzept zur Untersuchung sich überschneidender Machtverhältnisse wie Rassismus, Sexismus, Klassismus etc. Ursprünglich von Kimberlé Crenshaw eingeführt.}
}
\newglossaryentry{class}{
    name={\textit{class}},
    description={Sozial konstruierte Kategorie, die ökonomische und symbolische Ungleichheiten beschreibt. In dieser Arbeit kursiv gesetzt.}
}

\newglossaryentry{reactnative}{
  name={React Native},
  description={Ein Framework zur plattformübergreifenden Entwicklung mobiler Apps. Es erlaubt die Programmierung mit \gls{javascript} oder \gls{typescript}, wobei der Code nativ auf Android- und iOS-Geräten ausgeführt wird.}
}

\newglossaryentry{expo}{
  name={Expo},
  description={Ein Toolchain und Dienst, der die Entwicklung mit \gls{reactnative} vereinfacht. Expo stellt Werkzeuge zum Testen, Debuggen und Veröffentlichen von Apps bereit – ohne dass native Programmierkenntnisse erforderlich sind.}
}

\newglossaryentry{supabase}{
  name={Supabase},
  description={Ein Open-Source-Backend, das als Alternative zu Firebase dient. Es basiert auf einer \gls{datenbank} (PostgreSQL) und bietet Funktionen wie Authentifizierung, Datei-Hosting und \gls{rls}.}
}

\newglossaryentry{javascript}{
  name={JavaScript},
  description={Eine weit verbreitete Programmiersprache für Webentwicklung, die auch in mobilen Frameworks wie \gls{reactnative} verwendet wird. Sie ist dynamisch und flexibel, aber nicht typensicher.}
}

\newglossaryentry{typescript}{
  name={TypeScript},
  description={Eine von Microsoft entwickelte Programmiersprache, die auf \gls{javascript} basiert, aber zusätzliche statische Typisierung bietet. Sie erhöht die Wartbarkeit und Fehlervermeidung in grösseren Softwareprojekten.}
}

\newglossaryentry{python}{
  name={Python},
  description={Eine interpretierte, einfach lesbare Programmiersprache, die häufig in Wissenschaft, Datenanalyse und Automatisierung eingesetzt wird. Sie wurde auch zur Auswertung der App-Daten verwendet.}
}

\newglossaryentry{uuid}{
  name={UUID},
  description={Abkürzung für Universally Unique Identifier. Eine \textit{UUID} ist eine zufällig generierte Zeichenkette, die zur eindeutigen Identifikation eines Geräts oder Datensatzes dient, ohne personenbezogene Daten zu erfassen.},
  first={Universally Unique Identifier (\glsentrytext{uuid})}
}

\newglossaryentry{opensource}{
  name={Open-Source},
  description={Bezeichnet Software, deren Quellcode öffentlich einsehbar, veränderbar und frei verwendbar ist. \gls{supabase} und viele Komponenten von \gls{reactnative} und \gls{expo} sind Open-Source.}
}

\newglossaryentry{datenbank}{
  name={Datenbank},
  description={Ein digitales System zur strukturierten Speicherung, Abfrage und Verwaltung von Daten. In der App kommt eine relationale Datenbank zum Einsatz, die durch \gls{supabase} bereitgestellt wird.}
}

\newglossaryentry{rls}{
  name={RLS},
  description={Ein feingranulares Zugriffsmodell in einer \gls{datenbank}, das sicherstellt, dass Nutzer:innen nur jene Datenzeilen sehen oder ändern können, für die sie berechtigt sind. \gls{supabase} unterstützt RLS standardmässig.},
  first={Row-Level Security (\glsentrytext{rls})}
}

\newglossaryentry{github}{
  name={GitHub},
  description={Eine webbasierte Plattform zur Versionsverwaltung und Zusammenarbeit an Softwareprojekten. Sie basiert auf \gls{git} und wird häufig zur Entwicklung und Veröffentlichung von \gls{opensource}-Software verwendet}
}

\newglossaryentry{git}{
  name={Git},
  description={Ein verteiltes Versionskontrollsystem zur Nachverfolgung von Änderungen im Quellcode. Git ermöglicht es, Entwicklungsschritte lokal oder kollaborativ zu verwalten und ist Grundlage vieler Plattformen wie \gls{github}}
}

\newglossaryentry{frontend}{
  name={Frontend},
  description={Der Teil einer Software, der für Nutzer:innen sichtbar und direkt bedienbar ist – etwa die Benutzeroberfläche einer App. Wird meist in Kombination mit dem \gls{backend} verwendet.}
}

\newglossaryentry{backend}{
  name={Backend},
  description={Der Teil einer Software, der im Hintergrund läuft und Daten verarbeitet, speichert oder bereitstellt. In dieser Arbeit wird dafür \gls{supabase} verwendet.}
}

\newglossaryentry{framework}{
  name={Framework},
  description={Ein vorgefertigtes Gerüst für die Softwareentwicklung, das häufig genutzte Funktionen bereitstellt. \gls{reactnative} ist ein Beispiel für ein solches Framework.},
  plural={Frameworks}
}

\newglossaryentry{pushnotification}{
  name={Push-Benachrichtigung},
  description={Eine Mitteilung, die von einer App aktiv an das Gerät gesendet wird – auch wenn die App im Hintergrund läuft. In dieser Studie werden so die Teilnehmenden zur Beantwortung der Fragen aufgefordert.},
  plural={Push-Benachrichtigungen}
}

\newglossaryentry{ui}{
  name={UI},
  description={Die Benutzeroberfläche einer App, die für Nutzer:innen sichtbar und direkt bedienbar ist. In dieser Arbeit wird die Benutzeroberfläche der App als \gls{ui} bezeichnet.},
  first={User Interface (\glsentrytext{ui})}
}

\newglossaryentry{postgresql}{
  name={PostgreSQL},
  description={Eine relationale \gls{datenbank}, die als \gls{backend} für \gls{supabase} verwendet wird.}
}

