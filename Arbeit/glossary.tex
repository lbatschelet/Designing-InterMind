\newacronym{esm}{ESM}{Experience Smpling Method}
\newacronym{ema}{EMA}{Ecological Momentary Assessment}
\newacronym{gema}{GEMA}{Geographically Explicit Ecological Momentary Assessment}
\newacronym{drm}{DRM}{Day Reconstruction Method}
\newacronym{panas}{PANAS}{The Positive \& Negative Affect Schedule}
\newacronym{maihda}{MAIHDA}{Multilevel Analysis of Individual Heterogeneity and Discriminatory Accuracy}
\newacronym{cart}{CART}{Classification and Regression Trees}


\newglossaryentry{race}{
    name={\textit{race}},
    description={Eine im englischsprachigen Raum etablierte, gesellschaftlich konstruierte Kategorie, die rassifizierende Zugehörigkeiten beschreibt. In dieser Arbeit kursiv gesetzt, um ihre soziokulturelle Bedeutung zu betonen und sie vom biologistischen Begriff „Rasse“ abzugrenzen.}
}

\newglossaryentry{gender}{
    name={\textit{gender}},
    description={Bezeichnet die soziale Konstruktion von Geschlecht. Der Begriff verweist auf gesellschaftlich geprägte Vorstellungen und Erwartungen von Geschlechtsidentität. In dieser Arbeit kursiv gesetzt.}
}

\newglossaryentry{schwarz}{
    name={Schwarz},
    description={Politische Selbstbezeichnung von Menschen, die im Kontext rassistischer Machtverhältnisse positioniert werden. Grossgeschrieben zur Abgrenzung von farblichen Zuschreibungen.}
}

\newglossaryentry{intersektionalitaet}{
    name={Intersektionalität},
    description={Analytisches Konzept zur Untersuchung sich überschneidender Machtverhältnisse wie Rassismus, Sexismus, Klassismus etc. Ursprünglich von Kimberlé Crenshaw eingeführt.}
}
\newglossaryentry{class}{
    name={\textit{class}},
    description={Sozial konstruierte Kategorie, die ökonomische und symbolische Ungleichheiten beschreibt. In dieser Arbeit kursiv gesetzt.}
}

