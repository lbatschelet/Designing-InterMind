
\newacronym{esm}{ESM}{Experience Sampling Method}
\newacronym{ema}{EMA}{Ecological Momentary Assessment}
\newacronym{gema}{GEMA}{Geographically Explicit Ecological Momentary Assessment}
\newacronym{drm}{DRM}{Day Reconstruction Method}
\newacronym{panas}{PANAS}{The Positive \& Negative Affect Schedule}
\newacronym{maihda}{MAIHDA}{Multilevel Analysis of Individual Heterogeneity and Discriminatory Accuracy}
\newacronym[description={Intersectional MAIHDA -- siehe \gls[noindex]{maihda}}]{i-maihda}{I-MAIHDA}{Intersectional \gls[noindex]{maihda}}
\newacronym{cart}{CART}{Classification and Regression Trees}
\newacronym{json}{JSON}{JavaScript Object Notation}
\newacronym{dsg}{DSG}{Schweizer Datenschutzgesetz}
\newacronym{dsgvo}{DSGVO}{EuropäischeDatenschutz-Grundverordnung}
\newacronym{peqi}{PEQI}{Perceived Environmental Quality Indices}
\newacronym{news}{NEWS}{Neighborhood Environment Walkability Scale}
\newacronym{health}{HEALTH}{The Healthy Environments and Active Living for Translational Health Platform}
\newacronym{mvp}{MVP}{Minimum Viable Product}
\newacronym{cicd}{CI/CD}{Continuous Integration/Continuous Delivery}
\newacronym{esec}{ESec}{European Socio-economic Classification}
\newacronym{egp}{EGP}{Erikson–Goldthorpe–Portocarero-Klassenschema}
\newacronym{wemwbs}{WEMWBS}{Warwick-Edinburgh Mental Wellbeing Scale}
\newacronym{icc}{ICC}{Intra-Class Correlation}
\newacronym{pev}{PEV}{Proportional Explained Variance}





\newacronym{vgl}{vgl.}{vergleiche}
\newacronym{bspw}{bspw.}{beispielsweise}
\newacronym{gps}{GPS}{Global Positioning System}
\newacronym{zb}{z.\,B.}{zum Beispiel}
\newacronym{ua}{u.\,a.}{unter anderem}
\newacronym{etc}{etc.}{et cetera}
\newacronym{eng}{engl.}{englisch}
\newacronym{bzw}{bzw.}{beziehungsweise}


\newglossaryentry{race}{
    name={\textit{race}},
    description={Eine im englischsprachigen Raum etablierte, gesellschaftlich konstruierte Kategorie, die rassifizierende Zugehörigkeiten beschreibt. In dieser Arbeit kursiv gesetzt, um ihre soziokulturelle Bedeutung zu betonen und sie deutlich vom biologistischen Begriff \enquote{Rasse} abzugrenzen. Der Begriff verweist auf machtvolle Prozesse sozialer Differenzierung, Zugehörigkeit und Ausschluss, die historisch gewachsen sind und bis heute wirksam bleiben. Die Übertragung des Konzepts in europäische Kontexte ist umstritten: Aus Angst vor biologisierenden Implikationen und im Schatten nationaler Gewaltgeschichte (\gls{zb} Nationalsozialismus) wird \textit{race} oft durch vage Begriffe wie \enquote{Ethnizität} ersetzt oder gänzlich vermieden, was zur epistemischen Unsichtbarmachung rassifizierter Erfahrungen führen kann \parencite{bartelsPostcolonialFeminismIntersectionality2019}.},
    sort=race
}


\newglossaryentry{gender}{
    name={\textit{gender}},
    description={Eine gesellschaftlich konstruierte Kategorie, die auf soziale, kulturelle und politische Vorstellungen von Geschlecht verweist. \textit{gender} beschreibt normative Erwartungen an Geschlechtsidentität und Geschlechtsausdruck, die in spezifischen historischen und kulturellen Kontexten entstehen und sich verändern können. In dieser Arbeit kursiv gesetzt, um den Unterschied zum biologisierenden Begriff \enquote{sex} (\gls{eng}) \gls{bzw} \enquote{biologisches Geschlecht} hervorzuheben. Die Verwendung im Deutschen ist uneinheitlich: Während \textit{gender} in wissenschaftlichen Diskursen etabliert ist, wird es im Alltagsgebrauch häufig unscharf verwendet oder mit \enquote{Geschlecht} gleichgesetzt, was die soziale Konstruiertheit verwischt \parencite{butlerGenderTroubleFeminism1990}.},
    sort=gender
}

\newglossaryentry{schwarz}{
    name={Schwarz},
    description={Politische Selbstbezeichnung von Menschen, die im Kontext rassistischer Machtverhältnisse positioniert werden. Grossgeschrieben zur Abgrenzung von farblichen Zuschreibungen und um die Konstruktion von \enquote{Schwarzsein} als soziale Position zu betonen, nicht als biologische Eigenschaft. Die Schreibweise folgt postkolonialen und kritischen rassismustheoretischen Ansätzen, die die Erfahrung von Diskriminierung, Widerstand und Zugehörigkeit in den Vordergrund stellen \parencite{oguntoyeFarbeBekennenAfrodeutsche1986}.},
    sort=schwarz
}

\newglossaryentry{intersektionalitaet}{
    name={Intersektionalität},
    description={Analytisches Konzept zur Untersuchung sich überschneidender und wechselseitig verstärkender Machtverhältnisse wie Rassismus, Sexismus oder Klassismus. Entwickelt von Kimberlé Crenshaw zur Beschreibung der spezifischen Diskriminierungserfahrungen Schwarzer Frauen, die weder durch feministische noch durch antirassistische Theorien allein erfasst wurden. Der Ansatz betont, dass soziale Kategorien nicht additiv wirken, sondern in ihren Verflechtungen eigenständige Formen von Ungleichheit erzeugen. Heute wird er in einer Vielzahl disziplinärer und methodischer Kontexte angewandt, ist jedoch in seiner theoretischen Radikalität nicht unumstritten \parencite{crenshawMappingMarginsIntersectionality1991, collinsBlackFeministThought2002}.},
    sort=intersektionalitaet
}

\newglossaryentry{class}{
    name={\textit{class}},
    description={Eine gesellschaftlich konstruierte Kategorie, die ökonomische, kulturelle und symbolische Ungleichheiten beschreibt. \textit{class} verweist auf soziale Hierarchien, die sich aus Eigentum, Einkommen, Bildung, Beruf und anderen Ressourcen ergeben und historisch durch kapitalistische Strukturen geformt wurden. In dieser Arbeit kursiv gesetzt, um die soziale Konstruiertheit und die Differenz zum alltagssprachlichen Begriff \enquote{soziale Schicht} zu betonen. Die Übersetzung ins Deutsche ist umstritten, da Begriffe wie \enquote{Klasse} oder \enquote{Schicht} jeweils eigene theoretische Traditionen und politische Konnotationen mitbringen.},
    sort=class
}


\newglossaryentry{reactnative}{
  name={React Native},
  description={Ein Framework zur plattformübergreifenden Entwicklung mobiler Apps. Es erlaubt die Programmierung mit \gls[noindex]{javascript} oder \gls[noindex]{typescript}, wobei der Code nativ auf Android- und iOS-Geräten ausgeführt wird. \href{https://reactnative.dev/}{reactnative.dev}}
}

\newglossaryentry{expo}{
  name={Expo},
  description={Ein Toolchain und Dienst, der die Entwicklung mit \gls[noindex]{reactnative} vereinfacht. Expo stellt Werkzeuge zum Testen, Debuggen und Veröffentlichen von Apps bereit -- ohne dass native Programmierkenntnisse erforderlich sind. \href{https://expo.dev/}{expo.dev}}
}

\newglossaryentry{supabase}{
  name={Supabase},
  description={Ein Open-Source-Backend, das als Alternative zu Firebase dient. Es basiert auf einer \gls[noindex]{datenbank} (PostgreSQL) und bietet Funktionen wie Authentifizierung, Datei-Hosting und \gls[noindex]{rls}. \href{https://supabase.com/}{supabase.com}}
}

\newglossaryentry{javascript}{
  name={JavaScript},
  description={Eine weit verbreitete Programmiersprache für Webentwicklung, die auch in mobilen Frameworks wie \gls[noindex]{reactnative} verwendet wird. Sie ist dynamisch und flexibel, aber nicht typensicher.}
}

\newglossaryentry{typescript}{
  name={TypeScript},
  description={Eine von Microsoft entwickelte Programmiersprache, die auf \gls[noindex]{javascript} basiert, aber zusätzliche statische Typisierung bietet. Sie erhöht die Wartbarkeit und Fehlervermeidung in grösseren Softwareprojekten.}
}

\newglossaryentry{python}{
  name={Python},
  description={Eine interpretierte, einfach lesbare Programmiersprache, die häufig in Wissenschaft, Datenanalyse und Automatisierung eingesetzt wird. Sie wurde auch zur Auswertung der App-Daten verwendet.}
}

\newglossaryentry{uuid}{
  name={UUID},
  description={Abkürzung für Universally Unique Identifier. Eine \textit{UUID} ist eine zufällig generierte Zeichenkette, die zur eindeutigen Identifikation eines Geräts oder Datensatzes dient, ohne personenbezogene Daten zu erfassen.},
  first={Universally Unique Identifier (\glsentrytext{uuid})}
}

\newglossaryentry{opensource}{
  name={Open-Source},
  description={Bezeichnet Software, deren Quellcode öffentlich einsehbar, veränderbar und frei verwendbar ist. \gls[noindex]{supabase} und viele Komponenten von \gls[noindex]{reactnative} und \gls[noindex]{expo} sind Open-Source.}
}

\newglossaryentry{datenbank}{
  name={Datenbank},
  description={Ein digitales System zur strukturierten Speicherung, Abfrage und Verwaltung von Daten. In der App kommt eine relationale Datenbank zum Einsatz, die durch \gls[noindex]{supabase} bereitgestellt wird.}
}

\newglossaryentry{rls}{
  name={RLS},
  description={Ein feingranulares Zugriffsmodell in einer \gls[noindex]{datenbank}, das sicherstellt, dass Nutzer:innen nur jene Datenzeilen sehen oder ändern können, für die sie berechtigt sind. \gls[noindex]{supabase} unterstützt RLS standardmässig.},
  first={Row-Level Security (\glsentrytext{rls})}
}

\newglossaryentry{github}{
  name={GitHub},
  description={Eine webbasierte Plattform zur Versionsverwaltung und Zusammenarbeit an Softwareprojekten. Sie basiert auf \gls[noindex]{git} und wird häufig zur Entwicklung und Veröffentlichung von \gls[noindex]{opensource}-Software verwendet}
}

\newglossaryentry{git}{
  name={Git},
  description={Ein verteiltes Versionskontrollsystem zur Nachverfolgung von Änderungen im Quellcode. Git ermöglicht es, Entwicklungsschritte lokal oder kollaborativ zu verwalten und ist Grundlage vieler Plattformen wie \gls[noindex]{github}}
}

\newglossaryentry{frontend}{
  name={Frontend},
  description={Der Teil einer Software, der für Nutzer:innen sichtbar und direkt bedienbar ist -- etwa die Benutzeroberfläche einer App. Wird meist in Kombination mit dem \gls[noindex]{backend} verwendet.}
}

\newglossaryentry{backend}{
  name={Backend},
  description={Der Teil einer Software, der im Hintergrund läuft und Daten verarbeitet, speichert oder bereitstellt. In dieser Arbeit wird dafür \gls[noindex]{supabase} verwendet.}
}

\newglossaryentry{framework}{
  name={Framework},
  description={Ein vorgefertigtes Gerüst für die Softwareentwicklung, das häufig genutzte Funktionen bereitstellt. \gls[noindex]{reactnative} ist ein Beispiel für ein solches Framework.},
  plural={Frameworks}
}

\newglossaryentry{pushnotification}{
  name={Push-Benachrichtigung},
  description={Eine Mitteilung, die von einer App aktiv an das Gerät gesendet wird -- auch wenn die App im Hintergrund läuft. In dieser Studie werden so die Teilnehmenden zur Beantwortung der Fragen aufgefordert.},
  plural={Push-Benachrichtigungen}
}

\newglossaryentry{ui}{
  name={UI},
  description={Die Benutzeroberfläche einer App, die für Nutzer:innen sichtbar und direkt bedienbar ist. In dieser Arbeit wird die Benutzeroberfläche der App als \gls[noindex]{ui} bezeichnet.},
  first={User Interface (\glsentrytext{ui})}
}

\newglossaryentry{postgresql}{
  name={PostgreSQL},
  description={Eine relationale \gls[noindex]{datenbank}, die als \gls[noindex]{backend} für \gls[noindex]{supabase} verwendet wird.}
}

\newglossaryentry{urbanmind}{
  name={\textit{Urban Mind}},
  description={Eine mobile App zur Echtzeit-Erhebung subjektiven Wohlbefindens mittels \gls[noindex]{gema}. Die App erfasst affektive Zustände mehrfach täglich und verknüpft sie mit räumlichen Kontexten wie Naturerleben. \glslink[noindex]{intersektionalitaet}{Intersektionale} Perspektiven werden nicht systematisch berücksichtigt. Siehe \href{https://www.urbanmind.info/}{urbanmind.info}},
  first={Urban Mind},
  sort=urbanmind
}

\newglossaryentry{reliefmaps}{
  name={\textit{Relief Maps+}},
  description={Ein webbasiertes Tool zur retrospektiven und \glslink[noindex]{intersektionalitaet}{intersektionalen} Reflexion subjektiver Raumerfahrungen. Nutzer\genderstern innen verorten emotionale Bewertungen entlang von \glspl{identitaetsachse} auf einer Karte. Siehe \href{https://reliefmaps.upf.edu/}{reliefmaps.upf.edu}},
  first={Relief Maps+},
  sort=reliefmaps
}

\newglossaryentry{intermind}{
  name={\textit{InterMind}},
  description={Eine in dieser Arbeit entwickelte \gls[noindex]{opensource}-App zur \gls[noindex]{gema}-basierten Erhebung situativer Erfahrungen. Die Entwicklung wird in \cref{sec:entwicklung_app} beschrieben. Siehe \href{https://intermind.ch/}{intermind.ch}},
  sort=intermind
}

\newglossaryentry{identitaetsachse}{
  name={Identitätsachse},
  plural={Identitätsachsen},
  description={Begriff aus der \glslink[noindex]{intersektionalitaet}{intersektionalen} Theorie, der eine einzelne soziale Kategorie wie \gls[noindex]{gender}, \gls[noindex]{race}, \gls[noindex]{class}, sexuelle Orientierung, (Dis-)Ability oder Alter bezeichnet. Solche Achsen strukturieren gesellschaftliche Positionierungen und prägen Erfahrungen von Privilegierung oder Diskriminierung.}
}

\newglossaryentry{ios}{
  name={iOS},
  description={Eine mobile Betriebssystem-Plattform, die von Apple entwickelt wird. Sie wird hauptsächlich auf Geräten des iPhone- und iPad-Produktlinien verwendet.}
}

\newglossaryentry{android}{
  name={Android},
  description={Eine mobile Betriebssystem-Plattform, die von Google entwickelt wird. Sie wird hauptsächlich auf Geräten des Android-Produktlinien verwendet.}
}

\newglossaryentry{authentifizierung}{
  name={Authentifizierung},
  description={Vorgang zur Überprüfung der Identität eines Nutzers oder Geräts, typischerweise durch die Eingabe eines Passworts, Tokens oder durch kryptografische Verfahren. In der App erfolgt die Authentifizierung gerätebasiert mittels UUID, ohne Eingabe persönlicher Informationen}
}

\newglossaryentry{autorisierung}{
  name={Autorisierung},
  description={Festlegung von Zugriffsrechten auf bestimmte Ressourcen oder Daten nach erfolgreicher Authentifizierung. In diesem Projekt bedeutet dies, dass nur das eigene Gerät auf die jeweils verknüpften Datensätze zugreifen kann (Row-Level Security)}
}

\newglossaryentry{solid}{
  name={SOLID},
  description={Akronym für fünf grundlegende Prinzipien guter objektorientierter Softwarearchitektur: \textit{Single Responsibility, Open/Closed, Liskov Substitution, Interface Segregation, Dependency Inversion}. Die Prinzipien sollen verständliche, wartbare und erweiterbare Softwaresysteme ermöglichen \parencite{martinCleanArchitectureCraftsmans2018}}
}

\newglossaryentry{emulator}{
  name={Emulator},
  plural={Emulatoren},
  description={Softwareumgebung, die ein bestimmtes Betriebssystem oder Gerät auf einem anderen System simuliert, um Programme wie auf einem echten Gerät auszuführen. In der App-Entwicklung dienen Emulatoren insbesondere dem Testen von Anwendungen auf unterschiedlichen Bildschirmgrössen, Betriebssystemversionen und Gerätearchitekturen, ohne dass reale Geräte erforderlich sind}
}

\newglossaryentry{googleplayconsole}{
  name={Google Play Console},
  description={Plattform von \gls[noindex]{google} zur Verwaltung und Verteilung von \gls[noindex]{android}-Anwendungen. Sie ermöglicht die Veröffentlichung, das Testing und das Monitoring von Apps auf Geräten mit dem Betriebssystem \gls[noindex]{android}}
}

\newglossaryentry{testflight}{
  name={TestFlight},
  description={Offizielle Plattform von Apple zur Bereitstellung von \gls[noindex]{ios}-Apps für Betatests. Entwickler\genderstern innen können damit Vorabversionen ihrer Anwendungen an registrierte Testpersonen verteilen}
}

\newglossaryentry{java}{
  name={Java},
  description={Eine streng \glslink[noindex]{objektorientierung}{objektorientierte} Programmiersprache, die in vielen Bereichen der Softwareentwicklung eingesetzt wird.}
}

\newglossaryentry{objektorientierung}{
  name={Objektorientierung},
  description={Ein Programmierparadigma, das die Entwicklung von Software durch die Modellierung von Objekten und deren Interaktionen vereinfacht. In dieser Arbeit wird die \gls[noindex]{objektorientierung} verwendet, um die App strukturiert und wartbar zu halten.}
}

\newglossaryentry{githubissue}{
    name={GitHub-Issue},
    description={Ein integriertes Werkzeug zur Aufgaben- und Projektverwaltung auf der Plattform \gls[noindex]{github}. 
    GitHub-Issues dienen der strukturierten Erfassung, Diskussion und Nachverfolgung von Aufgaben, 
    Fehlern, neuen Funktionen oder allgemeinen Projektthemen. Sie können mit Labels, Meilensteinen 
    und Verantwortlichkeiten versehen werden, um Entwicklungsprozesse transparent und nachvollziehbar zu gestalten.},
    plural={GitHub-Issues}
}

\newglossaryentry{devops}{
  name={DevOps},
  description={Ein Konzept, das die Integration von Entwicklung und Operations zusammenführt, um schnellere und stabilere Softwareentwicklung zu ermöglichen. DevOps umfasst Tools und Prozesse, die die Automatisierung von Build, Test und Bereitstellung von Software unterstützen.}
}

\newglossaryentry{meta}{
    name={Meta},
    description={US-Technologiekonzern, \gls[noindex]{ua} Entwickler von Facebook, Instagram, WhatsApp und \gls[noindex]{reactnative}.}
}

\newglossaryentry{google}{
    name={Google},
    description={US-Technologiekonzern, \gls[noindex]{ua} Betreiber von Suchmaschine, Google Play Store, YouTube und Entwickler von \gls[noindex]{firebase}.}
}

\newglossaryentry{apple}{
    name={Apple},
    description={US-Technologiekonzern, Hersteller von iPhone und Betreiber des App Store.}
}

\newglossaryentry{firebase}{
    name={Firebase},
    description={Ein \gls[noindex]{backend}-as-a-Service von \gls[noindex]{google}, das Authentifizierung, Datenspeicherung und Schnittstellenbereitstellung integriert bereitstellt.}
}

\newglossaryentry{refactoring}{
  name={Refactoring},
  description={Der Prozess der Verbesserung der Struktur und Lesbarkeit von Code, ohne dass sich die Funktionalität ändert. Refactoring ist ein wichtiger Bestandteil der Softwareentwicklung, um Code-Qualität zu erhöhen und Wartbarkeit zu verbessern.}
}

\newglossaryentry{stratum}{
    name={Stratum},
    plural={Strata},
    description={Bezeichnung für eine Teilmenge einer Grundgesamtheit in der Statistik, gebildet nach gemeinsamen Merkmalen der darin enthaltenen Beobachtungen.}
}



