\section*{Arbeitsdefinitionen}

\subsection*{Well-Being}

In dieser Arbeit wird unter Well-Being das subjektive psychische Erleben von Personen verstanden, welches kognitive und affektive Komponenten umfasst und durch soziale sowie räumliche Kontexte beeinflusst werden kann. Die Definition orientiert sich an einem subjektiven, multidimensionalen Verständnis von Wohlbefinden, das in der Literatur auch unter dem Begriff "subjective well-being" (SWB) diskutiert wird \parencite{diener2009, kahnemanKrueger2006}. Dabei wird zwischen langfristiger Lebenszufriedenheit (life satisfaction) und kurzfristigeren affektiven Zuständen wie Zufriedenheit, Stress oder Sicherheit unterschieden \parencite{schwanenWellBeingContextEveryday2014}.

Well-Being wird hier nicht als stabile Persönlichkeitseigenschaft aufgefasst, sondern als dynamisches, kontextabhängiges Erleben, das mit alltäglichen Aktivitäten, sozialen Kontakten und räumlichen Situationen in Verbindung steht. Die Analyse stützt sich dabei nicht auf ecological momentary assessment (EMA) oder realzeitliche digitale Erhebungstechniken, sondern auf zeitpunktbezogene Selbstauskünfte, die auf spezifische Situationen Bezug nehmen. Damit wird ein Ansatz verfolgt, der zwar kontextsensitive Daten erlaubt, aber methodisch niedrigschwellig bleibt und bewusst auf invasive oder ortsabhängige Datenerhebung verzichtet.

Die Gestaltung der Skalen und Frageformulierungen orientiert sich an etablierten Operationalisierungen von subjektivem Wohlbefinden, ohne eine standardisierte Skala vollständig zu übernehmen. Der verwendete Ansatz umfasst sowohl affektive Zustände (z. B. "angstfrei", "inkludiert", "körperlich wohl") als auch soziale und physische Kontextwahrnehmungen, die in der Forschung als Prädiktoren für situatives Wohlbefinden identifiziert wurden \parencite{bautistaWhatWellbeingScoping2023, schwanenWellBeingContextEveryday2014}.

\subsection*{Intersektionalität}

Intersektionalität bezeichnet in dieser Arbeit die gleichzeitige Wirksamkeit und gegenseitige Durchdringung mehrerer sozialer Positionierungen – etwa Geschlecht, ethnische Zugehörigkeit, Behinderung, sexuelle Orientierung, soziale Klasse – in der Erfahrung von Individuen. Der Begriff geht zurück auf die Arbeiten von Crenshaw und wurde im Kontext der Black Feminist Theory entwickelt, um strukturelle Ungleichheiten sichtbar zu machen, die aus dem Zusammenspiel mehrerer Diskriminierungsachsen entstehen \parencite{crenshawMappingMarginsIntersectionality1991, bauerIntersectionalityQuantitativeResearch2021}.

In quantitativen Designs bedeutet ein intersektionaler Zugang, dass soziale Positionen nicht als additive Einzelkategorien betrachtet werden dürfen (z. B. "Frau" plus "migrantisch" plus "behindert"), sondern dass ihre Kombination eine eigene soziale Position mit spezifischen Effekten bildet. Diese Positionen können nicht durch die bloße Addition einzelner Effekte erfasst werden, sondern müssen als ko-konstituierte Konfigurationen begriffen werden \parencite{hancock2007, bauerIntersectionalityQuantitativeResearch2021}.

Das hier entwickelte Fragebogendesign folgt diesem Verständnis, indem es zwar einzelne Dimensionen sozialer Positionierung erfasst (z. B. Geschlecht, Behinderung, Ethnizität), deren Wirkungen jedoch nicht isoliert, sondern in Kombination analysiert werden. Es wird bewusst darauf verzichtet, nach individuellen Einschätzungen des Einflusses einzelner Merkmale auf das Wohlbefinden zu fragen. Stattdessen werden im Analyseprozess durch statistische Verfahren (z. B. Interaktionstermen, Mehrebenenmodelle) differenzierte Zusammenhänge zwischen kombinatorischen Gruppenpositionen und situativem Wohlbefinden untersucht \parencite{bauerIntersectionalityQuantitativeResearch2021}.

Der Fokus liegt dabei auf struktureller Ungleichheit und der Kontextabhängigkeit sozialer Positionen. Das bedeutet auch, dass die Wirkung intersektionaler Positionen nicht in allen Kontexten gleich sein muss. Die Interaktion mit räumlichen und sozialen Bedingungen – etwa Exklusion im öffentlichen Raum – ist zentraler Bestandteil der Analyse \parencite{rodo-de-zarateDevelopingGeographiesIntersectionality2014}.

\subsection*{Zusammenführung}

Die hier vorgeschlagenen Definitionen bilden die konzeptionelle Grundlage für die Entwicklung eines Erhebungsinstruments, das sowohl subjektives Wohlbefinden als auch soziale Positionierung erfasst, ohne die Komplexität intersektionaler Erfahrungen zu reduzieren. Die Entscheidung, auf eine realzeitliche Erhebung (z. B. mittels EMA) zu verzichten, folgt dabei nicht nur aus methodischen Überlegungen, sondern auch aus dem Ziel, möglichst barrierearme, partizipative und datenschutzfreundliche Forschung zu ermöglichen.
