% LTeX: language=de-CH

\chapter{Diskussion} \label{sec:diskussion}

\section{Potential und Grenzen des entwickelten Erhebungsinstruments}

Die Entwicklung von \textit{InterMind} war im Rahmen dieser Arbeit nicht nur ein technisches, sondern auch ein methodisches Experiment. Ziel war es, mit begrenzten Ressourcen ein Werkzeug zu schaffen, das situative, geolokalisierte Erhebungen zuverlässig durchführen kann -- und dabei die Grundprinzipien von Transparenz, Datenschutz und Anpassungsfähigkeit wahrt. Die im Kapitel skizzierten technischen Entscheidungen waren dabei stets auch methodische Abwägungen: Sie bestimmten nicht nur, wie die App funktioniert, sondern auch, welche Formen der Datenerhebung und -auswertung überhaupt möglich waren.

Besonders prägend war die Wahl eines bewusst reduzierten, clientseitig gesteuerten Systemdesigns. Diese Architektur minimierte Abhängigkeiten von externer Infrastruktur, reduzierte potenzielle Datenschutzrisiken und erlaubte eine transparente, vollständig nachvollziehbare Funktionsweise. Gleichzeitig bedeutete sie den Verzicht auf Funktionen, wie sie in komplexeren GEMA-Implementierungen üblich sind -- etwa geofence-basierte Trigger oder serverseitige Kontextlogiken. Dadurch blieb die App methodisch auf feste, vordefinierte Erhebungszeitpunkte beschränkt und konnte nicht adaptiv auf räumliche oder kontextuelle Veränderungen reagieren. Für explorative Pilotstudien wie die vorliegende war dies ausreichend, in längerfristigen oder gross angelegten Projekten wäre jedoch eine dynamischere, kontextsensitivere Architektur wünschenswert.

Die Entscheidung zur Open-Source-Veröffentlichung stellt einen zentralen Bestandteil des Projekts dar. Sie ermöglicht anderen Forschenden nicht nur die Nachnutzung des Codes, sondern schafft auch die Grundlage für kollaborative Weiterentwicklungen. Gleichzeitig machte die Erfahrung mit den App-Store-Gatekeeping-Prozessen deutlich, dass Offenheit allein keine Garantie für breite Zugänglichkeit ist: Die Distribution über zentrale Plattformen bleibt an kommerzielle und intransparente Strukturen gebunden, die auch nicht-kommerzielle, wissenschaftliche Projekte einschränken können. Hier zeigt sich ein strukturelles Spannungsfeld zwischen der offenen, gemeinschaftsorientierten Logik von Open-Source-Software und den geschlossenen, marktkontrollierten Ökosystemen der grossen Plattformanbieter.

Im Rückblick wird deutlich, dass die App-Entwicklung in dieser Form einerseits ein funktionierendes, forschungsnahes Werkzeug hervorgebracht hat, andererseits aber auch klare Grenzen aufweist. Diese liegen weniger in der Stabilität oder Bedienbarkeit, sondern vielmehr in der eingeschränkten Kontextanpassung, der fehlenden Echtzeitauswertung und der aufwändigen Anpassbarkeit für andere Forschungssettings. Zukünftige Iterationen könnten hier ansetzen -- etwa durch die Ergänzung serverseitiger Module, die Entwicklung eines webbasierten Dashboards für Monitoring und Feedback, oder die modularisierte Integration zusätzlicher Erhebungsmethoden.

Damit verdeutlicht \textit{InterMind} sowohl die Chancen als auch die Grenzen einer eigenständigen Entwicklung im Rahmen einer Abschlussarbeit: Sie eröffnet Handlungsspielräume, schafft technologische Unabhängigkeit im Entwicklungsprozess und macht Forschungsinfrastruktur transparent -- bleibt aber eingebettet in grössere, teils restriktive Strukturen, die den Handlungsspielraum letztlich mitbestimmen.

\section{Reflexion und Weiterentwicklungspotenzial des Fragebogens}

Der entwickelte Fragebogen erwies sich im Feld als grundsätzlich funktional und gut in den Ablauf der Studie integrierbar. Er erfüllte die Anforderung, situative Erhebungen in kurzer Zeit und mit geringer Belastung für die Teilnehmenden durchführen zu können. Gleichzeitig zeigte sich jedoch, dass diese Stärken teilweise mit methodischen Einbussen erkauft wurden, die den wissenschaftlichen Anspruch der Erhebung begrenzen.

Besonders deutlich wird dies bei der Auswahl der Items zur Erfassung situativen affektiven Wohlbefindens. Die gewählten Dimensionen -- darunter „generelles Wohlbefinden“, Zufriedenheit, Anspannung, Energie und Zugehörigkeit -- erlaubten zwar eine kompakte Erfassung, entstanden jedoch nicht aus einer stringenten theoretischen Modellierung heraus. Diese pragmatische Herangehensweise erleichterte zwar die Umsetzung im Rahmen einer Mehrfacherhebung, führte aber zu einer geringeren konzeptuellen Schärfe und erschwerte den direkten Vergleich mit bestehenden Studien.

Auch der Verzicht auf etablierte standardisierte Skalen hatte ambivalente Folgen. Er trug dazu bei, den Fragebogen schlank zu halten und die Akzeptanz bei den Teilnehmenden zu erhöhen, schränkte jedoch die Vergleichbarkeit der Daten und ihre Anschlussfähigkeit an bestehende Forschungsinstrumente ein. Eine gekürzte, modulare Integration validierter Skalen hätte hier einen Ausgleich zwischen Praktikabilität und methodischer Robustheit schaffen können.

Die mehrsprachige Umsetzung des Instruments war ein wichtiger Schritt in Richtung Zugänglichkeit, blieb jedoch ohne formalisierte Validierung durch muttersprachliche Expert:innen. Dadurch ist nicht auszuschliessen, dass inhaltliche Nuancen, insbesondere bei affektiven Zustandsbeschreibungen, zwischen den Sprachversionen leicht variierten. Diese Unsicherheiten verstärkten sich bei sensiblen Konzepten wie \gls[noindex]{race}, für das im deutschsprachigen Kontext keine etablierten, diskriminierungssensiblen Kategorien verfügbar sind. Die gewählte Operationalisierung über Geburts- und Aufenthaltsland senkte zwar die Erhebungsbarrieren, konnte die Komplexität rassifizierter Erfahrungen jedoch nur unvollständig erfassen.

Schliesslich war der Entwicklungsprozess des Fragebogens zwar iterativ angelegt und von kontinuierlichem Feedback begleitet, basierte jedoch nicht auf einem formalen Pretest mit einer breiten und divers zusammengesetzten Testgruppe. Dadurch wurden potenzielle Verständnisschwierigkeiten oder kulturelle Unschärfen nur in begrenztem Umfang sichtbar.

Insgesamt bleibt festzuhalten, dass der Fragebogen in seiner vorliegenden Form eine praktikable, aber methodisch eingeschränkte Lösung darstellt. Für zukünftige Studien bieten sich mehrere Ansatzpunkte zur Weiterentwicklung: eine engere theoretische Anbindung der Items, die gezielte Integration gekürzter validierter Skalen, ein systematischeres Übersetzungs- und Validierungsverfahren sowie umfassendere Pretests. Auf diese Weise liesse sich die inhaltliche Aussagekraft der Erhebung stärken, ohne die für hochfrequente Befragungen notwendige Niedrigschwelligkeit aufzugeben.

\section{Empfehlungen für weiterführende Forschung}

\subsection{Verbesserungsvorschläge zur Erhöhung der Teilnahmequote}

\subsection{Optimierung der intersektionalen Datenerhebung und Analyse}

\subsection{Integration qualitativer Verfahren}


% ohne technische details mal hier rüberkopiert:

% \section{Eigenständig, aber nicht unabhängig -- Entwicklung im Plattformzeitalter}

% Die Entwicklung von \gls{intermind} war mein erstes grösseres Projekt in \gls{typescript} und mit \gls{reactnative}. Die Umstellung vom strikt objektorientierten Denken in \gls{java} auf den dynamischeren, komponentenbasierten Ansatz war anspruchsvoll, aber enorm lehrreich. Insbesondere das konsequente Anwenden der \gls{solid}-Prinzipien half dabei, die Struktur der Anwendung nachvollziehbar zu halten -- gerade in einem neuen Ökosystem. Die App funktioniert stabil, sieht gut aus, und hat ihren Zweck erfüllt.

% Trotz einer bewussten Orientierung an Prinzipien wie \gls{solid} und einem grundlegenden Architekturkonzept zeigte sich im Verlauf der Entwicklung, dass eine noch systematischere Auseinandersetzung mit der Softwarearchitektur hilfreich gewesen wäre. Zwar wurde auf eine modulare Struktur geachtet, viele Designentscheidungen wurden jedoch eher situativ getroffen und nicht im Sinne eines übergeordneten Gesamtdesigns immer wieder überprüft. Gerade im weiteren Projektverlauf wäre es sinnvoll gewesen, gezielt zu früheren architektonischen Überlegungen zurückzukehren und diese zu reflektieren oder anzupassen.

% Methoden wie \emph{Test-Driven Development} hätten diesen Prozess zusätzlich stützen können, indem sie klare Schnittstellen und Verantwortlichkeiten frühzeitig erzwingen. Auch der Aufbau automatisierter Tests und eine kontinuierlich integrierte Codeanalyse hätten dazu beigetragen, Fehlerquellen frühzeitig zu identifizieren und die langfristige Wartbarkeit der Anwendung zu verbessern. Viele kleinere Schwächen im Code wurden zwar pragmatisch behoben, ein strukturierteres Qualitätsmanagement hätte jedoch die Notwendigkeit späterer Refactoring-Prozesse deutlich reduziert.

% In diesem Sinne reiht sich die App auch in eine typische Dynamik vieler \gls{opensource}-Projekte ein: Sie wurde aus einem konkreten Forschungsbedarf heraus entwickelt, funktioniert zuverlässig, ist öffentlich dokumentiert -- aber nicht in jedem Teilbereich optimal strukturiert. Durch die Offenlegung des Quellcodes besteht jedoch die Möglichkeit, dass andere Entwickler\genderstern innen auf dieser Grundlage aufbauen, Verbesserungsvorschläge einbringen oder eigene Erweiterungen umsetzen.

% \vspace{1em}

% Trotz stabiler Funktionalität und durchdachter Grundstruktur weist das entwickelte System klare Begrenzungen auf -- insbesondere im Hinblick auf die situative Reaktionsfähigkeit und Kontextanpassung. So verzichtet \gls{intermind} bewusst auf kontinuierliches Geotracking, automatisierte Trigger oder serverseitige Kontextlogiken, wie sie in anderen \acrshort{gema}-Systemen Anwendung finden.

% Ein Beispiel dafür bietet das im Rahmen einer kanadischen Studie zu Nationalparks entwickelte \acrshort{health}-Plattform \parencite{wrayHealthyEnvironmentsActive2025}. Die Dokumentation zu diesem Tool ist erst während der Entstehung dieser Arbeit als Preprint veröffentlicht worden. Die App wird derzeit exklusiv im Rahmen des \textit{ParkSeek}-Projekts\footnote{\href{https://parkseek.ca/}{parkseek.ca}} eingesetzt und ist nicht öffentlich zugänglich. Ihre zugrundeliegende Systemarchitektur erlaubt eine kontinuierliche Standorterfassung und serverseitige Kontextverarbeitung, wodurch komplexe Logiken wie geofence-basierte Trigger umgesetzt werden können. So lassen sich etwa Benachrichtigungen auslösen, wenn sich Teilnehmende über längere Zeit in spezifischen Umwelten aufhalten. Diese technisch anspruchsvolle Lösung erlaubt eine besonders enge Verzahnung zwischen räumlichem Verhalten und situativer Befragung, geht jedoch mit einem hohen Aufwand sowie erheblichen Anforderungen an Datenschutz, Datenmanagement und Infrastruktur einher.

% Im Rahmen eines Bachelorprojekts wäre die Implementierung eines derart umfassenden Systems weder zeitlich noch organisatorisch realistisch gewesen. Stattdessen wurde ein datensparsamer, clientseitig gesteuerter Ansatz gewählt, der mit begrenzten Mitteln eine funktionale, transparente und reflektierte Umsetzung ermöglicht. Die Entscheidung für ein reduziertes Systemdesign war damit nicht nur eine Frage des Aufwands, sondern auch ein bewusster Kompromiss zugunsten von Kontrollierbarkeit und Datenschutz.

% Eine weitere Limitation des aktuellen Systemdesigns liegt im Fehlen eines serverseitigen Dashboards oder einer integrierten Auswertungsoberfläche. Es besteht keine Möglichkeit, Rückmeldungen in Echtzeit zu visualisieren, aggregierte Antworten einzusehen oder Monitoring-Funktionen während der Erhebung zu nutzen. Solche Features wären insbesondere für die Steuerung längerer Erhebungsphasen, die Qualitätssicherung oder für Feedbackschleifen mit den Teilnehmenden von Vorteil gewesen. Ihre Umsetzung hätte jedoch zusätzliche Entwicklungsressourcen sowie eine komplexere Backend-Architektur vorausgesetzt. Gleichwohl bleibt die Möglichkeit bestehen, entsprechende Funktionen in zukünftigen Iterationen oder auf Basis der veröffentlichten Codebasis nachzurüsten.

% \vspace{1em}

% Die Erfahrungen rund um die Veröffentlichung in den App Stores hat zentrale Spannungsfelder digitaler Infrastruktur deutlich gemacht. Obwohl die App funktional einsatzbereit war und über TestFlight bzw. den Play Store zugänglich gemacht werden konnte, blieb die reguläre Veröffentlichung im Apple App Store aufgrund einer intransparenten Ablehnung verwehrt.

% Solche Prozesse offenbaren strukturelle Abhängigkeiten, die weit über dieses Projekt hinausgehen: Zwei gigantische multinationale Tech-Konzerne kontrollieren in weiten Teilen den Zugang zu digitaler Infrastruktur. Dabei wirken sie zugleich als Regelsetzer, Infrastrukturbetreiber und ökonomische Gatekeeper. Diese doppelte Rolle ist nicht demokratisch legitimiert, aber mit erheblicher Lenkungsmacht verbunden. Gerade nicht-kommerzielle, experimentelle oder aktivistische Projekte sind von diesen Kontrollmechanismen besonders betroffen, da sie sich nicht ohne Weiteres den geforderten Verwertungslogiken oder Standardprozessen unterwerfen.

% Auch wenn Open-Source-Prinzipien allein diese strukturellen Hürden nicht auflösen können, war die Entscheidung zur Veröffentlichung des Quellcodes dennoch zentral: Sie schafft Transparenz, ermöglicht Weiterentwicklung und signalisiert ein bewusstes Gegenmodell zu proprietären, intransparenten Systemen. Gleichzeitig offenbart sich hier ein grundlegendes Spannungsfeld: Die offene, zugängliche und gemeinschaftsorientierte Logik von \gls{opensource}-Software steht in einem scharfen Kontrast zu den geschlossenen, marktkontrollierten Strukturen kommerzieller Distributionsplattformen. Wer eine App entwickeln und öffentlich zugänglich machen will, ist faktisch gezwungen, sich diesen Plattformen zu unterwerfen.

% Die Arbeit an der App war dabei nicht nur funktional motiviert, sondern auch von der Erfahrung getragen, ein eigenes digitales Werkzeug gestalten zu können -- mit all seinen Herausforderungen, aber auch mit dem unmittelbaren Lernerfolg und der Freude am konkreten Entstehungsprozess. Gerade vor dem Hintergrund einer zunehmend von privatwirtschaftlichen Plattformen dominierten digitalen Infrastruktur bleibt die Fähigkeit, eigene Werkzeuge zu entwickeln, ein wichtiger Akt technischer Aneignung.

% auch hier nicht nochmal entwickelnd erklären, je nachdem acuh ganz sein lassen

% \section{Klar, verständlich, iterativ -- Der Weg zum finalen Fragebogen}

% Die sprachliche Gestaltung der Fragebogen-Items stellte im Entwicklungsprozess eine zentrale methodische Herausforderung dar. Ziel war es, die Befragung möglichst zugänglich, verständlich und gleichzeitig inhaltlich präzise zu gestalten. Da die Befragung explizit auf eine intersektionale Analyse abzielt, wurde besonderer Wert darauf gelegt, die sprachliche Zugänglichkeit möglichst breit zu gewährleisten. Folglich wurde der Fragebogen bewusst mehrsprachig konzipiert und auf Deutsch, Englisch sowie Französisch umgesetzt. Weitere Sprachversionen wären zwar aus Sicht der intersektionalen Zugänglichkeit wünschenswert gewesen, scheiterten jedoch am hohen Aufwand für qualitativ hochwertige und inhaltlich konsistente Übersetzungen.

% Ein grundsätzliches Anliegen war eine möglichst direkte, adressierende Sprache in der \enquote{Du}-Form, um einen niederschwelligen Zugang zur Befragung zu fördern und hierarchische Distanz zwischen Forschenden und Teilnehmenden zu reduzieren. Gleichzeitig mussten die Formulierungen prägnant, alltagsnah und schnell erfassbar sein, da insbesondere die situativen Erhebungen kurz gehalten werden sollten. Hier ergab sich ein methodischer Balanceakt: Einerseits sollte die Befragung leicht verständlich bleiben, andererseits mussten komplexe Konzepte in zugänglicher Sprache operationalisiert werden. So wurde \gls{bspw} das Konzept der \gls{intersektionalitaet} im Einführungsteil des Fragebogens erläutert, danach jedoch bewusst vermieden, um unnötige Barrieren zu reduzieren. Stattdessen wurden alternative Formulierungen wie \enquote{persönliche Merkmale} verwendet, die jedoch teilweise inhaltliche Unschärfen mit sich brachten. 

% Besonders deutlich wurde diese Herausforderung im Umgang mit dem Konzept \gls[noindex]{race}. Gerade im deutschsprachigen Kontext existieren hier nur schwer geeignete Begrifflichkeiten: Formulierungen wie \enquote{Rasse} oder \enquote{Ethnizität} sind entweder sprachlich ungebräuchlich, problematisch oder stark mit kolonialen und biologistischen Zuschreibungen assoziiert \parencite[\gls{vgl}][]{roigIntersectionalityEuropeDepoliticized2018}. Alternativ verwendete Begriffe wie \enquote{Herkunft} oder \enquote{Aussehen} sind wiederum unpräzise und greifen die Dimension rassifizierter Diskriminierung nur unvollständig auf.

% Die Übersetzung der Items erfolgte nicht wörtlich, sondern sinngemäss, wobei insbesondere bei affektiven Zustandsbeschreibungen semantische Abstimmungen zwischen den Sprachversionen vorgenommen wurden. Dabei wurden auch kulturelle Unterschiede in der Alltagsverwendung bestimmter Begriffe berücksichtigt. Dieser Kompromiss ermöglichte trotz beschränkter Ressourcen eine hinreichend konsistente Mehrsprachigkeit, brachte jedoch gewisse methodische Limitierungen hinsichtlich der Vergleichbarkeit der Sprachversionen mit sich.

% Der beschriebene Sprach- und Übersetzungsprozess war eingebettet in einen breiteren, iterativen Entwicklungsprozess, der sowohl auf der Analyse bestehender Literatur als auch auf kontinuierlichem Feedback basierte. Ausgangspunkt bildeten Studien wie die Urban-Mind-Studie \parencite{bakolisUrbanMindUsing2018}, deren methodische Ansätze zur Erhebung situativen Wohlbefindens und räumlicher Wahrnehmung als Orientierung dienten. Diese Ansätze wurden jedoch um eigene Überlegungen zur intersektionalen Erhebung sozialer Positionierung ergänzt und in mehreren Durchläufen kritisch reflektiert.

% Während der Testphase der App Entwicklung (Siehe \cref{sec:app_entwicklung_feldtest}) sind ebenfalls zahlreiche kleinere Rückmeldungen zu Formulierungen und sprachlichen Feinheiten eingegangen. Diese wurden laufend eingearbeitet. Ebenfalls wurde der Fragebogen mit der betreuenden Dozentin durchgegangen und danach ebenfalls nochmals entsprechend überarbeitet.
% Ein wiederkehrendes methodisches Kriterium bei diesen Diskussionen war stets, die Belastung für Teilnehmende so gering wie möglich zu halten, ohne zentrale Aspekte der Forschungsfrage zu vernachlässigen. Durch den iterativen Ansatz konnte die Perspektive potenzieller Befragter frühzeitig einbezogen werden, was zu einer praxisnahen Optimierung der Items und des Fragebogenaufbaus führte.

% Im Rückblick lassen sich einige Punkte identifizieren, die bei einer erneuten Durchführung anders gestaltet werden könnten. Die Auswahl der Wohlbefindensdimensionen erfolgte nicht vollständig theoriegeleitet; insbesondere das Item „generelles Wohlbefinden“ wirkt im Nachhinein wenig trennscharf. Eine stärkere konzeptuelle Fundierung der Items wäre sinnvoll, um die Aussagekraft einzelner Skalen zu erhöhen.

% Der Verzicht auf standardisierte Skalen ermöglichte eine kompakte Erhebung, schränkt jedoch Vergleichbarkeit und Validität ein. Eine modularisierte Integration validierter Instrumente -- etwa in gekürzter Form -- könnte eine tragfähige Alternative darstellen.

% Die mehrsprachige Umsetzung war aus methodischer Sicht wichtig, konnte jedoch mangels Ressourcen nicht vollständig abgesichert werden. Eine zusätzliche Validierung durch muttersprachliche Expert:innen wäre wünschenswert gewesen, um semantische Konsistenz über Sprachversionen hinweg besser zu gewährleisten.

% Insgesamt zeigen sich an mehreren Stellen Stellschrauben für eine künftige Weiterentwicklung -- etwa durch eine engere theoretische Anbindung, gezielte Pretests oder eine systematischere Überprüfung von Übersetzungen und Antwortformaten. Gleichzeitig hat sich der gewählte Zugang als praktikabel und kontextsensibel erwiesen, insbesondere im Hinblick auf Zugänglichkeit und situative Anschlussfähigkeit.

% Besonders herausfordernd war die Erfassung von \gls[noindex]{race}. Im europäischen Kontext existieren kaum etablierte Kategorien, die rassifizierte Zugehörigkeiten erfassen, ohne problematische koloniale oder biologistische Zuschreibungen zu reproduzieren \parencite[\gls{vgl}][]{roigIntersectionalityEuropeDepoliticized2018}. Anders als in der US-amerikanischen Tradition, in der standardisierte Selbstkategorisierungen weit verbreitet sind, fehlen im hiesigen Kontext praktikable, breit akzeptierte Formate für quantitative Erhebungen. 

% Vor diesem Hintergrund wurde aus pragmatischen Gründen lediglich erfasst, ob Teilnehmende aktuell in einem anderen Land leben als in jenem, in dem sie geboren wurden. Diese Lösung reduzierte die Erhebungsbarrieren, blieb jedoch analytisch begrenzt: Sie kann nur indirekt auf rassifizierte Erfahrungen hinweisen und wird der Komplexität intersektionaler Ungleichheiten nicht gerecht. Rückblickend wäre eine offene, selbstbezeichnungsbasierte Erhebung vorzuziehen gewesen, um dieser sozialen Dimension angemessen Sichtbarkeit zu verleihen.

% Vor diesem Hintergrund wurde ein eigener, stark reduzierter Item-Satz entwickelt, um zentrale Dimensionen des Wohlbefindens situativ abbilden zu können. Ausgewählt wurden fünf Dimensionen: generelles Wohlbefinden, Zufriedenheit, Anspannung, Energie und Zugehörigkeit. Die Antworten wurden über lineare Slider-Skalen erfasst, um eine schnelle und intuitive Bearbeitung zu ermöglichen. 

% Diese Lösung stellt jedoch einen methodischen Kompromiss dar. Die Auswahl der Dimensionen erfolgte nicht auf Basis eines validierten theoretischen Modells, sondern primär pragmatisch und unter der Prämisse minimaler Befragungsdauer. Entsprechend ist die Validität und Vergleichbarkeit der erhobenen Werte eingeschränkt. Rückblickend erweist sich dies als Schwachpunkt der Studie: Die Messung bleibt inhaltlich und konzeptuell weniger trennscharf als wünschenswert, und es fehlt an einer etablierten Referenz, um die Ergebnisse eindeutig einzuordnen. Eine künftige Weiterentwicklung sollte daher auf einer systematischeren Konzeptualisierung beruhen und -- sofern möglich -- gekürzte, validierte Instrumente integrieren, die auf die Anforderungen situativer Mehrfacherhebungen angepasst sind.


