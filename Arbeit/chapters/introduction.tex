\section{Einleitung}

\glqq Städte sind für alle da\grqq{} – so zumindest die idealisierte Vorstellung urbaner Gleichheit, wie sie oft in Leitbildern, Planungsstrategien und im Anspruch auf das \emph{Recht auf Stadt} nach \textcite{lefebvreDroitVille1967} propagiert wird. Doch wie erleben Menschen den urbanen Raum tatsächlich? Und wie beeinflusst ihre soziale Position – etwa hinsichtlich Alter, Geschlecht, Herkunft oder Gesundheit – ihr momentanes Wohlbefinden in bestimmten Umgebungen? Diese Fragen stehen im Zentrum dieser Bachelorarbeit, die sich der intersektionalen Analyse des unmittelbaren Wohlbefindens in städtischen Kontexten widmet.

Der Forschungsstand zum Zusammenhang von Umwelt und mentalem Wohlbefinden hat sich in den letzten Jahren stark erweitert. Projekte wie Urban Mind \parencite{bakolisUrbanMindUsing2018} oder Studien zu mentaler Gesundheit in Natur- und Stadträumen \parencite{bergouMentalHealthBenefits2022} zeigen eindrücklich, dass räumliche Merkmale das psychische Befinden beeinflussen. Auch Experience-Sampling-Methoden (EMA) haben sich als fruchtbar erwiesen, um solche Zusammenhänge in Echtzeit zu erfassen \parencite{shiffmanEcologicalMomentaryAssessment2008, stoneEcologicalMomentaryAssessment1994, kirchnerSpatiotemporalDeterminantsMental2016}. Diese Methoden erlauben es, situative, affektive Reaktionen auf räumliche Kontexte sichtbar zu machen – ein bedeutender Fortschritt gegenüber früheren, retrospektiven Ansätzen. Gleichzeitig existiert eine wachsende Literatur zu Fragen der Intersektionalität im Raum \parencite{crenshawMappingMarginsIntersectionality1991, rodo-de-zarateDevelopingGeographiesIntersectionality2014, rodo-de-zarateIntersectionalityFeministGeographies2018, rodo-de-zarateYoungLesbiansNegotiating2015}, die zeigt, wie Machtverhältnisse und soziale Kategorien wie Geschlecht, Klasse oder Sexualität räumlich wirksam werden. Diese Debatten haben zu innovativen methodischen Zugängen geführt, etwa den Relief Maps \parencite{rodo-de-zarateDevelopingGeographiesIntersectionality2014}, die den Zusammenhang von Ort, Macht und Subjektivität visualisieren.

In dieser Arbeit werden diese beiden Perspektiven zusammengeführt, um zu untersuchen, wie sich intersektionale Positionierungen auf das unmittelbare Wohlbefinden im Alltag auswirken – in Abhängigkeit konkreter räumlicher Situationen. Der Beitrag liegt somit in der Verbindung zweier bislang meist getrennt bearbeiteter Forschungsfelder: der Analyse räumlicher Umweltwirkungen auf das momentane Wohlbefinden und der intersektionalen Ungleichheitsforschung. Ziel ist es nicht, bestehende Felder grundlegend zu reformulieren, sondern eine methodisch neue Kombination zu erproben, die Perspektiven der kritischen Geographie mit digitalen, datengestützten Erhebungsverfahren zusammenbringt.

Persönlich motiviert ist die Arbeit durch das Interesse an der Verbindung kritischer Geographie mit digitalen Methoden sowie dem Wunsch, mit Hilfe eigens entwickelter Technologien neue Erkenntnisse über soziale Gerechtigkeit im Stadtraum zu generieren. Zudem soll die Forschung einen Beitrag zu aktuellen stadtgeographischen Debatten in Bern leisten, etwa im Zusammenhang mit Projekten zur Stadthitze \parencite{burgerModellingSpatialPattern2021} oder dem Bernometer.

Im Fokus steht die folgende Forschungsfrage:

\begin{quote}
\textbf{Wie beeinflussen räumliche Umgebungen das momentane Wohlbefinden intersektional positionierter Personen im Alltag?}
\end{quote}

Untersucht wird also nicht das langfristige subjektive Wohlbefinden, sondern die situative, affektive Wahrnehmung, wie sie etwa mit EMA-Methoden erhoben wird. Ziel ist es, sowohl Umweltmerkmale als auch soziale Positionierungen in Beziehung zum momentanen Wohlbefinden zu setzen – also zu analysieren, wie beispielsweise Alter, Geschlecht oder Bildungsstand in Kombination mit bestimmten räumlichen Eigenschaften (z.\,B. Grünraum, Verkehr, Atmosphäre) das Befinden beeinflussen. Die theoretische Grundlage bildet der MAIHDA-Ansatz ("Multilevel Analysis of Individual Heterogeneity and Discriminatory Accuracy") \parencite{grossModellingIntersectionalityQuantitative2023}, der als vielversprechende Methode zur Umsetzung intersektioneller Analysen gilt.

Zur Umsetzung wurde eine App entwickelt, die über einen mehrtägigen Zeitraum hinweg wiederholt Fragen zum momentanen Wohlbefinden, zur Umgebung sowie zu individuellen Merkmalen stellt. Die Daten werden in Echtzeit georeferenziert erfasst und anonym gespeichert. Die Erhebung erfolgt im Rahmen einer Pilotstudie mit Studierenden in Bern. Die App wurde bewusst als Open-Source-Tool entwickelt, um eine flexible Wiederverwendung und Weiterentwicklung zu ermöglichen. Mit geringem Aufwand lässt sie sich an andere Forschungskontexte anpassen und stellt so eine potenziell nachhaltige Infrastruktur für die Erhebung kontextualisierter Wohlbefindensdaten dar. Diese Offenheit und Modularität sind Teil eines bewusst gewählten Designs, das interdisziplinäre Anschlussfähigkeit und partizipative Weiterentwicklung fördern soll.

Die Arbeit ist wie folgt aufgebaut: Kapitel 2 diskutiert den theoretischen Rahmen mit Fokus auf intersektionale Ungleichheiten, Well-Being-Konzepte sowie EMA-Methoden. In Kapitel 3 wird das technische Vorgehen erläutert, insbesondere die App-Entwicklung, das Studiendesign und der methodische Zugriff über MAIHDA. Kapitel 4 präsentiert zentrale Ergebnisse zur Verteilung von Wohlbefinden in Abhängigkeit von Raum und Positionierung. In Kapitel 5 werden diese Befunde eingeordnet, kritisch diskutiert und Implikationen für Forschung und Praxis abgeleitet.

Diese Arbeit versteht sich als explorativer Beitrag zur Verbindung digitaler Werkzeuge mit kritischer Geographie. Sie erhebt nicht den Anspruch auf Repräsentativität, sondern zielt darauf, methodische Möglichkeiten und erste Hypothesen zur intersektionalen Analyse von Wohlbefinden im Raum aufzuzeigen.
