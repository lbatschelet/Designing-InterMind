\section{Einleitung}

„Städte sind für alle da“ – diese idealisierte Vorstellung urbaner Gleichheit, wie sie oft in Leitbildern, Planungsstrategien und insbesondere im Anspruch auf das \emph{Recht auf Stadt} nach \textcite{lefebvreDroitVille1967} propagiert wird, prägt aktuelle Debatten der kritischen Geographie. Doch wie erleben Menschen den urbanen Raum tatsächlich? Und wie beeinflusst ihre soziale Position – etwa hinsichtlich Alter, Geschlecht, Herkunft oder Gesundheit – ihr momentanes Wohlbefinden in bestimmten Umgebungen? Diese Fragen stehen im Zentrum der vorliegenden Bachelorarbeit, die sich der intersektionalen Analyse des unmittelbaren Wohlbefindens in urbanen Kontexten widmet.

In den letzten Jahren wurde der Zusammenhang zwischen räumlichen Umgebungen und mentalem Wohlbefinden zunehmend erforscht. Projekte wie Urban Mind \parencite{bakolisUrbanMindUsing2018} und Studien zur mentalen Gesundheit in Natur- und Stadträumen \parencite{bergouMentalHealthBenefits2022, hammoudSmartphonebasedEcologicalMomentary2024} zeigen, dass räumliche Merkmale erheblichen Einfluss auf das psychische Befinden nehmen. Vor allem Experience-Sampling-Methoden (EMA) haben sich dabei als besonders geeignet erwiesen, situative und affektive Reaktionen auf räumliche Kontexte in Echtzeit zu erfassen \parencite[vgl.][]{shiffmanEcologicalMomentaryAssessment2008, stoneEcologicalMomentaryAssessment1994, kirchnerSpatiotemporalDeterminantsMental2016}. Durch diese Methoden wird eine unmittelbare und kontextbezogene Datenerhebung ermöglicht, was einen wesentlichen Fortschritt gegenüber retrospektiven Ansätzen darstellt \parencite{kahnemanDevelopmentsMeasurementSubjective2006}.

Parallel dazu existiert eine umfangreiche Forschungsliteratur zur Intersektionalität und deren räumlichen Implikationen \parencite[vgl.][]{crenshawMappingMarginsIntersectionality1991, rodo-de-zarateDevelopingGeographiesIntersectionality2014, rodo-de-zarateYoungLesbiansNegotiating2015, rodo-de-zarateIntersectionalityFeministGeographies2018}. Diese Arbeiten verdeutlichen, wie unterschiedliche soziale Kategorien wie Geschlecht, Klasse oder ethnische Zugehörigkeit in räumlichen Kontexten miteinander verwoben sind und Ungleichheiten erzeugen oder verstärken können. Insbesondere methodische Innovationen wie die Relief Maps \parencite{rodo-de-zarateDevelopingGeographiesIntersectionality2014} erlauben eine Visualisierung und Analyse dieser komplexen Wechselwirkungen.

Diese Bachelorarbeit verbindet die beiden Perspektiven – EMA-basierte Wohlbefindensforschung und intersektionale Ungleichheitsanalyse – und untersucht, wie sich intersektionale Positionierungen konkret auf das unmittelbare Wohlbefinden in verschiedenen räumlichen Situationen auswirken. Ziel ist nicht eine fundamentale theoretische Neuorientierung, sondern die methodische Verbindung bislang getrennt betrachteter Ansätze, um neue Erkenntnisse an der Schnittstelle kritischer Geographie und digitaler Methoden zu generieren. 

Die persönliche Motivation für diese Arbeit speist sich aus einem Interesse an der Verknüpfung kritischer geographischer Ansätze mit digitalen Technologien sowie aus dem Wunsch, mit eigens entwickelten digitalen Werkzeugen neue Einblicke in Fragen sozialer Gerechtigkeit und Wohlbefinden im städtischen Alltag zu ermöglichen. Darüber hinaus soll die Arbeit konkrete Beiträge zu stadtgeographischen Diskussionen in Bern liefern, etwa im Kontext von Forschung zur Stadthitze \parencite{burgerModellingSpatialPattern2021} oder Projekten wie dem Bernometer, die auf detaillierte, raumbezogene Wohlbefindensdaten angewiesen sind.

Im Fokus steht dabei folgende Forschungsfrage:

\begin{quote}
\textbf{Wie beeinflussen räumliche Umgebungen das momentane Wohlbefinden intersektional positionierter Personen im Alltag?}
\end{quote}

Dabei geht es explizit nicht um langfristige subjektive Wohlbefindenswerte, sondern um situative, affektive Reaktionen, die mittels EMA erhoben werden. Ziel der Analyse ist es, Umweltmerkmale und soziale Positionierungen systematisch miteinander in Beziehung zu setzen und zu untersuchen, wie beispielsweise Alter, Geschlecht oder Bildungsstand zusammen mit spezifischen räumlichen Merkmalen (wie Grünflächen, Verkehrsdichte oder atmosphärische Faktoren) das momentane Wohlbefinden prägen. Als theoretisch-methodische Grundlage dient dabei der MAIHDA-Ansatz („Multilevel Analysis of Individual Heterogeneity and Discriminatory Accuracy“), der besonders geeignet ist, intersektionale Fragestellungen quantitativ abzubilden \parencite{grossModellingIntersectionalityQuantitative2023}.

Zur Erhebung der notwendigen Daten wurde die App \emph{InterMind}\footnote{\href{https://intermind.ch/app}{intermind.ch/app}} entwickelt, die über einen mehrtägigen Zeitraum hinweg kontinuierlich Daten zum momentanen Wohlbefinden, zur räumlichen Umgebung und zu individuellen sozialen Merkmalen erfasst. Diese Daten werden in Echtzeit georeferenziert erhoben und anonymisiert gespeichert. Im Rahmen einer Pilotstudie mit Studierenden der Universität Bern werden erste explorative Daten gesammelt. Ein wichtiger methodischer Aspekt ist, dass die App bewusst Open-Source entwickelt wurde, um eine einfache und flexible Anpassung an ähnliche Forschungskontexte zu ermöglichen und somit langfristig eine nachhaltige Infrastruktur für kontextualisierte Wohlbefindensdaten zu bieten.

Der Aufbau der Arbeit gliedert sich wie folgt: Kapitel 2 erörtert den theoretischen Rahmen hinsichtlich Intersektionalität, Well-Being-Konzepten sowie EMA-Methoden. Kapitel 3 beschreibt die technische und methodische Umsetzung, insbesondere die Entwicklung der App, das Studiendesign sowie den analytischen Zugriff mittels des MAIHDA-Ansatzes. Kapitel 4 präsentiert zentrale empirische Ergebnisse zur Beziehung von Wohlbefinden, Raum und intersektionaler Positionierung. Abschließend werden diese Ergebnisse in Kapitel 5 kritisch diskutiert, kontextualisiert und Implikationen für weitere Forschung und städtische Planungspraxis aufgezeigt.

Diese Arbeit versteht sich als explorativer Beitrag, der methodische Innovationen mit gesellschaftlich relevanten Fragestellungen verbindet. Sie erhebt nicht den Anspruch auf allgemeine Repräsentativität, sondern zielt darauf ab, erste Hypothesen und methodische Potenziale für zukünftige intersektionale Analysen des momentanen Wohlbefindens im Stadtraum aufzuzeigen.
