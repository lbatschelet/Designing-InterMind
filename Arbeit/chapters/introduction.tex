% LTeX: language=de-CH

\section{Einleitung} \label{sec:einleitung}

„Städte sind für alle da“ – diese Vorstellung urbaner Gleichheit wird oft in Leitbildern und Planungsstrategien bemüht. Sie knüpft an Debatten an, die auch durch den Anspruch auf das \emph{Recht auf Stadt} nach \textcite{lefebvreDroitVille1967} inspiriert sind, auch wenn Lefebvres ursprüngliches Konzept weitaus radikaler war und eine grundlegende Transformation urbanen Lebens forderte. Unabhängig von der theoretischen Tiefe dieser Forderung stellt sich jedoch die Frage: Wie erleben Menschen den urbanen Raum tatsächlich? Und wie beeinflusst ihre soziale Position – etwa hinsichtlich Alter, Geschlecht, Herkunft oder Gesundheit – ihr momentanes Wohlbefinden in bestimmten Umgebungen? Diese Fragen stehen im Zentrum der vorliegenden Bachelorarbeit, die sich der intersektionalen Analyse des unmittelbaren Wohlbefindens in alltäglichen Lebensräumen widmet.

Methoden zur Erfassung momentaner psychischer Zustände und Erfahrungen im Alltag, wie die \gls{esm} und insbesondere das \gls{ema}, wurden bereits in den 1990er Jahren konzipiert, vor allem in der Psychologie \parencite[vgl.][]{stoneEcologicalMomentaryAssessment1994, shiffmanEcologicalMomentaryAssessment2008}. Sie zielten darauf ab, kontextbezogene Daten zu erheben und Nachteile rein retrospektiver Ansätze zu überwinden \parencite{kahnemanDevelopmentsMeasurementSubjective2006}. Das volle Potenzial dieser Methoden, insbesondere für eine unmittelbare, georeferenzierte Datenerhebung in Echtzeit, entfaltete sich jedoch erst mit der Verbreitung von Smartphones. An der Schnittstelle von Stadtplanung und Psychologie wurden zudem Ansätze zur räumlich expliziten Erfassung von Alltagserfahrungen entwickelt, wie etwa das \gls{gema} \parencite[vgl.][]{kirchnerSpatiotemporalDeterminantsMental2016}. Seit etwa Mitte der 2010er Jahre ist eine deutliche Zunahme an Studien zu beobachten, welche die durch Smartphones erweiterten \gls{ema}/\gls{gema}-Möglichkeiten nutzen, um den Zusammenhang zwischen spezifischen räumlichen Umgebungen und psychischer Gesundheit bzw. Wohlbefinden detailliert zu untersuchen. Ein Beispiel hierfür ist das grossangelegte Projekt Urban Mind: Die Arbeiten von \textcite{bakolisUrbanMindUsing2018}, \textcite{bergouMentalHealthBenefits2022} und \textcite{hammoudSmartphonebasedEcologicalMomentary2024} nutzen diesen Ansatz bzw. dessen Methodik, um insbesondere den Einfluss von Grün- und Stadträumen auf die psychische Gesundheit zu analysieren. Diese Studien prägen den aktuellen Forschungstrend, situative affektive Reaktionen systematisch in Bezug auf räumliche Kontexte zu untersuchen.

Parallel dazu existiert eine umfangreiche Forschungsliteratur zur Intersektionalität und deren räumlichen Implikationen, massgeblich geprägt durch feministische und kritische Perspektiven \parencite[vgl.][]{crenshawMappingMarginsIntersectionality1991, rodo-de-zarateDevelopingGeographiesIntersectionality2014, rodo-de-zarateYoungLesbiansNegotiating2015, rodo-de-zarateIntersectionalityFeministGeographies2018}. Diese Arbeiten verdeutlichen, wie unterschiedliche soziale Kategorien wie Geschlecht, Klasse oder ethnische Zugehörigkeit in räumlichen Kontexten miteinander verwoben sind und Ungleichheiten erzeugen oder verstärken können. Insbesondere methodische Innovationen wie die Relief Maps \parencite{rodo-de-zarateDevelopingGeographiesIntersectionality2014} erlauben eine Visualisierung und Analyse dieser komplexen Wechselwirkungen.

Diese Arbeit verbindet die beiden Perspektiven: Sie nutzt die methodischen Möglichkeiten der smartphone-basierten Echtzeit-Datenerfassung, wie sie in der \gls{esm}/\gls{ema}-Forschung etabliert wurden, verknüpft diese jedoch explizit mit der intersektionalen Ungleichheitsanalyse. Der Fokus verschiebt sich dabei von einer rein 'ökologischen' Betrachtung oder einer engen Definition von 'psychischer Gesundheit' hin zu einer Untersuchung des \emph{situativen affektiven Wohlbefindens} in vielfältigen alltäglichen Umgebungen. Es wird untersucht, wie sich intersektionale Positionierungen konkret auf dieses situative Wohlbefinden auswirken. Ziel dieser Bachelorarbeit ist es, im Rahmen einer explorativen Pilotstudie das Potenzial dieser methodischen Adaption und Verknüpfung auszuloten: Es soll geprüft werden, ob und wie dieser Ansatz an der Schnittstelle von feministischer Sozial- und Kulturgeographie, Intersektionalitätsforschung und der Analyse digital erhobener Alltagsdaten erste Einblicke und Hypothesen generieren kann.

Die persönliche Motivation für diese Arbeit ergibt sich aus dem Wunsch, mit eigens entwickelten digitalen Werkzeugen neue Einblicke in Fragen sozialer Gerechtigkeit und Wohlbefinden im Alltag zu ermöglichen. Perspektivisch könnte der hier erprobte methodische Ansatz in weiterführenden Arbeiten dazu dienen, sozialräumliche Fragestellungen mit Themen wie Klimaanpassung oder -mitigation zu verbinden. Eine solche Verknüpfung könnte beispielsweise für den Berner Kontext relevant sein, etwa für die Forschung zur Stadthitze \parencite[vgl.][]{burgerModellingSpatialPattern2021} und für Projekte wie dem \emph{Bernometer}\footnote{\href{https://bernometer.unibe.ch/bernometer/}{bernometer.unibe.ch}}, die mit detaillierten raumbezogenen Daten zum Wohlbefinden weiter ausgebaut werden könnten.Im Fokus steht dabei folgende Forschungsfrage:

\begin{quote}
\emph{Wie beeinflussen räumliche Umgebungen das momentane Wohlbefinden intersektional positionierter Personen im Alltag?}
\end{quote}

Dabei geht es explizit nicht um langfristige subjektive Wohlbefindenswerte, sondern um die im Alltag erlebten situativen, affektiven Reaktionen. Ziel der Analyse ist es, aus einer intersektionalen Perspektive zu untersuchen, unter welchen Bedingungen und an welchen Orten sich Menschen zugehörig oder fremd fühlen. Es soll also ergründet werden, wie soziale Positionierungen und räumliche Kontexte zusammenwirken und das momentane Gefühl der (Nicht-)Zugehörigkeit beeinflussen. Als analytischer Ansatz zur quantitativen Untersuchung der zugrundeliegenden intersektionalen Muster dient \gls{maihda} nach \textcite{grossModellingIntersectionalityQuantitative2023}.

Zur Erhebung der für diese Arbeit notwendigen Daten wurde das digitale Werkzeug \emph{InterMind}\footnote{\href{https://intermind.ch/app}{intermind.ch/app}} entwickelt. Diese App ermöglicht es, Teilnehmende über einen festgelegten Zeitraum hinweg wiederholt zu befragen und ihre Antworten zusammen mit georeferenzierten Informationen in Echtzeit zu erfassen und anonymisiert zu speichern. Für die vorliegende Untersuchung werden im Rahmen einer Pilotstudie mit Studierenden der Universität Bern erste explorative Daten gesammelt. Die App selbst wurde bewusst Open-Source entwickelt, um eine flexible Anpassung an ähnliche Forschungskontexte zu ermöglichen und potenziell eine nachhaltige Infrastruktur für kontextualisierte Alltagsdaten zu bieten.


Der Aufbau der Arbeit gestaltet sich wie folgt: \Cref{sec:theoretischer_rahmen} entfaltet den theoretischen Rahmen und behandelt (i) Intersektionalität, (ii) räumliches Wohlbefinden sowie (iii) methodische Grundlagen der \gls{ema}- bzw.\ \gls{gema}-Forschung. \Cref{sec:methodik} beschreibt das methodische Vorgehen, insbesondere die Entwicklung der App, das Studiendesign und den analytischen Zugriff mittels des \gls{maihda}-Ansatzes. In \Cref{sec:ergebnisse} werden die zentralen empirischen Ergebnisse zur Beziehung von Wohlbefinden, Raum und intersektionaler Positionierung vorgestellt. Anschliessend diskutiert \Cref{sec:diskussion} diese Befunde, verortet sie im aktuellen Forschungsstand und leitet Implikationen für zukünftige Studien ab. Ein abschliessendes Fazit sowie ein Ausblick auf Weiterentwicklungspotenziale folgen in \Cref{sec:fazit}.


Diese Arbeit versteht sich als explorativer Beitrag, der methodische Innovationen mit gesellschaftlich relevanten Fragestellungen verbindet. Sie erhebt nicht den Anspruch auf allgemeine Repräsentativität, sondern zielt darauf ab, erste Hypothesen und methodische Potenziale für zukünftige intersektionale Analysen des momentanen Wohlbefindens in alltäglichen Lebensräumen aufzuzeigen.