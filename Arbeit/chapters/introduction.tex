% LTeX: language=de-CH
\chapter{Einleitung} \label{sec:einleitung}

Eine Parkbank am Rand eines kleinen Platzes. Beton unter den Füssen, ein Baum wirft etwas Schatten, Kinderstimmen im Hintergrund. Der Ort löst nicht bei allen dasselbe aus: Für manche bedeutet er Ruhe, für andere Anspannung oder Distanz. Solche situativen Emotionen entstehen im Zusammenspiel materieller Eigenschaften (Licht, Geräusche, Gerüche, Temperatur), sozialer Dynamiken und individueller Erfahrungen -- und sie sind durch soziale Positionierungen mitgeprägt. In dieser Arbeit rücke ich dieses situative, kontextgebundene Erleben -- \emph{affektives Wohlbefinden} -- in den Mittelpunkt.

Ich verorte die Arbeit in einer intersektionalen Perspektive, weil sie Unterschiede im Erleben nicht als Summe einzelner Merkmale versteht, sondern als Ergebnis verschränkter, machtvoll strukturierter Positionierungen. Damit rücke ich Konstellationen in den Blick, in denen Kategorien wie Geschlecht, Klasse oder Herkunft im räumlichen Kontext zusammenwirken und Erfahrungen prägen \parencite{crenshawMappingMarginsIntersectionality1991, valentineTheorizingResearchingIntersectionality2007}. In der Geographie zeigen feministische und kritisch-soziale Ansätze, wie sich solche Überschneidungen in alltäglichen Situationen materiell niederschlagen und soziale Ungleichheiten (re)produzieren \parencite{rodo-de-zarateDevelopingGeographiesIntersectionality2014, rodo-de-zarateIntersectionalityFeministGeographies2018, rodo-de-zarateIntersectionalitySpatialityEmotions2023}. Für diese Arbeit heisst das: Ich betrachte affektives Wohlbefinden als situatives, kontextgebundenes Erleben, das in der Wechselwirkung von räumlicher Materialität, sozialen Dynamiken und sozialer Positionierung entsteht -- und genau in dieser Verschränkung analysiert werden muss.

Zweitens orientiere ich mich an den \emph{affective geographies}. Sie begreifen Emotionen als verkörperte, relationale und räumlich situierte Phänomene, die zirkulieren und Zugehörigkeiten wie Distanzen herstellen \parencite{ahmedAffectiveEconomies2004}. Konzepte affektiver Atmosphären heben dabei die Spannungen von Materialität und Ideation, Bestimmtheit und Unbestimmtheit hervor und machen erfahrbar, wie Orte im Zusammenspiel von Körpern, Dingen und Sinneseindrücken wirken \parencite{andersonAffectiveAtmospheres2009}. Dieser Zugang ist für mein Vorhaben zentral, weil er den Gegenstand -- situativ-affektives Wohlbefinden -- präzise fasst, ohne ihn auf individuelle Präferenzen oder rein kognitive Bewertungen zu verkürzen, und zugleich eine explizit räumliche Analyse nahelegt \parencite{rodo-de-zarateIntersectionalitySpatialityEmotions2023}.

Drittens nehme ich eine kritisch-digitale Perspektive ein. Aus \emph{Data Feminism} folgt, dass nicht nur \emph{was} erhoben wird, sondern auch \emph{wie} und \emph{womit} eine forschungswesentliche, politische und ethische Entscheidung ist; Transparenz, Partizipation, Kontextsensibilität und Machtkritik werden damit zu Qualitätskriterien \parencite{dignazioDataFeminism2020}. Feministische Digitalgeographien zeigen zugleich, wie digitale Praktiken und Infrastrukturen ungleichheitsrelevante Einschreibungen tragen -- und warum reflektierte, situierte Datenerhebung erforderlich ist \parencite{elwoodFeministDigitalGeographies2018}. Debatten um digitale Souveränität verdeutlichen schliesslich, dass Kontrolle, Nachvollziehbarkeit und Gestaltbarkeit von Dateninfrastrukturen umkämpft sind und nicht allein technischen, sondern auch räumlich-politischen Logiken folgen \parencite{glaszeContestedSpatialitiesDigital2023}. Vor diesem Hintergrund setze ich auf eine offene, überprüfbare Infrastruktur, die Datensparsamkeit, klare Datenflüsse und Anpassbarkeit priorisiert.

Zusammengenommen bilden diese drei Perspektiven den Rahmen für mein Vorgehen. Intersektionalität lenkt den Blick auf Differenzen und Überschneidungen sozialer Positionierungen, die \emph{affective geographies} begründen, warum ich situative, räumlich gebundene Emotionen ins Zentrum rücke, und die kritisch-digitale Sicht definiert die Anforderungen an Erhebung und Verarbeitung -- Offenheit, Nachvollziehbarkeit und Datensparsamkeit. Daraus ergibt sich ein Forschungsdesign, das wiederholte, kontextnahe Erhebungen mit einer intersektionalen Auswertung auf einer offenen digitalen Grundlage verbindet. Entlang dieses Rahmens formuliere ich im Folgenden die Leitfrage sowie die drei dazugehörigen Teilfragen.



Vor diesem Hintergrund leite ich die Forschungsfragen ab. Sie rahmen die Arbeit inhaltlich, methodisch und infrastrukturell und bilden den roten Faden der folgenden Kapitel.

\begin{quote}
\emph{Wie lässt sich der Einfluss räumlicher Umgebungen auf das affektive Wohlbefinden intersektional positionierter Personen erfassen und analysieren?}
\end{quote}

\begin{enumerate}
    \item Welche Merkmale muss ein Erhebungsansatz aufweisen, um affektives Wohlbefinden intersektional positionierter Personen gemeinsam mit relevanten Kontextmerkmalen wiederholt in situ zu erfassen?
    \item Welche Anforderungen ergeben sich aus einer kritisch-digitalen Perspektive an ein Werkzeug, das solche Erhebungen ermöglicht, und wie realisiere ich diese Anforderungen in einer konkreten Umsetzung?
    \item Wie geeignet sind die in einer Pilotstudie erhobenen Daten für eine intersektionale Mehrebenenmodellierung?
\end{enumerate}

Methodisch verorte ich die Arbeit im Feld wiederholter, kontextgebundener Befragungen: Ich diskutiere, wie \gls[noindex]{ema}/\gls[noindex]{gema} die unmittelbare Erfassung situativer Emotionen ermöglichen und wie sich diese Logik mit intersektionalen Auswertungsansätzen verbinden lässt. Analytisch prüfe ich, ob die in der Pilotierung gewonnenen Daten die Grundvoraussetzungen einer intersektionalen Mehrebenenanalyse erfüllen.

Infrastrukturell entwickle ich mit \gls[noindex]{intermind}\footnote{\href{https://intermind.ch/app}{intermind.ch/app}} eine offene \gls[noindex]{gema}-Infrastruktur. Die App befragt Teilnehmende über einen festgelegten Zeitraum hinweg wiederholt und erfasst Antworten zusammen mit standortbezogenen Kontextinformationen in Echtzeit; alle Daten werden anonymisiert gespeichert. Die quelloffene Auslegung ermöglicht Anpassungen an andere Forschungskontexte und schafft eine Grundlage für eine transparente, langfristig nutzbare Infrastruktur zur Erhebung kontextualisierter Alltagsdaten.

Aus den Forschungsfragen leite ich die Anforderungen an den Fragebogen ab und entwickle darauf aufbauend eine kompakte Erhebung. Der Fragebogen bildet das thematische Feld präzise ab, hält die Teilnahmebelastung gering und schafft die Voraussetzung für eine intersektionale Auswertung.

In einer explorativen Pilotstudie erprobe ich das Zusammenspiel aus Infrastruktur, Erhebungsdesign und Auswertungspfad. Dabei prüfe ich, ob die erhobenen Daten die erforderliche Differenzierung und Qualität für eine intersektionale Mehrebenenanalyse aufweisen und wo die Grenzen des Ansatzes liegen. Die Pilotierung dient dem methodischen Machbarkeitsnachweis; inhaltliche Effektschätzungen und Generalisierungen sind nicht Ziel dieser Arbeit.

Um die Forschungsfragen zu beantworten, führe ich die Lesenden zunächst in \Cref{sec:theoretischer_rahmen} in zentrale Begriffe und Konzepte ein -- darunter (i) Intersektionalität als Analyseinstrument, (ii) affektives Wohlbefinden als räumlich situierte Erfahrung und (iii) kritisch-digitale Perspektiven auf Forschungsinfrastrukturen. Darauf aufbauend erläutere ich in \Cref{sec:methodik} das methodische Vorgehen: von den theoretischen Grundlagen wiederholter Befragung über konzeptionelle Entscheidungen bis hin zur Einordnung des gewählten Zugangs im Vergleich zu bestehenden Instrumenten. Die Entwicklung der App \gls[noindex]{intermind} und die zugrunde liegenden Anforderungen thematisiere ich in \Cref{sec:entwicklung_app}, bevor ich in \Cref{sec:fragebogenentwicklung} die Konstruktion des Fragebogens darstelle. Anschliessend widme ich mich in \Cref{sec:pilotstudie} der Durchführung und Analyse der explorativen Pilotstudie. Den Abschluss bildet \Cref{sec:diskussion}, in dem ich zentrale Befunde reflektiere, methodische Implikationen diskutiere und Perspektiven für künftige Forschung skizziere.

Mit dieser Arbeit leiste ich einen explorativen Beitrag, der methodische Innovationen mit gesellschaftlich relevanten Fragestellungen verbindet. Ich erhebe keinen Anspruch auf allgemeine Repräsentativität; im Zentrum steht, methodische Potenziale und erste Ansätze für zukünftige intersektionale Analysen des situativen Wohlbefindens in alltäglichen Umgebungen aufzuzeigen.
