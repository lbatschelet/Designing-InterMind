\section{Kontextspezifisch und alltagstauglich -- Entwicklung des Fragebogens}
\label{sec:fragebogenentwicklung}

Im Zentrum der vorliegenden Arbeit steht die Frage, \emph{wie räumliche Umgebungen das momentane Wohlbefinden intersektional positionierter Personen im Alltag beeinflussen}. Der entwickelte Fragebogen bildet hierfür das zentrale methodische Instrument, unabhängig von der technischen Umsetzung durch die App. Gleichzeitig diente die Fragebogenentwicklung auch dazu, die Flexibilität und Tauglichkeit der App zu validieren.

Konkret bestand die Herausforderung darin, zwei zentrale Aspekte abzudecken: einerseits grundlegende Merkmale zur Charakterisierung der Stichprobe (Baseline-Modul), andererseits das situative, affektive Wohlbefinden im unmittelbaren räumlichen und sozialen Kontext zu erfassen (\gls{ema}-Modul). Gleichzeitig sollte die Befragung möglichst kurz gehalten werden, um die Akzeptanz und Teilnahmebereitschaft langfristig zu sichern. Als Zielvorgabe wurde festgelegt, dass die Baseline maximal zehn Minuten, die wiederholten situativen Erhebungen jeweils höchstens drei Minuten dauern sollten.

Ein zusätzliches methodisches Kriterium bildete die Mehrsprachigkeit des Fragebogens. Diese Entscheidung erfolgte, um möglichst viele Teilnehmende zu erreichen und den Zugang breit zu ermöglichen. Realisiert wurde der Fragebogen in Deutsch, Englisch und Französisch. Weitere Sprachen wären aus methodischer Sicht wünschenswert gewesen, wurden jedoch aus praktischen Gründen nicht umgesetzt.

Die Aufteilung des Fragebogens in ein einmaliges Baseline-Modul und eine wiederholte situative Erhebung folgt direkt aus den methodischen Anforderungen der Forschungsfrage. Die Baseline dient primär der Charakterisierung der Stichprobe, um später differenzierte intersektionale Analysen vornehmen zu können. Die wiederholten situativen Fragen wiederum erfassen das affektive Wohlbefinden in konkreten Alltagskontexten und bilden somit den eigentlichen Kern der empirischen Datenerhebung.

Der vollständige Fragebogen ist im \cref{app:appendix_fragebogen} zu finden.

\subsection{Kontext schaffen -- Einmalige Eingangsbefragung}

Die einmalige Baseline-Erhebung (Siehe \cref{tab:baseline-fragen}) zielte darauf ab, die sozialen Positionierungen der Teilnehmenden möglichst differenziert zu erfassen. Berücksichtigt wurden dabei Merkmale wie Alter, \gls{gender}, sexuelle Orientierung, Behinderung sowie soziale Klasse (\gls{class}) \parencite{bauerIntersectionalityQuantitativeResearch2021}. Besonders herausfordernd gestaltete sich jedoch die Erfassung von \gls[noindex]{race}: Im europäischen Kontext fehlt es an etablierten, nicht-essentialisierenden Kategorien, die diesen Aspekt angemessen abbilden, ohne problematische koloniale oder biologistische Zuschreibungen zu reproduzieren \parencite[\gls{vgl}][]{roigIntersectionalityEuropeDepoliticized2018}. Angesichts der fehlenden nicht-essentialisierenden, praktikablen Kategorien zur Erfassung rassifizierter Zugehörigkeiten im europäischen Kontext -- etwa im Vergleich zur US-amerikanischen Tradition der Selbstkategorisierung -- wurde pragmatisch lediglich erhoben, ob Teilnehmende aktuell in einem anderen Land leben als in jenem, in dem sie geboren wurden. Rückblickend erscheint diese Entscheidung ungenügend, da sie dem Anspruch einer intersektionalen Analyse nicht gerecht wird. Eine offene, selbstbezeichnungsbasierte Erhebung wäre geeigneter gewesen, um dieser komplexen sozialen Dimension Sichtbarkeit zu verleihen.

Die soziale Klasse (\gls{class}) wurde über eine Kombination mehrerer sozioökonomischer Indikatoren erfasst: höchster Bildungsabschluss, aktuelle Beschäftigungssituation, Haushaltseinkommen sowie Anzahl der Haushaltsmitglieder und deren Einkommensbeitrag. Diese Konstruktion basiert nicht auf klassischen Schemata wie EGP oder ESeC, da deren Operationalisierung mit standardisierten Berufen und sozialstrukturellen Kategorien eine detailliertere Erhebung erfordert hätte, die im Rahmen dieses Fragebogens nicht praktikabel gewesen wäre \parencite{bihagenSocialClassEmployment2010}. Stattdessen wurde eine pragmatische, mehrdimensionale Annäherung gewählt, die zentrale Aspekte sozialer Lage abbildet, ohne den Fragebogen übermässig zu verlängern.

Zur Erfassung bereits erfahrener Diskriminierungen wurde zusätzlich eine Multiple-Choice-Frage eingesetzt, die sowohl das Vorhandensein als auch die Art der Diskriminierung erfasst.

Die Auswahl dieser Merkmale folgte einem pragmatischen Abwägungsprozess zwischen analytischer Relevanz, Umsetzbarkeit und dem Ziel, die Befragung möglichst kurz und zugänglich zu halten.

\subsection{Vom Ort zur Emotion -- situativ befragen}

Der situative Teil des Fragebogens (Siehe \cref{tab:wiederholte-fragen}) diente dazu, die unmittelbare räumliche und soziale Umgebung der Befragten systematisch zu erfassen, um deren Einfluss auf das momentane affektive Wohlbefinden abzubilden. Zur Beschreibung der räumlichen Situation wurde zunächst unterschieden, ob sich die Teilnehmenden drinnen oder draussen befanden, gefolgt von einer näheren Kategorisierung des Ortes (z.\,B. Zuhause, Arbeitsplatz, Café, Park, öffentlicher Verkehr). Weiter abgefragt wurden zentrale Umgebungsmerkmale: Geräuschkulisse (Lautstärke), Sichtbarkeit von Pflanzen oder Bäumen, Lebhaftigkeit sowie die subjektiv wahrgenommene Qualität des Ortes . Die soziale Umgebung wurde erfasst, indem Teilnehmende angaben, welche Personen sich in ihrer Nähe befanden und in welchem Verhältnis sie zu diesen standen.

Inspiration und teilweise Vorlage dieser Items bot die Urban-Mind-Studie \parencite{bakolisUrbanMindUsing2018}, die ebenfalls auf eine kurze, situative Beschreibung der unmittelbaren Umgebung setzt. Standardisierte Skalen zur subjektiven Umgebungsqualität (z.\,B. \acrfull{peqi} \parencite{bonaiutoPerceivedResidentialEnvironment2015}, \acrfull{news} \parencite{saelensNeighborhoodEnvironmentWalkability2018}) wurden als Referenzpunkt berücksichtigt, waren jedoch aufgrund ihrer Länge und Komplexität für die wiederholten Befragungen nicht praktikabel. Die gewählte kompakte Form stellt somit einen bewussten methodischen Kompromiss zwischen wissenschaftlicher Tiefe und praktischer Umsetzbarkeit dar.

Die Operationalisierung des situativen affektiven Wohlbefindens stellte eine besondere Herausforderung dar, da bestehende standardisierte Instrumente (z.\,B. PANAS \parencite{yountMeasuringMoodComparison2023}, WHO-5 \parencite{toppWHO5WellBeingIndex2015}, WEMWBS \parencite{tennantWarwickEdinburghMentalWellbeing2007}) primär für mittel- und langfristige Einschätzungen konzipiert sind und zudem mit zu vielen Items eine schnelle situative Erhebung erschweren würden. Aus diesem Grund erfolgte eine eigenständige, jedoch teilweise intuitive Auswahl von fünf zentralen Dimensionen des Wohlbefindens: generelles Wohlbefinden, Zufriedenheit, Anspannung, Energie und Zugehörigkeit.

Diese Items wurden mithilfe linearer Slider-Skalen operationalisiert, was eine schnelle Bearbeitung und höhere Flexibilität bei der Antwortermöglichung sicherstellen sollte. Allerdings bringt die Slider-Methode auch methodische Herausforderungen mit sich, insbesondere bezüglich der Interpretation und Vergleichbarkeit der Ergebnisse.

Ein zentrales Merkmal des situativen Fragebogens besteht zudem im Versuch, intersektionale Dimensionen auch auf der Ebene der aktuellen räumlichen Erfahrung sichtbar zu machen. Zwei Items wurden dabei explizit so formuliert, dass sie situative Einschätzungen der eigenen sozialen Verortung gegenüber einer als dominant wahrgenommenen Mehrheit erfassen. Zum einen wurde gefragt, ob das eigene Gefühl von Zugehörigkeit oder Fremdheit an einem Ort mit der wahrgenommenen gesellschaftlichen Positionierung in Verbindung steht. Zum anderen konnten Teilnehmende angeben, in welchen Merkmalen sie sich im Vergleich zu den anwesenden Personen als zugehörig oder nicht zugehörig empfanden.

Diese Fragen stellen eine bewusste methodische Erweiterung dar, um nicht nur strukturelle Positionierungen (wie in der Baseline) zu erfassen, sondern auch situative Wechselwirkungen zwischen Raum, Wahrnehmung und sozialer Differenz. Durch diesen intersektionalen Zugriff auf situative Erfahrungen wird die Reduktion komplexer Ungleichheitsverhältnisse auf rein numerische Merkmale gezielt unterlaufen. Ergänzt wurden diese Fragen durch zwei offene Fragen, in denen Teilnehmende weitere Gründe für ihr situatives Wohlbefinden oder Unwohlsein benennen konnten. Diese qualitativen Elemente ermöglichen es, affektive, kontextgebundene und nicht-standardisierte Faktoren sichtbar zu machen, die durch geschlossene Fragen nicht erfasst werden können.

\subsection{Klar, verständlich, iterativ -- Der Weg zum finalen Fragebogen}

Die sprachliche Gestaltung der Fragebogen-Items stellte im Entwicklungsprozess eine zentrale methodische Herausforderung dar. Ziel war es, die Befragung möglichst zugänglich, verständlich und gleichzeitig inhaltlich präzise zu gestalten. Da die Befragung explizit auf eine intersektionale Analyse abzielt, wurde besonderer Wert darauf gelegt, die sprachliche Zugänglichkeit möglichst breit zu gewährleisten. Folglich wurde der Fragebogen bewusst mehrsprachig konzipiert und auf Deutsch, Englisch sowie Französisch umgesetzt. Weitere Sprachversionen wären zwar aus Sicht der intersektionalen Zugänglichkeit wünschenswert gewesen, scheiterten jedoch am hohen Aufwand für qualitativ hochwertige und inhaltlich konsistente Übersetzungen.

Ein grundsätzliches Anliegen war eine möglichst direkte, adressierende Sprache in der \enquote{Du}-Form, um einen niederschwelligen Zugang zur Befragung zu fördern und hierarchische Distanz zwischen Forschenden und Teilnehmenden zu reduzieren. Gleichzeitig mussten die Formulierungen prägnant, alltagsnah und schnell erfassbar sein, da insbesondere die situativen Erhebungen kurz gehalten werden sollten. Hier ergab sich ein methodischer Balanceakt: Einerseits sollte die Befragung leicht verständlich bleiben, andererseits mussten komplexe Konzepte in zugänglicher Sprache operationalisiert werden. So wurde \gls{bspw} das Konzept der \gls{intersektionalitaet} im Einführungsteil des Fragebogens erläutert, danach jedoch bewusst vermieden, um unnötige Barrieren zu reduzieren. Stattdessen wurden alternative Formulierungen wie \enquote{persönliche Merkmale} verwendet, die jedoch teilweise inhaltliche Unschärfen mit sich brachten. 

Besonders deutlich wurde diese Herausforderung im Umgang mit dem Konzept \gls[noindex]{race}. Gerade im deutschsprachigen Kontext existieren hier nur schwer geeignete Begrifflichkeiten: Formulierungen wie \enquote{Rasse} oder \enquote{Ethnizität} sind entweder sprachlich ungebräuchlich, problematisch oder stark mit kolonialen und biologistischen Zuschreibungen assoziiert \parencite[\gls{vgl}][]{roigIntersectionalityEuropeDepoliticized2018}. Alternativ verwendete Begriffe wie \enquote{Herkunft} oder \enquote{Aussehen} sind wiederum unpräzise und greifen die Dimension rassifizierter Diskriminierung nur unvollständig auf.

Die Übersetzung der Items erfolgte nicht wörtlich, sondern sinngemäss, wobei insbesondere bei affektiven Zustandsbeschreibungen semantische Abstimmungen zwischen den Sprachversionen vorgenommen wurden. Dabei wurden auch kulturelle Unterschiede in der Alltagsverwendung bestimmter Begriffe berücksichtigt. Dieser Kompromiss ermöglichte trotz beschränkter Ressourcen eine hinreichend konsistente Mehrsprachigkeit, brachte jedoch gewisse methodische Limitierungen hinsichtlich der Vergleichbarkeit der Sprachversionen mit sich.

Der beschriebene Sprach- und Übersetzungsprozess war eingebettet in einen breiteren, iterativen Entwicklungsprozess, der sowohl auf der Analyse bestehender Literatur als auch auf kontinuierlichem Feedback basierte. Ausgangspunkt bildeten Studien wie die Urban-Mind-Studie \parencite{bakolisUrbanMindUsing2018}, deren methodische Ansätze zur Erhebung situativen Wohlbefindens und räumlicher Wahrnehmung als Orientierung dienten. Diese Ansätze wurden jedoch um eigene Überlegungen zur intersektionalen Erhebung sozialer Positionierung ergänzt und in mehreren Durchläufen kritisch reflektiert.

Während der Testphase der App Entwicklung (Siehe \cref{sec:app_entwicklung_feldtest}) sind ebenfalls zahlreiche kleinere Rückmeldungen zu Formulierungen und sprachlichen Feinheiten eingegangen. Diese wurden laufend eingearbeitet. Ebenfalls wurde der Fragebogen mit der betreuenden Dozentin durchgegangen und danach ebenfalls nochmals entsprechend überarbeitet.
Ein wiederkehrendes methodisches Kriterium bei diesen Diskussionen war stets, die Belastung für Teilnehmende so gering wie möglich zu halten, ohne zentrale Aspekte der Forschungsfrage zu vernachlässigen. Durch den iterativen Ansatz konnte die Perspektive potenzieller Befragter frühzeitig einbezogen werden, was zu einer praxisnahen Optimierung der Items und des Fragebogenaufbaus führte.

Im Rückblick lassen sich einige Punkte identifizieren, die bei einer erneuten Durchführung anders gestaltet werden könnten. Die Auswahl der Wohlbefindensdimensionen erfolgte nicht vollständig theoriegeleitet; insbesondere das Item „generelles Wohlbefinden“ wirkt im Nachhinein wenig trennscharf. Eine stärkere konzeptuelle Fundierung der Items wäre sinnvoll, um die Aussagekraft einzelner Skalen zu erhöhen.

Der Verzicht auf standardisierte Skalen ermöglichte eine kompakte Erhebung, schränkt jedoch Vergleichbarkeit und Validität ein. Eine modularisierte Integration validierter Instrumente -- etwa in gekürzter Form -- könnte eine tragfähige Alternative darstellen.

Die mehrsprachige Umsetzung war aus methodischer Sicht wichtig, konnte jedoch mangels Ressourcen nicht vollständig abgesichert werden. Eine zusätzliche Validierung durch muttersprachliche Expert:innen wäre wünschenswert gewesen, um semantische Konsistenz über Sprachversionen hinweg besser zu gewährleisten.

Insgesamt zeigen sich an mehreren Stellen Stellschrauben für eine künftige Weiterentwicklung -- etwa durch eine engere theoretische Anbindung, gezielte Pretests oder eine systematischere Überprüfung von Übersetzungen und Antwortformaten. Gleichzeitig hat sich der gewählte Zugang als praktikabel und kontextsensibel erwiesen, insbesondere im Hinblick auf Zugänglichkeit und situative Anschlussfähigkeit.



% Der Fragebogen besteht aus zwei zentralen Elementen:

% \paragraph{Baseline-Befragung (einmalig):}
% Hier wurden grundlegende demografische Variablen (Alter, Geschlecht, sexuelle Orientierung, Behinderung, sozioökonomischer Hintergrund) erhoben, um soziale Positionierungen für spätere intersektionale Analysen verfügbar zu machen. Fragen wurden bewusst offen oder mit freier Spezifikation gestaltet, um Normativität in den Antwortoptionen zu vermeiden und soziale Positionierungen differenziert erfassen zu können.

% \paragraph{Situative Befragung (wiederholt):}
% Der wiederholt eingesetzte Fragebogen besteht aus kurzen situativen Erhebungen, die mittels EMA durchgeführt wurden. Teilnehmer:innen wurden regelmässig aufgefordert, Fragen zu ihrem momentanen Wohlbefinden (z.\,B. empfundene Sicherheit, Zugehörigkeit, Entspannung) sowie zur aktuellen räumlichen und sozialen Umgebung zu beantworten. Hierbei kamen insbesondere Slider-Fragen zum Einsatz, um eine kontinuierliche Bewertung zu ermöglichen. Ergänzend wurde ein optionales Freitextfeld angeboten, das qualitative Kontextinformationen und subjektive Reflexionen zuliess.

% Der Begriff \emph{Operationalisierung} bezeichnet hierbei die konkrete Umsetzung theoretischer Konstrukte in messbare Indikatoren. Wohlbefinden wurde operationalisiert über sieben zentrale Dimensionen (u.\,a. Entspannung, Sicherheit, Zugehörigkeit), die in der Literatur als relevant identifiziert wurden. Räumliche und soziale Kontexte wurden mit Items operationalisiert, die beispielsweise aus der Forschung von Bakolis et al. (2018) übernommen und für die vorliegende Studie adaptiert wurden.

% \subsection{Theoretischer Hintergrund}

% Die Erhebung intersektionalen Wohlbefindens im Stadtraum erfordert eine methodische Annäherung, die sowohl soziale Machtverhältnisse als auch die situative Kontextabhängigkeit psychischen Erlebens berücksichtigt. Klassische Operationalisierungen von Wohlbefinden übersehen häufig diese Dimensionen und tendieren dazu, soziale Kategorien wie Geschlecht, Ethnizität oder Behinderung als additive Prädiktoren zu behandeln \parencite{bauerIntersectionalityQuantitativeResearch2021}. Um der relationalen Natur von Intersektionalität gerecht zu werden, muss die Konstruktion des Fragebogens vermeiden, diese Kategorien separat und unabhängig voneinander abzufragen. Stattdessen werden kombinierte soziale Positionierungen erhoben, um Muster intersektionaler Ungleichheit durch statistische Analyse sichtbar zu machen.

% Zugleich zeigt sich in der Literatur zum Wohlbefinden ein Mangel an standardisierten, kontextsensiblen Erhebungsinstrumenten \parencite{bautistaWhatWellbeingScoping2023}. Besonders dynamische und ortsgebundene Einflüsse auf das mentale Erleben, wie sie durch Alltagssituationen in unterschiedlichen Umgebungen entstehen, sind mit klassischen Befragungsdesigns schwer zu erfassen. Die Methode des \emph{Ecological Momentary Assessment} (EMA), bei der Personen mehrmals täglich kurze Selbstauskünfte im Alltag abgeben, bietet hier eine geeignete Alternative. Sie erlaubt die Erfassung momentanen Wohlbefindens in situ und in realen Kontexten, was sowohl die ökologische Validität als auch die Erfassbarkeit situativer Dynamiken erhöht \parencite{bakolisUrbanMindUsing2018}.

% \subsection{Struktur des Fragebogens}

% Der entwickelte Fragebogen gliedert sich in zwei komplementäre Abschnitte: eine einmalig zu erhebende \emph{Baseline-Komponente} und ein \emph{wiederholt} eingesetztes Modul zur situativen Erhebung.

% \paragraph{Baseline: Demografie und soziale Positionierung}

% Die Baseline-Komponente umfasst sowohl klassische demografische Variablen (z.\,B. Alter, Geschlecht, Einkommen) als auch kontextuelle Einschätzungen zum Wohn- und Arbeitsumfeld. Ziel ist es, einen Ausgangspunkt für die spätere Modellierung intersektionaler Positionen zu schaffen. Fragen nach Geschlecht, Behinderung oder sexueller Orientierung erfolgen entweder offen oder mit der Möglichkeit zur Spezifizierung, um normative Antwortoptionen zu vermeiden. Die Erhebung sozialer Positionen wird nicht direkt mit Fragen zur Wirkung auf Wohlbefinden verknüpft; diese werden analytisch modelliert, um dem Prinzip der Emergenz intersektionaler Effekte gerecht zu werden \parencite{bauerIntersectionalityQuantitativeResearch2021}.

% \paragraph{Wiederholte Erhebung: Kontextuelles Wohlbefinden}

% In der wiederholten Erhebung wird nach aktueller Aktivität, sozialem und physischem Kontext sowie subjektiver Kontextwahrnehmung gefragt. Diese Kontexteinschätzungen orientieren sich teilweise an der Urban Mind Studie \parencite{bakolisUrbanMindUsing2018}, etwa durch Items zu Sichtbarkeit von Natur, gefühlter Sicherheit oder Inklusion im aktuellen Umfeld.

% Das momentane Wohlbefinden wird mittels sieben Skalenitems erfasst, die zentrale Dimensionen wie Entspannung, Stress, Sicherheit, Zugehörigkeit und physisches Wohlbefinden abbilden. Alle Items verwenden eine 11-stufige Skala (0--10). Diese Form der Situationsbezogenheit erlaubt es, Wohlbefinden nicht als statische Eigenschaft, sondern als kontextsensitive Momentaufnahme zu analysieren -- eine methodische Umsetzung der Forderung nach Kontextualität in intersektionaler Forschung \parencite{rodo-de-zarateDevelopingGeographiesIntersectionality2014}.

% \paragraph{Optional: Freitextfeld}

% Zur Ergänzung der quantitativen Daten enthält jedes Wiederholungsmodul ein optionales Freitextfeld, in dem Teilnehmende beschreiben können, was ihr aktuelles Befinden beeinflusst. Diese qualitativen Daten ermöglichen es, mikrosoziale oder emotionale Einflüsse jenseits vorstrukturierter Kategorien zu erfassen.

% \subsection{Intersektionalität in der Analyse -- nicht im Item}

% Ein zentrales Designprinzip ist die Trennung von Datenerhebung und Analyseebene: Die Intersektionalität entsteht nicht durch direkte Fragen (z.\,B. \glqq Wie beeinflusst Ihr Geschlecht Ihr Wohlbefinden?\grqq), sondern durch die Verknüpfung kombinierter sozialer Positionen mit den erhobenen Kontext- und Befindlichkeitsdaten. Diese Praxis steht im Einklang mit Empfehlungen aus der systematischen Übersichtsarbeit von Bauer et al., die vor der Vereinfachung komplexer Machtverhältnisse durch additive Regressionsmodelle warnen \parencite{bauerIntersectionalityQuantitativeResearch2021}.

% \subsection{Methodologische Implikationen}

% Die erhobenen Daten eignen sich zur Analyse mittels Mehrebenenmodellen oder linearer Mixed Models, die individuelle Baseline-Merkmale und situative Kontexte in Beziehung setzen. Interaktionstermen zwischen sozialen Positionen (z.\,B. Geschlecht × Behinderung) und Kontextvariablen (z.\,B. öffentlich vs. privat) erlauben es, Unterschiede in der Wirkung des Kontexts auf das Wohlbefinden sichtbar zu machen.

% Zusätzlich erlaubt die georeferenzierte Erhebung eine Einbeziehung räumlicher Strukturen in die Analyse. Diese Kombination aus Kontext, Positionierung und Raum ist zentral für die Analyse intersektionaler Ungleichheiten im Stadtraum.

% \subsection{Zusammenfassung}

% Der entwickelte Fragebogen basiert auf dem Prinzip der \emph{analytischen Intersektionalität} und verbindet kontextuelle Situationsdiagnostik mit einem sensiblen, nicht-kategorisierenden Umgang mit sozialer Heterogenität. Die theoretische Grundlage aus intersektionaler Forschung und kritischer Gesundheitsgeographie wird durch eine methodisch stringente Implementierung in ein digitales Erhebungsinstrument überführt. Damit bietet der Fragebogen eine geeignete Grundlage, um intersektionales Wohlbefinden im Stadtraum empirisch zu untersuchen, ohne die Komplexität sozialer Machtverhältnisse zu vereinfachen.

