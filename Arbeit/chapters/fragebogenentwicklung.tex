\section*{Fragebogenentwicklung}

\subsection*{Theoretischer Hintergrund}

Die Erhebung intersektionalen Wohlbefindens im Stadtraum erfordert eine methodische Annäherung, die sowohl soziale Machtverhältnisse als auch die situative Kontextabhängigkeit psychischen Erlebens berücksichtigt. Klassische Operationalisierungen von Wohlbefinden übersehen häufig diese Dimensionen und tendieren dazu, soziale Kategorien wie Geschlecht, Ethnizität oder Behinderung als additive Prädiktoren zu behandeln \parencite{bauerIntersectionalityQuantitativeResearch2021}. Um der relationalen Natur von Intersektionalität gerecht zu werden, muss die Konstruktion des Fragebogens vermeiden, diese Kategorien separat und unabhängig voneinander abzufragen. Stattdessen werden kombinierte soziale Positionierungen erhoben, um Muster intersektionaler Ungleichheit durch statistische Analyse sichtbar zu machen.

Zugleich zeigt sich in der Literatur zum Wohlbefinden ein Mangel an standardisierten, kontextsensiblen Erhebungsinstrumenten \parencite{bautistaWhatWellbeingScoping2023}. Besonders dynamische und ortsgebundene Einflüsse auf das mentale Erleben, wie sie durch Alltagssituationen in unterschiedlichen Umgebungen entstehen, sind mit klassischen Befragungsdesigns schwer zu erfassen. Die Methode des \emph{Ecological Momentary Assessment} (EMA), bei der Personen mehrmals täglich kurze Selbstauskünfte im Alltag abgeben, bietet hier eine geeignete Alternative. Sie erlaubt die Erfassung momentanen Wohlbefindens in situ und in realen Kontexten, was sowohl die ökologische Validität als auch die Erfassbarkeit situativer Dynamiken erhöht \parencite{bakolisUrbanMindUsing2018}.

\subsection*{Struktur des Fragebogens}

Der entwickelte Fragebogen gliedert sich in zwei komplementäre Abschnitte: eine einmalig zu erhebende \emph{Baseline-Komponente} und ein \emph{wiederholt} eingesetztes Modul zur situativen Erhebung.

\paragraph{Baseline: Demografie und soziale Positionierung}

Die Baseline-Komponente umfasst sowohl klassische demografische Variablen (z.\,B. Alter, Geschlecht, Einkommen) als auch kontextuelle Einschätzungen zum Wohn- und Arbeitsumfeld. Ziel ist es, einen Ausgangspunkt für die spätere Modellierung intersektionaler Positionen zu schaffen. Fragen nach Geschlecht, Behinderung oder sexueller Orientierung erfolgen entweder offen oder mit der Möglichkeit zur Spezifizierung, um normative Antwortoptionen zu vermeiden. Die Erhebung sozialer Positionen wird nicht direkt mit Fragen zur Wirkung auf Wohlbefinden verknüpft; diese werden analytisch modelliert, um dem Prinzip der Emergenz intersektionaler Effekte gerecht zu werden \parencite{bauerIntersectionalityQuantitativeResearch2021}.

\paragraph{Wiederholte Erhebung: Kontextuelles Wohlbefinden}

In der wiederholten Erhebung wird nach aktueller Aktivität, sozialem und physischem Kontext sowie subjektiver Kontextwahrnehmung gefragt. Diese Kontexteinschätzungen orientieren sich teilweise an der Urban Mind Studie \parencite{bakolisUrbanMindUsing2018}, etwa durch Items zu Sichtbarkeit von Natur, gefühlter Sicherheit oder Inklusion im aktuellen Umfeld.

Das momentane Wohlbefinden wird mittels sieben Skalenitems erfasst, die zentrale Dimensionen wie Entspannung, Stress, Sicherheit, Zugehörigkeit und physisches Wohlbefinden abbilden. Alle Items verwenden eine 11-stufige Skala (0–10). Diese Form der Situationsbezogenheit erlaubt es, Wohlbefinden nicht als statische Eigenschaft, sondern als kontextsensitive Momentaufnahme zu analysieren – eine methodische Umsetzung der Forderung nach Kontextualität in intersektionaler Forschung \parencite{rodo-de-zarateDevelopingGeographiesIntersectionality2014}.

\paragraph{Optional: Freitextfeld}

Zur Ergänzung der quantitativen Daten enthält jedes Wiederholungsmodul ein optionales Freitextfeld, in dem Teilnehmende beschreiben können, was ihr aktuelles Befinden beeinflusst. Diese qualitativen Daten ermöglichen es, mikrosoziale oder emotionale Einflüsse jenseits vorstrukturierter Kategorien zu erfassen.

\subsection*{Intersektionalität in der Analyse – nicht im Item}

Ein zentrales Designprinzip ist die Trennung von Datenerhebung und Analyseebene: Die Intersektionalität entsteht nicht durch direkte Fragen (z.\,B. \glqq Wie beeinflusst Ihr Geschlecht Ihr Wohlbefinden?\grqq), sondern durch die Verknüpfung kombinierter sozialer Positionen mit den erhobenen Kontext- und Befindlichkeitsdaten. Diese Praxis steht im Einklang mit Empfehlungen aus der systematischen Übersichtsarbeit von Bauer et al., die vor der Vereinfachung komplexer Machtverhältnisse durch additive Regressionsmodelle warnen \parencite{bauerIntersectionalityQuantitativeResearch2021}.

\subsection*{Methodologische Implikationen}

Die erhobenen Daten eignen sich zur Analyse mittels Mehrebenenmodellen oder linearer Mixed Models, die individuelle Baseline-Merkmale und situative Kontexte in Beziehung setzen. Interaktionstermen zwischen sozialen Positionen (z.\,B. Geschlecht × Behinderung) und Kontextvariablen (z.\,B. öffentlich vs. privat) erlauben es, Unterschiede in der Wirkung des Kontexts auf das Wohlbefinden sichtbar zu machen.

Zusätzlich erlaubt die georeferenzierte Erhebung eine Einbeziehung räumlicher Strukturen in die Analyse. Diese Kombination aus Kontext, Positionierung und Raum ist zentral für die Analyse intersektionaler Ungleichheiten im Stadtraum.

\subsection*{Zusammenfassung}

Der entwickelte Fragebogen basiert auf dem Prinzip der \emph{analytischen Intersektionalität} und verbindet kontextuelle Situationsdiagnostik mit einem sensiblen, nicht-kategorisierenden Umgang mit sozialer Heterogenität. Die theoretische Grundlage aus intersektionaler Forschung und kritischer Gesundheitsgeographie wird durch eine methodisch stringente Implementierung in ein digitales Erhebungsinstrument überführt. Damit bietet der Fragebogen eine geeignete Grundlage, um intersektionales Wohlbefinden im Stadtraum empirisch zu untersuchen, ohne die Komplexität sozialer Machtverhältnisse zu vereinfachen.

