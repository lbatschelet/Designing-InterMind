% LTeX: language=de-CH

\chapter{Verflechtungen verstehen -- Begriffe und Konzepte} \label{sec:theoretischer_rahmen}

Dieses Kapitel legt den theoretischen Grundstein der Arbeit. Es führt in zentrale Begriffe und Konzepte ein, die das Erkenntnisinteresse leiten und das methodische Vorgehen rahmen. Ausgangspunkt ist die intersektionale Perspektive, die gesellschaftliche Unterschiede nicht isoliert, sondern in ihrer wechselseitigen Verflechtung analysiert. Im Anschluss wird das Konzept des affektiven Wohlbefindens als kontextabhängige, räumlich gebundene Erfahrung entfaltet. Ergänzt wird diese Analyse um eine digitale Perspektive, die fragt, wie Daten, digitale Infrastrukturen und technologische Gestaltungsprozesse gesellschaftliche Machtverhältnisse widerspiegeln und (re)produzieren. Zusammen bilden diese Perspektiven die Grundlage für ein Forschungsdesign, das soziale Positionierung, räumliche Kontexte, situative Erfahrungen und digitale Infrastrukturen in Beziehung setzt.

\section{Verwebte Unterschiede -- Intersektionalität als Analyseinstrument}

Gesellschaftliche Wirklichkeiten sind durchzogen von komplexen Ungleichheiten. Menschen erfahren soziale Benachteiligung selten entlang nur einer einzigen Achse -- vielmehr wirken verschiedene Differenzlinien wie \gls{race}, \gls{gender} oder \gls{class}\footnote{\gls{race}, \gls{gender} und \gls{class} werden in dieser Arbeit kursiv gesetzt, um auf ihre Bedeutung als gesellschaftlich konstruierte, aber wirkmächtige Kategorien hinzuweisen. \textit{race} verweist auf rassifizierende Zugehörigkeitszuschreibungen, die historisch gewachsen sind und soziale Ungleichheiten produzieren. \gls{gender} beschreibt die soziale Konstruktion von Geschlecht und verweist auf normative Vorstellungen von Weiblichkeit, Männlichkeit oder anderen Geschlechtsidentitäten. \gls{class} bezeichnet die soziale Konstruktion von sozialer Klasse und verweist auf normative Vorstellungen von Reichtum, Mittelklasse oder Armut. Die Begriffe werden im englischen Original verwendet, da adäquate deutsche Entsprechungen fehlen oder missverständlich sind \parencite[\gls{vgl}][]{hallRaceArticulationSocieties1980, butlerGenderTroubleFeminism1990}} häufig gleichzeitig und verstärken sich wechselseitig. Um diese Verflechtungen zu erfassen, bietet der intersektionale Ansatz einen theoretischen Rahmen, der Ungleichheitsverhältnisse nicht isoliert betrachtet, sondern ihre Überschneidungen und Wechselwirkungen in den Blick nimmt.

Geprägt wurde der Begriff der \gls{intersektionalitaet} von \textcite{crenshawMappingMarginsIntersectionality1991}, die auf die spezifischen Diskriminierungserfahrungen \glslink{schwarz}{Schwarzer}\footnote{„\gls{schwarz}“ wird in dieser Arbeit als politische Selbstbezeichnung \glslink{schwarz}{Schwarzer} Menschen mit grossem Anfangsbuchstaben verwendet. Der Begriff beschreibt keine biologische Eigenschaft, sondern eine soziale Positionierung im Kontext rassistischer Machtverhältnisse. Die Grossschreibung dient der Abgrenzung von äusserlichen Zuschreibungen \parencite{oguntoyeFarbeBekennenAfrodeutsche1986}.} Frauen aufmerksam machte. Sie argumentierte, dass bestehende feministische und antirassistische Theorien nicht ausreichten, um Mehrfachdiskriminierung zu erfassen, und entwickelte Intersektionalität als analytisches Instrument zur Beschreibung solcher überlagerten Ungleichheitsverhältnisse \parencite[\gls{vgl}][]{hancockWhenMultiplicationDoesnt2007}.

Ausgangspunkt dieser theoretischen Perspektive ist der  Black Feminist Thought, welcher unter anderen in den Arbeiten von \textcite{hooksAintWomanBlack1981}, \textcite{lordeSisterOutsiderEssays1984}, Kimberle~\textcite{crenshawMappingMarginsIntersectionality1991} und \textcite{collinsBlackFeministThought2002} ihren Ausdruck findet. Black Feminist Theory formulierte eine scharfe Kritik an traditionellen feministischen Ansätzen, denen vorgeworfen wurde, primär die Erfahrungen weisser, privilegierter Frauen ins Zentrum zu stellen und somit die Lebensrealitäten \glslink{schwarz}{Schwarzer} Frauen zu marginalisieren. \textcite{crenshawMappingMarginsIntersectionality1991} entwickelte das Konzept der Intersektionalität explizit als Reaktion auf die Unfähigkeit bestehender theoretischer Ansätze, die spezifischen Diskriminierungserfahrungen \glslink{schwarz}{Schwarzer} Frauen adäquat zu erfassen. Dabei verdeutlichte sie, dass Diskriminierung nicht als Summe einzelner, isolierter Erfahrungen verstanden werden könne, sondern als eigenständige Form sozialer Benachteiligung, die sich an der Überschneidung sozialer Kategorien wie \gls{race}, \gls{gender} und \gls{class} manifestiert.

Intersektionalität entwickelte sich somit nicht allein im akademischen Kontext, sondern ist stark verwurzelt in den politischen Kämpfen sozialer Bewegungen, insbesondere im Kontext feministischer, antirassistischer und antikapitalistischer Aktivismen der 1970er- und 1980er-Jahre \parencite{collinsBlackFeministThought2002}. Zentral für die theoretische Grundlage des intersektionalen Ansatzes ist die Anerkennung von Machtverhältnissen und sozialen Ungleichheiten als strukturell verankert und historisch bedingt. Gesellschaftliche Positionierungen werden als sozial konstruierte Kategorien verstanden, die immer in Verbindung mit bestehenden Machtsystemen wie Sexismus, Rassismus oder Klassismus betrachtet werden müssen. Audre Lorde und bell hooks betonten insbesondere die Rolle struktureller Unterdrückung und verdeutlichten, wie sich dominante Gesellschaftsstrukturen auf individueller Ebene reproduzieren und sich somit wechselseitig verstärken \parencite{collinsBlackFeministThought2002, hancockWhenMultiplicationDoesnt2007}.

Von der ursprünglich starken Fokussierung auf \textit{race} und \textit{gender} wurde das Konzept der Intersektionalität in den folgenden Jahrzehnten zunehmend erweitert und schliesst heute oft eine Vielzahl sozialer Positionierungen und Identitäten ein, darunter etwa Sexualität, Alter, Behinderung, Nationalität oder Religion \parencite{bauerIntersectionalityQuantitativeResearch2021, bowlegInvitedReflectionQuantifying2016}. Diese Erweiterung verdeutlicht die breite theoretische und empirische Anwendbarkeit von Intersektionalität als Analyseinstrument zur kritischen Untersuchung gesellschaftlicher Ungleichheiten und Diskriminierungserfahrungen. Intersektionalität hat sich somit nicht nur als theoretisches Konzept, sondern auch als methodische Grundlage etabliert, welche insbesondere in feministisch und sozialwissenschaftlich orientierten Diskursen verwendet wird, um die komplexen Wechselwirkungen gesellschaftlicher Machtverhältnisse zu analysieren.

\vspace{2em}

Die Anwendung intersektionaler Perspektiven auf räumliche Fragestellungen stellt eine zentrale Weiterentwicklung des ursprünglichen Konzepts der Intersektionalität dar. Seit den 2000er-Jahren etablierte sich eine eigenständige geographische Perspektive, die räumliche Kontextualität und situative Dimensionen sozialer Ungleichheiten explizit in den Mittelpunkt rückt \parencite{valentineTheorizingResearchingIntersectionality2007,rodo-de-zarateIntersectionalityFeministGeographies2018}.

Zentral für diesen Perspektivwechsel ist das Verständnis von Raum als gesellschaftlich produzierter Grösse. \textcite{lefebvreProductionLespace1974} argumentiert, dass Raum kein neutrales Behältnis ist, sondern als Produkt sozialer Praktiken und Beziehungen verstanden werden muss. Machtverhältnisse schreiben sich demnach in Raumstrukturen und Nutzungen ein und reproduzieren sich über diese. \textcite{foucaultEspacesAutres2004} erweitert diese Perspektive mit dem Konzept der Heterotopien: Räume spiegeln gesellschaftliche Normen nicht nur wider, sondern bieten auch die Möglichkeit ihrer Infragestellung und Verschiebung.

Auf dieser theoretischen Grundlage argumentiert \textcite{valentineTheorizingResearchingIntersectionality2007}, dass soziale Kategorien wie \textit{race}, \textit{gender} oder \textit{class} nicht unabhängig vom Raum wirken. Sie entfalten ihre Bedeutung erst im Zusammenspiel mit konkreten räumlichen Kontexten. Ungleichheiten sind somit nicht nur räumlich verteilt, sondern werden durch räumliche Anordnungen hervorgebracht und erfahrbar gemacht. Räume erzeugen je nach sozialer Positionierung unterschiedliche Bedeutungen, Zugänglichkeiten und emotionale Resonanzen -- etwa in Form von \textit{Safe Spaces} oder Zonen der Exklusion \parencite[\gls{vgl}][S.~548--549]{rodo-de-zarateIntersectionalityFeministGeographies2018}.

\textcite{mccallComplexityIntersectionality2005} unterscheidet drei methodische Zugänge zu Intersektionalität: Der \emph{interkategoriale} Ansatz vergleicht festgelegte soziale Kategorien miteinander, um deren Wechselwirkungen zu analysieren. Der \emph{intrakategoriale} Ansatz richtet den Blick auf Erfahrungen innerhalb einer einzelnen Kategorie, insbesondere dort, wo diese intern heterogen ist. Der \emph{antikategoriale} Ansatz hinterfragt die Stabilität und Nützlichkeit solcher Kategorien grundsätzlich. Diese Systematisierung wird auch in einzelnen geographischen Arbeiten aufgegriffen, um methodisch zu begründen, wie sich unterschiedliche Dimensionen sozialer Differenz in räumlichen Analysen verknüpfen lassen. McCall betont zudem, dass \gls{gender} nicht isoliert betrachtet werden kann, sondern als interdependente Kategorie zu verstehen ist, deren Wirkung nur im Zusammenspiel mit anderen Differenzachsen entsteht. Diese Wechselwirkungen sind zudem stets in spezifische räumliche und historische Kontexte eingebettet, die ihre Ausprägung und Bedeutung mitbestimmen.

Empirische Arbeiten in der Geographie operationalisieren diese theoretischen Ansätze auf unterschiedliche Weise: Beispielhaft nutzt \textcite{fensterRightGenderedCity2005} narrative, qualitativ-ethnographische Zugänge -- etwa Interviews --, um zu untersuchen, wie \gls{gender} und Raum zusammenwirken. \textcite{rodo-de-zarateDevelopingGeographiesIntersectionality2014} bringt partizipative Kartierungen und visuelle Instrumente wie die \emph{Relief Maps} ein; diese verbinden bewusst soziale Positionen, emotionale Dimensionen und Orte und visualisieren subjektive Erfahrungen räumlicher Ungleichheit. \textcite{mccallSpatialRoutesGender1998} verwendet quantitative, multilevel-statistische Analysen, um regionale Strukturen mit geschlechtsspezifischen Lohnunterschieden zu verbinden und räumliche Muster intersektionaler Disparitäten aufzudecken. 

Diese Vielfalt zeigt, dass Intersektionalität in der Geographie nicht nur theoretisch relevant ist, sondern auch methodisch greifbar wird -- über differenzierte Zugänge zur Analyse räumlicher Machtverhältnisse.


\vspace{2em}

Obwohl intersektionale Forschung historisch in qualitativen und aktivistischen Traditionen verankert ist, gewinnen quantitative Verfahren zunehmend an Relevanz, insbesondere in sozialpolitischen und raumplanerischen Kontexten \parencite{bauerIntersectionalityQuantitativeResearch2021}. Diese Verfahren bieten die Möglichkeit, strukturelle Muster intersektionaler Benachteiligung über grössere Stichproben sichtbar und empirisch überprüfbar zu machen.

Jedoch ist die Übertragung intersektionaler Theorien in quantitative Methoden mit erheblichen Herausforderungen verbunden. Zentral ist die Kritik, dass traditionelle statistische Verfahren soziale Kategorien oft eindimensional oder additiv behandeln, was der komplexen theoretischen Vorstellung intersektionaler Verschachtelungen nicht gerecht wird \parencite{hancockWhenMultiplicationDoesnt2007, bowlegInvitedReflectionQuantifying2016}. Insbesondere birgt die numerische Operationalisierung sozialer Identitäten die Gefahr, die Fluidität und Kontextabhängigkeit dieser Kategorien zu ignorieren und damit ungewollt jene komplexen Wechselwirkungen zu nivellieren, die intersektionale Ansätze ursprünglich sichtbar machen wollen \parencite{scottIntersectionalityQuantitativeMethods2017}.

Um diesen Herausforderungen zu begegnen, bedarf es einer reflexiven und kontextsensiblen Operationalisierung intersektionaler Kategorien. Dies beinhaltet, soziale Gruppen nicht als statische Entitäten zu behandeln, sondern ihre relationalen und kontextuellen Eigenschaften explizit zu berücksichtigen \parencite{rodo-de-zarateDevelopingGeographiesIntersectionality2014, websterCenteringSocialtechnicalRelations2021}.


\section{Gefühlte Orte -- Wohlbefinden als räumliche Erfahrung} % Titel überarbeiten da gefühl und nicht emotion

Momente des Wohlbefindens können sich innerhalb kürzester Zeit verändern -- sie entstehen im direkten Erleben und sind eng mit dem jeweiligen räumlichen und sozialen Kontext verknüpft. Affektives Wohlbefinden bezeichnet kurzfristige, situativ schwankende emotionale Zustände wie Zufriedenheit, Gelassenheit oder Anspannung. Im Gegensatz zu psychologischen Konzepten subjektiven Wohlbefindens, die oft langfristige Lebenszufriedenheit messen, fokussiert das Konzept des affektiven Wohlbefindens bewusst auf flüchtige und unmittelbar erlebte Emotionen.\footnote{In dieser Arbeit wird der Begriff \emph{Emotion(en)} verwendet, um zeitlich begrenzte, sozial und kulturell geformte, benennbare Zustände zu bezeichnen, die in spezifischen räumlichen Kontexten entstehen und bewusst wahrgenommen werden können \parencite{bondiIntroductionGeographysEmotional2006}. Der deutsche Begriff \enquote{Gefühl(e)} wird vermieden, da er unscharf ist und sowohl körperliche Empfindungen, Stimmungen als auch Emotionen meinen kann.} Diese sind stark vom aktuellen räumlichen und sozialen Kontext beeinflusst und reagieren besonders sensibel auf Veränderungen der unmittelbaren Umgebung \parencite{dodgeChallengeDefiningWellbeing2012}.

Innerhalb der Geographie gewann das Konzept des affektiven Wohlbefindens seit dem \emph{emotional turn} in den frühen 2000er-Jahren zunehmend an Bedeutung \parencite{hoSocialGeographyIII2024}. Dabei stehen Emotionen nicht mehr ausschliesslich als interne Zustände von Individuen im Vordergrund, sondern vielmehr deren Wechselwirkungen mit der räumlichen Umwelt. Orte werden aus dieser Perspektive nicht als passive Kulissen menschlichen Erlebens begriffen, sondern als aktive Bestandteile emotionaler Prozesse, die das Wohlbefinden direkt mitgestalten.

\textcite{ahmedAffectiveEconomies2004} argumentiert, dass Emotionen an Körper, Dinge und Orte \enquote{haften} und dadurch Zugehörigkeiten und Abgrenzungen affektiv stabilisieren. Diese Zirkulation ist historisch und machtvoll situiert: Rassistische und sexistische Bedeutungen strukturieren, \emph{wo} sich wer wie bewegen kann, welche Situationen als sicher, bedrohlich oder \enquote{normal} empfunden werden, und welche Orientierungen im Raum als selbstverständlich erscheinen. In diesem Sinn lässt sich \enquote{Whiteness} als räumliche Orientierung verstehen, die für einige Körper Wege öffnet und für andere versperrt \parencite{ahmedPhenomenologyWhiteness2007}. Entscheidend ist damit weniger eine rein individuelle Positionierung, sondern die in Orte und Praktiken eingeschriebene Wirksamkeit von Machtverhältnissen.\footnote{In dieser Arbeit wird bewusst der Begriff \emph{affektives Wohlbefinden} verwendet, um kurzfristige, situativ geprägte Emotionen in ihrer räumlichen und sozialen Situiertheit zu betonen und damit eine explizit machtkritische Perspektive einzunehmen. Damit wird eine Lesart von \emph{Affect} im rein präpersonalen oder vorsprachlichen Sinn vermieden, wie sie in Teilen der \emph{affective geographies} vertreten wird. Feministische Kritiken warnen, dass eine solche ontologisierte Auffassung von Affekt Machtverhältnisse entpolitisiert \parencite{hemmingsInvokingAffectCultural2005,andersonAffectiveAtmospheres2009}.}

Das Konzept der \emph{affective atmospheres} präzisiert diese räumliche Dimension. Atmosphären entstehen im Zusammenspiel räumlicher (\gls{zb} Sichtachsen, Beleuchtung, Dichte, Lärm, Überwachung) und sozialer Ordnungen (\gls{zb} Zugangsregime, informelle Normen) und prägen affektives Wohlbefinden situativ \parencite{andersonAffectiveAtmospheres2009}. So erklären sich Unterschiede in der Wahrnehmung desselben Ortes: Ein stark kontrollierter Eingangsbereich oder ein nächtlicher Platz kann für privilegierte Gruppen belebt und angenehm wirken, während er \gls{bspw} für marginalisierte Gruppen als belastend oder gefährlich erfahrbar ist. Damit schliesst diese Perspektive an die intersektionale Geographie an, in der soziale Kategorien ihre Bedeutung erst im konkreten räumlichen Kontext entfalten \parencite{valentineTheorizingResearchingIntersectionality2007,rodo-de-zarateIntersectionalityFeministGeographies2018}.

Zugehörigkeit ist dabei kein stabiler Zustand, sondern ein relationaler, umkämpfter Prozess, in dem Zugehörigkeitsansprüche und Grenzziehungen fortlaufend verhandelt werden. Wer sich wo zugehörig und wann sicher fühlen kann, ist Ergebnis solcher Aushandlungen und spiegelt ungleiche Machtverhältnisse wider. Gefühle von Sicherheit, Bedrohung oder Normalität strukturieren damit alltägliche Raumerfahrungen und prägen, welche Wege, Aufenthaltsorte und Begegnungen als möglich erscheinen \parencite{antonsichSearchingBelongingAnalytical2010,painGlobalizedFearEmotional2009}.

Atmosphären wirken nicht nur semantisch, sondern verkörpert: Sie werden durch Routinen, Haltungen und Bewegungen aufgenommen und als Gewohnheit gefestigt. Solche affektiven Praktiken erklären, warum bestimmte Umgebungen für manche Körper \enquote{selbstverständlich} funktionieren, während sie andere ermüden, anspannen oder abweisen. Damit verbinden Atmosphären Sinneswahrnehmung, Körperdispositionen und soziale Ordnung -- und schreiben sich in alltägliche Mobilitäten ein \parencite{mccormackEngineeringAffectiveAtmospheres2008,bissellPassengerMobilitiesAffective2010}.

Für die vorliegende Arbeit bedeutet das, affektives Wohlbefinden als räumlich \emph{und} sozial bedingtes, zugleich aber auch körperbezogenes Phänomen zu fassen. Menschen erleben denselben Ort unterschiedlich -- je nach sozialer Positionierung, biografischen Erfahrungen und aktuellen körperlichen Zuständen (etwa Müdigkeit, Schmerz, sensorische Empfindlichkeit), die sich in Routinen und Haltungen niederschlagen \parencite{mccormackEngineeringAffectiveAtmospheres2008,bissellPassengerMobilitiesAffective2010,ahmedPhenomenologyWhiteness2007}. So kann \gls{bspw} ein öffentlicher Platz mit grosser Lautstärke und starker sozialer Kontrolle für einige als Ort der Erholung und Interaktion erscheinen, während andere ihn als unsicher oder bedrohlich erleben \parencite{collectiveSafeSpaceReconceptualization2014}. Emotionales Erleben entsteht damit relational im Zusammenspiel von Körper, Atmosphäre und Machtverhältnissen -- es ist weder rein individuell noch aus dem Kontext herauslösbar.

Empirisch arbeiten affective geographies überwiegend qualitativ und situationsnah: mit go-along/walking interviews, die Wahrnehmungen in-situ erfassen \parencite{kusenbachStreetPhenomenologyGoAlong2003}, mit sensory ethnography, die Sinnesmodalitäten systematisch einbindet \parencite{pinkDoingSensoryEthnography2009}, sowie mit mobile methods, die affektive Dynamiken unterwegs untersuchen \parencite{buscherIntroductionMobileMethods2010}. Solche Zugänge machen atmosphärische Feinheiten sichtbar und verknüpfen sie mit sozialen Positionierungen; vergleichende Analysen zwischen unterschiedlich positionierten Gruppen am selben Ort bleiben jedoch häufig retrospektiv oder fallbasiert und sind seltener systematisch wiederholend angelegt \parencite{hoSocialGeographyIII2024}.

Die beschriebenen Unterschiede im affektiven Erleben werden in der Geographie auf unterschiedlichen Massstabsebenen (scales) untersucht. Ein Fokus auf körpernahe, individuelle Erlebnisse erlaubt es, feinste situative Veränderungen des Wohlbefindens zu erfassen und deren Zusammenhang mit unmittelbaren räumlichen und sozialen Kontexten zu analysieren. Auf einer meso-räumlichen Ebene geraten kollektive Atmosphären in Quartieren, Stadtteilen oder anderen lokalisierten Gemeinschaftsräumen in den Blick, während makro-räumliche Analysen nationale oder transnationale Strukturen einbeziehen, die emotionale Erfahrungen rahmen und begrenzen \parencite{howittScaleRelationMusical1998,marstonHumanGeographyScale2005}. Dieses skalierende Verständnis macht deutlich, dass affektives Wohlbefinden weder rein individuell noch vollständig lokal erklärbar ist, sondern immer in ein Geflecht aus Mikroerfahrungen, kollektiven Dynamiken und übergeordneten gesellschaftlich-räumlichen Strukturen eingebettet ist.

Die Geographie nutzt dieses skalierende Verständnis, um Fragen räumlicher Gerechtigkeit und sozialer Teilhabe zu untersuchen. Indem Mikroerfahrungen des Alltags mit kollektiven Dynamiken und übergeordneten gesellschaftlich-räumlichen Strukturen in Beziehung gesetzt werden, lassen sich ungleiche Verteilungen von Möglichkeiten, Sicherheit oder Zugang sichtbar machen. Damit wird affektives Wohlbefinden zu einem analytischen Zugang, der alltägliche emotionale Erfahrungen mit den Macht- und Ungleichheitsverhältnissen verknüpft, in die sie eingebettet sind.

Dieses theoretische Geflecht legt in-situ und wiederholte Erhebungen nahe, die Unterschiede innerhalb derselben Personen über Situationen hinweg sowie zwischen unterschiedlich positionierten Personen am selben Ort vergleichbar machen -- ohne dabei die situative Einbettung zu verlieren.

\section{Digitale Werkzeuge -- Data Feminism, Open Source und digitale Souveränität}
\label{sec:datafeminism}

Digitale Technologien strukturieren zunehmend gesellschaftliche Realitäten -- sie beeinflussen, was sichtbar wird, wie Wissen entsteht und wer daran teilhat. Wer Software entwickelt, Daten sammelt oder Infrastrukturen kontrolliert, gestaltet diese Prozesse aktiv mit. Digitale Technologien sind daher nie neutral, sondern Ausdruck bestehender Machtverhältnisse. Eine kritische Auseinandersetzung mit digitalen Technologien muss deshalb deren soziale und politische Dimension systematisch in den Blick nehmen.

Einen geeigneten theoretischen Rahmen hierfür bietet das Konzept des \textit{Data Feminism} von \textcite{dignazioDataFeminism2020}. Data Feminism hinterfragt vermeintliche Objektivität und Neutralität von Daten und Algorithmen, indem es deren Entstehungskontexte, Produktionsbedingungen und zugrunde liegende Machtverhältnisse offenlegt. Aus dieser Perspektive erscheinen Daten nicht als neutrale Fakten, sondern als gesellschaftliche Konstrukte, die Ausschlüsse produzieren, Hierarchien festigen oder marginalisierte Gruppen unsichtbar machen können \parencite{elwoodFeministDigitalGeographies2018}.

Digitale Infrastrukturen sind Ausdruck und Austragungsorte gesellschaftlicher Machtverhältnisse. Im Sinne feministischer Geographien lassen sich digitale Technologien als Räume verstehen, in denen Fragen von Sichtbarkeit, Teilhabe und Gerechtigkeit neu verhandelt werden \parencite{elwoodFeministDigitalGeographies2018}. Aus dieser Perspektive gewinnen datenbezogene Praktiken politische Relevanz, gerade dann, wenn sie hegemoniale Strukturen hinterfragen und eigene Infrastrukturen schaffen. So zeigen feministische Initiativen etwa im Kontext von Feminiziden, wie digitale Praktiken als Mittel widerständiger Raumpolitik fungieren können -- durch das Sichtbarmachen von Gewalt, das Erinnern und das Etablieren eigener Datenräume \parencite{dignazioGeographiesMissingData2024}.

Diese Beispiele zeigen, dass digitale Infrastrukturen nicht nur technische Artefakte, sondern politische Räume sind, in denen Fragen nach Kontrolle, Zugang und Gestaltungsmacht neu verhandelt werden. In wissenschaftlichen und politischen Debatten wird dieser Aushandlungsprozess zunehmend unter dem Begriff der digitalen Souveränität gefasst \parencite{glaszeContestedSpatialitiesDigital2023}.

Während digitale Souveränität in politischen Diskursen oft als nationale Strategie oder technische Fähigkeit verstanden wird, rückt eine geographische Perspektive ihre räumlichen Dimensionen in den Vordergrund. Politisch-geographische Arbeiten betonen, dass digitale Souveränität stets in räumliche Ordnungen eingebettet ist und durch diese hervorgebracht wird \parencite{glaszeContestedSpatialitiesDigital2023,zhangBordersBorderingSovereignty2023}. Digitale Infrastrukturen produzieren und transformieren dabei Grenzen auf unterschiedlichen Massstabsebenen -- von staatlich regulierten Datenflüssen und territorial verankerten Rechenzentren bis hin zu unsichtbaren Abgrenzungen innerhalb digitaler Plattformen oder geschlossener Kommunikationsgruppen. Solche \enquote{digitalen Grenzen} bestimmen nicht nur, wer auf welche Daten und Dienste zugreifen kann und wer davon ausgeschlossen ist, sondern prägen auch, wie digitale Räume genutzt, wahrgenommen und angeeignet werden. Digitale Souveränität ist damit kein ortloses Prinzip, sondern in konkreten räumlichen Praktiken, Infrastrukturen und Machtverhältnissen verankert.

In diesem Verständnis bezeichnet digitale Souveränität nicht allein die technische Fähigkeit, digitale Technologien autonom zu betreiben oder zu kontrollieren. Sie umfasst auch die kollektive Befähigung, digitale Infrastrukturen kritisch zu reflektieren, partizipativ zu gestalten und als Gemeingüter zugänglich zu machen \parencite{baackDataficationEmpowermentHow2015,glaszeContestedSpatialitiesDigital2023}. Diese Perspektive begreift technologische Gestaltung als sozialen und politischen Aushandlungsprozess, in dem Fragen von Sichtbarkeit, Teilhabe und Verantwortung neu verhandelt werden.

\gls{opensource}-Praktiken können in diesem Kontext als konkrete Werkzeuge einer relational verstandenen digitalen Souveränität gelesen werden. Sie ermöglichen kollektive Kontrolle über technische Systeme, fördern Transparenz und erlauben es, digitale Infrastrukturen partizipativ zu gestalten \parencite{gurumurthyDataBodiesNew2022}. Indem sie Wissensproduktion nachvollziehbar machen und gemeinschaftliche Weiterentwicklung erlauben, tragen sie zur Demokratisierung technischer Expertise bei \parencite{baackDataficationEmpowermentHow2015, pohleDigitalSovereignty2020}.

Die Offenheit digitaler Infrastrukturen schafft zudem Räume für methodische Innovationen. Wissenschaftliche Erkenntnisse und technisches Know-how bleiben nicht hinter proprietären Zugangsbeschränkungen verborgen, sondern werden öffentlich überprüfbar und weiterentwickelbar gemacht. Dies fördert die Reproduzierbarkeit wissenschaftlicher Ergebnisse und stärkt partizipative Forschungsansätze wie Citizen Science oder kollektive Wissensproduktion \parencite{fecherWhatDrivesAcademic2014}.

Eine Entscheidung für Offenheit und digitale Souveränität erfordert eine kontinuierliche Reflexion über zugrunde liegende Bedingungen, Herausforderungen und mögliche Ausschlüsse. Es gilt stets kritisch zu fragen, wer Zugang zu digitalen Infrastrukturen hat, wer von ihnen profitiert und wer ausgeschlossen bleibt. Gerade feministische Perspektiven betonen, dass Offenheit nicht automatisch Gleichheit bedeutet, sondern aktiv gestaltet und gegen hegemoniale Machtverhältnisse verteidigt werden muss \parencite{wilshireTimeRebootFeminism2024}.
