% LTeX: language=de-CH

\section{Verflechtungen verstehen -- Begriffe und Konzepte} \label{sec:theoretischer_rahmen}

Dieses Kapitel legt den theoretischen Grundstein der Arbeit. Es führt in zentrale Begriffe und Konzepte ein, die das Erkenntnisinteresse leiten und das methodische Vorgehen rahmen. Ausgangspunkt ist die intersektionale Perspektive, die gesellschaftliche Unterschiede nicht isoliert, sondern in ihrer wechselseitigen Verflechtung analysiert. Im Anschluss wird das Konzept des affektiven Wohlbefindens als kontextabhängige, räumlich gebundene Erfahrung entfaltet. Beide Perspektiven bilden die Grundlage für die Entwicklung eines Forschungsdesigns, das soziale Positionierung, räumliche Kontexte und situative Erfahrungen miteinander in Beziehung setzt.

\subsection{Verwebte Unterschiede -- Intersektionalität als Analyseinstrument}

Gesellschaftliche Wirklichkeiten sind durchzogen von komplexen Ungleichheiten. Menschen erfahren soziale Benachteiligung selten entlang nur einer einzigen Achse -- vielmehr wirken verschiedene Differenzlinien wie \gls{race}, \gls{gender} oder \gls{class}\footnote{\gls{race}, \gls{gender} und \gls{class} werden in dieser Arbeit kursiv gesetzt, um auf ihre Bedeutung als gesellschaftlich konstruierte, aber wirkmächtige Kategorien hinzuweisen. \textit{race} verweist auf rassifizierende Zugehörigkeitszuschreibungen, die historisch gewachsen sind und soziale Ungleichheiten produzieren. \gls{gender} beschreibt die soziale Konstruktion von Geschlecht und verweist auf normative Vorstellungen von Weiblichkeit, Männlichkeit oder anderen Geschlechtsidentitäten. \gls{class} bezeichnet die soziale Konstruktion von sozialer Klasse und verweist auf normative Vorstellungen von Reichtum, Mittelklasse oder Armut. Die Begriffe werden im englischen Original verwendet, da adäquate deutsche Entsprechungen fehlen oder missverständlich sind \parencite[\gls{vgl}][]{hallRaceArticulationSocieties1980, butlerGenderTroubleFeminism1990}} häufig gleichzeitig und verstärken sich wechselseitig. Um diese Verflechtungen zu erfassen, bietet der intersektionale Ansatz einen theoretischen Rahmen, der Ungleichheitsverhältnisse nicht isoliert betrachtet, sondern ihre Überschneidungen und Wechselwirkungen in den Blick nimmt.

Geprägt wurde der Begriff der \gls{intersektionalitaet} von \textcite{crenshawMappingMarginsIntersectionality1991}, die auf die spezifischen Diskriminierungserfahrungen \glslink{schwarz}{Schwarzer}\footnote{„\gls{schwarz}“ wird in dieser Arbeit als politische Selbstbezeichnung \glslink{schwarz}{Schwarzer} Menschen mit grossem Anfangsbuchstaben verwendet. Der Begriff beschreibt keine biologische Eigenschaft, sondern eine soziale Positionierung im Kontext rassistischer Machtverhältnisse. Die Grossschreibung dient der Abgrenzung von äusserlichen Zuschreibungen \parencite{oguntoyeFarbeBekennenAfrodeutsche1986}.} Frauen aufmerksam machte. Sie argumentierte, dass bestehende feministische und antirassistische Theorien nicht ausreichten, um Mehrfachdiskriminierung zu erfassen, und entwickelte Intersektionalität als analytisches Instrument zur Beschreibung solcher überlagerten Ungleichheitsverhältnisse \parencite[\gls{vgl}][]{hancockWhenMultiplicationDoesnt2007}.

Ausgangspunkt dieser theoretischen Perspektive ist die Black Feminist Theory, welche unter anderen in den Arbeiten von \textcite{hooksAintWomanBlack1981}, \textcite{lordeSisterOutsiderEssays1984}, Kimberle~\textcite{crenshawMappingMarginsIntersectionality1991} und \textcite{collinsBlackFeministThought2002} ihren Ausdruck findet. Black Feminist Theory formulierte eine scharfe Kritik an traditionellen feministischen Ansätzen, denen vorgeworfen wurde, primär die Erfahrungen weisser, privilegierter Frauen ins Zentrum zu stellen und somit die Lebensrealitäten \glslink{schwarz}{Schwarzer} Frauen zu marginalisieren. \textcite{crenshawMappingMarginsIntersectionality1991} entwickelte das Konzept der Intersektionalität explizit als Reaktion auf die Unfähigkeit bestehender theoretischer Ansätze, die spezifischen Diskriminierungserfahrungen \glslink{schwarz}{Schwarzer} Frauen adäquat zu erfassen. Dabei verdeutlichte sie, dass Diskriminierung nicht als Summe einzelner, isolierter Erfahrungen verstanden werden könne, sondern als eigenständige Form sozialer Benachteiligung, die sich an der Überschneidung sozialer Kategorien wie \gls{race}, \gls{gender} und \gls{class} manifestiert.

Intersektionalität entwickelte sich somit nicht allein im akademischen Kontext, sondern ist stark verwurzelt in den politischen Kämpfen sozialer Bewegungen, insbesondere im Kontext feministischer, antirassistischer und antikapitalistischer Aktivismen der 1970er- und 1980er-Jahre \parencite{collinsBlackFeministThought2002}. Zentral für die theoretische Grundlage des intersektionalen Ansatzes ist die Anerkennung von Machtverhältnissen und sozialen Ungleichheiten als strukturell verankert und historisch bedingt. Gesellschaftliche Positionierungen werden als sozial konstruierte Kategorien verstanden, die immer in Verbindung mit bestehenden Machtsystemen wie Sexismus, Rassismus oder Klassismus betrachtet werden müssen. Audre Lorde und Bell Hooks betonten insbesondere die Rolle struktureller Unterdrückung und verdeutlichten, wie sich dominante Gesellschaftsstrukturen auf individueller Ebene reproduzieren und sich somit wechselseitig verstärken \parencite{collinsBlackFeministThought2002, hancockWhenMultiplicationDoesnt2007}.

Von der ursprünglich starken Fokussierung auf \textit{race} und \textit{gender} wurde das Konzept der Intersektionalität in den folgenden Jahrzehnten zunehmend erweitert und schliesst heute eine Vielzahl sozialer Positionierungen und Identitäten ein, darunter etwa Sexualität, Alter, Behinderung, Nationalität oder Religion \parencite{bauerIntersectionalityQuantitativeResearch2021, bowlegInvitedReflectionQuantifying2016}. Diese Erweiterung verdeutlicht die breite theoretische und empirische Anwendbarkeit von Intersektionalität als Analyseinstrument zur kritischen Untersuchung gesellschaftlicher Ungleichheiten und Diskriminierungserfahrungen. Intersektionalität hat sich somit nicht nur als theoretisches Konzept, sondern auch als methodische Grundlage etabliert, welche insbesondere in feministisch und sozialwissenschaftlich orientierten Diskursen verwendet wird, um die komplexen Wechselwirkungen gesellschaftlicher Machtverhältnisse zu analysieren.

\vspace{2em}

Die Anwendung intersektionaler Perspektiven auf räumliche Fragestellungen stellt eine zentrale Weiterentwicklung des ursprünglichen Konzepts der Intersektionalität dar. Seit den 2000er-Jahren etablierte sich eine eigenständige geographische Perspektive, die räumliche Kontextualität und situative Dimensionen sozialer Ungleichheiten explizit in den Mittelpunkt rückt \parencite{valentineTheorizingResearchingIntersectionality2007, rodo-de-zarateIntersectionalityFeministGeographies2018}.

Zentral für diesen Perspektivwechsel ist das Verständnis von Raum als gesellschaftlich produzierter Grösse. \textcite{lefebvreProductionLespace1974} argumentierte, dass Raum kein neutrales Behältnis sei, sondern als Produkt sozialer Praktiken und Beziehungen verstanden werden müsse. Machtverhältnisse schreiben sich demnach in Raumstrukturen und Nutzungen ein und reproduzieren sich über diese. \textcite{foucaultEspacesAutres2004} erweitert diese Perspektive mit seinem Konzept der Heterotopien: Räume spiegeln gesellschaftliche Normen nicht nur wider, sondern bieten auch die Möglichkeit ihrer Infragestellung und Verschiebung.

Auf dieser theoretischen Grundlage argumentiert \textcite{valentineTheorizingResearchingIntersectionality2007}, dass soziale Kategorien wie \textit{race}, \textit{gender} oder \textit{class} nicht unabhängig vom Raum wirken. Sie entfalten ihre Bedeutung erst im Zusammenspiel mit konkreten räumlichen Kontexten. Ungleichheiten sind somit nicht nur räumlich verteilt, sondern werden durch räumliche Anordnungen hervorgebracht und erfahrbar gemacht. Räume erzeugen je nach sozialer Positionierung unterschiedliche Bedeutungen, Zugänglichkeiten und emotionale Resonanzen -- etwa in Form von \textit{Safe Spaces} oder Zonen der Exklusion \parencite[\gls{vgl}][S.~548--549]{rodo-de-zarateIntersectionalityFeministGeographies2018}.

An diese Überlegungen knüpft \textcite{rodo-de-zarateDevelopingGeographiesIntersectionality2014} mit dem Konzept der „Relief Maps“ an. Dieses Instrument ermöglicht es, relationale Überlagerungen von sozialen Positionierungen, räumlichen Kontexten und emotionalen Erfahrungen systematisch zu erfassen. Indem subjektive Bewertungen mit spezifischen Orten und sozialen Kategorien verknüpft werden, lassen sich Machtverhältnisse sichtbar machen und ihre räumliche Wirksamkeit nachvollziehbar darstellen.

Die räumlich-intersektionale Perspektive erlaubt es somit, das Zusammenspiel von sozialer Identität, Macht und Raum differenziert zu analysieren. Raum erscheint dabei nicht nur als passiver Hintergrund sozialer Prozesse, sondern als aktiver Mitproduzent sozialer Differenz \parencite{rodo-de-zarateIntersectionalityFeministGeographies2018}.

\vspace{2em}

Obwohl intersektionale Forschung historisch in qualitativen und aktivistischen Traditionen verankert ist, gewinnen quantitative Verfahren zunehmend an Relevanz, insbesondere in sozialpolitischen und raumplanerischen Kontexten \parencite{bauerIntersectionalityQuantitativeResearch2021}. Diese Verfahren bieten die Möglichkeit, strukturelle Muster intersektionaler Benachteiligung über grössere Stichproben sichtbar und empirisch überprüfbar zu machen.

Jedoch ist die Übertragung intersektionaler Theorien in quantitative Methoden mit erheblichen Herausforderungen verbunden. Zentral ist die Kritik, dass traditionelle statistische Verfahren soziale Kategorien oft eindimensional oder additiv behandeln, was der komplexen theoretischen Vorstellung intersektionaler Verschachtelungen nicht gerecht wird \parencite{hancockWhenMultiplicationDoesnt2007, bowlegInvitedReflectionQuantifying2016}. Insbesondere birgt die numerische Operationalisierung sozialer Identitäten die Gefahr, die Fluidität und Kontextabhängigkeit dieser Kategorien zu ignorieren und damit ungewollt jene komplexen Wechselwirkungen zu nivellieren, die intersektionale Ansätze ursprünglich sichtbar machen wollen \parencite{scottIntersectionalityQuantitativeMethods2017}.

Um diesen Herausforderungen zu begegnen, bedarf es einer reflexiven und kontextsensiblen Operationalisierung intersektionaler Kategorien. Dies beinhaltet, soziale Gruppen nicht als statische Entitäten zu behandeln, sondern ihre relationalen und kontextuellen Eigenschaften explizit zu berücksichtigen \parencite{rodo-de-zarateDevelopingGeographiesIntersectionality2014, websterCenteringSocialtechnicalRelations2021}.


\subsection{Gefühlte Orte -- Wohlbefinden als räumliche Erfahrung}

Affektives Wohlbefinden bezeichnet kurzfristige, situativ schwankende emotionale Zustände wie Zufriedenheit, Gelassenheit oder Anspannung. Im Gegensatz zu psychologischen Konzepten subjektiven Wohlbefindens, die oft langfristige Lebenszufriedenheit messen, fokussiert das Konzept des affektiven Wohlbefindens bewusst auf flüchtige und unmittelbar erlebte Gefühlslagen. Diese sind stark vom aktuellen räumlichen und sozialen Kontext beeinflusst und reagieren besonders sensibel auf Veränderungen der unmittelbaren Umgebung \parencite{dodgeChallengeDefiningWellbeing2012}.

Innerhalb der Geographie gewann das Konzept des affektiven Wohlbefindens seit dem \emph{emotional turn} in den frühen 2000er-Jahren zunehmend an Bedeutung \parencite{hoSocialGeographyIII2024}. Dabei stehen Emotionen nicht mehr ausschliesslich als interne Zustände von Individuen im Vordergrund, sondern vielmehr deren Wechselwirkungen mit der räumlichen Umwelt. Orte werden aus dieser Perspektive nicht als passive Kulissen menschlichen Erlebens begriffen, sondern als aktive Bestandteile emotionaler Prozesse, die das Wohlbefinden direkt mitgestalten.

\textcite{ahmedAffectiveEconomies2004} beschreibt in ihrer Theorie der \emph{affective economies}, wie Emotionen sich an Orte, Objekte oder Personen heften und dadurch kollektive Atmosphären erzeugen. Emotionen sind demnach weniger als individuelles Eigentum zu verstehen, sondern vielmehr als soziale und räumliche Dynamiken, die zwischen Menschen und ihrer Umwelt zirkulieren und somit soziale Beziehungen und räumliche Zugehörigkeiten strukturieren. \textcite{andersonAffectiveAtmospheres2009} konkretisiert diesen Gedanken mit dem Konzept der affektiven Atmosphären: Orte werden durchzogen von Stimmungen, die kollektiv wahrgenommen werden, ohne jedoch klar greifbar oder individuell lokalisierbar zu sein. \textcite{hoSocialGeographyIII2024} ergänzt diese Perspektive durch den Hinweis, dass solche Atmosphären und Emotionen stets in gesellschaftliche Machtverhältnisse eingebettet sind. Die Wahrnehmung und Bewertung eines Ortes ist somit immer auch eine Frage sozialer Positionierung und gesellschaftlicher Normen.

Für die vorliegende Arbeit bedeutet das, affektives Wohlbefinden explizit als räumlich und sozial bedingtes Phänomen zu verstehen. Unterschiedliche gesellschaftliche Gruppen erleben denselben Ort verschieden, abhängig von ihrer sozialen Positionierung und den damit verbundenen Erwartungen oder Erfahrungen. So kann beispielsweise ein öffentlicher Platz mit grosser Lautstärke und starker sozialer Kontrolle für privilegierte Gruppen ein Raum der Erholung und Interaktion sein, während er für marginalisierte Gruppen Unsicherheit oder sogar Angst bedeutet \parencite{collectiveSafeSpaceReconceptualization2014}. Diese Beobachtung verdeutlicht, wie emotionales Erleben unmittelbar mit gesellschaftlichen Strukturen und räumlichen Bedingungen verknüpft ist.

Die Geographie nutzt diese Einsichten zunehmend für eine kritische Analyse räumlicher Gerechtigkeit und sozialer Teilhabe. Fragen nach dem affektiven Wohlbefinden ermöglichen es, Mikroerfahrungen des Alltags systematisch mit Makrostrukturen sozialer Ungleichheit in Verbindung zu setzen. Zunehmend werden daher Methoden wie \acrfull{esm} oder \acrfull{ema} eingesetzt, um das flüchtige und situative Erleben im realen Alltag direkt und kontextbezogen zu erfassen \parencite{songRecallBiasIntegrating2025}. Solche Ansätze erlauben nicht nur eine allgemeinere Aussage über Wohlbefinden, sondern machen dessen räumliche und soziale Verteilung sichtbar.

\subsection{Digitale Werkzeuge -- Data Feminism, Open Source und digitale Souveränität}

Digitale Technologien strukturieren zunehmend gesellschaftliche Realitäten -- sie beeinflussen, was sichtbar wird, wie Wissen entsteht und wer daran teilhat. Wer Software entwickelt, Daten sammelt oder Infrastrukturen kontrolliert, gestaltet diese Prozesse aktiv mit. Digitale Technologien sind daher nie neutral, sondern Ausdruck bestehender Machtverhältnisse. Eine kritische Auseinandersetzung mit digitalen Technologien muss deshalb deren soziale und politische Dimension systematisch in den Blick nehmen.

Einen geeigneten theoretischen Rahmen hierfür bietet das Konzept des \textit{Data Feminism} von \textcite{dignazioDataFeminism2020}. Data Feminism hinterfragt vermeintliche Objektivität und Neutralität von Daten und Algorithmen, indem es deren Entstehungskontexte, Produktionsbedingungen und zugrunde liegende Machtverhältnisse offenlegt. Aus dieser Perspektive erscheinen Daten nicht als neutrale Fakten, sondern als gesellschaftliche Konstrukte, die Ausschlüsse produzieren, Hierarchien festigen oder marginalisierte Gruppen unsichtbar machen können \parencite{elwoodFeministDigitalGeographies2018}.

Digitale Infrastrukturen sind Ausdruck und Austragungsorte gesellschaftlicher Machtverhältnisse. Im Sinne feministischer Geographien lassen sich digitale Technologien als Räume verstehen, in denen Fragen von Sichtbarkeit, Teilhabe und Gerechtigkeit neu verhandelt werden \parencite{elwoodFeministDigitalGeographies2018}. Aus dieser Perspektive gewinnen datenbezogene Praktiken politische Relevanz, gerade dann, wenn sie hegemoniale Strukturen hinterfragen und eigene Infrastrukturen schaffen. So zeigen feministische Initiativen etwa im Kontext von Feminiziden, wie digitale Praktiken als Mittel widerständiger Raumpolitik fungieren können -- durch das Sichtbarmachen von Gewalt, das Erinnern und das Etablieren eigener Datenräume \parencite{dignazioGeographiesMissingData2024}.

Diese Beispiele zeigen, dass digitale Infrastrukturen nicht nur technische Artefakte, sondern politische Räume sind, in denen Fragen nach Kontrolle, Zugang und Gestaltungsmacht neu verhandelt werden. In der wissenschaftlichen und politischen Debatte wird dieser Aushandlungsprozess zunehmend unter dem Begriff der digitalen Souveränität gefasst.

Digitale Souveränität bezeichnet in diesem Zusammenhang nicht bloss die technische Fähigkeit, digitale Technologien autonom zu nutzen oder zu kontrollieren. Vielmehr geht es um die kollektive Befähigung, digitale Infrastrukturen kritisch zu reflektieren, gesellschaftlich verantwortungsvoll zu gestalten und als Gemeingüter zugänglich zu machen \parencite{baackDataficationEmpowermentHow2015, glaszeContestedSpatialitiesDigital2023}. Damit rückt eine Perspektive in den Vordergrund, die technologische Gestaltung als sozialen und politischen Aushandlungsprozess begreift.

\gls{opensource}-Praktiken können in diesem Kontext als konkrete Werkzeuge einer relational verstandenen digitalen Souveränität gelesen werden. Sie ermöglichen kollektive Kontrolle über technische Systeme, fördern Transparenz und erlauben es, digitale Infrastrukturen partizipativ zu gestalten \parencite{gurumurthyDataBodiesNew2022}. Indem sie Wissensproduktion nachvollziehbar machen und gemeinschaftliche Weiterentwicklung erlauben, tragen sie zur Demokratisierung technischer Expertise bei \parencite{baackDataficationEmpowermentHow2015, pohleDigitalSovereignty2020}.

Die Offenheit digitaler Infrastrukturen schafft zudem Räume für methodische Innovationen. Wissenschaftliche Erkenntnisse und technisches Know-how bleiben nicht hinter proprietären Zugangsbeschränkungen verborgen, sondern werden öffentlich überprüfbar und weiterentwickelbar gemacht. Dies fördert die Reproduzierbarkeit wissenschaftlicher Ergebnisse und stärkt partizipative Forschungsansätze wie Citizen Science oder kollektive Wissensproduktion \parencite{fecherWhatDrivesAcademic2014}.

Eine Entscheidung für Offenheit und digitale Souveränität erfordert eine kontinuierliche Reflexion über zugrunde liegende Bedingungen, Herausforderungen und mögliche Ausschlüsse. Es gilt stets kritisch zu fragen, wer Zugang zu digitalen Infrastrukturen hat, wer von ihnen profitiert und wer ausgeschlossen bleibt. Gerade feministische Perspektiven betonen, dass Offenheit nicht automatisch Gleichheit bedeutet, sondern aktiv gestaltet und gegen hegemoniale Machtverhältnisse verteidigt werden muss \parencite{wilshireTimeRebootFeminism2024}.
