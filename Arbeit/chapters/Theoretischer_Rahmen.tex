% LTeX: language=de-CH

\section{Theoretischer Rahmen} \label{sec:theoretischer_rahmen}

\subsection{Intersektionalität in der quantitativen Forschung}

\subsubsection{Definition und Ursprünge der Intersektionalität}

Der Begriff der Intersektionalität wurde ursprünglich von Kimberlé \textcite{crenshawMappingMarginsIntersectionality1991} geprägt und verweist auf die Überlagerung und wechselseitige Verstärkung unterschiedlicher Formen von Diskriminierung, insbesondere im Kontext von \emph{race} und \emph{gender} \parencite{hancockWhenMultiplicationDoesnt2007}. Ausgangspunkt dieser theoretischen Perspektive ist die Black Feminist Theory, welche unter anderen in den Arbeiten von Crenshaw sowie Patricia Hill \textcite{collinsBlackFeministThought2002}, Audre Lorde und bell hooks ihren Ausdruck findet. Black Feminist Theory formulierte eine scharfe Kritik an traditionellen feministischen Ansätzen, denen vorgeworfen wurde, primär die Erfahrungen weisser, privilegierter Frauen ins Zentrum zu stellen und somit die Lebensrealitäten Schwarzer Frauen zu marginalisieren \parencite{collinsBlackFeministThought2002}. Kimberlé Crenshaw entwickelte das Konzept der Intersektionalität explizit als Reaktion auf die Unfähigkeit bestehender theoretischer Ansätze, die spezifischen Diskriminierungserfahrungen Schwarzer Frauen adäquat zu erfassen. Dabei verdeutlichte sie, dass Diskriminierung nicht als Summe einzelner, isolierter Erfahrungen verstanden werden könne, sondern als eigenständige Form sozialer Benachteiligung, die sich an der Überschneidung sozialer Kategorien wie \emph{race}, \emph{gender} und Klasse manifestiert \parencite{crenshawMappingMarginsIntersectionality1991}.

Intersektionalität entwickelte sich somit nicht allein im akademischen Kontext, sondern ist stark verwurzelt in den politischen Kämpfen sozialer Bewegungen, insbesondere im Kontext feministischer, antirassistischer und antikapitalistischer Aktivismen der 1970er- und 1980er-Jahre \parencite{collinsBlackFeministThought2002}. Zentral für die theoretische Grundlage des intersektionalen Ansatzes ist die Anerkennung von Machtverhältnissen und sozialen Ungleichheiten als strukturell verankert und historisch bedingt. Gesellschaftliche Positionierungen wie \emph{gender}, \emph{race} oder soziale Klasse werden hierbei als sozial konstruierte Kategorien verstanden, die immer in Verbindung mit bestehenden Machtsystemen wie Sexismus, Rassismus oder Klassismus betrachtet werden müssen. Audre Lorde und bell hooks betonten insbesondere die Rolle struktureller Unterdrückung und verdeutlichten, wie sich dominante Gesellschaftsstrukturen auf individueller Ebene reproduzieren und sich somit wechselseitig verstärken \parencite{collinsBlackFeministThought2002, hancockWhenMultiplicationDoesnt2007}.

Von der ursprünglich starken Fokussierung auf \emph{race} und \emph{gender} wurde das Konzept der Intersektionalität in den folgenden Jahrzehnten zunehmend erweitert und schliesst heute eine Vielzahl sozialer Positionierungen und Identitäten ein, darunter etwa Sexualität, Alter, Behinderung, Nationalität oder Religion \parencite{bauerIntersectionalityQuantitativeResearch2021, bowlegInvitedReflectionQuantifying2016}. Diese Erweiterung verdeutlicht die breite theoretische und empirische Anwendbarkeit von Intersektionalität als Analyseinstrument zur kritischen Untersuchung gesellschaftlicher Ungleichheiten und Diskriminierungserfahrungen. Intersektionalität hat sich somit nicht nur als theoretisches Konzept, sondern auch als methodische Grundlage etabliert, welche insbesondere in feministischen, sozialwissenschaftlichen und zunehmend auch in quantitativ orientierten Diskursen verwendet wird, um die komplexen Wechselwirkungen gesellschaftlicher Macht


\subsubsection{Quantitative Ansätze und ihre Herausforderungen}

Quantitative Forschungsmethoden gewinnen in der intersektionalen Forschung zunehmend an Bedeutung, wobei unterschiedliche theoretische und methodische Ansätze verfolgt werden \parencite{bauerIntersectionalityQuantitativeResearch2021}. Quantitative Verfahren bieten das Potenzial, systematische Strukturen und Muster von Interaktionen zwischen sozialen Kategorien empirisch sichtbar und statistisch überprüfbar zu machen. Gleichzeitig ist jedoch die methodische Umsetzung intersektionaler Analysen mit erheblichen Herausforderungen verbunden.

Ein grundlegendes Spannungsfeld ergibt sich aus der Integration der theoretischen Prämissen der Intersektionalität mit den technischen Anforderungen quantitativer Analysen. So kritisieren \textcite{hancockWhenMultiplicationDoesnt2007} eindimensionale und additive statistische Modelle, welche soziale Kategorien als unabhängige Variablen betrachten und lediglich deren einzelne Haupteffekte untersuchen. Solche Modelle laufen Gefahr, die Kernannahme der Intersektionalität, wonach soziale Kategorien stets miteinander verschränkt sind, unzureichend abzubilden \parencite{bowlegInvitedReflectionQuantifying2016, bauerIntersectionalityQuantitativeResearch2021}. In der Praxis wird Intersektionalität häufig auf einfache Interaktionseffekte in Regressionsmodellen reduziert, was die Gefahr einer Fehlinterpretation oder Vereinfachung komplexer sozialer Realitäten birgt \parencite{bauerIntersectionalityQuantitativeResearch2021, scottIntersectionalityQuantitativeMethods2017}.

Um diesen Herausforderungen zu begegnen, wurden unterschiedliche methodische Ansätze entwickelt. Ein vielversprechender Ansatz ist die \acrfull{maihda}, welche es erlaubt, intersektionale Effekte differenziert abzubilden, indem sie systematisch Varianzen innerhalb und zwischen sozialen Positionen quantifiziert \parencite{grossModellingIntersectionalityQuantitative2023}. \gls{maihda} bietet insbesondere die Möglichkeit, eine grosse Anzahl sozialer Positionierungen gleichzeitig zu betrachten, ohne diese auf blosse Interaktionsterme in klassischen Regressionsmodellen zu reduzieren \parencite{bauerIntersectionalityQuantitativeResearch2021}. Eine weitere Perspektive eröffnen sogenannte \emph{Decision-Tree}-Verfahren wie Klassifikations- und Regressionsbäume (\acrshort{cart}), welche explorativ heterogene Muster innerhalb intersektionaler Gruppen offenlegen können. Diese Verfahren sind jedoch durch ihre Datenabhängigkeit und eingeschränkte Reproduzierbarkeit begrenzt \parencite{bauerIntersectionalityQuantitativeResearch2021}.

Ein zentraler methodologischer Diskussionspunkt betrifft zudem die Wahl zwischen einer sogenannten \emph{intercategorical} und einer \emph{intracategorical} Herangehensweise, ein Konzept, welches ursprünglich von \textcite{mccallComplexityIntersectionality2005} eingeführt wurde. Während intercategorical-Analysen Unterschiede zwischen verschiedenen sozialen Gruppen vergleichen, fokussieren intracategorical-Ansätze auf die Analyse von Heterogenität und Prozessen innerhalb einer spezifischen Gruppe oder eines spezifischen sozialen Schnittpunkts \parencite{bauerAdvancingQuantitativeIntersectionality2019}. Die Wahl der jeweiligen Herangehensweise beeinflusst massgeblich die Operationalisierung intersektionaler Kategorien sowie die darauf basierenden statistischen Methoden und Auswertungen.

Eine zusätzliche Herausforderung stellt die kontextsensitive Operationalisierung und Messung intersektionaler Kategorien dar. Hierfür reicht eine blosse Festlegung statischer sozialer Gruppen nicht aus, da soziale Kategorien, wie etwa \emph{race} oder \emph{gender}, stets kontextabhängig und multidimensional konstruiert werden. \textcite{rodo-de-zarateDevelopingGeographiesIntersectionality2014, rodo-de-zarateYoungLesbiansNegotiating2015} argumentieren, dass quantitative Verfahren entsprechend flexibilisiert und angepasst werden müssen, um der Dynamik und Fluidität sozialer Identitäten gerecht zu werden. Dies erfordert eine hohe theoretische Reflexivität und methodologische Sensibilität, insbesondere bezüglich der Validität verwendeter Messinstrumente sowie der Interpretation statistischer Ergebnisse \parencite{bauerIntersectionalityQuantitativeResearch2021, websterCenteringSocialtechnicalRelations2021}.

Trotz der genannten Herausforderungen bieten quantitative Verfahren jedoch bedeutende Chancen für die intersektionale Forschung. Sie ermöglichen es, sozialstrukturelle Ungleichheiten empirisch sichtbar zu machen, grössere Stichproben systematisch zu untersuchen und somit evidenzbasierte Handlungsempfehlungen abzuleiten. Um quantitative Methoden adäquat für intersektionale Analysen nutzen zu können, ist es jedoch zwingend erforderlich, methodische Innovationen aktiv weiterzuentwickeln sowie eine kritische und reflektierte Anwendung der verfügbaren statistischen Verfahren sicherzustellen \parencite{bauerIntersectionalityQuantitativeResearch2021, bauerAdvancingQuantitativeIntersectionality2019, scottIntersectionalityQuantitativeMethods2017}.


\subsection{Räumliche Umgebung und momentanes Wohlbefinden}

\subsubsection{Umweltfaktoren und deren Einfluss auf Wohlbefinden}

Die unmittelbare räumliche Umgebung beeinflusst das menschliche Wohlbefinden erheblich. Die Bedeutung der Umgebung für psychisches und emotionales Wohlbefinden wird zunehmend sowohl in der Geographie als auch in der Psychologie und Stadtplanung untersucht. Dabei spielen Faktoren wie natürliche Umgebung, urbane Infrastruktur, soziale Interaktionen und die subjektive Wahrnehmung dieser Faktoren eine zentrale Rolle \parencite{birenboimInfluenceUrbanEnvironments2018, hammoudSmartphonebasedEcologicalMomentary2024, chenPerceivedUrbanEnvironment2025}.

\textcite{birenboimInfluenceUrbanEnvironments2018} untersuchte in einer umfassenden Studie mit Hilfe einer \acrfull{esm}, wie verschiedene urbane Kontexte (z.\,B. Parkanlagen, Verkehrsinfrastrukturen und öffentliche Plätze) subjektive momentane Erfahrungen wie Sicherheit, Komfort und Freude beeinflussen. Auf Grundlage von über 5000 individuellen Befragungen zeigt sich, dass Umweltfaktoren unmittelbar und signifikant das Wohlbefinden im Alltag prägen. Vor allem Merkmale wie räumliche Charakteristika (z.\,B. öffentliche versus private Räume), die Anwesenheit anderer Menschen sowie die Art der Aktivität beeinflussen das momentane Erleben erheblich. Überraschenderweise wurden stabile Persönlichkeitsfaktoren hingegen als kaum relevant für die Variabilität momentanen Wohlbefindens identifiziert \parencite{birenboimInfluenceUrbanEnvironments2018}. Diese Ergebnisse unterstreichen die Bedeutung der spezifischen situativen und räum

Ähnliche Befunde liefert eine Studie von \textcite{mascherekMeadowsAsphaltRoad2025}, die mit einer geografischen ökologischen Momentaufnahme (\acrfull{gema}) arbeitete. In dieser Studie wurden Daten aus drei deutschen Metropolregionen analysiert, wobei sich zeigte, dass Wetterbedingungen (insbesondere Sonnenschein), soziale Begleitung und Mobilität bedeutend stärkere Effekte auf momentanes affektives Wohlbefinden hatten als die blosse Verfügbarkeit urbaner Grünflächen. Diese Ergebnisse weisen darauf hin, dass nicht nur die physischen Merkmale eines Ortes, sondern auch kontextuelle und soziale Faktoren die momentane Stimmung und das Wohlbefinden entscheidend prägen \parencite{mascherekMeadowsAsphaltRoad2025}.

Ebenso konnte eine Untersuchung von \textcite{hammoudSmartphonebasedEcologicalMomentary2024} mittels Smartphone-basierter \gls{esm} nachweisen, dass insbesondere die Vielfalt natürlicher Elemente wie Pflanzen, Bäume und Tierwelt mit höheren Werten mentalen Wohlbefindens verbunden ist. Diese Studie betont explizit die Bedeutung ökologischer Diversität für die öffentliche psychische Gesundheit und hebt die Notwendigkeit hervor, biodiversitätsreiche Orte in städtischen Umgebungen zu fördern, um das Wohlbefinden der Bevölkerung nachhaltig zu verbessern.

\textcite{chenPerceivedUrbanEnvironment2025} bestätigen diese Perspektive und ergänzen, dass bestimmte urbane Infrastrukturen wie Cafés, Kultur- und Bildungseinrichtungen sowie öffentliche Räume langfristiges Wohlbefinden fördern, wohingegen natürliche und ruhige Umgebungen vor allem mit kurzfristigem, momentanen Wohlbefinden positiv korrelieren. Umgekehrt wirkten sich belebte und bewegungsintensive urbane Kontexte negativ auf momentanes Wohlbefinden aus. Diese Studie hebt die Komplexität der räumlichen Einflussfaktoren hervor und fordert eine differenzierte Betrachtung der Wechselwirkungen zwischen räumlichen Merkmalen und psychischem Wohlbefinden.

Insgesamt zeigt sich in der Literatur, dass die räumliche Umgebung unmittelbar und deutlich Einfluss auf das momentane Wohlbefinden ausübt, wobei natürliche, soziale und infrastrukturelle Charakteristika der Umgebung besonders bedeutsam sind. Methodische Fortschritte wie \gls{esm} und \gls{gema} erlauben eine immer präzisere Erfassung dieser räumlich-temporalen Dynamiken, die für eine ganzheitliche Betrachtung von Wohlbefinden und Stadtentwicklung unverzichtbar sind \parencite{kirchnerSpatiotemporalDeterminantsMental2016, cookeMeasuringWellBeingReview2016}.

\subsubsection{Bedeutung intersektionaler Perspektiven auf räumliches Wohlbefinden}

Während die Bedeutung der räumlichen Umgebung für das Wohlbefinden inzwischen gut dokumentiert ist, besteht weiterhin Forschungsbedarf hinsichtlich intersektionaler Perspektiven, die individuelle Erfahrungen in ihren sozial differenzierten Kontexten analysieren. Bisherige Untersuchungen räumlicher Einflüsse auf Wohlbefinden berücksichtigen häufig nicht ausreichend, dass Individuen aufgrund ihrer sozialen Positionierungen wie Gender, Ethnizität, sozioökonomischem Status oder Alter sehr unterschiedliche Erfahrungen in denselben räumlichen Kontexten machen können \parencite{rodo-de-zarateDevelopingGeographiesIntersectionality2014, rodo-de-zarateIntersectionalityFeministGeographies2018}.

Intersektionale Ansätze ermöglichen es, nicht nur räumliche, sondern auch soziale Differenzierungen präzise zu erfassen und so differenzierte Aussagen darüber zu treffen, wie räumliche Umgebungen je nach sozialer Positionierung unterschiedlich wahrgenommen werden. Beispielsweise zeigen Studien zu jungen lesbischen Frauen, dass öffentliche Räume nicht generell als sicher oder unsicher empfunden werden, sondern diese Empfindungen stark von der Überschneidung individueller sozialer Positionierungen und der räumlichen Umgebung abhängen \parencite{rodo-de-zarateYoungLesbiansNegotiating2015}. Räumliche Orte können somit gleichzeitig als befreiend und bedrohlich wahrgenommen werden, je nachdem, wie sich spezifische soziale Identitäten mit räumlichen Kontexten verschränken.

Auch hinsichtlich der psychischen Gesundheit und des Wohlbefindens verdeutlicht eine intersektionale Perspektive, dass Effekte der räumlichen Umgebung nicht universell sind, sondern von sozialen Machtverhältnissen abhängen, welche sich in bestimmten Orten materialisieren. Intersektionale Forschung betont, dass soziale Ungleichheiten und Diskriminierungen sich räumlich manifestieren und dadurch spezifische Belastungen und Herausforderungen für marginalisierte Gruppen entstehen \parencite{websterCenteringSocialtechnicalRelations2021}. Ein solcher Ansatz erweitert die räumliche Wohlbefindensforschung um eine kritische Reflexion sozialer Ungleichheiten und trägt somit dazu bei, eine differenziertere und gesellschaftlich relevantere Forschungsperspektive zu entwickeln.

Zusammenfassend lässt sich festhalten, dass intersektionale Perspektiven essenzielle Ergänzungen zur Erforschung räumlicher Einflüsse auf das Wohlbefinden darstellen. Durch die Verbindung von räumlichen und sozialen Differenzierungen können spezifische Erfahrungen und Bedürfnisse unterschiedlicher Bevölkerungsgruppen sichtbar gemacht und in der Gestaltung sozial gerechter und inklusiver Städte berücksichtigt werden.

