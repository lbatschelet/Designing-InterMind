% LTeX: language=de-CH

\section{Theoretischer Rahmen} \label{sec:theoretischer_rahmen}

\subsection{Intersektionalität als analytische Perspektive}

\subsubsection{Begriff und Ursprung}

Der Begriff der \gls{intersektionalitaet} wurde ursprünglich von Kimberlé \textcite{crenshawMappingMarginsIntersectionality1991} geprägt und verweist auf die Überlagerung und wechselseitige Verstärkung unterschiedlicher Formen von Diskriminierung, insbesondere im Kontext von \gls{race} und \gls{gender}\footnote{\gls{race}, \gls{gender} und \gls{class} werden in dieser Arbeit kursiv gesetzt, um auf ihre Bedeutung als gesellschaftlich konstruierte, aber wirkmächtige Kategorien hinzuweisen. \textit{race} verweist auf rassifizierende Zugehörigkeitszuschreibungen, die historisch gewachsen sind und soziale Ungleichheiten produzieren. \gls{gender} beschreibt die soziale Konstruktion von Geschlecht und verweist auf normative Vorstellungen von Weiblichkeit, Männlichkeit oder anderen Geschlechtsidentitäten. \gls{class} bezeichnet die soziale Konstruktion von sozialer Klasse und verweist auf normative Vorstellungen von Reichtum, Mittelklasse oder Armut. Die Begriffe werden im englischen Original verwendet, da adäquate deutsche Entsprechungen fehlen oder missverständlich sind \parencite[vgl.][]{hallRaceArticulationSocieties1980, butlerGenderTroubleFeminism1990}} \parencite{hancockWhenMultiplicationDoesnt2007}.

Ausgangspunkt dieser theoretischen Perspektive ist die Black Feminist Theory, welche unter anderen in den Arbeiten von Kimberlé \textcite{crenshawMappingMarginsIntersectionality1991} sowie Patricia Hill \textcite{collinsBlackFeministThought2002}, Audre Lorde und Bell Hooks ihren Ausdruck findet. Black Feminist Theory formulierte eine scharfe Kritik an traditionellen feministischen Ansätzen, denen vorgeworfen wurde, primär die Erfahrungen weisser, privilegierter Frauen ins Zentrum zu stellen und somit die Lebensrealitäten Schwarzer\footnote{„\gls{schwarz}“ wird in dieser Arbeit als politische Selbstbezeichnung Schwarzer Menschen mit grossem Anfangsbuchstaben verwendet. Der Begriff beschreibt keine biologische Eigenschaft, sondern eine soziale Positionierung im Kontext rassistischer Machtverhältnisse. Die Grossschreibung dient der Abgrenzung von äusserlichen Zuschreibungen \parencite{oguntoyeFarbeBekennenAfrodeutsche1986}.} Frauen zu marginalisieren \parencite{collinsBlackFeministThought2002}. \textcite{crenshawMappingMarginsIntersectionality1991} entwickelte das Konzept der Intersektionalität explizit als Reaktion auf die Unfähigkeit bestehender theoretischer Ansätze, die spezifischen Diskriminierungserfahrungen Schwarzer Frauen adäquat zu erfassen. Dabei verdeutlichte sie, dass Diskriminierung nicht als Summe einzelner, isolierter Erfahrungen verstanden werden könne, sondern als eigenständige Form sozialer Benachteiligung, die sich an der Überschneidung sozialer Kategorien wie \textit{race}, \textit{gender} und \textit{class} manifestiert.

Intersektionalität entwickelte sich somit nicht allein im akademischen Kontext, sondern ist stark verwurzelt in den politischen Kämpfen sozialer Bewegungen, insbesondere im Kontext feministischer, antirassistischer und antikapitalistischer Aktivismen der 1970er- und 1980er-Jahre \parencite{collinsBlackFeministThought2002}. Zentral für die theoretische Grundlage des intersektionalen Ansatzes ist die Anerkennung von Machtverhältnissen und sozialen Ungleichheiten als strukturell verankert und historisch bedingt. Gesellschaftliche Positionierungen werden als sozial konstruierte Kategorien verstanden, die immer in Verbindung mit bestehenden Machtsystemen wie Sexismus, Rassismus oder Klassismus betrachtet werden müssen. Audre Lorde und Bell Hooks betonten insbesondere die Rolle struktureller Unterdrückung und verdeutlichten, wie sich dominante Gesellschaftsstrukturen auf individueller Ebene reproduzieren und sich somit wechselseitig verstärken \parencite{collinsBlackFeministThought2002, hancockWhenMultiplicationDoesnt2007}.

Von der ursprünglich starken Fokussierung auf \textit{race} und \textit{gender} wurde das Konzept der Intersektionalität in den folgenden Jahrzehnten zunehmend erweitert und schliesst heute eine Vielzahl sozialer Positionierungen und Identitäten ein, darunter etwa Sexualität, Alter, Behinderung, Nationalität oder Religion \parencite{bauerIntersectionalityQuantitativeResearch2021, bowlegInvitedReflectionQuantifying2016}. Diese Erweiterung verdeutlicht die breite theoretische und empirische Anwendbarkeit von Intersektionalität als Analyseinstrument zur kritischen Untersuchung gesellschaftlicher Ungleichheiten und Diskriminierungserfahrungen. Intersektionalität hat sich somit nicht nur als theoretisches Konzept, sondern auch als methodische Grundlage etabliert, welche insbesondere in feministisch und sozialwissenschaftlich orientierten Diskursen verwendet wird, um die komplexen Wechselwirkungen gesellschaftlicher Machtverhältnisse zu analysieren.


\subsubsection{Intersektionalität und Raum}

Die Anwendung intersektionaler Perspektiven auf räumliche Fragestellungen stellt eine zentrale Erweiterung des ursprünglichen Konzepts der Intersektionalität dar. So etablierte sich seit den 2000er-Jahren zunehmend eine eigenständige geographische Perspektive, die räumliche Kontextualität und situative Dimensionen sozialer Ungleichheiten explizit in den Mittelpunkt rückt \parencite{valentineTheorizingResearchingIntersectionality2007, rodo-de-zarateIntersectionalityFeministGeographies2018}.

Bezugnehmend auf Henri \textcite{lefebvreProductionLespace1974} wird argumentiert, dass Raum kein neutrales Behältnis ist, sondern als gesellschaftlich erzeugtes Produkt verstanden werden muss. Er entsteht durch soziale Praktiken und Beziehungen und ist durchzogen von Machtverhältnissen, die sich in seiner Struktur und Nutzung manifestieren. Diese Perspektive wurde von Michel \textcite{foucaultEspacesAutres2004} weiterentwickelt, der in seiner Theorie der „Heterotopien“ darauf hinweist, dass Räume gesellschaftliche Normen nicht nur widerspiegeln, sondern auch deren Infragestellung und Verschiebung ermöglichen. Raum wird somit nicht nur als Hintergrund, sondern als aktiver Bestandteil sozialer Prozesse verstanden.

Aufbauend darauf argumentiert Gill \textcite{valentineTheorizingResearchingIntersectionality2007} in diesem Zusammenhang, dass soziale Kategorien erst in spezifischen räumlichen Kontexten wirksam werden. Ihrer Argumentation zufolge sind soziale Ungleichheiten nicht nur über Raum verteilt, sondern werden durch räumliche Anordnungen überhaupt erst hervorgebracht und erfahrbar gemacht. Anhand von Beispielen wie \textit{Safe Spaces} oder Zonen der Exklusion zeigt sie auf, dass Räume je nach sozialer Positionierung unterschiedliche Bedeutungen, Zugänglichkeiten und emotionale Resonanzen erzeugen \parencite[vgl.][S. 548 - 549]{rodo-de-zarateIntersectionalityFeministGeographies2018}.

Maria \textcite{rodo-de-zarateDevelopingGeographiesIntersectionality2014, rodo-de-zarateYoungLesbiansNegotiating2015} entwickelt zur Analyse dieser räumlichen Intersektionalität das Konzept der „Relief Maps“. Dieses Instrument erlaubt es, relationale Überlagerungen von sozialen Positionierungen, räumlichen Kontexten und emotionalen Erfahrungen systematisch zu erfassen. Indem emotionale Bewertungen mit sozialen Kategorien und konkreten Orten verbunden werden, lassen sich Machtverhältnisse visuell darstellen und deren räumliche Wirkung besser verstehen.

Diese räumlich-intersektionale Perspektive trägt dazu bei, Wechselwirkungen zwischen sozialer Identität, Macht und Raum differenzierter zu analysieren. Raum wird so nicht nur als Kontext, sondern als Mitproduzent sozialer Differenz verstanden \parencite{rodo-de-zarateIntersectionalityFeministGeographies2018}.

\subsubsection{Quantitative Ansätze der Intersektionalitätsforschung}

Obwohl intersektionale Forschung historisch in qualitativen und aktivistischen Traditionen verankert ist, gewinnen quantitative Verfahren zunehmend an Relevanz, insbesondere in sozialpolitischen und raumplanerischen Kontexten \parencite{bauerIntersectionalityQuantitativeResearch2021}. Diese Verfahren bieten die Möglichkeit, strukturelle Muster intersektionaler Benachteiligung über grössere Stichproben sichtbar und empirisch überprüfbar zu machen.

Jedoch ist die Übertragung intersektionaler Theorien in quantitative Methoden mit erheblichen Herausforderungen verbunden. Zentral ist die Kritik, dass traditionelle statistische Verfahren soziale Kategorien oft eindimensional oder additiv behandeln, was der komplexen theoretischen Vorstellung intersektionaler Verschachtelungen nicht gerecht wird \parencite{hancockWhenMultiplicationDoesnt2007, bowlegInvitedReflectionQuantifying2016}. Insbesondere birgt die numerische Operationalisierung sozialer Identitäten die Gefahr, die Fluidität und Kontextabhängigkeit dieser Kategorien zu ignorieren und damit ungewollt jene komplexen Wechselwirkungen zu nivellieren, die intersektionale Ansätze ursprünglich sichtbar machen wollen \parencite{scottIntersectionalityQuantitativeMethods2017}.

Um diesen Herausforderungen zu begegnen, bedarf es einer reflexiven und kontextsensiblen Operationalisierung intersektionaler Kategorien. Dies beinhaltet, soziale Gruppen nicht als statische Entitäten zu behandeln, sondern ihre relationalen und kontextuellen Eigenschaften explizit zu berücksichtigen \parencite{rodo-de-zarateDevelopingGeographiesIntersectionality2014, websterCenteringSocialtechnicalRelations2021}.


\subsection{Affektives Wohlbefinden}

Mit dem \textit{emotional turn} hat sich die Geographie seit den frühen 2000er‑Jahren intensiv der Frage gewidmet, wie Emotionen in und durch Räume entstehen \parencite{hoSocialGeographyIII2024}. Im Zentrum steht dabei das Konzept des affektiven Wohlbefindens: Es bezeichnet kurzfristige, situativ schwankende Emotionen wie Freude, Gelassenheit oder Anspannung, die besonders sensibel auf Kontexteinflüsse reagieren. 

Eine zentrale Einsicht dieses Paradigmenwechsels besteht darin, dass Emotionen nicht ausschliesslich im Inneren von Individuen entstehen, sondern durch ihr in einer Wechselwirkung zwischen Körpern, Dingen und Orten soziale Wirklichkeit mitgestalten. Sara \textcite{ahmedAffectiveEconomies2004} beschreibt diese Dynamik als \emph{affective economies}: Emotionen „haften“ an Gegenständen, Räumen oder Personengruppen und erzeugen dadurch Grenzziehungen zwischen Eigenem und Fremdem. Ben \textcite{andersonAffectiveAtmospheres2009} greift diesen Gedanken auf und spricht von "atmosphärischen Stimmungsschichten", die Orte wie ein kaum sichtbarer Schleier durchziehen und das Erleben aller Anwesenden prägen. Elaine \textcite{hoSocialGeographyIII2024} betont zudem, dass sich Emotionen stets in bereits bestehende Macht‑ und Ungleichheitsverhältnisse einschreiben.

Für diese Arbeit bedeutet das: Affektives Wohlbefinden entsteht nicht im luftleeren Raum, sondern im Zusammenspiel von materieller Ausstattung (etwa Vegetation oder Lärmpegel), sozialer Situation und den historisch gewachsenen Bedeutungen, die einem Ort anhaften.


%TODO: Material-turn
