% LTeX: language=de-CH

\section{Theoretischer Rahmen} \label{sec:theoretischer_rahmen}

\subsection{Intersektionalität in der quantitativen Forschung}

\subsubsection{Definition und Ursprünge der Intersektionalität}

Der Begriff der \gls{intersektionalitaet} wurde ursprünglich von Kimberlé \textcite{crenshawMappingMarginsIntersectionality1991} geprägt und verweist auf die Überlagerung und wechselseitige Verstärkung unterschiedlicher Formen von Diskriminierung, insbesondere im Kontext von \gls{race} und \gls{gender}\footnote{\gls{race} und \gls{gender} werden in dieser Arbeit kursiv gesetzt, um auf ihre Bedeutung als gesellschaftlich konstruierte, aber wirkmächtige Kategorien hinzuweisen. \textit{race} verweist auf rassifizierende Zugehörigkeitszuschreibungen, die historisch gewachsen sind und soziale Ungleichheiten produzieren. \gls{gender} beschreibt die soziale Konstruktion von Geschlecht und verweist auf normative Vorstellungen von Weiblichkeit, Männlichkeit oder anderen Geschlechtsidentitäten. Die Begriffe werden im englischen Original verwendet, da adäquate deutsche Entsprechungen fehlen oder missverständlich sind \parencite[vgl.][]{hallRaceArticulationSocieties1980, butlerGenderTroubleFeminism1990}} \parencite{hancockWhenMultiplicationDoesnt2007}.
Ausgangspunkt dieser theoretischen Perspektive ist die Black Feminist Theory, welche unter anderen in den Arbeiten von Crenshaw sowie Patricia Hill \textcite{collinsBlackFeministThought2002}, Audre Lorde und Bell Hooks ihren Ausdruck findet. Black Feminist Theory formulierte eine scharfe Kritik an traditionellen feministischen Ansätzen, denen vorgeworfen wurde, primär die Erfahrungen weisser, privilegierter Frauen ins Zentrum zu stellen und somit die Lebensrealitäten Schwarzer\footnote{„\gls{schwarz}“ wird in dieser Arbeit als politische Selbstbezeichnung Schwarzer Menschen mit grossem Anfangsbuchstaben verwendet. Der Begriff beschreibt keine biologische Eigenschaft, sondern eine soziale Positionierung im Kontext rassistischer Machtverhältnisse. Die Grossschreibung dient der Abgrenzung von farblichen oder äusserlichen Zuschreibungen \parencite[vgl.][]{oguntoyeFarbeBekennenAfrodeutsche1986}.} Frauen zu marginalisieren \parencite{collinsBlackFeministThought2002}. Crenshaw entwickelte das Konzept der Intersektionalität explizit als Reaktion auf die Unfähigkeit bestehender theoretischer Ansätze, die spezifischen Diskriminierungserfahrungen Schwarzer Frauen adäquat zu erfassen. Dabei verdeutlichte sie, dass Diskriminierung nicht als Summe einzelner, isolierter Erfahrungen verstanden werden könne, sondern als eigenständige Form sozialer Benachteiligung, die sich an der Überschneidung sozialer Kategorien wie \gls{race}, \gls{gender} und \gls{class} manifestiert \parencite{crenshawMappingMarginsIntersectionality1991}.

Intersektionalität entwickelte sich somit nicht allein im akademischen Kontext, sondern ist stark verwurzelt in den politischen Kämpfen sozialer Bewegungen, insbesondere im Kontext feministischer, antirassistischer und antikapitalistischer Aktivismen der 1970er- und 1980er-Jahre \parencite{collinsBlackFeministThought2002}. Zentral für die theoretische Grundlage des intersektionalen Ansatzes ist die Anerkennung von Machtverhältnissen und sozialen Ungleichheiten als strukturell verankert und historisch bedingt. Gesellschaftliche Positionierungen wie \textit{gender}, \textit{race} oder \textit{class} werden hierbei als sozial konstruierte Kategorien verstanden, die immer in Verbindung mit bestehenden Machtsystemen wie Sexismus, Rassismus oder Klassismus betrachtet werden müssen. Audre Lorde und Bell Hooks betonten insbesondere die Rolle struktureller Unterdrückung und verdeutlichten, wie sich dominante Gesellschaftsstrukturen auf individueller Ebene reproduzieren und sich somit wechselseitig verstärken \parencite{collinsBlackFeministThought2002, hancockWhenMultiplicationDoesnt2007}.

Von der ursprünglich starken Fokussierung auf \textit{race} und \textit{gender} wurde das Konzept der Intersektionalität in den folgenden Jahrzehnten zunehmend erweitert und schliesst heute eine Vielzahl sozialer Positionierungen und Identitäten ein, darunter etwa Sexualität, Alter, Behinderung, Nationalität oder Religion \parencite{bauerIntersectionalityQuantitativeResearch2021, bowlegInvitedReflectionQuantifying2016}. Diese Erweiterung verdeutlicht die breite theoretische und empirische Anwendbarkeit von Intersektionalität als Analyseinstrument zur kritischen Untersuchung gesellschaftlicher Ungleichheiten und Diskriminierungserfahrungen. Intersektionalität hat sich somit nicht nur als theoretisches Konzept, sondern auch als methodische Grundlage etabliert, welche insbesondere in feministischen, sozialwissenschaftlichen und zunehmend auch in quantitativ orientierten Diskursen verwendet wird, um die komplexen Wechselwirkungen gesellschaftlicher Machtverhältnisse zu analysieren.


\subsubsection{Quantitative Ansätze und ihre Herausforderungen}

Quantitative Forschungsmethoden gewinnen in der intersektionalen Forschung zunehmend an Bedeutung, wobei unterschiedliche theoretische und methodische Ansätze verfolgt werden \parencite{bauerIntersectionalityQuantitativeResearch2021}. Quantitative Verfahren bieten das Potenzial, systematische Strukturen und Muster von Interaktionen zwischen sozialen Kategorien empirisch sichtbar und statistisch überprüfbar zu machen. Gleichzeitig ist jedoch die methodische Umsetzung intersektionaler Analysen mit erheblichen Herausforderungen verbunden.

Ein grundlegendes Spannungsfeld ergibt sich aus der Integration der theoretischen Prämissen der Intersektionalität mit den technischen Anforderungen quantitativer Analysen. So kritisieren \textcite{hancockWhenMultiplicationDoesnt2007} eindimensionale und additive statistische Modelle, welche soziale Kategorien als unabhängige Variablen betrachten und lediglich deren einzelne Haupteffekte untersuchen. Solche Modelle laufen Gefahr, die Kernannahme der Intersektionalität, wonach soziale Kategorien stets miteinander verschränkt sind, unzureichend abzubilden \parencite{bowlegInvitedReflectionQuantifying2016, bauerIntersectionalityQuantitativeResearch2021}. In der Praxis wird Intersektionalität häufig auf einfache Interaktionseffekte in Regressionsmodellen reduziert, was die Gefahr einer Fehlinterpretation oder Vereinfachung komplexer sozialer Realitäten birgt \parencite{bauerIntersectionalityQuantitativeResearch2021, scottIntersectionalityQuantitativeMethods2017}.

Besonders zentral ist in diesem Zusammenhang die Frage nach der kontextsensitiven Operationalisierung intersektionaler Kategorien. Eine blosse Festlegung statischer sozialer Gruppen reicht nicht aus, da soziale Kategorien wie \textit{race} oder \textit{gender} stets kontextabhängig und multidimensional konstruiert sind. Maria \textcite{rodo-de-zarateDevelopingGeographiesIntersectionality2014} argumentiert, dass quantitative Verfahren entsprechend flexibilisiert und angepasst werden müssen, um der Dynamik und Fluidität sozialer Identitäten gerecht zu werden. Dies erfordert eine hohe theoretische Reflexivität und methodologische Sensibilität, insbesondere bezüglich der Validität verwendeter Messinstrumente sowie der Interpretation statistischer Ergebnisse \parencite{bauerIntersectionalityQuantitativeResearch2021, websterCenteringSocialtechnicalRelations2021}.

Um diesen Herausforderungen zu begegnen, wurden unterschiedliche methodische Ansätze entwickelt. Ein vielversprechender Ansatz ist die \acrfull{maihda}, welche es erlaubt, intersektionale Effekte differenziert abzubilden, indem sie systematisch Varianzen innerhalb und zwischen sozialen Positionen quantifiziert \parencite{evansTutorialConductingIntersectional2024, grossModellingIntersectionalityQuantitative2023, axelssonfiskChronicObstructivePulmonary2018}. \gls{maihda} bietet insbesondere die Möglichkeit, eine grosse Anzahl sozialer Positionierungen gleichzeitig zu betrachten, ohne diese auf blosse Interaktionsterme in klassischen Regressionsmodellen zu reduzieren \parencite{bauerIntersectionalityQuantitativeResearch2021}.

Ein Beispiel für die praktische Anwendung von \gls{maihda} liefert die Studie von \textcite{axelssonfiskChronicObstructivePulmonary2018}, in der die Verteilung von Lungenerkrankungen (COPD) unter mehr als 2{,}4 Millionen Menschen in Schweden untersucht wurde. Die Forschenden kombinierten dabei soziale Merkmale wie Alter, Geschlecht, Einkommen, Bildung, Familienstand und Migrationsstatus zu 96 verschiedenen sozialen Gruppen (sogenannten intersektionalen Strata). Durch die MAIHDA-Analyse konnte sichtbar gemacht werden, welche dieser Gruppen besonders stark oder schwach von der Erkrankung betroffen waren. Besonders auffällig war etwa, dass alleinlebende ältere Frauen mit niedrigem Einkommen und geringer Bildung ein bis zu 49-mal höheres Erkrankungsrisiko hatten als junge Männer mit hoher Bildung, gutem Einkommen und stabiler Partnerschaft. Die Studie zeigt exemplarisch, wie mit MAIHDA komplexe soziale Ungleichheiten im Gesundheitsbereich nicht nur statistisch erfasst, sondern auch differenziert beschrieben werden können. Gleichzeitig zeigte sich, dass viele dieser Unterschiede nicht durch besondere Wechselwirkungen (Interaktionen), sondern vor allem durch die Summe sozialer Nachteile erklärbar sind. 

Eine weitere Perspektive eröffnen sogenannte \emph{Decision-Tree}-Verfahren wie Klassifikations- und Regressionsbäume (\acrshort{cart}), welche explorativ heterogene Muster innerhalb intersektionaler Gruppen offenlegen können. Diese Verfahren sind insbesondere geeignet, wenn lineare Modellannahmen nicht greifen oder wenn vermutet wird, dass Schwellenwerte oder spezifische Kombinationen sozialer Merkmale entscheidend sind. CART-Verfahren sind jedoch durch ihre Datenabhängigkeit, begrenzte Generalisierbarkeit und eingeschränkte Reproduzierbarkeit limitiert \parencite{bauerIntersectionalityQuantitativeResearch2021}.

Ein zentraler methodologischer Diskussionspunkt betrifft zudem die Frage, wie intersektionale Kategorien in quantitativen Analysen konzeptualisiert werden. Leslie \textcite{mccallComplexityIntersectionality2005} entwickelte hierfür eine dreiteilige Typologie, die zwischen einer zwischenkategorialen (\textit{intercategorical}) und einer innerkategorialen (\textit{intracategorical}) Herangehensweise unterscheidet. Während zwischenkategoriale Analysen Unterschiede zwischen verschiedenen sozialen Gruppen systematisch vergleichen, richten sich innerkategoriale Ansätze auf die Untersuchung von Heterogenität und Dynamiken innerhalb einer bestimmten Gruppe oder sozialen Positionierung \parencite{bauerAdvancingQuantitativeIntersectionality2019}. Die Wahl der jeweiligen Herangehensweise beeinflusst massgeblich die Operationalisierung intersektionaler Kategorien sowie die Auswahl und Interpretation quantitativer Verfahren.

Trotz der genannten Herausforderungen bieten quantitative Verfahren jedoch bedeutende Chancen für die intersektionale Forschung. Sie ermöglichen es, sozialstrukturelle Ungleichheiten empirisch sichtbar zu machen, grössere Stichproben systematisch zu untersuchen und somit evidenzbasierte Handlungsempfehlungen abzuleiten. Um quantitative Methoden adäquat für intersektionale Analysen nutzen zu können, ist es jedoch zwingend erforderlich, methodische Innovationen aktiv weiterzuentwickeln sowie eine kritische und reflektierte Anwendung der verfügbaren statistischen Verfahren sicherzustellen \parencite{bauerIntersectionalityQuantitativeResearch2021, bauerAdvancingQuantitativeIntersectionality2019, scottIntersectionalityQuantitativeMethods2017}.


\subsection{Räumliche Umgebung und affektives Wohlbefinden}

\subsubsection{Affektives Wohlbefinden}

Mit dem sogenannten \emph{emotional turn} hat sich die Geographie seit den frühen 2000er‑Jahren intensiv der Frage gewidmet, wie Emotionen und Affekte in und durch Räume entstehen \parencite{hoSocialGeographyIII2024}. Im Zentrum steht dabei das Konzept des affektiven Wohlbefindens: Es bezeichnet kurzfristige, situativ schwankende Gefühlslagen wie Freude, Gelassenheit oder Anspannung, die besonders sensibel auf Kontexteinflüsse reagieren. 

Die zentrale Einsicht dieses Paradigmenwechsels besteht darin, dass solche Gefühle nicht ausschliesslich im Inneren von Individuen entstehen, sondern durch ihr Zirkulieren zwischen Körpern, Dingen und Orten soziale Wirklichkeit mitgestalten. Sara \textcite{ahmedAffectiveEconomies2004} beschreibt diese Dynamik als \emph{affective economies}: Gefühle „haften“ an Gegenständen, Räumen oder Personengruppen und erzeugen dadurch Grenzziehungen zwischen Eigenem und Fremdem. Anderson greift diesen Gedanken auf und spricht von atmosphärischen Stimmungsschichten, die Orte wie ein kaum sichtbarer Schleier durchziehen und das Erleben aller Anwesenden prägen \parencite{andersonAffectiveAtmospheres2009}. Elaine \textcite{hoSocialGeographyIII2024} fasst die aktuelle Forschung zu diesen affektiven Raumqualitäten zusammen und betont, dass sich Gefühle stets in bestehende Macht‑ und Ungleichheitsverhältnisse einschreiben.

Für diese Arbeit bedeutet das: Affektives Wohlbefinden entsteht nicht im luftleeren Raum, sondern im Zusammenspiel von materieller Ausstattung (etwa Vegetation oder Lärmpegel), sozialer Situation und den historisch gewachsenen Bedeutungen, die einem Ort anhaften.


\subsubsection{Methodische Zugänge und empirische Befunde}

Erste Versuche, affektives Wohlbefinden empirisch zu erfassen, stützten sich auf klassische Befragungen oder Laborexperimente – Methoden, die die situative Eingebundenheit emotionaler Zustände jedoch nur unzureichend abbilden konnten \parencite{kirchnerSpatiotemporalDeterminantsMental2016}. Einen grundlegenden methodischen Fortschritt brachte die Entwicklung der \acrfull{esm}, bei der Teilnehmende mehrmals täglich per Smartphone kurze Angaben zu ihrer momentanen Stimmung, Tätigkeit und sozialen Situation machen. Aufbauend darauf verbindet die \acrfull{gema} diese subjektiven Angaben mit GPS‑Daten sowie externen Umweltdaten – etwa Vegetationsanteil, Verkehrsaufkommen oder Wetter – und erlaubt so eine präzise räumlich‑zeitliche Analyse affektiver Erfahrungen \parencite{kirchnerSpatiotemporalDeterminantsMental2016}.

Auf Grundlage dieser Instrumente zeigen sich in der Literatur drei wiederkehrende Befundlinien:

Erstens deuten zahlreiche Studien auf eine positive Wirkung naturräumlicher Qualitäten hin. \textcite{birenboimInfluenceUrbanEnvironments2018} analysierten über 5'000 \gls{esm}-Einträge aus verschiedenen urbanen Kontexten und zeigten, dass Parkanlagen, baumgesäumte Strassen und visuell offene Plätze das Sicherheits- und Komfortempfinden erhöhen, während stark befahrene Verkehrsachsen gegenteilige Effekte zeigen. Eine \gls{gema}-Studie von \textcite{mascherekMeadowsAsphaltRoad2025} ergänzt, dass Sonnenschein, das Unterwegssein mit Freund:innen und aktive Mobilität stärkere Effekte auf affektives Wohlbefinden haben als die blosse Präsenz von Grünflächen. \textcite{hammoudSmartphonebasedEcologicalMomentary2024} heben insbesondere die Rolle ökologischer Vielfalt hervor: Je abwechslungsreicher die natürliche Ausstattung eines Orts (etwa durch Bäume, Pflanzen oder Vogelgesang), desto höher die berichteten Werte des Wohlbefindens. Auch unter pandemiebedingten Einschränkungen blieben solche Effekte bestehen: Während der COVID‑19‑Lockdowns gewannen Aufenthalte in Grün- und Blauräumen für viele Menschen therapeutische Bedeutung \parencite{doughtyTherapeuticLandscapesCOVID192023}.

Zweitens zeigen sich klare Differenzen darin, wie Menschen Räume affektiv erleben – je nach sozialer Positionierung. So beschreibt \textcite{shakerSayingNothingSaying2021}, wie alltägliche Fahrten mit dem öffentlichen Verkehr für junge Muslim\genderstern innen in Amsterdam durch subtile Blicke oder Gesten zu Räumen latenter Bedrohung werden können. \textcite{hallHatescapeRelationalGeography2019} dokumentieren ähnliche Erfahrungen bei Menschen mit Behinderung, für die bestimmte urbane Räume durch wiederholte Mikroaggressionen zu sogenannten Hatescapes werden. Diese empirischen Arbeiten unterstreichen die Notwendigkeit, Macht- und Ungleichheitsverhältnisse systematisch in die Analyse affektiver Raumwahrnehmung einzubeziehen \parencite{hoSocialGeographyIII2024}.

Drittens wird in der Literatur zunehmend zwischen kurzfristigem affektivem Erleben und langfristiger Lebenszufriedenheit unterschieden. \textcite{chenPerceivedUrbanEnvironment2025} zeigen, dass ruhige, naturnahe Orte vor allem das unmittelbare emotionale Erleben verbessern, während kulturelle Angebote, Bildungseinrichtungen oder gastronomische Infrastruktur eher mit langfristiger Zufriedenheit assoziiert sind. Lärmintensive und hochverdichtete städtische Räume stehen hingegen konsistent im negativen Zusammenhang mit affektivem Wohlbefinden.

Insgesamt belegt die empirische Forschung, dass affektives Wohlbefinden ein sensibles Produkt aus räumlichen, sozialen und situativen Faktoren ist. Dank \gls{esm} und \gls{gema} lassen sich diese komplexen Zusammenhänge inzwischen so erfassen, dass Stadtplanung und Public Health gezielt auf sie reagieren können – etwa durch die Förderung biodiversitätsreicher Grünflächen, sozial nutzbarer Rückzugsorte oder durch Massnahmen zur Lärmminderung an belasteten Verkehrsräumen \parencite{cookeMeasuringWellBeingReview2016}.

\subsection{Bedeutung intersektionaler Perspektiven auf affektives Wohlbefinden}

Die empirische Literatur zeigt überzeugend, dass räumliche Kontexte das affektive Wohlbefinden prägen. Weniger gut verstanden ist jedoch, wie sich diese situativen Gefühlslagen mit sozialen Machtachsen – etwa \textit{gender}, \textit{race} und \textit{class} – verschränken. Genau hier setzt die intersektionale Geographie an: Sie verknüpft emotionale Raumforschung mit der Analyse mehrfacher Ungleichheiten und fragt, wie Orte für verschiedene Körper zugleich befreiend oder belastend sein können \parencite{rodo-de-zarateIntersectionalityFeministGeographies2018}. 

Rodó‑de‑Zárate entwickelt dafür das Instrument der \textit{Relief Maps}, das soziale Positionierungen, geographische Orte und subjektives (Un‑)Behagen in einer Matrix zusammenführt \parencite{rodo-de-zarateDevelopingGeographiesIntersectionality2014}. Ihre Studien mit jungen lesbischen Frauen zeigen, dass Sicherheitsempfinden im öffentlichen Raum nicht im Ort selbst liegt, sondern in der relationalen Überlagerung räumlicher Strukturen mit sozialen Kategorien \textcite{rodo-de-zarateYoungLesbiansNegotiating2015}. In einer jüngeren Arbeit betont sie, dass Emotionen—verstanden als (Dis)comforts—räumliche Indikatoren für intersectionale Machtgeometrien sind: Wer welchen Platz „bewohnen“ kann, entscheidet sich an der Schnittstelle von Gefühl und Raum \parencite{rodo-de-zarateIntersectionalitySpatialityEmotions2023}. 

Diese Befunde machen deutlich, dass affektives Wohlbefinden räumlich ungleich verteilt ist. Es entsteht in situativen Atmosphären, die für marginalisierte Gruppen von Mikroaggressionen, Blicken oder architektonischen Barrieren durchzogen sein können \parencite{websterCenteringSocialtechnicalRelations2021}. Eine intersektionale Perspektive legt somit offen, warum dieselbe Grünfläche für eine Person Ruhe stiftet, für eine andere aber Stress auslöst. Sie erweitert die Forschung zu affektiven Atmosphären um eine machtanalytische Dimension und liefert praxisnahe Hinweise für eine Stadtplanung, die soziale Gerechtigkeit und emotionale Zugänglichkeit gleichermassen berücksichtigt.


