\section{Literaturübersicht}

\subsection{Einführung und Definition des Begriffs Wohlbefinden}

Wohlbefinden bezeichnet einen Zustand, der über die Abwesenheit von Krankheit oder Beschwerden hinausgeht und sowohl psychische als auch physische Dimensionen umfasst. Während physisches Wohlbefinden oft durch objektive Indikatoren wie Gesundheitsstatus oder körperliche Funktionen erfasst wird, bezeichnet psychisches Wohlbefinden häufig subjektive Empfindungen, wie Zufriedenheit, Glück oder positive Stimmungslagen \parencite{kahnemanDevelopmentsMeasurementSubjective2006a}. Im Kontext aktueller sozialwissenschaftlicher und gesundheitsbezogener Forschung gewinnt die Erfassung subjektiver Zustände zunehmend an Bedeutung, da diese die erlebte Lebensqualität und Gesundheit ganzheitlicher abbilden. Dabei unterscheidet die Forschung häufig zwischen langfristigem („Lebenszufriedenheit“) und kurzfristigem, sogenanntem momentanem Wohlbefinden, das die affektiven Zustände von Personen unmittelbar, während alltäglicher Situationen beschreibt \parencite{chenPerceivedUrbanEnvironment2025}.

Die präzise quantitative Erfassung von momentanen Wohlbefindenszuständen erfordert methodische Ansätze, die Erinnerungsverzerrungen und retrospektive Verzerrungen möglichst vermeiden. Retrospektive Bewertungen wie klassische Lebenszufriedenheits-Skalen sind stark anfällig für Verzerrungen durch kontextuelle Einflüsse oder die selektive Erinnerung besonders intensiver Momente \parencite{kahnemanDevelopmentsMeasurementSubjective2006a}. Daher etablierte sich in den letzten Jahrzehnten zunehmend die Nutzung von Methoden, die Wohlbefinden unmittelbar im Moment der Erfahrung messen.

\subsection{Quantitative Methoden zur Messung von momentanen Wohlbefindenszuständen}

Die am häufigsten eingesetzten quantitativen Ansätze zur Erfassung momentanen Wohlbefindens basieren auf der \acrfull{esm}, auch bekannt als \acrfull{ema}. Beide Ansätze erfassen Wohlbefinden in Echtzeit, indem Studienteilnehmende mehrmals täglich auf kurze standardisierte Fragen über ihr aktuelles emotionales Erleben reagieren. So nutzten beispielsweise \textcite{birenboimInfluenceUrbanEnvironments2018} ein \acrshort{ema}-basiertes Smartphone-Design, bei dem Teilnehmende über acht Monate hinweg viermal täglich Fragen zu vier spezifischen affektiven Dimensionen (Sicherheit, Glück, Komfort, Ärger) beantworteten. Die quantitative Erfassung erfolgte mittels kurzer numerischer Skalen, welche die aktuelle Intensität dieser Zustände unmittelbar abfragten.

Ähnlich arbeiteten \textcite{chenPerceivedUrbanEnvironment2025} in einer Studie im japanischen Kashiwa, wo mittels einer Smartphone-App momentanes Wohlbefinden zusammen mit kontextspezifischen Umgebungsmerkmalen erhoben wurde. Dabei wurden affektive Zustände explizit im Kontext verschiedener urbaner Settings (z.B. Cafés, öffentliche Grünflächen oder belebte Plätze) erfasst und anschliessend mittels Strukturgleichungsmodellen quantitativ analysiert. Diese Methode ermöglichte es den Forschenden, sowohl kurzfristige Effekte von Umgebungen auf das Wohlbefinden als auch langfristige Zusammenhänge differenziert darzustellen.

Eine methodische Weiterentwicklung stellt das sogenannte \acrfull{gema} dar. Dabei wird die klassische \acrshort{ema} mit räumlichen Kontextinformationen kombiniert, die durch GPS-Daten automatisiert erhoben werden. \textcite{kirchnerSpatiotemporalDeterminantsMental2016} beschreiben, dass dieser Ansatz es ermöglicht, nicht nur subjektive affektive Zustände, sondern auch deren situative und räumliche Kontextfaktoren explizit quantitativ zu modellieren. So konnten \textcite{mascherekMeadowsAsphaltRoad2025} mit einer \acrshort{gema}-basierten Studie in drei deutschen Grossräumen nachweisen, dass externe Faktoren wie Sonnenschein, soziale Begleitung und Mobilität messbar positive Effekte auf die momentane Stimmung haben. In dieser Studie wurden einfache numerische Skalen eingesetzt, um die Intensität der momentanen affektiven Zustände zu quantifizieren, während gleichzeitig automatisierte räumliche Kontextinformationen über GPS erfasst wurden.

Ein innovativer methodischer Ansatz findet sich auch in der Arbeit von \textcite{hammoudSmartphonebasedEcologicalMomentary2024}, die in der „Urban Mind“-Studie über 41,000 Einzelfälle analysierten. Neben klassischen affektiven Items wurde in dieser Studie zusätzlich eine selbstberichtete Einschätzung der lokalen Biodiversität erhoben, wobei Teilnehmende angaben, wie viele unterschiedliche natürliche Elemente (z.B. Vogelarten, Baumarten) sie aktuell wahrnehmen konnten. Die Ergebnisse zeigten dabei einen signifikanten Zusammenhang zwischen wahrgenommener Biodiversität und momentaner psychischer Befindlichkeit. Methodisch wurden hier gemischte lineare Modelle eingesetzt, die es erlaubten, komplexe Wechselwirkungen zwischen individuellen, situativen und räumlichen Variablen quantitativ abzubilden.

Insgesamt zeigt sich, dass quantitative Studien zum momentanen Wohlbefinden typischerweise einfache numerische Skalen verwenden, um die affektive Dimension präzise, aber dennoch niedrigschwellig zu erfassen. Dies dient einerseits dazu, die Belastung der Teilnehmenden gering zu halten und andererseits methodisch verlässliche und validierbare Daten zu erzeugen. Dennoch betonen \textcite{cookeMeasuringWellBeingReview2016}, dass bei der Auswahl der Messinstrumente sorgfältig auf deren theoretische und psychometrische Qualität geachtet werden muss, da viele Instrumente in ihrer Aussagekraft stark variieren.

Zusammenfassend lässt sich feststellen, dass die quantitative Erfassung von momentanen Wohlbefindenszuständen durch vielfältige methodische Ansätze geprägt ist. Während einfache EMA-Methoden primär auf subjektiven Selbstauskünften basieren, erlauben erweiterte Ansätze wie GEMA und die Integration weiterer sensorischer oder umweltbezogener Variablen eine deutlich differenziertere und räumlich-kontextuelle Analyse. Die statistische Auswertung dieser Daten erfolgt zunehmend mittels komplexer, hierarchisch verschachtelter Modelle, um intra- und interindividuelle Varianzen sowie zeitliche Dynamiken adäquat abzubilden. Trotz ihrer methodischen Diversität eint alle diese Ansätze das Ziel, Wohlbefinden möglichst unmittelbar, präzise und valide zu erfassen und damit ein tieferes Verständnis für die situativen Determinanten subjektiven Wohlbefindens zu ermöglichen.

\subsection{Quantitative Methoden zur Erfassung und Analyse von Intersektionalität}

Intersektionalität beschreibt das Zusammenspiel mehrerer sozialer Kategorien, wie Geschlecht, ethnische Zugehörigkeit oder soziale Schicht, und deren Rolle bei der Entstehung und Verstärkung sozialer Ungleichheiten. Ursprünglich ein theoretisches Konzept, wurde Intersektionalität zunächst überwiegend qualitativ untersucht. Erst in jüngerer Zeit werden zunehmend quantitative Ansätze genutzt, um intersektionale Zusammenhänge systematisch und empirisch zu erfassen \parencite{bauerIntersectionalityQuantitativeResearch2021}. Die Operationalisierung intersektionaler Ansätze stellt dabei spezifische methodische Herausforderungen dar, da klassische statistische Verfahren häufig nicht in der Lage sind, die komplexe, multidimensionale Natur intersektionaler Effekte angemessen abzubilden \parencite{bowlegInvitedReflectionQuantifying2016, scottIntersectionalityQuantitativeMethods2017}.

Die systematische Übersichtsarbeit von \textcite{bauerIntersectionalityQuantitativeResearch2021} zeigt, dass quantitative Analysen intersektionaler Zusammenhänge in den letzten Jahren stark zugenommen haben, gleichzeitig aber oft nicht ausreichend theoretisch fundiert sind. Viele quantitative Studien verwenden noch immer einfache statistische Verfahren, die Intersektionalität fälschlicherweise als additive oder einfache multiplikative Effekte sozialer Kategorien modellieren. Dadurch wird das eigentliche Kernprinzip intersektionaler Theorie, nämlich die kontextabhängige, dynamische Verschränkung sozialer Positionen, häufig vernachlässigt oder sogar fehlinterpretiert \parencite{bowlegInvitedReflectionQuantifying2016}.

Ein häufig eingesetzter methodischer Ansatz in der quantitativen Intersektionalitätsforschung sind multiple Regressionsmodelle, die sogenannte Interaktionsterme zwischen verschiedenen sozialen Kategorien verwenden, um deren kombinierten Einfluss abzubilden. Wie \textcite{scottIntersectionalityQuantitativeMethods2017} zeigen, kann dieser Ansatz zwar erste Einblicke in die Wechselwirkungen sozialer Kategorien bieten, doch weisen sie gleichzeitig darauf hin, dass diese Modelle oft nur begrenzt geeignet sind, komplexere und vor allem kontextspezifische Muster intersektionaler Benachteiligung zu erfassen. Besonders problematisch ist dabei, dass Interaktionsterme in linearen Modellen typischerweise additive Interpretationen nahelegen und in nichtlinearen Modellen, wie etwa logistischen Regressionen, komplexe multiplikative Effekte entstehen, deren Interpretation deutlich anspruchsvoller ist.

Um diesen methodischen Grenzen besser zu begegnen, setzen Forschende zunehmend auf Multilevel-Modelle. Diese Modelle bieten den Vorteil, dass sie mehrere Ebenen (zum Beispiel individuelle Merkmale und strukturelle Kontextfaktoren) gleichzeitig berücksichtigen können. Ein besonders vielversprechender Multilevel-Ansatz ist der sogenannte MAIHDA-Ansatz („Multilevel Analysis of Individual Heterogeneity and Discriminatory Accuracy“), der von \textcite{grossModellingIntersectionalityQuantitative2023} ausführlich diskutiert wird. Diese Methode erlaubt es, nicht nur Unterschiede zwischen verschiedenen intersektionalen Gruppen (wie beispielsweise „schwarze Frauen“ oder „weiße Männer“) abzubilden, sondern auch Unterschiede innerhalb dieser Gruppen statistisch zu modellieren. Dadurch können Forschende präziser analysieren, wie heterogen Erfahrungen innerhalb vermeintlich homogener intersektionaler Gruppen tatsächlich sind. Dies trägt wesentlich dazu bei, dass intersektionale Analysen sowohl theoretisch konsistenter als auch empirisch differenzierter durchgeführt werden können \parencite{grossModellingIntersectionalityQuantitative2023}.

Zusätzlich zu diesen Multilevel-Ansätzen bieten sich kausale Mediationsanalysen an, die \textcite{bauerAdvancingQuantitativeIntersectionality2019} als innovative Methode hervorheben. Diese Analysen ermöglichen es, direkte und indirekte Mechanismen intersektionaler Benachteiligungen genauer zu untersuchen. Bauer und Scheim (2019) beschreiben dabei insbesondere eine dreifache kausale Zerlegung („three-way decomposition“), die verwendet werden kann, um zu unterscheiden, ob Effekte bestimmter sozialer Kategorien (z.,B. Geschlecht oder Ethnizität) direkt, indirekt oder interaktiv auf Zielvariablen wie Gesundheit oder Diskriminierung wirken. Durch diese Methode wird es möglich, komplexe Kausalpfade intersektionaler Benachteiligungen sichtbar zu machen und dadurch ein tiefergehendes Verständnis für diese Prozesse zu entwickeln.

Neben regressions- und mediationsbasierten Verfahren werden explorative statistische Verfahren wie Entscheidungsbäume oder Clusteranalysen ebenfalls zunehmend eingesetzt, um intersektionale Datenstrukturen zu untersuchen \parencite{bauerIntersectionalityQuantitativeResearch2021}. Solche explorativen Verfahren bieten den Vorteil, dass sie theoretisch weniger vorstrukturierte Analysen erlauben und es Forschenden ermöglichen, komplexe und unerwartete intersektionale Muster empirisch sichtbar zu machen. Dennoch mahnen \textcite{bauerIntersectionalityQuantitativeResearch2021} an, dass gerade diese datengetriebenen Ansätze theoretische Aspekte der Intersektionalität nicht vernachlässigen dürfen und kritisch reflektiert eingesetzt werden sollten.

Trotz des methodischen Fortschritts gibt es weiterhin zentrale Herausforderungen bei der quantitativen Analyse intersektionaler Zusammenhänge. Viele quantitative Studien riskieren durch eine mangelnde theoretische Reflexion oder eine unzureichende methodische Fundierung, zentrale Prinzipien der Intersektionalität falsch oder nur oberflächlich umzusetzen. Insbesondere werden Intersektionalitätseffekte häufig auf einfache statistische Interaktionen reduziert, wodurch die theoretische und politische Dimension des Konzepts verloren gehen könnte \parencite{bowlegInvitedReflectionQuantifying2016,bauerIntersectionalityQuantitativeResearch2021}.

Um diese Herausforderungen künftig besser zu adressieren, empfehlen \textcite{bauerIntersectionalityQuantitativeResearch2021} sowie \textcite{bauerAdvancingQuantitativeIntersectionality2019} eine systematischere Verzahnung theoretischer und methodischer Perspektiven in der quantitativen Forschung. Notwendig sei eine klarere Berichterstattung sowie eine intensivere methodische Reflexion darüber, wie und warum bestimmte methodische Ansätze ausgewählt und implementiert wurden.

Zusammenfassend lässt sich feststellen, dass die quantitative Forschung zur Intersektionalität in den letzten Jahren deutliche Fortschritte gemacht hat, insbesondere durch die Entwicklung neuerer statistischer Verfahren wie MAIHDA und kausaler Mediationsanalyse. Gleichwohl bedarf es einer kontinuierlichen methodischen Weiterentwicklung sowie einer stärkeren theoretischen Reflexion quantitativer Forschungsansätze, um die komplexe, multidimensionale Natur intersektionaler Ungleichheiten präziser und umfassender zu erfassen.
