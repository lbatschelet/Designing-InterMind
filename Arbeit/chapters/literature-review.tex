\section{Quantitative Erfassung von moment­anem Wohlbefinden: Methoden und empirische Befunde}

Die quantitative Erfassung momentaner emotionaler Zustände im Alltag hat in den letzten zwei Jahrzehnten zunehmend Bedeutung erlangt. Anders als klassische retrospektive Bewertungen („Lebenszufriedenheit“) erfassen solche Echtzeitmethoden flüchtige affektive Zustände unmittelbar im Alltag und minimieren dadurch Erinnerungs- und Bewertungsverzerrungen \citep{kahnemanDevelopmentsMeasurementSubjective2006a}. Im Folgenden werden zentrale methodische Zugänge vorgestellt und deren Anwendung anhand empirischer Studien erläutert.

\paragraph{Experience Sampling Method (ESM) und Ecological Momentary Assessment (EMA).}
Die am weitesten verbreitete Methode zur Echtzeiterfassung ist das Experience Sampling (ESM), häufig auch Ecological Momentary Assessment (EMA) genannt. Hierbei beantworten Teilnehmende mehrmals täglich kurze, standardisierte Fragen zu ihrem aktuellen emotionalen Zustand. In einer umfangreichen Studie von \citet{birenboimInfluenceUrbanEnvironments2018} berichteten 91 Teilnehmende über einen Zeitraum von acht Monaten hinweg per Smartphone-App viermal täglich ihr momentanes Wohlbefinden in den Dimensionen Sicherheit, Glück, Komfort und Ärger. Methodisch typisch für EMA sind einfache, numerische Skalen oder Kurzitems zur Erfassung emotionaler Zustände. Ähnlich arbeiteten \citet{chenPerceivedUrbanEnvironment2025}, die mittels einer App in Japan Teilnehmer aufforderten, ortsbezogen (z.B. in Cafés oder Parks) Befragungen auszulösen, um situative Einflüsse auf momentanes Wohlbefinden zu erfassen.

\paragraph{Geographically Explicit Ecological Momentary Assessment (GEMA).}
GEMA erweitert die EMA-Methode durch das automatische Erfassen räumlicher Kontextdaten mittels GPS-Tracking. Kirchner und Shiffman (2016) erläutern, dass durch die Integration räumlicher Informationen eine detaillierte Modellierung möglich ist, bei der Umwelteinflüsse auf momentanes Wohlbefinden unmittelbar mit subjektiven Daten verknüpft werden können. Mascherek et al. (2025) nutzen diese Methode, um den Einfluss urbaner Grünräume, Wetterbedingungen und sozialer Interaktionen auf das Wohlbefinden in drei deutschen Städten zu untersuchen. Sie zeigen, dass dynamische Strukturgleichungsmodelle (DSEM) besonders geeignet sind, um solche komplex verschachtelten Datensätze auszuwerten, da sie zeitliche und räumliche Abhängigkeiten gleichzeitig berücksichtigen können.

\paragraph{Day Reconstruction Method (DRM).}
Die Day Reconstruction Method (DRM) bietet eine retrospektive Alternative zu EMA. Hier rekonstruieren Befragte ihren vorherigen Tag in Episoden und geben für jede Episode ihr affektives Befinden retrospektiv an. Obwohl die DRM retrospektiv ist, zeigen Kahneman und Krueger (2006), dass diese Methode Ergebnisse liefert, die stark mit denen von Echtzeit-Erhebungen übereinstimmen, aber mit deutlich geringerem Aufwand verbunden sind.

\paragraph{Sensorbasierte Kombinationen und Biodiversitätsindikatoren.}
Ein neuerer methodischer Trend kombiniert EMA mit sensorischen oder umweltbezogenen Zusatzdaten. \citet{hammoudSmartphonebasedEcologicalMomentary2024} verwendeten etwa Biodiversitätsindikatoren in ihrer Studie, um den Einfluss der Vielfalt natürlicher Elemente auf momentanes Wohlbefinden zu messen. Ihre Ergebnisse deuten darauf hin, dass eine höhere Artenvielfalt, wie etwa verschiedene Pflanzenarten und Vogelarten, das Wohlbefinden signifikant positiv beeinflussen.

\paragraph{Verwendete Messinstrumente.}
Die Auswahl geeigneter Messinstrumente ist entscheidend für die Validität momentaner Wohlbefindensmessungen. Laut einer umfassenden Übersicht von \citet{cookeMeasuringWellBeingReview2016} existieren zahlreiche Instrumente mit unterschiedlicher theoretischer Grundlage und unterschiedlicher Skalenlänge. Für EMA-Studien sind besonders ultrakurze Skalen oder Kurzformen bekannter Instrumente wie PANAS gebräuchlich, da sie die Belastung der Befragten minimieren.

\paragraph{Statistische Auswertungsmethoden.}
Zur statistischen Auswertung der durch EMA oder GEMA gewonnenen Daten kommen häufig hierarchische lineare Modelle (Multilevel Models) oder dynamische Strukturgleichungsmodelle (DSEM) zum Einsatz. Solche Methoden erlauben, intra- und interindividuelle Varianzen gleichzeitig zu modellieren und bieten zudem die Möglichkeit, komplexe zeitliche Abhängigkeiten und Interaktionen zwischen Variablen zu erfassen \citep{mascherekMeadowsAsphaltRoad2025, hammoudSmartphonebasedEcologicalMomentary2024, chenPerceivedUrbanEnvironment2025}.

\paragraph{Fazit und Forschungslücken.}
Insgesamt zeigen aktuelle methodische Ansätze, dass situative und räumliche Kontextvariablen oft mehr zur Erklärung momentanen Wohlbefindens beitragen als stabile individuelle Faktoren. Dennoch verbleiben methodische Herausforderungen, etwa hinsichtlich der optimalen Messfrequenz, des Datenschutzes bei GPS-Daten und der Generalisierbarkeit der Befunde. Zukünftige Forschungen könnten daher insbesondere auf diversere Stichproben und die langfristige Integration sensorischer Daten fokussieren, um die Robustheit und Übertragbarkeit der bisherigen Ergebnisse weiter zu prüfen.
