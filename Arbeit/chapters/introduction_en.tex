\section{Introduction}
“Cities are for everyone” – this notion of urban equality is often invoked in mission statements and planning strategies. It connects to debates inspired by the claim to the \emph{Right to the City} according to \textcite{lefebvreDroitVille1967}, although Lefebvre's original concept was far more radical and demanded a fundamental transformation of urban life. Regardless of the theoretical depth of this demand, however, the question arises: How do people actually experience urban space? And how does their social position – for example, regarding age, gender, origin, or health – influence their momentary well-being in specific environments? These questions are central to this Bachelor's thesis, which is dedicated to the intersectional analysis of immediate well-being in everyday living spaces.

Methods for capturing momentary psychological states and experiences in everyday life, such as \gls{esm} and particularly \gls{ema}, were already conceived in the 1990s, primarily in psychology \parencite[cf.][]{stoneEcologicalMomentaryAssessment1994, shiffmanEcologicalMomentaryAssessment2008}. They aimed to collect context-related data and overcome the disadvantages of purely retrospective approaches \parencite{kahnemanDevelopmentsMeasurementSubjective2006}. However, the full potential of these methods, especially for immediate, georeferenced data collection in real-time, only unfolded with the proliferation of smartphones. Furthermore, at the interface of urban planning and psychology, approaches for the spatially explicit capture of everyday experiences were developed, such as \gls{gema} \parencite[cf.][]{kirchnerSpatiotemporalDeterminantsMental2016}. Since around the mid-2010s, a significant increase in studies has been observed that utilize the smartphone-enhanced \gls{ema}/\gls{gema} possibilities to investigate the relationship between specific spatial environments and mental health or well-being in detail. An example of this is the large-scale project Urban Mind: The works by \textcite{bakolisUrbanMindUsing2018}, \textcite{bergouMentalHealthBenefits2022}, and \textcite{hammoudSmartphonebasedEcologicalMomentary2024} use this approach or its methodology, respectively, to analyze in particular the influence of green and urban spaces on mental health. These studies shape the current research trend of systematically examining situational affective responses in relation to spatial contexts.

In parallel, there is an extensive research literature on intersectionality and its spatial implications, significantly shaped by feminist and critical perspectives \parencite[cf.][]{crenshawMappingMarginsIntersectionality1991, rodo-de-zarateDevelopingGeographiesIntersectionality2014, rodo-de-zarateYoungLesbiansNegotiating2015, rodo-de-zarateIntersectionalityFeministGeographies2018}. These works illustrate how different social categories such as gender, class, or ethnic background are intertwined in spatial contexts and can create or reinforce inequalities. In particular, methodological innovations such as Relief Maps \parencite{rodo-de-zarateDevelopingGeographiesIntersectionality2014} allow for the visualization and analysis of these complex interactions.

This thesis connects these two perspectives: It utilizes the methodological possibilities of smartphone-based real-time data collection, as established in \gls{esm}/\gls{ema} research, but explicitly links them with intersectional inequality analysis. The focus thereby shifts from a purely 'ecological' perspective or a narrow definition of 'mental health' towards an investigation of \emph{situational affective well-being} in diverse everyday environments. It investigates how intersectional positionings specifically affect this situational well-being. The aim of this Bachelor's thesis is to explore the potential of this methodological adaptation and linkage within the framework of an exploratory pilot study: The intention is to examine whether and how this approach, at the intersection of feminist social and cultural geography, intersectionality research, and the analysis of digitally collected everyday data, can generate initial insights and hypotheses.

The personal motivation for this work stems from the desire to enable new insights into questions of social justice and well-being in everyday life using self-developed digital tools. Prospectively, the methodological approach tested here could serve in further work to connect socio-spatial questions with topics such as climate adaptation or mitigation. Such a connection could be relevant, for example, in the Bernese context, for research on urban heat \parencite[cf.][]{burgerModellingSpatialPattern2021} and for projects like the \emph{Bernometer}\footnote{\href{[https://bernometer.unibe.ch/bernometer/](https://bernometer.unibe.ch/bernometer/)}{bernometer.unibe.ch}}, which could be further expanded with detailed spatially referenced data on well-being. The focus is on the following research question:

\begin{quote}
\emph{How do spatial environments influence the momentary well-being of intersectionally positioned individuals in everyday life?}
\end{quote}

This explicitly does not concern long-term subjective well-being values, but rather the situational, affective reactions experienced in everyday life. The aim of the analysis is to investigate from an intersectional perspective under what conditions and in which places people feel a sense of belonging or alienation. Thus, it aims to explore how social positionings and spatial contexts interact and influence the momentary feeling of (non-)belonging. The analytical approach for the quantitative investigation of the underlying intersectional patterns is \gls{maihda} according to \textcite{grossModellingIntersectionalityQuantitative2023}.

To collect the data necessary for this work, the digital tool \emph{InterMind}\footnote{\href{[https://intermind.ch/app](https://intermind.ch/app)}{intermind.ch/app}} was developed. This app allows participants to be surveyed repeatedly over a defined period, and their responses, along with georeferenced information, to be captured and stored anonymously in real-time. For the present study, initial exploratory data are being collected within the framework of a pilot study with students from the University of Bern. The app itself was intentionally developed as Open Source to enable flexible adaptation to similar research contexts and potentially offer a sustainable infrastructure for contextualized everyday data.

The structure of the thesis is as follows: Chapter 2 discusses the theoretical framework regarding intersectionality, well-being concepts, and \gls{ema} methods. Chapter 3 describes the technical and methodological implementation, particularly the development of the app, the study design, and the analytical approach using \gls{maihda}. Chapter 4 presents key empirical findings on the relationship between well-being, space, and intersectional positioning. Finally, Chapter 5 discusses and contextualizes these findings and outlines implications for further research.

This work positions itself as an exploratory contribution that connects methodological innovations with socially relevant questions. It does not claim general representativeness but aims to reveal initial hypotheses and methodological potential for future intersectional analyses of momentary well-being in everyday living spaces.