% LTeX: language=de-CH

\section{Ergebnisse} \label{sec:ergebnisse}

\subsection{Beschreibung des Datensatzes}



\begin{figure}[htbp]
    \centering
    \includegraphics[width=10cm]{analysis/plots/survey_counts.pdf}
    \caption{Aufteilung nach Anzahl abgeschlossener Umfragen pro Person}
    \label{fig:survey_counts}
\end{figure}

\begin{figure}[htbp]
    \centering
    \includegraphics[width=5cm]{analysis/plots/cat_dist_indoors_outdoors.pdf}
    \caption{Aufteilung nach Drinnen / Draussen}
    \label{fig:cat_dist_indoors_outdoors}
\end{figure}

\begin{figure}[htbp]
    \centering
    \includegraphics[width=10cm]{analysis/plots/cat_dist_people_with_you.pdf}
    \caption{Aufteilung nach Personen um mich}
    \label{fig:cat_dist_people_with_you}
\end{figure}

\begin{figure}[htbp]
    \centering
    \includegraphics[width=10cm]{analysis/plots/cat_dist_activity.pdf}
    \caption{Aufteilung nach Aktivität}
    \label{fig:cat_dist_activity}
\end{figure}

\begin{figure}[htbp]
    \centering
    \includegraphics[width=10cm]{analysis/plots/cat_dist_location_category.pdf}
    \caption{Aufteilung nach Ort}
    \label{fig:cat_dist_location_category}
\end{figure}


\subsubsection{Umfang und demografische Merkmale}

\subsubsection{Qualitative Rückmeldungen der Teilnehmenden}

\subsection{Intersektionale Analysen am erhobenen Material}

\subsubsection{Exemplarische Analysen mittels MAIHDA}
\label{sec:pilot_maihda}

In einem Probelauf wurde geprüft, ob die vorliegenden Daten eine intersektionale Multilevel-Analyse nach dem \gls{maihda}-Ansatz zulassen. Konkret sollte untersucht werden, ob (a) genügend Beobachtungen je intersektionalem Stratum vorhanden sind und (b) die zwischenstratale Varianz gross genug ist, um stabile Random-Effects-Schätzungen zu erhalten.

\paragraph{Iterative Spezifikation der Strata}
Ausgangspunkt war ein Set über alle erhobenen Achsen: Biologisches Geschlecht, Soziales Geschlecht, Sexuelle Orientierung, Ausbildungsstufe, Gruppiertes Äquivalenz-Einkommen, Anstellungsverhältnis, Geburtsland, Vorhandene Behinderungen.

Die Kombination dieser Merkmale ergab 20 unterschiedliche Strata. Die Zellgrössen (Anzahl Personen pro Stratum) sind allerdings sehr klein (siehe \cref{tab:zellgroessen_alle_achsen}).

\begin{table}[h]
    \centering
    \begin{tabular}{rl}
        count & 20 \\
        mean & 1.25 \\
        std & 0.55 \\
        min & 1 \\
        max & 3 \\
    \end{tabular}
    \caption{Zellgrössen pro Stratum mit allen Achsen}
    \label{tab:zellgroessen_alle_achsen}
\end{table}

Mit diesem Set von Strata ist die Modellierung nicht möglich, da die Zellgrössen zu klein sind und eine gute Schätzung der Varianzanteile nicht möglich ist.

Um die Modellierbarkeit zu erhöhen, wurde das Stratum anschliessend auf zwei theoretisch zentrale Achsen reduziert: Biologisches Geschlecht und Alter.

Dies führte zu insgesamt $6$ Strata, mit folgenden Zellgrössen:

\begin{table}[h]
    \centering
    \begin{tabular}{rl}
        count & 6 \\
        mean & 4.17 \\
        std & 4.7 \\
        min & 1 \\
        max & 12 \\
    \end{tabular}
    \caption{Zellgrössen pro Stratum mit reduzierten Achsen}
    \label{tab:zellgroessen_reduzierte_achsen}
\end{table}

Auch hier gibt es noch einzelne Strata mit weniger als 3 Beobachtungen.

\begin{table}[h]
    \centering
    \begin{tabular}{lll}
        Gender & Altersgruppe & Anzahl \\
        \hline
        man & 16 – 25 & 2 \\
        trans man & 56 – 65 & 1 \\
        man & missing & 1 \\
        woman & 26 – 35 & 1 \\
    \end{tabular}
    \caption{Strata mit weniger als 3 Beobachtungen}
    \label{tab:zellgroessen_reduzierte_achsen}
\end{table}

Trotzdem wurde versucht, mit diesem Set von Strata eine MAIHDA-Analyse durchzuführen.

Für jedes kontinuierliche Outcome (\texttt{sense\_of\_belonging}, \texttt{environmen\_pleasure}, \texttt{environment\_lively}, \texttt{environment\_nature}, \texttt{environment\_noise}) wurde ein zweistufiges MAIHDA-Setting geschätzt:
\begin{description}
    \item[Modell~1A (Nullmodell):] Zufallsinterzept auf Stratum-Ebene, keine festen Effekte der Achsen.
    \item[Modell~1B (Additives Modell):] Zusätzlich feste Haupteffekte von \texttt{age\_group} und \texttt{gender}; der verbleibende Stratum-Random-Effect wird als Interaktionsanteil interpretiert.
\end{description}
Aufgrund der geringen Zellgrössen wurde kein zusätzlicher Random-Intercept auf Personenebene modelliert. Die Schätzung erfolgte mittels \texttt{statsmodels.mixedlm} (MLE, Optimierer \texttt{lbfgs}).


\subsection{Varianzzerlegung (VPC, PCV)}
Die geschätzten Varianzanteile zwischen Strata (Variance Partition Coefficient, VPC) waren durchgängig extrem klein. Beispielhaft:

\begin{center}
\begin{tabular}{lrrrr}
\toprule
Outcome & VPC$_{\text{Null}}$ & VPC$_{\text{add}}$ & PCV & $n$ (Zeilen) \\
\midrule
\texttt{sense\_of\_belonging}      & $1.24\times 10^{-5}$ & $1.15\times 10^{-6}$ & 90.7\% & 106 \\
\texttt{environmen\_pleasure}      & $\approx 0$          & $4.40\times 10^{-7}$ & --      & 106 \\
\texttt{environment\_lively}       & $2.95\times 10^{-4}$ & $3.43\times 10^{-5}$ & 88.2\%  & 106 \\
\texttt{environment\_nature}       & $1.03\times 10^{-2}$ & $1.12\times 10^{-6}$ & 99.99\% & 106 \\
\texttt{environment\_noise}        & $1.95\times 10^{-2}$ & $2.98\times 10^{-5}$ & 99.85\% & 106 \\
\bottomrule
\end{tabular}
\end{center}

Die nahezu Null liegenden VPCs belegen, dass (a) die Outcomes sich zwischen den Strata kaum unterscheiden und (b) die Stratum-Varianz im Modell auf Null \emph{geschrumpft} wird. PCV-Werte sind bei einem praktisch Null-VPC im Nullmodell numerisch instabil (z.\,B.\ negative oder extrem grosse Werte) und daher nicht interpretierbar.

\subsection{Schlussfolgerung}
Die Pilotanalyse zeigt, dass mit den vorliegenden Daten keine sinnvolle MAIHDA-Varianzzerlegung durchführbar ist. Gründe:
\begin{enumerate}
    \item \textbf{Zu kleine Strata-Zellgrössen}: Die meisten Strata enthalten nur eine Person bzw.\ sehr wenige Beobachtungen.
    \item \textbf{Geringe zwischenstratale Varianz}: Die betrachteten Outcomes variieren kaum zwischen den (reduzierten) Strata.
    \item \textbf{Numerische Instabilität}: Die Random-Effects-Kovarianzmatrix wird singular; die Schätzung kollabiert auf Randlösungen.
\end{enumerate}

\paragraph{Implikation für die weitere Analyse.}
Für die Beantwortung der Forschungsfrage (Einfluss situativer Umweltfaktoren auf affektives Wohlbefinden) bietet es sich an, die Umweltvariablen als Level-1-Prädiktoren in einem vereinfachten Modell (z.\,B.\ lineares Modell mit cluster-robusten Standardfehlern nach Person oder ein Mixed Model nur mit Personen-Random-Intercept) zu analysieren. Intersectionale Unterschiede können vorerst über feste Effekte (z.\,B.\ \texttt{C(age\_group)}, \texttt{C(gender)}) kontrolliert werden. Eine vollwertige MAIHDA-Anwendung ist erst mit grösserer Stichprobe und ausreichenden Zellgrössen pro Stratum sinnvoll.

\subsubsection{Illustration möglicher Zusammenhänge zwischen Umwelt und Wohlbefinden}

\subsection{Interpretation der explorativen Befunde}
