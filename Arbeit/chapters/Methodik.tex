% LTeX: language=de-CH

\section{Ein eigener Zugang -- methodisch und angewandt} \label{sec:methodik}

In diesem Kapitel wird der methodische Zugang dieser Arbeit im Rahmen bestehender Ansätze zur Erhebung situativer Daten verortet. Dabei geht es zunächst um die begriffliche Einordnung der verwendeten Erhebungslogik und ihre Abgrenzung gegenüber verwandten Verfahren. Anschliessend werden bestehende digitale Werkzeuge vorgestellt, die ähnliche Zielsetzungen verfolgen. Die vergleichende Analyse dient dazu, Gemeinsamkeiten, Unterschiede und Leerstellen sichtbar zu machen und die eigene Herangehensweise im Anschluss präzise zu positionieren.


Die konkreten technischen und inhaltlichen Umsetzungen -- etwa die Entwicklung der App (\cref{sec:entwicklung_app}) oder die Gestaltung des Fragebogens (\cref{sec:fragebogenentwicklung}) -- werden in den folgenden Kapiteln ausführlich dargestellt.


\subsection{Situationen erfassen -- Wiederholte Befragung mit \acrshort{esm}, \acrshort{ema} und \acrshort{gema}}

Die systematische Erhebung von affektivem Wohlbefinden erfordert Methoden, die subjektive Erfahrungen möglichst unmittelbar und kontextspezifisch erfassen. Retrospektive Selbstauskünfte sind hierfür nur begrenzt geeignet, da sie Verzerrungen durch selektive Erinnerung oder nachträgliche Neubewertung unterliegen \parencite[\textit{Recall Bias},][]{kahnemanDevelopmentsMeasurementSubjective2006}. Um solche Verzerrungen zu vermeiden, wurde bereits in den 1980er-Jahren die \emph{\acrfull{esm}} entwickelt. Dieses Verfahren basiert auf der mehrfach wiederholten Erhebung subjektiver Zustände im Alltag -- etwa durch zeitlich zufällig verteilte Aufforderungen an Teilnehmende, ihre momentane Stimmung, Tätigkeit oder Umgebung zu protokollieren \parencite{csikszentmihalyiValidityReliabilityExperienceSampling1987}. Ziel ist es, das Erleben möglichst nah am Zeitpunkt der Erfahrung und im natürlichen Kontext zu erfassen. Typisch für \acrshort{esm} sind kurze, wiederholte Abfragen zu spezifischen psychologischen Konstrukten, die Verzerrungen minimieren und einen Einblick in die dynamischen Prozesse individuellen Erlebens erlauben.

Während \acrshort{esm} ursprünglich primär als psychologisches Messinstrument konzipiert war, wurde der Ansatz in den 1990er-Jahren durch die \emph{\acrfull{ema}}\footnote{Der Begriff \textit{ecological} verweist hierbei nicht auf natürliche Umgebungen, sondern auf die Wechselbeziehungen zwischen Lebewesen und ihrer jeweiligen Umwelt -- unabhängig davon, ob diese natürlich, sozial oder technisch geprägt ist.} methodologisch erweitert. \acrshort{ema} bezeichnet die unmittelbare Erhebung subjektiven Erlebens, erweitert diese jedoch explizit um physiologische, verhaltensbezogene und kontextuelle Daten -- etwa mittels mobiler Geräte, integrierter Sensorik oder digitaler Tagebuchsysteme \parencite{shiffmanEcologicalMomentaryAssessment2008}. \acrshort{ema} erlaubt dadurch eine umfassendere Erfassung individueller Zustände und deren Kontextbedingungen. Im Gegensatz zu \acrshort{esm} ist \acrshort{ema} damit methodologisch offener für die Integration verschiedenster Datenquellen und Analyseebenen.

Mit der zunehmenden Verbreitung von \gls{gps}-fähigen Endgeräten wurde \acrshort{ema} in den 2010er-Jahren durch das Konzept der \emph{\acrfull{gema}} ergänzt. \acrshort{gema} kombiniert subjektive Momentaufnahmen mit objektiven, räumlich verortbaren Kontextinformationen wie Standort, Wetterbedingungen, Lärmpegel oder Bebauungsstruktur \parencite{kirchnerSpatiotemporalDeterminantsMental2016}. Im Unterschied zu \acrshort{ema} legt \acrshort{gema} besonderen Wert auf die explizite räumliche Kontextualisierung der erhobenen Daten. Dabei werden subjektive Erfahrungen nicht nur als zeitlich-situativ, sondern explizit als räumlich-situiert betrachtet. Entscheidend ist hierbei die Möglichkeit, affektives Erleben in direkten Bezug zum spezifischen räumlich-materiellen Kontext zu setzen und dadurch differenzierte Aussagen über räumliche Einflüsse auf das Erleben zu ermöglichen. \acrshort{gema} erlaubt dadurch eine komplexere Analyse der Wechselwirkungen zwischen individuellen Erfahrungen und räumlicher Umgebung und öffnet die methodologische Perspektive für interdisziplinäre, insbesondere geographische Fragestellungen.

Die vorliegende Arbeit folgt diesem methodischen Paradigma. Ziel ist es, situativ-affektive Zustände im Raum nicht nur als individuelle, sondern explizit als kontextuell-räumlich bedingte Erfahrungen zu erfassen. Zu diesem Zweck wurde eine eigene Smartphone-Applikation (\gls{intermind}) entwickelt, die Teilnehmende mehrmals täglich auffordert, eine kurze Selbsteinschätzung ihres momentanen Wohlbefindens und ihrer Umgebung vorzunehmen. Gleichzeitig werden automatisiert Geodaten gespeichert, sodass jede Beobachtung in ihrer konkreten räumlichen Verortung analysiert werden kann. Anders als bei klassischen \acrshort{gema}-Studien, die häufig spezifische Umweltmerkmale quantifizieren, liegt der Fokus der vorliegenden Arbeit auf einer relationalen Betrachtung von Raum und subjektivem Erleben.

Die Entscheidung für ein solches Studiendesign bringt gegenüber querschnittbasierten Verfahren mehrere methodische Vorteile mit sich. Erstens reduziert die wiederholte intraindividuelle Erhebung Verzerrungen durch retrospektive Einschätzungen und erlaubt eine präzisere Erfassung situativer Schwankungen \parencite{randallDevelopmentTrialMobile2013}. Zweitens ermöglicht sie eine Kontrolle individueller Basisniveaus, was insbesondere für intersektionale Analysen relevant ist, die sowohl zwischen als auch innerhalb von Personen Differenzierungen vornehmen. Drittens erlaubt die Kombination von Echtzeitbefragung und Geodatenanalyse eine kontextsensitive Modellierung der Beziehungen zwischen affektivem Zustand und Umgebung -- im Sinne eines relationalen, ökologisch verstandenen Raumbegriffs \parencite{mascherekMeadowsAsphaltRoad2025}.


\subsection{Anknüpfen und Abgrenzen -- Vergleich mit bestehenden Instrumenten}

Die im Rahmen dieser Arbeit entwickelte App bewegt sich im Spannungsfeld zweier methodischer Herangehensweisen: der Echtzeiterhebung räumlich kontextualisierter affektiver Zustände (wie bei \gls{urbanmind}) und der explizit intersektionalen Analyse subjektiver Raumwahrnehmungen (wie bei \textit{Relief Maps+}). Beide bestehenden Instrumente bilden zentrale Referenzpunkte für die Konzeption des eigenen Ansatzes, da sie jeweils zentrale Teilaspekte adressieren: Während \gls{urbanmind} eine räumlich verortete Echtzeiterhebung subjektiven Wohlbefindens umsetzt, fokussiert \textit{Relief Maps+} auf eine reflexive, intersektionale Kartierung räumlicher Erfahrung.

Die Auswahl dieser beiden Werkzeuge erfolgte zum einen aufgrund ihrer inhaltlichen Nähe zum eigenen Untersuchungsinteresse, zum anderen auch aus praktischer Zugänglichkeit: Zum Zeitpunkt des Projektstarts war \gls{urbanmind} eines der wenigen öffentlich zugänglichen \acrshort{gema}-Tools, das bereits in wissenschaftlichen Studien eingesetzt wurde.\footnote{Die Dokumentation eines weiteren vielversprechenden Tools \textit{\acrfull{health}} \parencite{wrayHealthyEnvironmentsActive2025} wurde während der Entstehung dieser Arbeit als Preprint veröffentlicht.} \textit{Relief Maps+} wiederum ist der einzige bekannte Ansatz, der intersektionale Raumwahrnehmungen systematisch operationalisiert und erschien durch seine Kombination aus Emotionalität, Raumbezug und Identitätsachsen besonders anschlussfähig für das vorliegende Projekt.

Der folgende Vergleich dient dazu, methodische Gemeinsamkeiten und Unterschiede herauszuarbeiten und den eigenen methodischen Zugang klar zu positionieren.


\subsubsection*{\gls{urbanmind}: Ein vielseitiges, aber nicht quelloffenes Werkzeug}

\gls{urbanmind} ist ein exemplarisches Werkzeug zur Anwendung von \acrshort{gema}: Es kombiniert standardisierte Echtzeiterhebungen subjektiven Wohlbefindens mit automatisiert erfassten Geodaten und erlaubt so die kontextsensitive Analyse psychischer Gesundheit im Alltag \parencite{bakolisUrbanMindUsing2018}. Die zugrunde liegende Smartphone-App wird in verschiedenen Studien eingesetzt und kann flexibel an unterschiedliche Forschungsfragen angepasst werden.

\gls{urbanmind} wurde in mehreren Studien eingesetzt, um Zusammenhänge zwischen Umweltfaktoren und psychischer Gesundheit zu analysieren: So zeigten \textcite{bakolisUrbanMindUsing2018}, dass natürliche Elemente wie Himmel, Wasser oder Grünflächen kurzfristig das Wohlbefinden steigern können, \textcite{bergouMentalHealthBenefits2022} belegten vergleichbare Effekte für Aufenthalte an Flüssen und Kanälen, \textcite{hammoudLonelyCrowdInvestigating2021} identifizierten Zusammenhänge zwischen sozialer Dichte, dem Gefühl sozialer Inklusion und situativer Einsamkeit, und \textcite{hammoudSmartphonebasedEcologicalMomentary2022} fanden Hinweise darauf, dass Vögel die psychische Verfassung -- auch bei Personen mit Depressionen -- nachhaltig verbessern können.

Während diese Studien wichtige Beiträge zur Analyse kontextueller Einflüsse auf psychisches Wohlbefinden leisten, bleibt eine explizit intersektionale Perspektive bislang unberücksichtigt. Zwar erlaubt die App die Erfassung zentraler demografischer Merkmale, dieses Potenzial wurde in den vorliegenden Auswertungen jedoch nicht genutzt -- obwohl entsprechende Analysen innerhalb der bestehenden Infrastruktur prinzipiell möglich wären.


\begin{figure}[h]
    \centering
    \begin{minipage}[t]{0.38\textwidth}
        \centering
        \includegraphics[width=\textwidth]{Arbeit/images/urban_mind01.jpeg}
        \caption{Screenshot einer typischen Frageseite aus der \gls{urbanmind}-App}
        \label{fig:urban_mind_screenshot_1}
    \end{minipage}
    \hspace{0.1\textwidth}
    \begin{minipage}[t]{0.38\textwidth}
        \centering
        \includegraphics[width=\textwidth]{Arbeit/images/urban_mind_report.jpg}
        \caption{Screenshot eines individuellen Reports aus der \gls{urbanmind}-App}
        \label{fig:urban_mind_report}
    \end{minipage}
\end{figure}

\gls{urbanmind} zeichnet sich durch eine einfache und ansprechend gestaltete Benutzeroberfläche aus, die eine niedrige Einstiegshürde für die Teilnehmenden bietet (siehe \cref{fig:urban_mind_screenshot_1}). Im Mittelpunkt stehen kurze, tägliche Befragungen, die etwa drei Minuten dauern und die Teilnehmenden abhängig von der konkreten Studie \gls{bspw} zu ihrem momentanen Wohlbefinden, aktuellen Tätigkeiten sowie ihrer direkten räumlichen und sozialen Umgebung befragen. Diese Befragungen in den meisten Studien drei Mal täglich über eine Dauer von zwei Wochen und werden via Push-Benachrichtigung ausgelöst. Teilnehmende haben jeweils eine Stunde Zeit, um die Befragung abzuschliessen.

Zusätzlich zu den standardisierten Fragebogen-Items erfasst die App kontinuierlich Standortdaten mittels \gls{gps} sowie optional Gesundheits- und Aktivitätsdaten (\gls{zb} Schrittzahl, zurückgelegte Distanzen), sofern die Teilnehmenden diese Datenerfassung explizit freigeben. Weiter bietet \gls{urbanmind} die Möglichkeit, kurze Audioaufnahmen und Fotografien zu teilen. Diese Mediendateien können nicht nur für wissenschaftliche Analysen, sondern auch für künstlerische Zwecke und Öffentlichkeitsarbeit verwendet werden.

Diese Praxis wirft jedoch kritische Fragen hinsichtlich Datenschutz und informierter Einwilligung auf -- insbesondere da besonders sensible Daten wie kontinuierliche Standortverläufe und Gesundheitsinformationen betroffen sind. Hinzu kommt, dass die Teilnehmenden ihre Zustimmung nicht differenziert nach Verwendungszweck (\gls{zb} Forschung, Kunst, Social Media) geben können, sondern pauschal für alle vorgesehenen Nutzungen. Informationen zur tatsächlichen Verwendung der Daten sind zudem nicht durchgängig transparent oder direkt in der App zugänglich, sondern teilweise nur über ergänzende Webseiten auffindbar.

Eine weitere Besonderheit der App sind individuelle Reports, die Teilnehmenden automatisch und grafisch ansprechend Rückmeldungen über ihre Interaktionen mit der Umwelt geben. So wird \gls{bspw} am Ende der Studiendauer dargestellt, bei wie vielen Befragungen die Teilnehmenden in Kontakt mit natürlichen Elementen waren und wie sich dies auf verschiedene Aspekte des persönlichen Wohlbefindens auswirkte (siehe \Cref{fig:urban_mind_report}). Dies dient sowohl der Reflexion über das eigene Alltagsverhalten als auch der Motivation, längerfristig an der Studie teilzunehmen.

Trotz seiner vielseitigen und benutzerfreundlichen Gestaltung weist \gls{urbanmind} einige Einschränkungen auf: Teilnehmende haben \gls{bspw} keine Möglichkeit, ihre erhobenen Rohdaten direkt zu exportieren, und auch die Löschung persönlicher Daten erfordert den expliziten Kontakt mit dem jeweiligen Forschungsteam. Zudem ist der Quellcode der App nicht öffentlich zugänglich -- eine unabhängige Prüfung oder Weiterentwicklung der technischen Infrastruktur ist somit nicht möglich.

Gerade dieser Mangel an Transparenz und Offenheit markiert eine zentrale Leerstelle im bestehenden Tool-Ökosystem -- und bildet einen wesentlichen Ausgangspunkt für die hier entwickelte Anwendung.

\subsubsection*{Relief Maps+: Reflexive und intersektionale Kartierung retrospektiver Erfahrungen}

Im Unterschied zu \gls{urbanmind} verfolgt \gls{reliefmaps}\footnote{Siehe \href{https://reliefmaps.upf.edu/}{reliefmaps.upf.edu}} einen qualitativ-reflexiven Ansatz, der retrospektiv subjektive Erfahrungen intersektional positioniert sichtbar macht. Aufbauend auf der ursprünglichen Version der „Relief Maps“ \parencite{rodo-de-zarateDevelopingGeographiesIntersectionality2014} integriert die digitale Anwendung drei miteinander verschränkte Dimensionen -- geografische Orte, soziale Identitäten und emotionale Bewertungen -- und legt dabei besonderen Wert auf die Förderung individueller Selbstreflexion und kollektiver Sichtbarmachung diskriminierender Raumstrukturen. % wenn ich das so schreibe muss ich das auch begründen, der letzte halbsatz ist sehr krass so formuliert

Zu Beginn des Erhebungsprozesses erstellen Nutzer\genderstern innen einen Avatar auf Basis intersektional relevanter Merkmale wie \gls{gender}, Sexualität, \gls{class}, Herkunft, Körperbild oder (Dis-)Ability. Darauf aufbauend reflektieren sie in mehreren Schritten über emotionale Erfahrungen in verschiedenen Raumkategorien wie „öffentliche Räume“, „Gesundheitseinrichtungen“ oder „virtuelle Räume“ (siehe \cref{fig:relief_maps_plus_screenshot_1}). Für jede Achse sozialer Positionierung können in einem nächsten Schritt Orte je nach erfahrenem (Un-)Wohlsein als unterdrückend, kontrovers, neutral oder entlastend klassifiziert werden. Ergänzend können Orte direkt auf einer Karte verortet und mit freien Kommentaren sowie Emotionslabels wie „Angst“, „Sicherheit“ oder „Empowerment“ versehen werden. Diese Funktion fördert eine dichte, kontextualisierte Beschreibung subjektiver Erlebnisse, die sich nicht auf standardisierte Itemskalen reduzieren lässt.

\begin{figure}[htbp]
    \centering
    \includegraphics[width=\textwidth]{Arbeit/images/reliefmap.png}
    \caption{Beispielhafte Ausgabe aus dem Relief Maps+ Tool}
    \label{fig:relief_maps_plus_screenshot_1}
\end{figure}

Ein zentrales methodisches Merkmal von Relief Maps+ ist der Versuch, die emotionale Wirkung sozialer Machtverhältnisse räumlich darstellbar zu machen -- ohne diese in eindimensionale Kausalbeziehungen zu überführen. Die Nutzer\genderstern innen bewerten ihre Erfahrungen explizit entlang einzelner \glspl{identitaetsachse}. Gleichzeitig zeigt sich hier eine zentrale methodologische Spannung: Die isolierte Betrachtung einzelner Diskriminierungsachsen widerspricht dem Grundgedanken intersektionaler Analyse, der gerade auf die Verwobenheit und Gleichzeitigkeit verschiedener Machtverhältnisse verweist. Eine konsequente intersektionale Operationalisierung bleibt damit methodisch herausfordernd.

Einige technische Merkmale von Relief Maps+ sind auch im Hinblick auf die Entwicklung eigener Tools relevant. Die browserbasierte Anwendung erlaubt es Forschenden, eigenständig Projekte zu erstellen und auszuwerten. Allerdings ist der Zugang derzeit stark auf den katalanischen Kontext zugeschnitten: Verfügbare Sprachen sind Katalanisch, Spanisch und Englisch; Optionen zur Erweiterung oder Lokalisierung sind nicht dokumentiert. Da der Quellcode nicht öffentlich zugänglich ist, bleiben Fragen zur Anpassbarkeit, Wiederverwendbarkeit und langfristigen Wartbarkeit offen. Aus methodischer Sicht stellt sich somit die Frage, inwiefern die Software übertragbar ist auf andere sprachliche, kulturelle und geografische Kontexte.

Trotz dieser Einschränkung eröffnet Relief Maps+ wichtige Potenziale: Die bewusste Integration von Reflexivität, die aktive Beteiligung der Nutzer\genderstern innen an der Interpretation ihrer eigenen Erfahrungen sowie die Sichtbarmachung räumlich kontextualisierter Ungleichheiten markieren einen innovativen Zugang für intersektionale, subjektzentrierte Geographien. Die methodische Fundierung des Tools beruht auf einem iterativen Validierungsprozess unter Einbezug feministischer, queerer und dekolonialer Perspektiven \parencite{luizdesouzaSpiralValidationProcess2025}.



\subsection{Mehr Infrastruktur als Innovation}

Die im Rahmen dieser Arbeit entwickelte App \gls{intermind} versteht sich als offen zugängliches und flexibel einsetzbares Werkzeug für Studien im Rahmen der \acrshort{gema}. Ziel war es, eine technisch eigenständige, quelloffene Infrastruktur bereitzustellen, die eine situative, geolokalisierte Erhebung affektiven Wohlbefindens ermöglicht -- ein Instrument, das in dieser Form bislang nicht allgemein verfügbar war. Die Entwicklung orientierte sich in Teilen an bestehenden Tools wie \gls{urbanmind}, insbesondere was das Interface-Design und die Nutzerführung betrifft, basiert jedoch auf einer unabhängig konzipierten Codebasis und wurde vollständig neu implementiert.

Die App selbst ist methodisch nicht innovativ im engeren Sinne, sondern stellt eine robuste, anpassbare Plattform dar, die für verschiedenste \acrshort{gema}-Studien konfiguriert werden kann. Ihr modularer Aufbau erlaubt die Integration beliebiger Fragebögen und Fragetypen -- einschliesslich Freitextfeldern, Schiebereglern oder Mehrfachantworten. Damit kann das System flexibel an unterschiedliche Forschungskontexte angepasst und in zukünftigen Studien weiterverwendet werden.

Im Zentrum der vorliegenden Arbeit steht ein spezifisch entwickelter Fragebogen, der auf der App zum Einsatz kommt. Dieser kombiniert klassische \gls{ema}-Items zur situativen Erfassung von Kontext und Wohlbefinden mit explizit intersektional angelegten Fragen. Dabei werden Dimensionen wie Geschlecht, Herkunft oder sozioökonomischer Status getrennt erfasst -- ein Ansatz, der zwar theoretisch nicht vollständig der Idee intersektionaler Verwobenheit entspricht, aber eine quantitative Auswertbarkeit ermöglicht. Gleichzeitig bleibt durch offene Antwortformate Raum für reflexive Auseinandersetzung mit der eigenen Erfahrung in konkreten räumlichen Situationen.

