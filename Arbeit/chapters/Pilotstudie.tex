\chapter{Pilotstudie}
\label{sec:pilotstudie}

Dieses Kapitel prüft, ob das in dieser Arbeit entwickelte Erhebungsinstrument und der dazugehörige Fragebogen (\gls[noindex]{vgl} \cref{sec:entwicklung_app,sec:fragebogenentwicklung}) geeignet ist, Daten zu generieren, die sich für eine intersektional-quantitative Analyse nutzen lassen.

Als Testfall dient die folgende Überprüfungsfrage:
\begin{quote}
\emph{Wie beeinflussen räumliche Umgebungen das momentane Wohlbefinden intersektional positionierter Personen im Alltag?}
\end{quote}

Die Frage ist bewusst allgemein formuliert, da sie in dieser Pilotstudie nicht vollständig beantwortet, sondern methodisch erprobt wird. Ziel ist es zu untersuchen, ob die erhobenen Daten eine statistische Auswertung grundsätzlich zulassen und welche praktischen, technischen und konzeptionellen Herausforderungen dabei sichtbar werden.

\section{Stichprobe}

Die Datenerhebung fand im Rahmen der einführenden Exkursion Recht auf Stadt im ersten Studienjahr des Bachelorstudiengangs Geographie an der Universität Bern im Mai 2025 statt. Zu Beginn jedes der insgesamt vier Exkursionstage erfolgte eine Einladung zur freiwilligen Teilnahme an der Studie -- beim ersten Termin von mir persönlich, an den folgenden Terminen durch die Exkursionsleitenden. Für jede teilnehmende Person begann die Erhebungsphase mit einer einmaligen Baseline-Befragung und dauerte ab diesem Zeitpunkt sieben Tage.

\subsection*{Demographische Daten aus der Baseline Befragung}

Insgesamt wurden rund \num{80} Personen zur Teilnahme eingeladen. \num{24} davon schlossen mindestens die einmalige Baseline-Befragung vollständig ab und wurden in die Stichprobe aufgenommen. \num{8} begonnene, aber nicht abgeschlossene Baseline-Befragungen wurden ausgeschlossen. Ebenfalls ausgeschlossen wurden \num{6} während der Erhebungsphase begonnene, aber nicht abgeschlossene Momentaufnahmen. Die endgültige Stichprobe umfasst somit \num{25} Personen. \cref{tab:kreuztabelle_abs} zeigt die Verteilung von sozialem Geschlecht und Altersgruppe.

\begin{table}[h]
    \centering
    \caption{Kreuztabelle: Soziales Geschlecht und Altersgruppe (absolute Häufigkeiten)}
    \label{tab:kreuztabelle_abs}
    \begin{tabular}{lccccS}
    \toprule
    \textbf{Geschlecht} & 16--25 & 26--35 & 56--65 & Keine Angabe & {Gesamt} \\
    \midrule
    Mann       & 12 &  2 & 0 & 1 & 15 \\
    Trans Mann &  0 &  0 & 1 & 0 &  1 \\
    Frau       &  8 &  1 & 0 & 0 &  9 \\
    \midrule
    \textbf{Gesamt} & 20 & 3 & 1 & 1 & 25 \\
    \bottomrule
    \end{tabular}
\end{table}

Die Mehrheit der Teilnehmenden verfügt über eine \emph{Matura oder ein gleichwertiges Abschlusszeugnis} (\num{23}; \SI{92}{\percent}), zwei Personen (\SI{8}{\percent}) besitzen einen Hochschulabschluss. Der überwiegende Teil ist als \emph{Student\genderstern in oder Schüler\genderstern in} erwerbstätig (\num{22}; \SI{88}{\percent}), drei Personen (\SI{12}{\percent}) sind angestellt. Die grosse Mehrheit wurde im gleichen Land geboren, in dem sie derzeit lebt (\num{17}; \SI{68}{\percent}), \num{7} Personen (\SI{28}{\percent}) nicht; eine Person (\SI{4}{\percent}) machte keine Angabe.  
Alle Personen gaben keine vorhandene Behinderung an (\num{25}; \SI{100}{\percent}).

Bezüglich der sexuellen Orientierung gaben \num{17} Personen (\SI{68}{\percent}) \emph{hetero} an, jeweils drei (\SI{12}{\percent}) \emph{homosexuell} oder \emph{bisexuell}, und je eine Person (\SI{4}{\percent}) \emph{asexuell} bzw. \emph{queer}. 

Beim gruppierten Äquivalenzeinkommen entfallen \num{8} Personen (\SI{32}{\percent}) auf die Kategorie \emph{Sehr niedrig}, \num{6} (\SI{24}{\percent}) machten keine Angabe, \num{5} (\SI{20}{\percent}) gehören zur Kategorie \emph{Hoch}, \num{4} (\SI{16}{\percent}) zu \emph{Niedrig} und \num{2} (\SI{8}{\percent}) zu \emph{Sehr hoch}.

Die hier gewählte Darstellung trennt die einzelnen Merkmale bewusst auf, um die Zusammensetzung der Stichprobe transparent zu machen. Methodisch betrachtet widerspricht diese Entzerrung jedoch einem intersektionalen Ansatz, da \glspl[noindex]{identitaetsachse} in isolierte Kategorien zerlegt werden. Die vollständige Übersicht über die Angaben aus der Baseline Befragung ist im \cref{app:appendix_demographics}.

\subsection*{Momentaufnahmen}

Insgesamt wurden \num{106} vollständig abgeschlossene Momentaufnahmen erhoben.  
Die Verteilung der Anzahl abgeschlossener Momentaufnahmen pro Person ist in \cref{fig:survey_counts} dargestellt.

Die Verteilungen der während der Momentaufnahmen angegebenen Tätigkeiten und Aufenthaltsortkategorien sind in \cref{fig:survey_activities} bzw. \cref{fig:survey_locations} dargestellt.  
Innenräume (\(n = \num{54}; \SI{51}{\percent}\)) und Aussenräume (\(n = \num{52}; \SI{49}{\percent}\)) waren nahezu gleich häufig vertreten.

Das soziale Umfeld variierte: Etwa ein Drittel der Momentaufnahmen wurde allein durchgeführt (\(n = \num{37}; \SI{35}{\percent}\)), ein weiteres Drittel in Gegenwart von Freund\genderstern innen (\(n = \num{28}; \SI{26}{\percent}\)). Weitere Angaben betrafen die Anwesenheit von Fremden (\(n = \num{10}; \SI{9}{\percent}\)), Kolleg\genderstern innen (\(n = \num{8}; \SI{8}{\percent}\)) oder Kombinationen dieser Gruppen (\gls[noindex]{vgl} \cref{app:people_table}).

\begin{figure}[h]
    \centering
    \includegraphics[width=8cm]{analysis/plots/survey_counts.pdf}
    \caption{Verteilung der Anzahl abgeschlossener Momentaufnahmen pro Person}
    \label{fig:survey_counts}
\end{figure}

\begin{figure}[h]
    \centering
    \includegraphics[width=10cm]{analysis/plots/cat_dist_activity.pdf}
    \caption{Tätigkeit während der Momentaufnahme}
    \label{fig:survey_activities}
\end{figure}

\begin{figure}[h]
    \centering
    \includegraphics[width=10cm]{analysis/plots/cat_dist_location_category.pdf}
    \caption{Aufenthaltsortkategorie während der Momentaufnahme}
    \label{fig:survey_locations}
\end{figure}


\section{Quantitativ-intersektional analysieren -- Ein Widerspruch?}

Wie im theoretischen Rahmen zu Intersektionalität (\cref{sec:theoretischer_rahmen}) dargelegt, besteht eine grundlegende Spannung zwischen den theoretischen Ansprüchen intersektionaler Forschung und den Anforderungen quantitativer Analyseverfahren. Während Intersektionalität auf die komplexe, relationale und kontextabhängige Überlagerung sozialer Kategorien abzielt, verlangen statistische Modelle in der Regel klar definierte, operationalisierte Variablen. Damit einher geht die Gefahr, fluid-dynamische Identitäten in starre Kategorien zu übersetzen und deren soziale Konstruiertheit zu verschleiern \parencite{hancockWhenMultiplicationDoesnt2007, bowlegInvitedReflectionQuantifying2016}. Hinzu kommt, dass viele herkömmliche Verfahren additive oder eindimensionale Effekte modellieren, wodurch genau jene Interdependenzen und Wechselwirkungen nivelliert werden, die intersektionale Ansätze sichtbar machen wollen \parencite{scottIntersectionalityQuantitativeMethods2017}.

Diese methodische Spannung ist nicht nur ein technisches Problem, sondern berührt den Kern intersektionaler Forschung: Die Gefahr, sozial konstruierte Kategorien wie feste, unveränderliche Eigenschaften zu behandeln, steht im Widerspruch zu ihrem theoretischen Verständnis als historisch, räumlich und sozial wandelbare Konstrukte. Jede quantitative Operationalisierung muss daher reflexiv mit diesen Grenzen umgehen und das Risiko methodischer Vereinfachungen offenlegen \parencite{rodo-de-zarateDevelopingGeographiesIntersectionality2014, websterCenteringSocialtechnicalRelations2021}.

Vor diesem Hintergrund wird in dieser Pilotanalyse ein \acrfull{maihda}\footnote{In der Literatur finden sich sowohl die Bezeichnungen \emph{\acrfull{maihda}} als auch \emph{\acrfull{i-maihda}} \parencite{evansTutorialConductingIntersectional2024}. Beide Begriffe beziehen sich auf das gleiche statistische Verfahren; die Bezeichnung mit vorangestelltem „I“ hebt den Bezug zu intersektionaler Theorie explizit hervor. In dieser Arbeit wird \emph{\gls{i-maihda}} verwendet, um diesen theoretischen Rahmen klar zu signalisieren.} eingesetzt. \gls{i-maihda} ist ein flexibles, mehrstufiges Analysemodell, das Daten in Gruppen („\glspl{stratum}“) verschachtelt, die sich aus der Kombination mehrerer sozialer Merkmale ergeben. Jede Person gehört genau zu einem solchen sozialen \gls{stratum}. Innerhalb eines sozialen \gls{stratum} können sich die Werte der untersuchten Variablen (\gls[noindex]{zb} Wohlbefinden) zwischen Personen unterscheiden, während sich gleichzeitig Unterschiede zwischen den sozialen \glspl{stratum} selbst zeigen.

Aus statistischer Sicht handelt es sich um ein hierarchisches Modell, das mindestens zwei Ebenen umfasst: \textit{Level 1} sind die einzelnen Beobachtungen, \textit{Level 2} die sozialen \glspl{stratum}. \gls{i-maihda} schätzt, wie sich die Gesamtvarianz -- also die Streuung der Messwerte im gesamten Datensatz -- auf unterschiedliche Ebenen verteilt. Dabei wird getrennt zwischen Varianz, die zwischen den sozialen \glspl{stratum} liegt, und Varianz, die innerhalb der sozialen \glspl{stratum} entsteht. Diese Zerlegung erlaubt es zu erkennen, in welchem Ausmass die Kombination sozialer Merkmale systematische Unterschiede im Outcome erklärt und wie viel der Unterschiede auf individuelle oder situative Faktoren zurückzuführen ist. In grossen Datensätzen ermöglicht dieser Ansatz die Modellierung komplexer Strata mit zahlreichen kombinierten Merkmalen, da pro sozielem \gls{stratum} genügend Beobachtungen vorliegen, um stabile und präzise Schätzungen zu erhalten.

Der zentrale Vorteil von \gls{i-maihda} gegenüber klassischen Regressionsmodellen liegt darin, dass nicht nur einzelne Haupteffekte und ausgewählte Interaktionsterme berücksichtigt werden, sondern jede Merkmalskombination als eigenständige Analyseeinheit behandelt wird \parencite{hancockWhenMultiplicationDoesnt2007, bowlegInvitedReflectionQuantifying2016}. Zudem ermöglicht \gls{i-maihda} die Berechnung der sogenannten „diskriminatorischen Genauigkeit“ -- ein Mass dafür, wie trennscharf die gewählten sozialen \glspl{stratum} das Outcome im jeweiligen Kontext erklären \parencite{evansTutorialConductingIntersectional2024}.

Trotz dieser Stärken ist \gls{i-maihda} kein ursprünglich intersektionales Verfahren. Es ist aus der epidemiologischen Mehrebenenanalyse hervorgegangen und wurde nicht primär entwickelt, um intersektionale Theorien oder Machtverhältnisse theoretisch zu adressieren. Seine intersektionale Anschlussfähigkeit entsteht erst durch eine bewusste, theoriegeleitete Auswahl der Merkmale, eine reflektierte Modellierung und die Einbettung der Ergebnisse in einen sozialen und politischen Kontext \parencite{grossModellingIntersectionalityQuantitative2023}. In diesem Sinne kann \gls{i-maihda} helfen, die eingangs skizzierte Spannung zwischen theoretischem Anspruch und quantitativer Operationalisierung zu verringern -- sie jedoch nicht vollständig auflösen.

\section{Exemplarische Analyse: Testlauf mit I-MAIHDA}

Für die hier vorliegende Pilotstudie wird ein zweistufiges \gls{i-maihda}-Modell eingesetzt, bei dem die Personenebene nicht separat modelliert wird. Stattdessen werden die einzelnen Momentaufnahmen mit den zeitinvarianten demografischen Baseline-Merkmalen der jeweiligen Teilnehmenden verknüpft und unmittelbar den sozialen \glspl{stratum} zugeordnet. Diese Vorgehensweise ist einerseits durch die geringe Zahl wiederholter Befragungen pro Person bedingt, die keine verlässliche Modellierung intrapersonaler Varianz zulässt. Andererseits führt das spezifische Rekrutierungssetting -- eine freiwillige Teilnahme im Rahmen einer einführenden Exkursion im ersten Studienjahr des Bachelorstudiengangs Geographie -- zu einer vergleichsweise homogenen Stichprobe. Dies reduziert die Anzahl unterschiedlicher \glspl{stratum} zusätzlich und erschwert damit die Anwendung komplexerer Mehrebenenstrukturen.

\begin{itemize}
\item Level 1: einzelne Momentaufnahmen (einschliesslich angehängter zeitinvarianter Personenmerkmale),
\item Level 2: soziale \glspl{stratum}, gebildet aus Kombinationen sozialer Positionierungen (\gls[noindex]{zb} \gls[noindex]{gender}, \gls[noindex]{class}, \gls[noindex]{etc}).
\end{itemize}

In einem Setting mit ausreichend wiederholten Messungen und diversen sozialen \glspl{stratum} liesse sich dieser Ansatz um eine zusätzliche Personenebene erweitern. Ein solches dreistufiges Design würde es erlauben, intrapersonale Varianz zwischen Messzeitpunkten (Level 1), stabile Unterschiede zwischen Personen (Level 2) und kollektive Unterschiede zwischen sozialen \glspl{stratum} (Level 3) gleichzeitig zu modellieren, ohne komplexe Interdependenzen vollständig in additive Haupteffekte aufzulösen:

\begin{itemize}
\item Level 1: einzelne Momentaufnahmen der Teilnehmenden (mehrere Messzeitpunkte pro Person),
\item Level 2: Personenebene (zur Modellierung wiederholter Messungen),
\item Level 3: intersektionale \glspl{stratum}, gebildet aus Kombinationen sozialer Positionierungen.
\end{itemize}

\vspace{1em}

In einem Probelauf wurde geprüft, ob die vorliegenden Daten eine intersektionale Multilevel-Analyse nach dem \gls{i-maihda}-Ansatz zulassen. Konkret sollte untersucht werden, ob (a) genügend Beobachtungen je intersektionalem \gls{stratum} vorhanden sind und (b) die zwischenstratale Varianz gross genug ist, um stabile Random-Effects-Schätzungen zu erhalten.

\paragraph{Iterative Spezifikation der \glspl{stratum}}
Ausgangspunkt war ein Set über alle erhobenen Achsen: Biologisches Geschlecht, Soziales Geschlecht, Sexuelle Orientierung, Ausbildungsstufe, Gruppiertes Äquivalenz-Einkommen, Anstellungsverhältnis, Geburtsland, Vorhandene Behinderungen.

Die Kombination dieser Merkmale ergab 20 unterschiedliche \glspl{stratum}. Die Zellgrössen (Anzahl Personen pro \gls{stratum}) sind allerdings sehr klein (siehe \cref{tab:zellgroessen_alle_achsen}).

\begin{table}[h]
    \centering
    \begin{tabular}{rl}
        count & 20 \\
        mean & 1.25 \\
        std & 0.55 \\
        min & 1 \\
        max & 3 \\
    \end{tabular}
    \caption{Zellgrössen pro \gls{stratum} mit allen Achsen}
    \label{tab:zellgroessen_alle_achsen}
\end{table}

Mit diesem Set von \glspl{stratum} ist die Modellierung nicht möglich, da die Zellgrössen zu klein sind und eine gute Schätzung der Varianzanteile nicht möglich ist.

Um die Modellierbarkeit zu erhöhen, wurde das \gls{stratum} anschliessend auf zwei theoretisch zentrale Achsen reduziert: Biologisches Geschlecht und Alter.

Dies führte zu insgesamt $6$ \glspl{stratum}, mit folgenden Zellgrössen:

\begin{table}[h]
    \centering
    \begin{tabular}{rl}
        count & 6 \\
        mean & 4.17 \\
        std & 4.7 \\
        min & 1 \\
        max & 12 \\
    \end{tabular}
    \caption{Zellgrössen pro \gls{stratum} mit reduzierten Achsen}
    \label{tab:zellgroessen_reduzierte_achsen}
\end{table}

Auch hier gibt es noch einzelne \glspl{stratum} mit weniger als 3 Beobachtungen.

\begin{table}[h]
    \centering
    \begin{tabular}{lll}
        Gender & Altersgruppe & Anzahl \\
        \hline
        man & 16 -- 25 & 2 \\
        trans man & 56 -- 65 & 1 \\
        man & missing & 1 \\
        woman & 26 -- 35 & 1 \\
    \end{tabular}
    \caption{\glspl{stratum} mit weniger als 3 Beobachtungen}
    \label{tab:zellgroessen_reduzierte_achsen_kleine_strata}
\end{table}

Trotzdem wurde versucht, mit diesem Set von \glspl{stratum} eine \gls{i-maihda}-Analyse durchzuführen.

Für jedes kontinuierliche Outcome (\texttt{sense\_of\_belonging}, \texttt{environmen\_pleasure}, \texttt{environment\_lively}, \texttt{environment\_nature}, \texttt{environment\_noise}) wurde ein zweistufiges \gls{i-maihda}-Setting geschätzt:
\begin{description}
    \item[Modell~1A (Nullmodell):] Zufallsinterzept auf \gls{stratum}-Ebene, keine festen Effekte der Achsen.
    \item[Modell~1B (Additives Modell):] Zusätzlich feste Haupteffekte von \texttt{age\_group} und \texttt{gender}; der verbleibende \gls{stratum}-Random-Effect wird als Interaktionsanteil interpretiert.
\end{description}
Aufgrund der geringen Zellgrössen wurde kein zusätzlicher Random-Intercept auf Personenebene modelliert. Die Schätzung erfolgte mittels \texttt{statsmodels.mixedlm} (MLE, Optimierer \texttt{lbfgs}).


Die geschätzten Varianzanteile zwischen \glspl{stratum} (Variance Partition Coefficient, VPC) waren durchgängig extrem klein. Beispielhaft:

\begin{center}
\begin{tabular}{lrrrr}
\toprule
Outcome & VPC$_{\text{Null}}$ & VPC$_{\text{add}}$ & PCV & $n$ (Zeilen) \\
\midrule
\texttt{sense\_of\_belonging}      & $1.24\times 10^{-5}$ & $1.15\times 10^{-6}$ & 90.7\% & 106 \\
\texttt{environmen\_pleasure}      & $\approx 0$          & $4.40\times 10^{-7}$ & --      & 106 \\
\texttt{environment\_lively}       & $2.95\times 10^{-4}$ & $3.43\times 10^{-5}$ & 88.2\%  & 106 \\
\texttt{environment\_nature}       & $1.03\times 10^{-2}$ & $1.12\times 10^{-6}$ & 99.99\% & 106 \\
\texttt{environment\_noise}        & $1.95\times 10^{-2}$ & $2.98\times 10^{-5}$ & 99.85\% & 106 \\
\bottomrule
\end{tabular}
\end{center}

Die nahezu Null liegenden VPCs belegen, dass (a) die Outcomes sich zwischen den \glspl{stratum} kaum unterscheiden und (b) die \gls{stratum}-Varianz im Modell auf Null \emph{geschrumpft} wird. PCV-Werte sind bei einem praktisch Null-VPC im Nullmodell numerisch instabil (z.\,B.\ negative oder extrem grosse Werte) und daher nicht interpretierbar.

\section{Methodische Einordnung und Reflexion}
% - Kurz zu bestehenden quantitativen intersektionalen Methoden (\gls{i-maihda} als Standard).
% - Warum es (noch) nicht funktioniert hat (Datenmenge, Struktur).
% - Was man methodisch gelernt hat (z.B. Anforderungen an Zellgrössen, Datenqualität).
% - Brücke zur Diskussion: Welche Anpassungen für künftige Studien nötig wären.

\section{Schlussfolgerung zur Pilotstudie}
% - Zusammenfassung der wichtigsten Punkte.
% - Klarstellung: Wert liegt in der methodischen Erprobung, nicht in inhaltlichen Ergebnissen.
% - Ausblick: Wie dieses Instrument in Zukunft nutzbar ist.



% \section{Im Ausprobieren lernen -- Methodischer Zugang der Studie} \label{sec:methodenentwickelnd}

% Diese Arbeit ist als methodenentwickelnde Studie konzipiert. Im Zentrum steht die Entwicklung eines digitalen Erhebungsinstruments, das die situative Erfassung von affektivem Wohlbefinden mit einer intersektionalen Analyse verknüpft. Ziel ist es, einen vollständigen methodischen Workflow zu entwerfen -- bestehend aus einem spezifisch konzipierten Fragebogen, einer Smartphone-Applikation zur standortbezogenen Datenerhebung sowie einer vorbereiteten Analysestruktur für für eine intersektionale Modellierung.

% Im Unterschied zu klassischen empirischen Studien liegt der Fokus auf der konzeptionellen und technischen Umsetzbarkeit des Ansatzes. Die wenigen im Rahmen der Pilotstudie erhobenen Daten dienen ausschliesslich der Erprobung und exemplarischen Durchführung des methodischen Prozesses -- sie erlauben aufgrund der geringen Stichprobengrösse keine  Aussagen über Zusammenhänge zwischen Umgebung, intersektionaler Positionierung und Wohlbefinden.

% Die Durchführung einer \acrshort{maihda}-Analyse erfolgt demnach lediglich zu illustrativen Zwecken. Sie diente dazu, die Struktur des Modells zu testen, die Anforderungen an die Datenqualität und -quantität zu reflektieren und das methodische Zusammenspiel von Erhebungsdesign und Analyseansatz zu überprüfen. Auch andere Auswertungsschritte -- etwa deskriptive Statistiken oder Visualisierungen -- verfolgen keine analytische Zielsetzung im engeren Sinn, sondern dienen der Überprüfung der Funktionsfähigkeit des entwickelten Instruments.

% Die methodische Reflexion dieser exemplarischen Anwendung bildet einen zentralen Teil der Arbeit. Sie erlaubt erste Einschätzungen dazu, welche praktischen, technischen oder konzeptionellen Herausforderungen bei der Umsetzung auftreten und wo Anpassungen für künftige Studien notwendig wären. Der wissenschaftliche Mehrwert der Arbeit liegt entsprechend nicht in empirischen Erkenntnissen, sondern in der Bereitstellung und kritischen Diskussion eines erprobten methodischen Zugangs, der für zukünftige Forschungsvorhaben adaptiert und weiterentwickelt werden kann.

% \subsection{Ablauf und Durchführung der Datenerhebung}

% Die Datenerhebung fand im Rahmen der einführenden Exkursion Recht auf Stadt im ersten Studienjahr des Bachelorstudiengangs Geographie an der Universität Bern im Mai 2025 statt. Zu Beginn jedes der insgesamt vier Exkursionstage erfolgte eine Einladung zur freiwilligen Teilnahme an der Studie -- beim ersten Termin von mir persönlich, an den folgenden Terminen durch die Exkursionsleitenden. Für jede teilnehmende Person begann die Erhebungsphase mit einer einmaligen Baseline-Befragung und dauerte ab diesem Zeitpunkt sieben Tage.


% \section{Limitationen und Herausforderungen der Datenerhebung}

% \subsection{Geringe Rücklaufquote und mögliche Ursachen}

% \subsection{Auswirkungen auf die Datenqualität und Analyse}

% LTeX: language=de-CH

% \section{Ergebnisse} \label{sec:ergebnisse}

% \subsection*{Beschreibung der Stichprobe}

% Die Stichprobe des Probelaufs umfasst insgesamt 24 Personen. Die Mehrheit gehört der Altersgruppe \emph{16--25 Jahre} an (n = 20; 80\,\%). Nur wenige Personen entfallen auf die Gruppen \emph{26--35 Jahre} (n = 3; 12\,\%) und \emph{56--65 Jahre} (n = 1; 4\,\%); eine Person machte keine Altersangabe.

% Bezüglich des \emph{sozialen Geschlechts} gaben 15 Personen (60\,\%) an, \emph{Mann} zu sein, 9 Personen (36\,\%) identifizierten sich als \emph{Frau}, und eine Person (4\,\%) als \emph{trans Mann}. Als biologisches Geschlecht gaben 16 Personen (64\,\%) an, \emph{männlich} zu sein, 8 (32\,\%) an, \emph{weiblich} zu sein, eine Person machte keine Angabe.

% Die \cref{tab:kreuztabelle_abs} zeigt die Verteilung von sozialem Geschlecht und Altersgruppe (absolute Häufigkeiten).

% \begin{table}[H]
%    \centering
%    \caption{Kreuztabelle: Soziales Geschlecht und Altersgruppe (absolute Häufigkeiten)}
%    \label{tab:kreuztabelle_abs}
%    \begin{tabular}{lccccc}
%    \toprule
%    \textbf{Geschlecht} & 16--25 & 26--35 & 56--65 & Keine Angabe & Gesamt \\
%    \midrule
%    Mann       & 12 & 2 & 0 & 1 & 15 \\
%    Trans Mann &  0 & 0 & 1 & 0 & 1  \\
%    Frau       &  8 & 1 & 0 & 0 & 9  \\
%    \midrule
%    \textbf{Gesamt} & 20 & 3 & 1 & 1 & 25 \\
%    \bottomrule
%    \end{tabular}
% \end{table}

% Weitere soziodemografische Merkmale der Teilnehmenden umfassen u.\,a. sexuelle Orientierung, Bildungsstand, Erwerbsstatus, Haushaltseinkommen, Haushaltsstruktur und -finanzierung, sowie Erfahrungen mit Diskriminierung. Eine vollständige Übersicht über die Verteilung dieser Merkmale findet sich in \cref{app:appendix_demographics}.


% Im Rahmen der Pilotstudie wurden insgesamt 106 Momentaufnahmen erhoben, verteilt auf die 25 Teilnehmenden. Die Anzahl abgeschlossener Befragungen pro Person variierte dabei erheblich (\textit{M} = 4{,}2; \textit{SD} = 2{,}9; \textit{Min} = 1; \textit{Max} = 12), was auf eine ungleichmässige Nutzung der App innerhalb der Teilnehmendengruppe hinweist (siehe \cref{fig:survey_counts}).

% Die Tätigkeiten, die während der Beantwortung der Umfrage durchgeführt wurden, decken ein breites Spektrum ab. Am häufigsten gaben Teilnehmende an, zu arbeiten oder zu studieren (n = 48, 43\,\%), gefolgt von Freizeit- und Entspannungsaktivitäten (n = 19, 22\,\%) und Reisen bzw. Pendeln (n = 9, 10\,\%). Weitere Angaben umfassten etwa Kochen, Medienkonsum, soziale Aktivitäten oder Kombinationen mehrerer Aktivitäten.

% Bezüglich des Aufenthaltsortes befanden sich die meisten Personen zum Zeitpunkt der Umfrage entweder an einer Bildungsinstitution (n = 38, 35\,\%) oder zu Hause (n = 28, 27\,\%). Weitere häufig genannte Orte waren unterwegs zu Fuss, per Fahrrad oder im Auto (n = 11, 11\,\%), öffentliche Verkehrsmittel (n = 8, 7\,\%) sowie die Wohnung anderer Personen oder Parks und Grünflächen (\gls[noindex]{vgl} \cref{app:location_table}). Die Aufenthaltsorte verteilten sich dabei nahezu gleichmässig auf Innenräume (n = 54, 51\,\%) und Aussenräume (n = 52, 49\,\%).

% Auch die soziale Situation während der Umfrage war sehr unterschiedlich: Ein Drittel der Momentaufnahmen wurde allein durchgeführt (n = 37, 35\,\%), ein weiteres Drittel in Gegenwart von Freund*innen (n = 28, 26\,\%). Weitere häufige Angaben betrafen die Anwesenheit von Fremden (n = 10, 9\,\%), Kolleg*innen (n = 8, 8\,\%) oder verschiedenen Kombinationen dieser Gruppen (\gls[noindex]{vgl} \cref{app:people_table}).

%Diese Vielfalt an Tätigkeiten, Kontexten und sozialen Situationen zeigt das Potenzial des Erhebungsinstruments, subjektives Wohlbefinden in unterschiedlichen Alltagssituationen ökologisch valide zu erfassen.



% \begin{figure}[htbp]
%     \centering
%     \includegraphics[width=8cm]{analysis/plots/survey_counts.pdf}
%     \caption{Aufteilung nach Anzahl abgeschlossener Umfragen pro Person}
%     \label{fig:survey_counts}
% \end{figure}


%\subsection*{Exemplarische Analysen mittels \gls{i-maihda}}
%\label{sec:pilot_maihda}

% In einem Probelauf wurde geprüft, ob die vorliegenden Daten eine intersektionale Multilevel-Analyse nach dem \gls{i-maihda}-Ansatz zulassen. Konkret sollte untersucht werden, ob (a) genügend Beobachtungen je intersektionalem \gls{stratum} vorhanden sind und (b) die zwischenstratale Varianz gross genug ist, um stabile Random-Effects-Schätzungen zu erhalten.

%\paragraph{Iterative Spezifikation der \glspl{stratum}}
% Ausgangspunkt war ein Set über alle erhobenen Achsen: Biologisches Geschlecht, Soziales Geschlecht, Sexuelle Orientierung, Ausbildungsstufe, Gruppiertes Äquivalenz-Einkommen, Anstellungsverhältnis, Geburtsland, Vorhandene Behinderungen.

% Die Kombination dieser Merkmale ergab 20 unterschiedliche \glspl{stratum}. Die Zellgrössen (Anzahl Personen pro \gls{stratum}) sind allerdings sehr klein (siehe \cref{tab:zellgroessen_alle_achsen}).

% \begin{table}[h]
%     \centering
%     \begin{tabular}{rl}
%        count & 20 \\
%        mean & 1.25 \\
%        std & 0.55 \\
%        min & 1 \\
%        max & 3 \\
%    \end{tabular}
%    \caption{Zellgrössen pro \gls{stratum} mit allen Achsen}
%    \label{tab:zellgroessen_alle_achsen}
% \end{table}

% Mit diesem Set von \glspl{stratum} ist die Modellierung nicht möglich, da die Zellgrössen zu klein sind und eine gute Schätzung der Varianzanteile nicht möglich ist.

% Um die Modellierbarkeit zu erhöhen, wurde das \gls{stratum} anschliessend auf zwei theoretisch zentrale Achsen reduziert: Biologisches Geschlecht und Alter.

% Dies führte zu insgesamt $6$ \glspl{stratum}, mit folgenden Zellgrössen:

% \begin{table}[h]
%     \centering
%     \begin{tabular}{rl}
%         count & 6 \\
%        mean & 4.17 \\
%        std & 4.7 \\
%        min & 1 \\
%        max & 12 \\
%    \end{tabular}
%    \caption{Zellgrössen pro \gls{stratum} mit reduzierten Achsen}
%    \label{tab:zellgroessen_reduzierte_achsen}
% \end{table}

% Auch hier gibt es noch einzelne \glspl{stratum} mit weniger als 3 Beobachtungen.

% \begin{table}[h]
%     \centering
%     \begin{tabular}{lll}
%         Gender & Altersgruppe & Anzahl \\
%         \hline
%         man & 16 -- 25 & 2 \\
%         trans man & 56 -- 65 & 1 \\
%         man & missing & 1 \\
%         woman & 26 -- 35 & 1 \\
%     \end{tabular}
%     \caption{\glspl{stratum} mit weniger als 3 Beobachtungen}
%     \label{tab:zellgroessen_reduzierte_achsen_kleine_strata}
% \end{table}

%Trotzdem wurde versucht, mit diesem Set von \glspl{stratum} eine \gls{i-maihda}-Analyse durchzuführen.

% Für jedes kontinuierliche Outcome (\texttt{sense\_of\_belonging}, \texttt{environmen\_pleasure}, \texttt{environment\_lively}, \texttt{environment\_nature}, \texttt{environment\_noise}) wurde ein zweistufiges \gls{i-maihda}-Setting geschätzt:
% \begin{description}
%     \item[Modell~1A (Nullmodell):] Zufallsinterzept auf \gls{stratum}-Ebene, keine festen Effekte der Achsen.
%     \item[Modell~1B (Additives Modell):] Zusätzlich feste Haupteffekte von \texttt{age\_group} und \texttt{gender}; der verbleibende \gls{stratum}-Random-Effect wird als Interaktionsanteil interpretiert.
% \end{description}
% Aufgrund der geringen Zellgrössen wurde kein zusätzlicher Random-Intercept auf Personenebene modelliert. Die Schätzung erfolgte mittels \texttt{statsmodels.mixedlm} (MLE, Optimierer \texttt{lbfgs}).


%Die geschätzten Varianzanteile zwischen \glspl{stratum} (Variance Partition Coefficient, VPC) waren durchgängig extrem klein. Beispielhaft:

% \begin{center}
% \begin{tabular}{lrrrr}
% \toprule
% Outcome & VPC$_{\text{Null}}$ & VPC$_{\text{add}}$ & PCV & $n$ (Zeilen) \\
% \midrule
% \texttt{sense\_of\_belonging}      & $1.24\times 10^{-5}$ & $1.15\times 10^{-6}$ & 90.7\% & 106 \\
% \texttt{environmen\_pleasure}      & $\approx 0$          & $4.40\times 10^{-7}$ & --      & 106 \\
% \texttt{environment\_lively}       & $2.95\times 10^{-4}$ & $3.43\times 10^{-5}$ & 88.2\%  & 106 \\
% \texttt{environment\_nature}       & $1.03\times 10^{-2}$ & $1.12\times 10^{-6}$ & 99.99\% & 106 \\
% \texttt{environment\_noise}        & $1.95\times 10^{-2}$ & $2.98\times 10^{-5}$ & 99.85\% & 106 \\
% \bottomrule
% \end{tabular}
% \end{center}

% Die nahezu Null liegenden VPCs belegen, dass (a) die Outcomes sich zwischen den \glspl{stratum} kaum unterscheiden und (b) die \gls{stratum}-Varianz im Modell auf Null \emph{geschrumpft} wird. PCV-Werte sind bei einem praktisch Null-VPC im Nullmodell numerisch instabil (z.\,B.\ negative oder extrem grosse Werte) und daher nicht interpretierbar.

% \section{Schlussfolgerung}
% Die Pilotanalyse zeigt, dass mit den vorliegenden Daten keine sinnvolle \gls{i-maihda}-Varianzzerlegung durchführbar ist. Gründe:
% \begin{enumerate}
%     \item \textbf{Zu kleine \glspl{stratum}-Zellgrössen}: Die meisten \glspl{stratum} enthalten nur eine Person bzw.\ sehr wenige Beobachtungen.
%     \item \textbf{Geringe zwischenstratale Varianz}: Die betrachteten Outcomes variieren kaum zwischen den (reduzierten) \glspl{stratum}.
%    \item \textbf{Numerische Instabilität}: Die Random-Effects-Kovarianzmatrix wird singular; die Schätzung kollabiert auf Randlösungen.
% \end{enumerate}

% \paragraph{Implikation für die weitere Analyse.}
% Für die Beantwortung der Forschungsfrage (Einfluss situativer Umweltfaktoren auf affektives Wohlbefinden) bietet es sich an, die Umweltvariablen als Level-1-Prädiktoren in einem vereinfachten Modell (z.\,B.\ lineares Modell mit cluster-robusten Standardfehlern nach Person oder ein Mixed Model nur mit Personen-Random-Intercept) zu analysieren. Intersectionale Unterschiede können vorerst über feste Effekte (z.\,B.\ \texttt{C(age\_group)}, \texttt{C(gender)}) kontrolliert werden. Eine vollwertige \gls{i-maihda}-Anwendung ist erst mit grösserer Stichprobe und ausreichenden Zellgrössen pro \gls{stratum} sinnvoll.



