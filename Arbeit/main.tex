\documentclass{template}
\usepackage{array}
\usepackage{graphicx}
\usepackage{float}
\usepackage{csquotes}
\usepackage[automake,acronym, toc]{glossaries-extra}

\hypersetup{
    pdftitle={Designing InterMind -- Entwicklung eines kritisch-digitalen Forschungsdesigns zur raumbezogenen Erfassung intersektional situierten (Un-)Wohlbefindens},
    pdfauthor={Lukas Batschelet},
    pdfsubject={Bachelorarbeit},
    pdfkeywords={Intersektionalität, Wohlbefinden, App, Erhebung, Geographie} 
}




\makeglossaries
%TC:ignore

\newacronym{esm}{ESM}{Experience Sampling Method}
\newacronym{ema}{EMA}{Ecological Momentary Assessment}
\newacronym{gema}{GEMA}{Geographically Explicit Ecological Momentary Assessment}
\newacronym{drm}{DRM}{Day Reconstruction Method}
\newacronym{panas}{PANAS}{The Positive \& Negative Affect Schedule}
\newacronym{maihda}{MAIHDA}{Multilevel Analysis of Individual Heterogeneity and Discriminatory Accuracy}
\newacronym[description={Intersectional MAIHDA -- siehe \gls[noindex]{maihda}}]{i-maihda}{I-MAIHDA}{Intersectional \gls[noindex]{maihda}}
\newacronym{cart}{CART}{Classification and Regression Trees}
\newacronym{json}{JSON}{JavaScript Object Notation}
\newacronym{dsg}{DSG}{Schweizer Datenschutzgesetz}
\newacronym{dsgvo}{DSGVO}{EuropäischeDatenschutz-Grundverordnung}
\newacronym{peqi}{PEQI}{Perceived Environmental Quality Indices}
\newacronym{news}{NEWS}{Neighborhood Environment Walkability Scale}
\newacronym{health}{HEALTH}{The Healthy Environments and Active Living for Translational Health Platform}
\newacronym{mvp}{MVP}{Minimum Viable Product}
\newacronym{cicd}{CI/CD}{Continuous Integration/Continuous Delivery}
\newacronym{esec}{ESec}{European Socio-economic Classification}
\newacronym{egp}{EGP}{Erikson–Goldthorpe–Portocarero-Klassenschema}
\newacronym{wemwbs}{WEMWBS}{Warwick-Edinburgh Mental Wellbeing Scale}
\newacronym{icc}{ICC}{Intra-Class Correlation}
\newacronym{pev}{PEV}{Proportional Explained Variance}





\newacronym{vgl}{vgl.}{vergleiche}
\newacronym{bspw}{bspw.}{beispielsweise}
\newacronym{gps}{GPS}{Global Positioning System}
\newacronym{zb}{z.\,B.}{zum Beispiel}
\newacronym{ua}{u.\,a.}{unter anderem}
\newacronym{etc}{etc.}{et cetera}
\newacronym{eng}{engl.}{englisch}
\newacronym{bzw}{bzw.}{beziehungsweise}
\newacronym{hiv}{HIV}{Human Immunodeficiency Virus}
\newacronym{gis}{GIS}{Geographic Information System}


\newglossaryentry{race}{
    name={\textit{race}},
    description={Eine im englischsprachigen Raum etablierte, gesellschaftlich konstruierte Kategorie, die rassifizierende Zugehörigkeiten beschreibt. In dieser Arbeit kursiv gesetzt, um ihre soziokulturelle Bedeutung zu betonen und sie deutlich vom biologistischen Begriff \enquote{Rasse} abzugrenzen. Der Begriff verweist auf machtvolle Prozesse sozialer Differenzierung, Zugehörigkeit und Ausschluss, die historisch gewachsen sind und bis heute wirksam bleiben. Die Übertragung des Konzepts in europäische Kontexte ist umstritten: Aus Angst vor biologisierenden Implikationen und im Schatten nationaler Gewaltgeschichte (\gls{zb} Nationalsozialismus) wird \textit{race} oft durch vage Begriffe wie \enquote{Ethnizität} ersetzt oder gänzlich vermieden, was zur epistemischen Unsichtbarmachung rassifizierter Erfahrungen führen kann \parencite{bartelsPostcolonialFeminismIntersectionality2019}.},
    sort=race
}


\newglossaryentry{gender}{
    name={\textit{gender}},
    description={Eine gesellschaftlich konstruierte Kategorie, die auf soziale, kulturelle und politische Vorstellungen von Geschlecht verweist. \textit{gender} beschreibt normative Erwartungen an Geschlechtsidentität und Geschlechtsausdruck, die in spezifischen historischen und kulturellen Kontexten entstehen und sich verändern können. In dieser Arbeit kursiv gesetzt, um den Unterschied zum biologisierenden Begriff \enquote{sex} (\gls{eng}) \gls{bzw} \enquote{biologisches Geschlecht} hervorzuheben. Die Verwendung im Deutschen ist uneinheitlich: Während \textit{gender} in wissenschaftlichen Diskursen etabliert ist, wird es im Alltagsgebrauch häufig unscharf verwendet oder mit \enquote{Geschlecht} gleichgesetzt, was die soziale Konstruiertheit verwischt \parencite{butlerGenderTroubleFeminism1990}.},
    sort=gender
}

\newglossaryentry{schwarz}{
    name={\textit{Schwarz}},
    description={Politische Selbstbezeichnung von Menschen, die im Kontext rassistischer Machtverhältnisse positioniert werden. Grossgeschrieben zur Abgrenzung von farblichen Zuschreibungen und um die Konstruktion von \enquote{Schwarzsein} als soziale Position zu betonen, nicht als biologische Eigenschaft. Die Schreibweise folgt postkolonialen und kritischen rassismustheoretischen Ansätzen, die die Erfahrung von Diskriminierung, Widerstand und Zugehörigkeit in den Vordergrund stellen \parencite{oguntoyeFarbeBekennenAfrodeutsche1986}.},
    sort=schwarz
}

\newglossaryentry{intersektionalitaet}{
    name={Intersektionalität},
    description={Analytisches Konzept zur Untersuchung sich überschneidender und wechselseitig verstärkender Machtverhältnisse wie Rassismus, Sexismus oder Klassismus. Entwickelt von Kimberlé Crenshaw zur Beschreibung der spezifischen Diskriminierungserfahrungen Schwarzer Frauen, die weder durch feministische noch durch antirassistische Theorien allein erfasst wurden. Der Ansatz betont, dass soziale Kategorien nicht additiv wirken, sondern in ihren Verflechtungen eigenständige Formen von Ungleichheit erzeugen. Heute wird er in einer Vielzahl disziplinärer und methodischer Kontexte angewandt, ist jedoch in seiner theoretischen Radikalität nicht unumstritten \parencite{crenshawMappingMarginsIntersectionality1991, collinsBlackFeministThought2002}.},
    sort=intersektionalitaet
}

\newglossaryentry{class}{
    name={\textit{class}},
    description={Eine gesellschaftlich konstruierte Kategorie, die ökonomische, kulturelle und symbolische Ungleichheiten beschreibt. \textit{class} verweist auf soziale Hierarchien, die sich aus Eigentum, Einkommen, Bildung, Beruf und anderen Ressourcen ergeben und historisch durch kapitalistische Strukturen geformt wurden. In dieser Arbeit kursiv gesetzt, um die soziale Konstruiertheit und die Differenz zum alltagssprachlichen Begriff \enquote{soziale Schicht} zu betonen. Die Übersetzung ins Deutsche ist umstritten, da Begriffe wie \enquote{Klasse} oder \enquote{Schicht} jeweils eigene theoretische Traditionen und politische Konnotationen mitbringen.},
    sort=class
}


\newglossaryentry{reactnative}{
  name={React Native},
  description={Ein Framework zur plattformübergreifenden Entwicklung mobiler Apps. Es erlaubt die Programmierung mit \gls[noindex]{javascript} oder \gls[noindex]{typescript}, wobei der Code nativ auf Android- und iOS-Geräten ausgeführt wird. \href{https://reactnative.dev/}{reactnative.dev}}
}

\newglossaryentry{expo}{
  name={Expo},
  description={Ein Toolchain und Dienst, der die Entwicklung mit \gls[noindex]{reactnative} vereinfacht. Expo stellt Werkzeuge zum Testen, Debuggen und Veröffentlichen von Apps bereit -- ohne dass native Programmierkenntnisse erforderlich sind. \href{https://expo.dev/}{expo.dev}}
}

\newglossaryentry{supabase}{
  name={Supabase},
  description={Ein Open-Source-Backend, das als Alternative zu Firebase dient. Es basiert auf einer \gls[noindex]{tech:db} (PostgreSQL) und bietet Funktionen wie Authentifizierung, Datei-Hosting und \gls[noindex]{rls}. \href{https://supabase.com/}{supabase.com}}
}

\newglossaryentry{javascript}{
  name={JavaScript},
  description={Eine weit verbreitete Programmiersprache für Webentwicklung, die auch in mobilen Frameworks wie \gls[noindex]{reactnative} verwendet wird. Sie ist dynamisch und flexibel, aber nicht typensicher.}
}

\newglossaryentry{typescript}{
  name={TypeScript},
  description={Eine von Microsoft entwickelte Programmiersprache, die auf \gls[noindex]{javascript} basiert, aber zusätzliche statische Typisierung bietet. Sie erhöht die Wartbarkeit und Fehlervermeidung in grösseren Softwareprojekten.}
}

\newglossaryentry{python}{
  name={Python},
  description={Eine interpretierte, einfach lesbare Programmiersprache, die häufig in Wissenschaft, Datenanalyse und Automatisierung eingesetzt wird. Sie wurde auch zur Auswertung der App-Daten verwendet.}
}

\newglossaryentry{uuid}{
  name={UUID},
  description={Abkürzung für Universally Unique Identifier. Eine \textit{UUID} ist eine zufällig generierte Zeichenkette, die zur eindeutigen Identifikation eines Geräts oder Datensatzes dient, ohne personenbezogene Daten zu erfassen.},
  first={Universally Unique Identifier (\glsentrytext{uuid})}
}

\newglossaryentry{opensource}{
  name={Open-Source},
  description={Bezeichnet Software, deren Quellcode öffentlich einsehbar, veränderbar und frei verwendbar ist. \gls[noindex]{supabase} und viele Komponenten von \gls[noindex]{reactnative} und \gls[noindex]{expo} sind Open-Source.}
}

\newglossaryentry{tech:db}{
  name={Datenbank},
  text={Datenbank},
  sort={datenbank},
  plural={Datenbanken},
  description={System zur strukturierten Speicherung, Verwaltung und Abfrage von Daten}
}


\newglossaryentry{rls}{
  name={RLS},
  description={Ein feingranulares Zugriffsmodell in einer \gls[noindex]{tech:db}, das sicherstellt, dass Nutzer:innen nur jene Datenzeilen sehen oder ändern können, für die sie berechtigt sind. \gls[noindex]{supabase} unterstützt RLS standardmässig.},
  first={Row-Level Security (\glsentrytext{rls})}
}

\newglossaryentry{github}{
  name={GitHub},
  description={Eine webbasierte Plattform zur Versionsverwaltung und Zusammenarbeit an Softwareprojekten. Sie basiert auf \gls[noindex]{git} und wird häufig zur Entwicklung und Veröffentlichung von \gls[noindex]{opensource}-Software verwendet}
}

\newglossaryentry{git}{
  name={Git},
  description={Ein verteiltes Versionskontrollsystem zur Nachverfolgung von Änderungen im Quellcode. Git ermöglicht es, Entwicklungsschritte lokal oder kollaborativ zu verwalten und ist Grundlage vieler Plattformen wie \gls[noindex]{github}}
}

\newglossaryentry{frontend}{
  name={Frontend},
  description={Der Teil einer Software, der für Nutzer:innen sichtbar und direkt bedienbar ist -- etwa die Benutzeroberfläche einer App. Wird meist in Kombination mit dem \gls[noindex]{backend} verwendet.}
}

\newglossaryentry{backend}{
  name={Backend},
  description={Der Teil einer Software, der im Hintergrund läuft und Daten verarbeitet, speichert oder bereitstellt. In dieser Arbeit wird dafür \gls[noindex]{supabase} verwendet.}
}

\newglossaryentry{framework}{
  name={Framework},
  description={Ein vorgefertigtes Gerüst für die Softwareentwicklung, das häufig genutzte Funktionen bereitstellt. \gls[noindex]{reactnative} ist ein Beispiel für ein solches Framework.},
  plural={Frameworks}
}

\newglossaryentry{pushnotification}{
  name={Push-Benachrichtigung},
  description={Eine Mitteilung, die von einer App aktiv an das Gerät gesendet wird -- auch wenn die App im Hintergrund läuft. In dieser Studie werden so die Teilnehmenden zur Beantwortung der Fragen aufgefordert.},
  plural={Push-Benachrichtigungen}
}

\newglossaryentry{ui}{
  name={UI},
  description={Die Benutzeroberfläche einer App, die für Nutzer:innen sichtbar und direkt bedienbar ist. In dieser Arbeit wird die Benutzeroberfläche der App als \gls[noindex]{ui} bezeichnet.},
  first={User Interface (\glsentrytext{ui})}
}

\newglossaryentry{postgresql}{
  name={PostgreSQL},
  description={Eine relationale \gls[noindex]{tech:db}, die als \gls[noindex]{backend} für \gls[noindex]{supabase} verwendet wird.}
}

\newglossaryentry{urbanmind}{
  name={\textit{Urban Mind}},
  description={Eine mobile App zur Echtzeit-Erhebung subjektiven Wohlbefindens mittels \gls[noindex]{gema}. Die App erfasst affektive Zustände mehrfach täglich und verknüpft sie mit räumlichen Kontexten wie Naturerleben. \glslink[noindex]{intersektionalitaet}{Intersektionale} Perspektiven werden nicht systematisch berücksichtigt. Siehe \href{https://www.urbanmind.info/}{urbanmind.info}},
  first={Urban Mind},
  sort=urbanmind
}

\newglossaryentry{reliefmaps}{
  name={\textit{Relief Maps+}},
  description={Ein webbasiertes Tool zur retrospektiven und \glslink[noindex]{intersektionalitaet}{intersektionalen} Reflexion subjektiver Raumerfahrungen. Nutzer\genderstern innen verorten emotionale Bewertungen entlang von \glspl{identitaetsachse} auf einer Karte. Siehe \href{https://reliefmaps.upf.edu/}{reliefmaps.upf.edu}},
  first={Relief Maps+},
  sort=reliefmaps
}

\newglossaryentry{intermind}{
  name={\textit{InterMind}},
  description={Eine in dieser Arbeit entwickelte \gls[noindex]{opensource}-App zur \gls[noindex]{gema}-basierten Erhebung situativer Erfahrungen. Die Entwicklung wird in \cref{sec:entwicklung_app} beschrieben. Siehe \href{https://intermind.ch/}{intermind.ch}},
  sort=intermind
}

\newglossaryentry{identitaetsachse}{
  name={Identitätsachse},
  plural={Identitätsachsen},
  description={Begriff aus der \glslink[noindex]{intersektionalitaet}{intersektionalen} Theorie, der eine einzelne soziale Kategorie wie \gls[noindex]{gender}, \gls[noindex]{race}, \gls[noindex]{class}, sexuelle Orientierung, (Dis-)Ability oder Alter bezeichnet. Solche Achsen strukturieren gesellschaftliche Positionierungen und prägen Erfahrungen von Privilegierung oder Diskriminierung.}
}

\newglossaryentry{ios}{
  name={iOS},
  description={Eine mobile Betriebssystem-Plattform, die von Apple entwickelt wird. Sie wird hauptsächlich auf Geräten des iPhone- und iPad-Produktlinien verwendet.}
}

\newglossaryentry{android}{
  name={Android},
  description={Eine mobile Betriebssystem-Plattform, die von Google entwickelt wird. Sie wird hauptsächlich auf Geräten des Android-Produktlinien verwendet.}
}

\newglossaryentry{authentifizierung}{
  name={Authentifizierung},
  description={Vorgang zur Überprüfung der Identität eines Nutzers oder Geräts, typischerweise durch die Eingabe eines Passworts, Tokens oder durch kryptografische Verfahren. In der App erfolgt die Authentifizierung gerätebasiert mittels UUID, ohne Eingabe persönlicher Informationen}
}

\newglossaryentry{autorisierung}{
  name={Autorisierung},
  description={Festlegung von Zugriffsrechten auf bestimmte Ressourcen oder Daten nach erfolgreicher Authentifizierung. In diesem Projekt bedeutet dies, dass nur das eigene Gerät auf die jeweils verknüpften Datensätze zugreifen kann (Row-Level Security)}
}

\newglossaryentry{solid}{
  name={SOLID},
  description={Akronym für fünf grundlegende Prinzipien guter objektorientierter Softwarearchitektur: \textit{Single Responsibility, Open/Closed, Liskov Substitution, Interface Segregation, Dependency Inversion}. Die Prinzipien sollen verständliche, wartbare und erweiterbare Softwaresysteme ermöglichen \parencite{martinCleanArchitectureCraftsmans2018}}
}

\newglossaryentry{emulator}{
  name={Emulator},
  plural={Emulatoren},
  description={Softwareumgebung, die ein bestimmtes Betriebssystem oder Gerät auf einem anderen System simuliert, um Programme wie auf einem echten Gerät auszuführen. In der App-Entwicklung dienen Emulatoren insbesondere dem Testen von Anwendungen auf unterschiedlichen Bildschirmgrössen, Betriebssystemversionen und Gerätearchitekturen, ohne dass reale Geräte erforderlich sind}
}

\newglossaryentry{googleplayconsole}{
  name={Google Play Console},
  description={Plattform von \gls[noindex]{google} zur Verwaltung und Verteilung von \gls[noindex]{android}-Anwendungen. Sie ermöglicht die Veröffentlichung, das Testing und das Monitoring von Apps auf Geräten mit dem Betriebssystem \gls[noindex]{android}}
}

\newglossaryentry{testflight}{
  name={TestFlight},
  description={Offizielle Plattform von Apple zur Bereitstellung von \gls[noindex]{ios}-Apps für Betatests. Entwickler\genderstern innen können damit Vorabversionen ihrer Anwendungen an registrierte Testpersonen verteilen}
}

\newglossaryentry{java}{
  name={Java},
  description={Eine streng \glslink[noindex]{objektorientierung}{objektorientierte} Programmiersprache, die in vielen Bereichen der Softwareentwicklung eingesetzt wird.}
}

\newglossaryentry{objektorientierung}{
  name={Objektorientierung},
  description={Ein Programmierparadigma, das die Entwicklung von Software durch die Modellierung von Objekten und deren Interaktionen vereinfacht. In dieser Arbeit wird die \gls[noindex]{objektorientierung} verwendet, um die App strukturiert und wartbar zu halten.}
}

\newglossaryentry{githubissue}{
    name={GitHub-Issue},
    description={Ein integriertes Werkzeug zur Aufgaben- und Projektverwaltung auf der Plattform \gls[noindex]{github}. 
    GitHub-Issues dienen der strukturierten Erfassung, Diskussion und Nachverfolgung von Aufgaben, 
    Fehlern, neuen Funktionen oder allgemeinen Projektthemen. Sie können mit Labels, Meilensteinen 
    und Verantwortlichkeiten versehen werden, um Entwicklungsprozesse transparent und nachvollziehbar zu gestalten.},
    plural={GitHub-Issues}
}

\newglossaryentry{devops}{
  name={DevOps},
  description={Ein Konzept, das die Integration von Entwicklung und Operations zusammenführt, um schnellere und stabilere Softwareentwicklung zu ermöglichen. DevOps umfasst Tools und Prozesse, die die Automatisierung von Build, Test und Bereitstellung von Software unterstützen.}
}

\newglossaryentry{meta}{
    name={Meta},
    description={US-Technologiekonzern, \gls[noindex]{ua} Entwickler von Facebook, Instagram, WhatsApp und \gls[noindex]{reactnative}.}
}

\newglossaryentry{google}{
    name={Google},
    description={US-Technologiekonzern, \gls[noindex]{ua} Betreiber von Suchmaschine, Google Play Store, YouTube und Entwickler von \gls[noindex]{firebase}.}
}

\newglossaryentry{apple}{
    name={Apple},
    description={US-Technologiekonzern, Hersteller von iPhone und Betreiber des App Store.}
}

\newglossaryentry{firebase}{
    name={Firebase},
    description={Ein \gls[noindex]{backend}-as-a-Service von \gls[noindex]{google}, das Authentifizierung, Datenspeicherung und Schnittstellenbereitstellung integriert bereitstellt.}
}

\newglossaryentry{refactoring}{
  name={Refactoring},
  description={Der Prozess der Verbesserung der Struktur und Lesbarkeit von Code, ohne dass sich die Funktionalität ändert. Refactoring ist ein wichtiger Bestandteil der Softwareentwicklung, um Code-Qualität zu erhöhen und Wartbarkeit zu verbessern.}
}

\newglossaryentry{stratum}{
    name={Stratum},
    plural={Strata},
    description={Bezeichnung für eine Teilmenge einer Grundgesamtheit in der Statistik, gebildet nach gemeinsamen Merkmalen der darin enthaltenen Beobachtungen.}
}

\newglossaryentry{lic:agpl}{
  name={AGPL-3.0},
  text={AGPL},
  sort={agpl},
  plural={AGPL-Lizenzen},
  description={Freie \gls[noindex]{lic:copyleft}-Lizenz (GNU \emph{Affero General Public License}, Version~3),
  entwickelt für netzwerkbasierte Software. Sie verlangt, dass bei Bereitstellung
  über ein Netzwerk der vollständige Quellcode einschliesslich aller Änderungen unter
  derselben Lizenz zugänglich gemacht wird.}
}

\newglossaryentry{tech:sql}{
  name={SQL},
  text={SQL},
  sort={sql},
  description={\emph{Structured Query Language}, eine standardisierte Abfragesprache zur Definition, Abfrage und Manipulation von Daten in relationalen \glspl[noindex]{tech:db}.}
}

\newglossaryentry{lic:copyleft}{
    name={Copyleft},
    description={Lizenzprinzip, bei dem Software oder Texte nicht nur frei genutzt und verändert werden dürfen, sondern auch jede abgeleitete Arbeit wiederum unter derselben oder einer kompatiblen offenen Lizenz veröffentlicht werden muss. Copyleft macht Offenheit damit verpflichtend und unterscheidet sich von permissiven Open-Source-Lizenzen, die diese Weitergabepflicht nicht kennen.}
}



\newglossaryentry{lic:cc-by-sa}{
    name={CC BY-SA~4.0},
    description={\emph{Creative Commons Attribution-ShareAlike 4.0 International}. Eine \gls[noindex]{lic:copyleft}-Lizenz für Texte, Bilder oder andere Werke. Sie erlaubt die Nutzung und Bearbeitung unter der Bedingung, dass die Urheberschaft genannt wird und abgeleitete Werke unter denselben Lizenzbedingungen weitergegeben werden.}
}



%TC:endignore



\AddBibFile{BA_Lukas_Batschelet.bib}


\begin{document}
\pagenumbering{gobble}

\includepdf[pages=1]{Arbeit/Titelblatt/Titelblatt_digital.pdf}

% \begin{titlepage}
% \sffamily
% \raggedright
% 
% \vspace*{1cm}
% \huge
% \textbf{Designing InterMind}
% 
% \vspace{0.5cm}
% \large
% Entwicklung eines kritisch-digitalen Forschungsdesigns zur raumbezogenen Erfassung intersektional situierten (Un-)Wohlbefindens
% 
% \vspace{1.5cm}
% \textbf{Lukas Batschelet}
% 
% \vspace{0.5cm}
% Matrikel-Nr. 16-499-733
% 
% \vfill
% \normalsize
% Bachelorarbeit der Philosophisch-naturwissenschaftlichen Fakultät der Universität Bern
% 
% \vspace{0.8cm}
% Betreut durch Prof. Dr. Carolin Schurr und Dr. Moritz Gubler
% 
% 
% Geographisches Institut \\
% Unit für Sozial- und Kulturgeographie\\
% Bern, \today
% 
% \end{titlepage}



\clearpage


\chapter*{Abstract}

Diese Bachelorarbeit entwickelt und dokumentiert ein kritisch-digitales Forschungsdesign zur raumbezogenen Erfassung situierten (Un\nobreakdash"=)Wohlbefindens aus intersektionaler Perspektive. Theoretisch knüpft die Arbeit an \emph{emotional geographies} und Intersektionalitätsforschung an und versteht Wohlbefinden als situierte Erfahrung, die durch räumliche und soziale Lagen mitgeprägt wird. Methodisch verbindet sie den \glsxtrshort{ema}/\glsxtrshort{gema}-Ansätze mit einer offenen, nachvollziehbaren Forschungsinfrastruktur: Mit der eigens entwickelten App InterMind werden wiederholte, geolokalisierte Befragungen im Alltag erhoben; Code und Workflows sind quelloffen und auf Transparenz, Nachnutzbarkeit und Datenschutz ausgelegt. Eine explorative Pilotstudie dient der Prüfung von Machbarkeit, Erhebungslogistik, Datenqualität und Nutzererfahrung sowie der Frage, ob die Datenstruktur für intersektionale Mehrebenenmodelle (\glsxtrshort[noindex]{i-maihda}) geeignet ist. Die Pilotstudie verfolgt ausdrücklich keinen inhaltlichen Wirksamkeitsnachweis; sie validiert Prozesse und macht methodische Grenzen sichtbar. Der Beitrag der Arbeit liegt in (1) der Bereitstellung einer quelloffenen  Erhebungsplattform für kontextsensitive Wiederholungsmessungen (\glsxtrshort{ema}/\glsxtrshort{gema}), (2) der Dokumentation eines reproduzierbaren Analysepfads und (3) einer reflektierten methodischen Einordnung, die Anschlussstellen für grössere, heterogenere Stichproben und weiterführende Mixed-Methods-Studien aufzeigt.


\vfill
\noindent
Diese Arbeit ist lizenziert unter einer 
\href{https://creativecommons.org/licenses/by-sa/4.0/}
{CC BY-SA 4.0 Lizenz}. \ccbysa


\clearpage

\pagenumbering{roman}

\begingroup
  \fontsize{10.4pt}{1.0em}\selectfont
  \setlength{\parskip}{0pt}
  \setlength{\itemsep}{0pt plus 0.02pt}
  \tableofcontents\null
\endgroup

% common abkürzungen beim ersten mal nicht ausführen
\glsunset{vgl}
\glsunset{bspw}
\glsunset{gps}
\glsunset{zb}
\glsunset{ua}
\glsunset{etc}
\glsunset{hiv}
\glsunset{gis}
\glsunset{kap}
\glsunset{s}

\glsadd{vgl}
\glsadd{bspw}
\glsadd{gps}
\glsadd{zb}
\glsadd{ua}
\glsadd{etc}
\glsadd{hiv}
\glsadd{gis}
\glsadd{kap}
\glsadd{s}

% Glossar ohne erzwungenen Seitenumbruch
\begingroup
  \fontsize{10.4pt}{1.0em}\selectfont
  \setstretch{1.2}
  \let\clearpage\relax
  \let\cleardoublepage\relax
  \printglossary[
    type=\acronymtype,
    title=Abkürzungsverzeichnis,
    toctitle=Abkürzungsverzeichnis,
    nonumberlist,
    nogroupskip
  ]
\endgroup



\begingroup
  \let\clearpage\relax
  \let\cleardoublepage\relax
  \listoffigures
  \listoftables
\endgroup


\clearpage
\pagenumbering{arabic}
% LTeX: language=de-CH
\chapter{Einleitung} \label{sec:einleitung}

Eine Parkbank steht am Rande eines kleinen Platzes. Unter den Füssen Beton, ein Baum wirft etwas Schatten, im Hintergrund sind Kinderstimmen zu hören. Doch dieser Ort löst nicht bei allen dasselbe aus. Für manche bedeutet er Ruhe, für andere Anspannung oder Distanz. Solche situativen Emotionen entstehen im Zusammenspiel materieller Eigenschaften (Licht, Geräusche, Gerüche, Temperatur), sozialer Dynamiken und individueller Erfahrungen -- und sie sind durch soziale Positionierungen mitgeprägt. In dieser Arbeit rücke ich dieses situative, kontextgebundene Erleben in den Mittelpunkt.

In der Geographie haben sich seit den 2000er-Jahren \emph{emotional} und \emph{affective geographies} herausgebildet, die alltägliche Gefühle und ihre räumlichen Dimensionen untersuchen \parencite{hoSocialGeographyIII2024}. Während \emph{emotional geographies} stärker auf bewusst artikulierte Gefühle fokussieren, betonen \emph{affective geographies} die körperlich-vorsprachliche Dimension von Empfindungen. Emotionen und Affekte werden dabei als verkörperte, relationale und situierte Phänomene verstanden, die Orte prägen und durch Orte geprägt werden. 

Die Trennung zwischen Affekt und Emotion ist jedoch umstritten. Feministische Autor\genderstern innen kritisieren, dass \emph{affective geographies} häufig zu wenig machtkritisch argumentieren, weil sie Affekte als vor- oder unbewusst behandeln und dadurch soziale Positionierungen und Ungleichheiten in den Hintergrund geraten \parencite{bondiIntroductionGeographysEmotional2006,rodo-de-zarateIntersectionalitySpatialityEmotions2023}. Diese Arbeit greift daher Ansätze auf, die affektives Erleben explizit mit intersektionalen Machtverhältnissen zusammendenken.

Intersektionale Ansätze wiederum zeigen, dass solche Erfahrungen nicht entlang einzelner sozialer Merkmale -- etwa \gls[noindex]{gender}, \emph{Alter} oder \gls[noindex]{class}\footnote{Soziale Kategorien wie \gls[noindex]{race}, \gls[noindex]{gender}, \gls[noindex]{class}, \emph{Alter} oder \emph{Behinderung} werden in dieser Arbeit kursiv gesetzt, um ihre Bedeutung als sozial konstruierte, wandelbare und gesellschaftlich wirkungsmächtige Kategorien hervorzuheben. Die Begriffe \gls[noindex]{race}, \gls[noindex]{gender} und \gls[noindex]{class} werden zudem in englischer Sprache verwendet, da ihre deutschen Übersetzungen in der wissenschaftlichen Debatte umstritten sind. Ausführlichere Erläuterungen finden sich im \nameref{sec:glossary}.} -- verstanden werden können, sondern durch deren Verschränkung und die daraus entstehenden sozialen Positionierungen. Der Begriff der \gls{intersektionalitaet} macht sichtbar, dass Diskriminierungen und Privilegien nicht additiv nebeneinanderstehen, sondern sich in ihrer Verschränkung gegenseitig verstärken, überlagern oder abschwächen. In der Geographie dominieren bislang qualitative Arbeiten, die solche Differenzen und Machtverhältnisse detailliert sichtbar machen, während systematische quantitative Umsetzungen selten sind.

Gerade im Zusammenspiel dieser beiden Perspektiven ergibt sich eine forschungspraktische Herausforderung: Zwar gibt es zahlreiche qualitativ ausgerichtete Studien, die affektive Erfahrungen und intersektionale Positionierungen in alltäglichen Situationen beschreiben, jedoch fehlen bislang methodische Ansätze, mit denen sich solche Dynamiken systematisch und quantitativ erfassen lassen. Nur wenige Versuche existieren, affektives Wohlbefinden in seiner räumlich-situativen und intersektionalen Dimension zugleich empirisch zu analysieren. Genau hier setzt diese Arbeit an: Sie entwickelt ein Verfahren zur Erhebung kontextgebundener Alltagsdaten und prüft ihre Eignung für intersektionale quantitative Auswertungen.

\vspace{1em}

Aus dieser forschungspraktischen Leerstelle ergeben sich die zentrale Forschungsfrage dieser Arbeit:
\begin{quote}
\emph{Wie lässt sich der Einfluss räumlicher Umgebungen auf das affektive Wohlbefinden intersektional positionierter Personen erfassen und analysieren?}
\end{quote}

Zur Bearbeitung dieser Leitfrage werden drei spezifische Teilfragen formuliert, die deren Beantwortung aus methodischer, infrastruktureller und empirischer Perspektive vorbereiten:

\begin{enumerate}
    \item Wie muss ein Erhebungsansatz gestaltet sein, um affektives Wohlbefinden intersektional positionierter Personen gemeinsam mit relevanten Kontextmerkmalen wiederholt in situ zu erfassen?
    \item Welche Anforderungen ergeben sich aus einer kritisch-digitalen Perspektive an eine Infrastruktur, die solche Erhebungen ermöglicht, und wie lassen sie diese praktisch umsetzen?
    \item Wie geeignet sind die in einer Pilotstudie erhobenen Daten für eine intersektionale Mehrebenenmodellierung?
\end{enumerate}

Die Beantwortung dieser Fragen erfordert einen Ansatz, der wiederholte Befragungen im Alltag mit räumlichem Bezug ermöglicht. Methodisch hat sich dafür in anderen Disziplinen die \gls{ema}-Methode etabliert, die darauf abzielt, Erfahrungen möglichst unmittelbar im jeweiligen Kontext zu erfassen. Erweiterungen im Sinne der \gls{gema}-Methode beziehen zusätzlich situative Umgebungsbedingungen ein und ermöglichen so beispielsweise, über Standortdaten auch Einflüsse wie Temperatur sichtbar zu machen. In der Geographie wurden diese Methoden bislang jedoch nur vereinzelt angewendet, obwohl sie ein hohes Potenzial bieten, den Einfluss räumlicher Kontexte systematisch zu untersuchen.

Für die Durchführung von \gls{ema}- und \gls{gema}-Studien existieren verschiedene digitale Infrastrukturen. Ich zeige in dieser Arbeit auf, dass diese entweder proprietär sind oder ihre Datenverarbeitung nur teilweise nachvollziehbar ist. Für die Erhebung sensibler Daten stellt dies eine zentrale Einschränkung dar: Wenn unklar bleibt, wie Daten gespeichert, verarbeitet oder weitergegeben werden, widerspricht dies den Prinzipien von Transparenz, Datensparsamkeit und Nachvollziehbarkeit. Aus einer kritisch-digitalen Perspektive ist deshalb eine offene Infrastruktur erforderlich, die den gesamten Datenfluss überprüfbar macht und die Kontrolle über die erhobenen Daten sowohl bei den Teilnehmenden als auch bei den Forschenden belässt.

Vor diesem Hintergrund habe ich in dieser Arbeit mit der App \gls[noindex]{intermind}\footnote{\href{https://intermind.ch/app}{intermind.ch/app}} eine quelloffene Infrastruktur für \gls{ema}- und \gls{gema}-Studien entwickelt. \gls[noindex]{intermind} ermöglicht wiederholte Befragungen im Alltag, erfasst Standortdaten und speichert Daten anonymisiert. Die offene Auslegung schafft die Grundlage für eine langfristig nutzbare, überprüfbare Infrastruktur zur Erhebung kontextualisierter Alltagsdaten.

In einer explorativen Pilotstudie erprobe ich das Zusammenspiel aus Infrastruktur, Erhebungsdesign und Auswertungspfad. Dabei prüfe ich, ob die erhobenen Daten die nötige Differenzierung und Qualität für eine intersektionale Mehrebenenanalyse aufweisen und wo die Grenzen des Ansatzes liegen. Ziel ist der methodische Machbarkeitsnachweis, nicht die Generalisierung inhaltlicher Effekte.

Der Aufbau der Arbeit folgt einer klaren Abfolge von theoretischer Rahmung, methodischer Herleitung und empirischer Umsetzung. In \Cref{sec:theoretischer_rahmen} führe ich zentrale Konzepte ein: \gls{intersektionalitaet} als Analyseperspektive, affektives Wohlbefinden als Gegenstand sowie kritisch-digitale Ansätze als Leitlinie für die Gestaltung der Forschungsinfrastruktur. Darauf aufbauend verorte ich die Arbeit in \Cref{sec:methodik} im Feld wiederholter Alltagsbefragungen und diskutiere, wie sich diese mit intersektionalen Auswertungsansätzen verbinden lassen. 

Im Anschluss wende ich mich der praktischen Umsetzung zu: In \Cref{sec:entwicklung_app} beschreibe ich die Entwicklung der offenen App \gls[noindex]{intermind}, bevor ich in \Cref{sec:fragebogenentwicklung} die Konstruktion des Fragebogens darstelle. Die Durchführung und Auswertung der Pilotstudie präsentiere ich in \Cref{sec:pilotstudie}. Den Abschluss bildet \Cref{sec:diskussion}, in der ich zentrale Befunde reflektiere, methodische Implikationen diskutiere und Perspektiven für künftige Forschung skizziere.

Mit dieser Arbeit ziele ich darauf, methodische Potenziale einer intersektionalen, kontextnahen Erhebung von Wohlbefinden sichtbar zu machen und eine Grundlage für künftige Anwendungen zu schaffen.




\clearpage

% LTeX: language=de-CH

\chapter{Verflechtungen verstehen -- Begriffe und Konzepte} \label{sec:theoretischer_rahmen}

In diesem Kapitel führe ich in die zentralen Begriffe und Konzepte ein, die das Erkenntnisinteresse leiten und das methodische Vorgehen rahmen. Ausgangspunkt ist eine intersektionale Perspektive, mit der ich gesellschaftliche Unterschiede nicht isoliert, sondern in ihrer wechselseitigen Verflechtung analysiere. Anschliessend entfalte ich das Konzept des situierten (Un\nobreakdash-)Wohlbefindens als kontextabhängige, räumlich gebundene Erfahrung. Ergänzend nehme ich eine digitale Perspektive ein, die fragt, wie Daten, digitale Infrastrukturen und technologische Gestaltungsprozesse gesellschaftliche Machtverhältnisse widerspiegeln und (re)produzieren. Zusammen stelle ich diese Perspektiven als Grundlage für ein Forschungsdesign vor, das soziale Positionierung, räumliche Kontexte, situative Erfahrungen und digitale Infrastrukturen in Beziehung setzt.

\section{Verwebte Unterschiede -- Intersektionalität als Analyseinstrument}

Gesellschaftliche Wirklichkeiten sind durchzogen von komplexen Ungleichheiten. Menschen erfahren soziale Benachteiligung selten entlang nur einer einzigen Achse -- vielmehr wirken verschiedene Differenzlinien wie \gls{race}, \gls{gender} oder \gls{class} häufig gleichzeitig und verstärken sich wechselseitig. So kann die Erfahrung einer migrantischen Frau auf dem Arbeitsmarkt nicht einfach in \enquote{sexistische} und \enquote{rassistische} Diskriminierung zerlegt werden. Ihre Benachteiligung ergibt sich vielmehr aus der spezifischen Verwobenheit dieser Positionierungen, die durch keine einzelne Kategorie vollständig erfasst wird. In dieser Arbeit beziehe ich mich deshalb auf einen intersektionalen Ansatz, um diese Verflechtungen zu erfassen und einen Rahmen zu produzieren, der Ungleichheitsverhältnisse nicht isoliert betrachtet, sondern ihre Überschneidungen und Wechselwirkungen berücksichtigt.

Geprägt wird der Begriff der \gls{intersektionalitaet} von \textcite{crenshawMappingMarginsIntersectionality1991}, die auf die spezifischen Diskriminierungserfahrungen \emph{\glslink{schwarz}{Schwarzer}}\footnotemark Frauen aufmerksam macht. In ihrer Analyse von Antidiskriminierungsklagen im US-amerikanischen Arbeitsrecht zeigt sie, dass \emph{\glslink{schwarz}{Schwarze}} Frauen häufig keinen Rechtsschutz erhielten. Gerichte verhandelten Diskriminierung entweder als \emph{\gls{gender} discrimination} oder als \emph{\gls{race} discrimination}, jedoch nicht in der Verwobenheit beider Kategorien. Wurde eine Klage als \emph{\gls{gender} discrimination} geprüft, erfolgte der Vergleich mit weissen Frauen; blieben diese unbetroffen, galt die Klage als unbegründet. Wurde sie als \emph{\gls{race} discrimination} geprüft, erfolgte der Vergleich mit \emph{\glslink{schwarz}{Schwarzen}} Männern; auch hier verschwanden die spezifischen Benachteiligungen \emph{\glslink{schwarz}{Schwarzer}} Frauen. Ihre Erfahrungen fielen damit durch die Raster der bestehenden Rechtskategorien und blieben juristisch unsichtbar. Crenshaw argumentiert, dass feministische und antirassistische Theorien in ähnlicher Weise unzureichend sind, um Mehrfachdiskriminierung zu erfassen, und entwickelt \gls{intersektionalitaet} als analytisches Instrument zur Beschreibung solcher überlagerten Ungleichheitsverhältnisse \parencite[\gls{vgl}][]{hancockWhenMultiplicationDoesnt2007}.

\footnotetext{Ich schreibe den Begriff \enquote{\gls{schwarz}} mit grossem Anfangsbuchstaben und verwende ihn als politische Selbstbezeichnung von Menschen, die im Kontext rassistischer Machtverhältnisse positioniert werden. Der Begriff bezeichnet keine biologistische Eigenschaft, sondern eine soziale Positionierung; die Grossschreibung dient der Abgrenzung von äusserlichen Zuschreibungen \parencite{oguntoyeFarbeBekennenAfrodeutsche1986}. }

Ausgangspunkt dieser theoretischen Perspektive ist der Black Feminist Thought, welcher unter anderen in den Arbeiten von \textcite{hooksAintWomanBlack1981}, \textcite{lordeSisterOutsiderEssays1984}, Kimberle~\textcite{crenshawMappingMarginsIntersectionality1991} und \textcite{collinsBlackFeministThought2002} ihren Ausdruck findet. Black Feminist Thought formuliert eine scharfe Kritik an traditionellen feministischen Ansätzen, denen vorgeworfen wird, primär die Erfahrungen weisser, privilegierter Frauen ins Zentrum zu stellen und somit die Lebensrealitäten \emph{\glslink{schwarz}{Schwarzer}} Frauen zu marginalisieren. \textcite{crenshawMappingMarginsIntersectionality1991} entwickelt das Konzept der \gls{intersektionalitaet} explizit als Reaktion auf die Unfähigkeit bestehender theoretischer Ansätze, die spezifischen Diskriminierungserfahrungen \emph{\glslink{schwarz}{Schwarzer}} Frauen adäquat zu erfassen. Dabei verdeutlicht sie, dass Diskriminierung nicht als Summe einzelner, isolierter Erfahrungen verstanden werden kann, sondern als eigenständige Form sozialer Benachteiligung, die sich an der Überschneidung sozialer Kategorien wie \gls{race} und \gls{gender} manifestiert.

\gls{intersektionalitaet} entwickelte sich nicht allein im akademischen Kontext, sondern ist eng mit den politischen Kämpfen sozialer Bewegungen der 1970er- und 1980er-Jahre verbunden, insbesondere im Umfeld feministischer, antirassistischer und antikapitalistischer Strömungen \parencite{collinsBlackFeministThought2002}. Diese Bewegungen machten sichtbar, dass unterschiedliche Formen sozialer Ungleichheit nicht isoliert nebeneinander existieren, sondern in ihrer Verwobenheit erfahrbar werden. Damit legten sie die Grundlage für eine Perspektive, die gesellschaftliche Differenzen nicht als additive Kategorien, sondern als strukturell verknüpfte Machtverhältnisse versteht.

Zentral für die theoretische Fundierung des intersektionalen Ansatzes ist die Einsicht, dass soziale Positionierungen historisch gewachsen und gesellschaftlich konstruiert sind. Kategorien wie \gls{gender}, \gls{race} oder \gls{class} können daher nicht ohne Bezug auf die Macht- und Herrschaftsordnungen verstanden werden, in denen sie entstanden sind. Sie wirken nicht nur beschreibend, sondern ordnen Zugänge zu Ressourcen, Rechten und gesellschaftlicher Teilhabe.

Autorinnen wie Audre Lorde und bell hooks verdeutlichten, dass diese strukturellen Machtverhältnisse nicht abstrakt bleiben, sondern konkrete Auswirkungen auf individuelle Lebensrealitäten haben. Lorde betonte die Bedeutung von Differenz als Quelle von Wissen und Widerstand, während hooks die alltägliche Reproduktion patriarchaler und rassistischer Herrschaftsverhältnisse analysierte \parencite{collinsBlackFeministThought2002, hancockWhenMultiplicationDoesnt2007}. Damit trugen sie entscheidend dazu bei, Intersektionalität als kritisches Instrument zu etablieren, das sowohl strukturelle Dimensionen von Ungleichheit als auch subjektive Erfahrungen in den Blick nimmt.

Von der ursprünglich starken Fokussierung auf \textit{race} und \textit{gender} wird das Konzept in den folgenden Jahrzehnten zunehmend erweitert und schliesst heute oft eine Vielzahl sozialer Positionierungen und Identitäten ein, darunter etwa \emph{Sexualität}, \emph{Alter}, \emph{Behinderung}, \emph{Nationalität} oder \emph{Religion} \parencite{bauerIntersectionalityQuantitativeResearch2021, bowlegInvitedReflectionQuantifying2016}. Diese Erweiterung verdeutlicht die breite theoretische und empirische Anwendbarkeit von \gls{intersektionalitaet} als Analyseinstrument zur kritischen Untersuchung gesellschaftlicher Ungleichheiten und Diskriminierungserfahrungen. \gls{intersektionalitaet} hat sich somit nicht nur als theoretisches Konzept, sondern auch als methodische Grundlage etabliert, welche insbesondere in feministisch und sozialwissenschaftlich orientierten Diskursen verwendet wird, um die komplexen Wechselwirkungen gesellschaftlicher Machtverhältnisse zu analysieren.

\vspace{2em}

Die Anwendung intersektionaler Perspektiven auf räumliche Fragestellungen stellt eine zentrale Weiterentwicklung des ursprünglichen Konzepts der \gls{intersektionalitaet} dar. Seit den 2000er-Jahren etablierte sich eine eigenständige geographische Perspektive, die räumliche Kontextualität und situative Dimensionen sozialer Ungleichheiten explizit in den Mittelpunkt rückt \parencite{valentineTheorizingResearchingIntersectionality2007,rodo-de-zarateIntersectionalityFeministGeographies2018}.

Zentral für diesen Perspektivwechsel ist das Verständnis von Raum als gesellschaftliches Produkt. \textcite{lefebvreProductionLespace1974} betont, dass Raum kein neutrales Behältnis ist, in dem soziale Prozesse einfach stattfinden, sondern ein Produkt sozialer Praktiken, Aushandlungen und Machtbeziehungen. Raum entsteht durch Planung, Nutzung und alltägliche Routinen -- etwa durch Wohnungs- und Stadtbaupolitik, durch Verkehrs- und Infrastrukturen oder durch symbolische Markierungen wie Namen, Grenzen und Symbole. Machtverhältnisse schreiben sich in diese Strukturen ein und wirken dadurch stabilisierend: Wer Zugang zu bestimmten Räumen hat, wer ausgeschlossen bleibt oder wie Räume bewertet werden, reproduziert gesellschaftliche Hierarchien.

\textcite{foucaultEspacesAutres2004} erweitert diese Perspektive mit dem Konzept der Heterotopien. Damit bezeichnet er Räume, die gesellschaftliche Normen zugleich widerspiegeln und infrage stellen. Solche Räume sind ambivalent: Sie können dominante Ordnungen stabilisieren, indem sie Abweichungen räumlich „einschliessen“ (wie etwa Gefängnisse, Kliniken oder Kasernen), oder sie können alternative Formen des Zusammenlebens sichtbar machen (wie etwa Gärten, Festivals oder subkulturelle Treffpunkte). Heterotopien machen deutlich, dass Räume nicht nur materielle Anordnungen sind, sondern gesellschaftliche Ordnungen verkörpern und potenziell auch verschieben können.

Auf dieser theoretischen Grundlage argumentiert \textcite{valentineTheorizingResearchingIntersectionality2007}, dass soziale Kategorien nicht unabhängig vom Raum wirken. Ihre Bedeutung entfaltet sich erst im Zusammenspiel mit konkreten räumlichen Kontexten. Valentine zeigt dies am Beispiel muslimischer Frauen in britischen Städten. Ihre Erfahrungen im öffentlichen Raum sind nicht überall gleich, sondern variieren je nach Ort und sozialer Situation: In bestimmten Strassen oder Nachbarschaften sind sie aufgrund sichtbarer religiöser Zugehörigkeit -- etwa durch das Tragen eines Kopftuchs -- rassistischen und sexistischen Anfeindungen ausgesetzt. Dieselben Frauen erleben in anderen Kontexten, zum Beispiel in Moscheen, Community-Zentren oder stärker divers geprägten Quartieren, Sicherheit, Zugehörigkeit und Anerkennung. Entscheidend ist damit nicht allein die soziale Positionierung, sondern deren situative Übersetzung in räumliche Erfahrungen. Der Raum fungiert nicht als neutraler Hintergrund, sondern als aktiver Vermittler, der Zugehörigkeit ermöglichen oder ausschliessen kann. Ungleichheiten sind somit nicht nur verteilt im Raum, sondern werden durch räumliche Anordnungen hervorgebracht und für unterschiedliche Gruppen in spezifischer Weise erfahrbar gemacht.

\textcite{mccallComplexityIntersectionality2005} unterscheidet drei methodische Zugänge zu \gls{intersektionalitaet}. Der \emph{interkategoriale} Ansatz vergleicht festgelegte soziale Kategorien miteinander, um deren Überschneidungen sichtbar zu machen -- etwa indem Lohnunterschiede zwischen \emph{\glslink{schwarz}{Schwarzen}} Frauen, weissen Frauen, \emph{\glslink{schwarz}{Schwarzen}} Männern und weissen Männern analysiert werden. Der \emph{intrakategoriale} Ansatz richtet den Blick auf Unterschiede innerhalb einer einzelnen Kategorie, insbesondere dort, wo diese intern heterogen ist. So kann etwa untersucht werden, wie sich die Erfahrungen von Frauen unterscheiden, je nachdem ob sie gleichzeitig rassistische oder klassistische Diskriminierung erfahren. Der \emph{antikategoriale} Ansatz schliesslich stellt die Stabilität und analytische Nützlichkeit solcher Kategorien grundsätzlich infrage und fragt, ob festgelegte Identitätsachsen nicht selbst Teil des Problems sind.

Diese Systematisierung hat auch in geographischen Arbeiten Bedeutung erlangt, da sie methodisch begründet, wie sich verschiedene Dimensionen sozialer Differenz in räumlichen Analysen miteinander verknüpfen lassen. McCall betont zudem, dass \gls{gender} nicht isoliert betrachtet werden kann, sondern als interdependente Kategorie zu verstehen ist, deren Wirkung nur im Zusammenspiel mit anderen Differenzachsen entsteht \parencite{mccallSpatialRoutesGender1998}. Diese Wechselwirkungen sind wiederum stets in spezifische räumliche und historische Kontexte eingebettet, die ihre Ausprägung und Bedeutung prägen.

Empirische Arbeiten in der Geographie operationalisieren diese theoretischen Ansätze auf unterschiedliche Weise. Ein frühes Beispiel liefert \textcite{mccallSpatialRoutesGender1998}, die mit multilevel-statistischen Analysen regionale Strukturen und geschlechtsspezifische Lohnunterschiede verknüpft. Auch wenn ihre Arbeit der expliziten intersektionalen Wende in der Geographie noch vorausgeht, verdeutlicht sie, wie sich interkategoriale Zugänge nutzen lassen, um räumliche Muster sozialer Disparitäten sichtbar zu machen. \textcite{fensterRightGenderedCity2005} entwickelt diese Perspektive weiter, indem sie narrative und ethnographische Methoden einsetzt, um alltägliche Erfahrungen von Frauen in städtischen Kontexten zu untersuchen. Sie zeigt, dass das Konzept des „Right to the City“ patriarchale Machtverhältnisse unzureichend berücksichtigt und dass Zugehörigkeit und Teilhabe durch geschlechtsspezifische Ausschlüsse strukturiert sind. Mit einem dezidiert intersektionalen Anspruch führt \textcite{rodo-de-zarateDevelopingGeographiesIntersectionality2014} schliesslich ein Instrument ein, das soziale Positionierungen, emotionale Dimensionen und Orte systematisch miteinander verbindet. Die \emph{Relief Maps} ermöglichen es, subjektive Erfahrungen räumlicher Ungleichheit nicht nur zu erfassen, sondern auch visuell darzustellen und vergleichbar zu machen.

\vspace{2em}

Obwohl intersektionale Forschung historisch in qualitativen und aktivistischen Traditionen verankert ist, gewinnen quantitative Verfahren zunehmend an Relevanz, insbesondere in sozialpolitischen und raumplanerischen Kontexten \parencite{bauerIntersectionalityQuantitativeResearch2021}. Diese Verfahren bieten die Möglichkeit, strukturelle Muster intersektionaler Benachteiligung über grössere Stichproben sichtbar und empirisch überprüfbar zu machen.

Neuere methodische Entwicklungen wie \gls{i-maihda} \parencite[\gls{ua}][]{evansMultilevelApproachModeling2018,bellExtendingIntersectionalMultilevel2023} versuchen, dieser Herausforderung zu begegnen, indem sie intersektionale Positionierungen nicht als feste Gruppenmerkmale behandeln, sondern als dynamische, verschachtelte Konstellationen modellieren. Solche Ansätze zeigen, dass auch quantitative Forschung produktiv an intersektionale Theorien anschliessen kann. Gerade in der Geographie eröffnet dies die Möglichkeit, intersektionale Ungleichheiten nicht nur statistisch nachzuzeichnen, sondern auch in ihrer räumlichen Dimension sichtbar zu machen.

Jedoch ist die Übertragung intersektionaler Theorien in quantitative Methoden mit erheblichen Herausforderungen verbunden. Zentral ist die Kritik, dass traditionelle statistische Verfahren soziale Kategorien oft eindimensional oder additiv behandeln, was der komplexen theoretischen Vorstellung intersektionaler Verschachtelungen nicht gerecht wird \parencite{hancockWhenMultiplicationDoesnt2007, bowlegInvitedReflectionQuantifying2016}. Insbesondere birgt die numerische Operationalisierung sozialer Identitäten die Gefahr, die Fluidität und Kontextabhängigkeit dieser Kategorien zu ignorieren und damit ungewollt jene komplexen Wechselwirkungen zu nivellieren, die intersektionale Ansätze ursprünglich sichtbar machen wollen \parencite{scottIntersectionalityQuantitativeMethods2017}.

Um diesen Herausforderungen zu begegnen, bedarf es einer reflexiven und kontextsensiblen Operationalisierung intersektionaler Kategorien. Dies beinhaltet, soziale Gruppen nicht als statische Entitäten zu behandeln, sondern ihre relationalen und kontextuellen Eigenschaften explizit zu berücksichtigen \parencite{rodo-de-zarateDevelopingGeographiesIntersectionality2014, websterCenteringSocialtechnicalRelations2021}.


\section{Gefühlte Orte -- Situiertes (Un-)Wohlbefinden als räumliche Erfahrung}

(Un\nobreakdash-)Wohlbefinden ist flüchtig und kontextabhängig. In diesem Abschnitt versuche ich zu entwickeln, wie Situationen entstehen, in denen Orte, Praktiken, Atmosphären und Positionierungen (Un\nobreakdash-)Wohlbefinden ermöglichen oder begrenzen.

In den Sozial- und Gesundheitswissenschaften kursieren unterschiedliche Konzepte von Wohlbefinden, die jeweils eigene theoretische Setzungen und politische Implikationen mittragen. Der aus der Psychologie stammende Begriff \enquote{Subjektives Wohlbefinden} wird häufig über standardisierte Skalen erfasst und als individuelle Eigenschaft begriffen. Kritische sozialwissenschaftliche und feministische Perspektiven -- einschliesslich geographischer Arbeiten -- weisen darauf hin, dass dieses Verständnis hochgradig individualisiert ist und zu einer neoliberalen Regierungstechnik werden kann: Es verlagert Verantwortung auf Einzelne und blendet strukturelle Ungleichheiten, zeitliche und räumliche Skalen sowie Relationen aus \parencite{atkinsonToxicEffectsSubjective2021}. Sichtbar wird dies in politischen Wohlbefindens-Indizes wie dem \emph{World Happiness Report} oder nationalen Befragungen, die Lebenszufriedenheit über Einzelfragen messen und politische Steuerung an individuelle Bewertungen koppeln. Auch Corporate-Wellbeing-Programme oder Public-Health-Strategien, die auf Resilienztrainings und Lifestyle-Optimierung setzen, illustrieren diese Tendenz: Sie fördern Anpassung an bestehende Strukturen, anstatt ungleiche Lebensbedingungen oder diskriminierende Machtverhältnisse zu problematisieren \parencite[\gls{vgl}]{atkinsonToxicEffectsSubjective2021}. Diese Kritik motiviert eine stärker relationale, räumlich-kontextualisierte und machtsensible Fassung von Wohlbefinden.

Das Konzept des \enquote{affektiven Wohlbefindens} ist zentral in den \emph{affective geographies} verankert \parencite{hoSocialGeographyIII2024}. Wohlbefinden wird hier nicht als rein innerer Zustand verstanden, sondern als Ergebnis von räumlichen Anordnungen, sozialen Praktiken und Atmosphären. \textcite{ahmedAffectiveEconomies2004} analysiert dies am Beispiel politischer Diskurse in den USA, etwa auf Webseiten der \emph{Aryan Nations}. Sie zeigt, wie Emotionen wie Hass oder Angst nicht einfach \enquote{individuell} entstehen, sondern als \enquote{affective economies} zwischen Körpern, Symbolen und Orten zirkulieren. Indem etwa Migration oder Diversität mit Bedrohung verknüpft wird, \enquote{haften} Emotionen an bestimmten Körpern und Orten -- und schaffen dadurch Zugehörigkeiten für \enquote{weisse} einerseits und Abgrenzungen gegenüber \enquote{Anderen} andererseits.

Zugleich problematisiert \textcite{hemmingsInvokingAffectCultural2005} den sogenannten \emph{affective turn} in den Kultur- und Sozialwissenschaften. Sie kritisiert insbesondere die Tendenz, Affekte als vorsprachliche, universelle Intensitäten zu deuten. Eine solche Lesart, so Hemmings, droht Unterschiede und Machtverhältnisse auszublenden, weil sie affektive Erfahrungen von historischen und sozialen Kontexten ablöst. Für diese Arbeit ist deshalb entscheidend, affektive Dimensionen mitzudenken, ohne sie zu naturalisieren: Affekte werden hier als historisch und sozial situierte Relationen verstanden, die in konkreten räumlichen Konstellationen (Un\nobreakdash-)Wohlbefinden hervorbringen.

\textcite{smithWhichBeingWellbeing2018} entwickeln als Gegenentwurf zu individualisierten Konzepten das Verständnis eines \enquote{intra-aktiven Wohlbefindens}. Sie greifen auf Karen Barads (\citeyear{baradMeetingUniverseHalfway2007}) Theorie des Agentiellen Realismus zurück, die davon ausgeht, dass Entitäten nicht unabhängig voneinander existieren und erst nachträglich in Relation treten, sondern dass sie durch materielle und diskursive \emph{Intra-Aktionen} überhaupt erst entstehen. Diese Perspektive verschiebt den Blick: Wohlbefinden ist nicht länger eine Eigenschaft isolierter Subjekte, sondern ein Effekt von Verflechtungen zwischen Menschen, Atmosphären, Infrastrukturen, Technologien und Dingen. Damit rückt eine \emph{more-than-human}-Lesart ins Zentrum, die die Mitwirkung materieller Umwelten und nicht-menschlicher Akteure ernst nimmt. 

Für die geographische Forschung eröffnet dieses Konzept neue Anschlussmöglichkeiten, etwa indem auch die Gestaltung von Stadträumen, die Materialität von Wohnumgebungen oder die Rolle digitaler Infrastrukturen als konstitutiv für Erfahrungen von (Un\nobreakdash-)Wohlbefinden verstanden werden können. Zugleich bleibt dieser Zugang sprachlich schwer zugänglich und in der Wellbeing-Forschung bislang wenig etabliert, was seine Übertragung in empirische Studien erschwert.

Vor diesem Hintergrund verwende ich in dieser Arbeit den Begriff \emph{situiertes (Un\nobreakdash-)Wohlbefinden}. Er knüpft an \textcite{leeUnderstandingDisruptedParticipation2021} an, die den Begriff in der Analyse von Sport- und Freizeitpraktiken eingeführt haben, sowie an feministische Epistemologien situierten Wissens \parencite{harawaySituatedKnowledgesScience1988}. Mit dieser Begriffswahl möchte ich Wohlbefinden nicht als innere, universelle Eigenschaft oder rein individuelles Affektgeschehen fassen, sondern als relationales, machtsensibles Erleben, das in spezifischen räumlichen und sozialen Konstellationen hervorgebracht wird. 

Die Schreibweise \enquote{(Un\nobreakdash-)} macht sichtbar, dass negative Erfahrungen wie Ausschluss, Angst oder Unsicherheit analytisch gleichwertig zu positiven Momenten von Zugehörigkeit oder Sicherheit sind und nicht lediglich als Abweichungen von einem \enquote{normalen} Zustand von Wohlbefinden verstanden werden dürfen. 

Der Begriff \emph{situiert} verweist in dreifacher Hinsicht auf die theoretische Rahmung dieser Arbeit: Erstens betont er die Verkörperung und Kontextgebundenheit von Erfahrung, die nicht unabhängig von Orten, Atmosphären oder Praktiken gedacht werden kann. Zweitens rückt er die Rolle von Machtverhältnissen und intersektionaler Positionierungen in den Vordergrund, durch die Wohlbefinden ermöglicht oder eingeschränkt wird. Drittens markiert er eine erkenntnistheoretische Haltung, wie sie \textcite{harawaySituatedKnowledgesScience1988} formuliert: Wissen -- und in diesem Fall Erfahrung -- ist stets perspektivisch, partiell und situiert, niemals neutral oder allumfassend. 

Mit dem Konzept des \emph{sitiuierten (Un\nobreakdash-)Wohlbefindens} ziele ich darauf, eine begriffliche Brücke zu schlagen: zwischen affektiven und atmosphärischen Dynamiken, den politischen Dimensionen von Macht und Zugehörigkeit sowie einer epistemologischen Reflexion über die Bedingungen, unter denen Wohlbefinden überhaupt erfahrbar und analysierbar wird.

\vspace{1em}

Aus dieser begrifflichen Herleitung folgt für meine Arbeit: Wenn (Un\nobreakdash-)Wohlbefinden als \emph{situiert} verstanden wird, muss gezeigt werden, \emph{wie} Situationen entstehen. Ich greife dafür das Konzept kollektiver Stimmungen und Atmosphären auf, das eine analytische Ebene eröffnet, über die das Zusammenspiel von Orten, Praktiken und Machtverhältnissen erfahrbar wird und konkrete Verkörperungen von (Un\nobreakdash-)Wohlbefinden sichtbar werden.

\textcite{andersonAffectiveAtmospheres2009} versteht unter \emph{affective atmospheres} kollektive, nicht vollständig repräsentierbare Stimmungen, die im Zusammenspiel räumlicher Faktoren (\gls{zb} Sichtachsen, Beleuchtung, Dichte, Lärm, Überwachung) und sozialer Ordnungen (\gls{zb} Zugangsregime, informelle Normen) entstehen. Atmosphären sind dabei nicht einfach die Summe dieser Elemente, sondern überschreiten sie, indem sie als schwer fassbare Qualitäten wirken, die zugleich materiell gebunden und flüchtig, bestimmt und unbestimmt sind. Sie prägen situativ (Un\nobreakdash-)Wohlbefinden und erklären, warum derselbe Ort für unterschiedliche Gruppen gegensätzlich wirken kann. Ein stark kontrollierter Eingangsbereich oder ein nächtlicher Platz mag für privilegierte Gruppen belebt und angenehm erscheinen, während er für marginalisierte Gruppen als belastend oder bedrohlich erfahrbar ist. Diese Differenzen sind nicht allein individualpsychologisch, sondern in intersektionalen Machtverhältnissen verankert.

Solche Machtverhältnisse lassen sich über Zugehörigkeitsordnungen analytisch fassen. \textcite{antonsichSearchingBelongingAnalytical2010} versteht Zugehörigkeit nicht als stabilen Status, sondern als relationalen, umkämpften Prozess und unterscheidet zwischen \emph{place-belongingness} (Gefühle des Dazugehörens) und \emph{politics of belonging} (Regeln und Grenzziehungen, die Zugehörigkeit herstellen und begrenzen). \textcite{painGlobalizedFearEmotional2009} zeigt in diesem Sinn, wie Sicherheit und Unsicherheit affektiv-geopolitisch produziert werden -- durch Überwachung, Kontrolle, mediale Diskurse oder lokale Praktiken. Solche Prozesse lassen sich als \emph{bordering} fassen \parencite[\gls{vgl}][]{yuval-davisBelongingPoliticsBelonging2006}: Sie materialisieren sich in Blicken, räumlichen Markierungen oder scheinbar neutralen Routinen des Zugangs und wirken damit unmittelbar in Atmosphären hinein. Zugehörigkeit und Ausschluss werden so nicht abstrakt, sondern leiblich-situativ erfahrbar und strukturieren (Un\nobreakdash-)Wohlbefinden.

Besonders prägnant illustriert dies \textcite{ahmedPhenomenologyWhiteness2007} mit einer phänomenologischen Analyse von \emph{Whiteness} im alltäglichen Raum. Sie argumentiert, dass Räume in mehrheitlich weissen Gesellschaften implizit auf weisse Körper ausgerichtet sind: Bewegungen wie das Betreten öffentlicher Gebäude verlaufen für sie \enquote{unmarkiert} und selbstverständlich. Nicht-weisse Körper dagegen stossen in denselben Situationen auf Widerstände -- etwa durch häufigere Polizeikontrollen, Blicke oder subtile Praktiken der Exklusion. Während weisse Subjekte sich im öffentlichen Raum mit einem Gefühl von Selbstverständlichkeit bewegen können, erfahren nicht-weisse Subjekte dieselben Orte als potenziell feindlich oder begrenzend. \emph{Whiteness} erscheint damit nicht nur als soziale Position, sondern als räumliche Orientierung, die alltägliche Handlungsräume strukturiert und die affektive Erfahrung von Vertrautheit oder Bedrohung ungleich verteilt.

\textcite{bissellPassengerMobilitiesAffective2010} zeigt am Beispiel von Zugfahrten, wie Atmosphären nicht nur als diffuse Stimmungen, sondern als leiblich-situative Kräfte wirksam werden. In überfüllten oder besonders lauten Waggons verdichten sich etwa Gereiztheit und Anspannung, sodass Fahrgäste bewusst bestimmte Abteile meiden, länger stehen oder Umwege in Kauf nehmen. Solche Atmosphären entstehen aus der Verwobenheit körperlicher Dispositionen (Müdigkeit, Schmerz, sensorische Empfindlichkeit) mit materiellen Arrangements (Enge, Sitzordnung, Überwachung) und sozialen Erwartungen und prägen dadurch mikro-leibliche Praktiken des Alltags. Damit macht Bissell Atmosphären körperlich und situativ fassbar: Wohlbefinden im Raum lässt sich nicht nur über strukturelle Zugehörigkeitsordnungen begreifen, sondern auch über verkörperte Rhythmen und Affekte, die in Bewegung, Stillstand und alltäglichen Routinen entstehen.

Für die vorliegende Arbeit ziehe ich daraus drei analytische Konsequenzen: (1) (Un\nobreakdash-)Wohlbefinden ist als \emph{kontextabhängiger Effekt} konkreter räumlich-sozialer Konstellationen zu untersuchen, nicht als innere Eigenschaft von Individuen; (2) Unterschiede im Erleben sind \emph{macht- und positionssensibel} zu lesen, das heisst intersektional verortet und durch Zugehörigkeitsordnungen strukturiert; (3) empirische Zugänge müssen \emph{verkörperte, situative Dynamiken} erfassen, ohne die situative Einbettung zu verlieren.

\vspace{1em}

Situiertes (Un\nobreakdash-)Wohlbefinden wird in der Geographie auf unterschiedlichen Massstabsebenen (scales) untersucht. Ein Fokus auf körpernahe, individuelle Erlebnisse erlaubt es, feinste situative Veränderungen des Wohlbefindens zu erfassen -- etwa wie sich Stress oder Entspannung im direkten Kontakt mit spezifischen Orten oder Personen zeigt. Auf einer meso-räumlichen Ebene geraten kollektive Atmosphären in Quartieren, Stadtteilen oder anderen lokalisierten Gemeinschaftsräumen in den Blick, zum Beispiel Nachbarschaften, in denen Sicherheit, Lärm oder soziale Dichte das gemeinsame Erleben prägen. Makro-räumliche Analysen beziehen hingegen nationale oder transnationale Strukturen ein -- etwa migrationspolitische Rahmenbedingungen oder globale Ungleichheitsordnungen --, die emotionale Erfahrungen rahmen und begrenzen \parencite{howittScaleRelationMusical1998,marstonHumanGeographyScale2005}. Dieses skalierende Verständnis macht deutlich, dass situiertes (Un\nobreakdash-)Wohlbefinden weder rein individuell noch vollständig lokal erklärbar ist, sondern immer in ein Geflecht aus Mikroerfahrungen, kollektiven Dynamiken und übergeordneten gesellschaftlich-räumlichen Strukturen eingebettet ist.

Die Geographie nutzt dieses skalierende Verständnis, um Fragen räumlicher Gerechtigkeit und sozialer Teilhabe zu untersuchen. Indem Mikroerfahrungen des Alltags mit kollektiven Dynamiken und übergeordneten gesellschaftlich-räumlichen Strukturen in Beziehung gesetzt werden, lassen sich ungleiche Verteilungen von Möglichkeiten, Sicherheit oder Zugang sichtbar machen. Damit wird situiertes (Un\nobreakdash-)Wohlbefinden zu einem analytischen Zugang, der alltägliche emotionale Erfahrungen mit den Macht- und Ungleichheitsverhältnissen verknüpft, in die sie eingebettet sind.

\section{Digitale Werkzeuge -- Data Feminism, Open Source und digitale Souveränität}
\label{sec:datafeminism}

Digitale Technologien strukturieren zunehmend gesellschaftliche Realitäten -- sie beeinflussen, was sichtbar wird, wie Wissen entsteht und wer daran teilhat. Wer Software verwendet oder entwickelt, Daten sammelt oder Infrastrukturen kontrolliert, gestaltet diese Prozesse aktiv mit. Digitale Technologien sind daher nie neutral, sondern Ausdruck bestehender Machtverhältnisse. Eine kritische Auseinandersetzung mit digitalen Technologien und Infrastrukturen muss deshalb deren soziale und politische Dimension systematisch in den Blick nehmen.

Einen geeigneten theoretischen Rahmen hierfür bietet das Konzept des \textit{Data Feminism} von \textcite{dignazioDataFeminism2020}. Ausgangspunkt ist die Einsicht, dass Daten nie neutral sind, sondern stets Ausdruck gesellschaftlicher Machtverhältnisse \parencite[\gls{vgl}][\gls{s}~53ff.]{dignazioDataFeminism2020}. Sie entstehen nicht in einem Vakuum, sondern in konkreten sozialen, politischen und ökonomischen Kontexten, die prägen, was überhaupt als \enquote{Daten} gilt, wie sie erhoben werden und welche Fragen damit gestellt werden können. Data Feminism fordert deshalb, diese Bedingungen systematisch offenzulegen. Nur so wird sichtbar, dass Daten nicht einfach objektive Abbilder einer Realität darstellen, sondern mitbestimmen, was sichtbar wird und was unsichtbar bleibt. In diesem Sinne haben Daten eine konstitutive Funktion: Sie machen bestimmte Erfahrungen zähl- und vergleichbar, während andere Perspektiven ausgeblendet oder gar unsichtbar gemacht werden.

Damit knüpft Data Feminism an intersektionale Analysen an, die verdeutlichen, dass Ungleichheiten nicht entlang einer einzigen sozialen Kategorie erklärt werden können, sondern sich über mehrere \glslink{identitaetsachse}{Achsen} wie \gls{gender}, \gls{class}, \gls{race} oder \emph{Alter} verschränken \parencite[\gls{vgl}][\gls{s}~131ff.]{dignazioDataFeminism2020}. Genau diese Verwobenheit macht sichtbar, dass auch in Datenpraktiken Ausschlüsse und Hierarchien reproduziert werden: Wer gezählt wird, wer eine eigene Kategorie erhält, wessen Erfahrungen in Zahlen übersetzt werden und wessen nicht -- all das ist Ergebnis sozialer Aushandlungen und Machtbeziehungen. Die zentrale Erkenntnis von Data Feminism liegt darin, dass Daten zweischneidig sind: Sie können zur Stabilisierung bestehender Ungleichheiten beitragen, etwa indem sie dominante Kategorien unhinterfragt fortschreiben \parencite[\gls{vgl}][\gls{s}~27]{dignazioDataFeminism2020}. Gleichzeitig können sie aber auch als Werkzeuge genutzt werden, um Machtasymmetrien offenzulegen, marginalisierte Perspektiven sichtbar zu machen und gerechtere Wissensordnungen zu schaffen.

Für die eigene Arbeit bedeutet das, dass Daten nicht einfach als technische Ressource betrachtet werden dürfen, sondern als soziale Praxis, die kritisch reflektiert werden muss. Data Feminism sensibilisiert dafür, dass auch methodische Entscheidungen -- welche Variablen erhoben werden, welche Kategorisierungen vorgenommen werden, wie Ergebnisse dargestellt werden -- nie rein technisch sind, sondern normative Annahmen transportieren. Diese Perspektive eröffnet die Möglichkeit, Datenerhebung nicht nur kritisch zu hinterfragen, sondern auch aktiv so zu gestalten, dass vielfältige Erfahrungen sichtbar werden und bestehende Ausschlüsse nicht weiter verstärkt werden. Damit liefert das Konzept eine wichtige Grundlage für eine Forschung, die soziale Gerechtigkeit nicht nur thematisiert, sondern auch methodisch einlöst.

\vspace{1em}

Digitale Infrastrukturen sind nicht nur technische Systeme, sondern Ausdruck und Austragungsorte gesellschaftlicher Machtverhältnisse. Im Anschluss an feministische Geographien lassen sie sich als Räume verstehen, in denen Fragen von Sichtbarkeit, Teilhabe und Gerechtigkeit immer wieder neu verhandelt werden \parencite{elwoodFeministDigitalGeographies2018}. Datenpraktiken und digitale Technologien erhalten damit eine politische Dimension: Sie strukturieren, wer Zugang hat, welche Perspektiven sichtbar werden und wie Wissen in Infrastrukturen eingeschrieben ist.

So zeigen \textcite{dignazioGeographiesMissingData2024} anhand einer Untersuchung von 33 zivilgesellschaftlichen Monitoring-Initiativen gegen Feminizide in 15 Ländern, wie digitale Praktiken zur Gegen-Datenproduktion eingesetzt werden können. Diese Initiativen reagieren auf systematische Leerstellen staatlicher und institutioneller Statistiken, in denen Gewalt an Frauen und queeren Personen unsichtbar bleibt. Indem sie eigene Datenbanken aufbauen, einzelne Fälle dokumentieren und in transnationale Netzwerke einspeisen, schaffen sie Räume der Erinnerung und Solidarität. Diese Praktiken machen deutlich, dass digitale Infrastrukturen nicht nur Gewalt sichtbar machen, sondern auch als Mittel widerständiger Raumpolitik fungieren können: Sie eröffnen Orte, an denen alternative Wissensordnungen etabliert werden und gesellschaftliche Machtverhältnisse herausgefordert werden.

Diese Initiativen verdeutlichen, dass digitale Infrastrukturen nicht nur Daten speichern oder bereitstellen, sondern selbst zu Schauplätzen politischer Auseinandersetzungen werden. Genau hier setzt die Debatte um digitale Souveränität an: Sie beschreibt den Anspruch, digitale Technologien nicht einfach als externe Vorgaben hinzunehmen, sondern die Bedingungen ihrer Nutzung und Gestaltung selbst mitzubestimmen. Während politische Diskurse den Begriff oft auf nationale Unabhängigkeit oder technologische Leistungsfähigkeit verengen -- etwa im Aufbau eigener Rechenzentren oder der Regulierung von Datenflüssen --, betonen politisch-geographische Ansätze, dass digitale Souveränität in spezifischen räumlichen Ordnungen hervorgebracht und verhandelt wird \parencite{glaszeContestedSpatialitiesDigital2023}. Sie ist damit weder ein klar umrissener Rechtsbegriff noch eine rein technische Fähigkeit, sondern ein umkämpftes, diskursives Konzept, das unterschiedliche normative Ansprüche bündelt \parencite{pohleDigitalSovereignty2020}. In dieser Perspektive geht es nicht nur um staatliche Kontrolle, sondern ebenso um Fragen des Zugangs, der Teilhabe und der kollektiven Befähigung, digitale Infrastrukturen kritisch zu reflektieren, partizipativ zu gestalten und als Gemeingüter zugänglich zu machen.

So zeigen \textcite{glaszeContestedSpatialitiesDigital2023}, dass digitale Souveränität in unterschiedlichen Kontexten -- etwa in der EU und in Russland -- zwar auf ähnliche Leitbilder verweist, in der praktischen Umsetzung jedoch stark divergiert. Klassische Vorstellungen von Souveränität, Territorialität und staatlicher Kontrolle konfigurieren sich im digitalen Raum neu: Cloud-Computing, Plattformökonomien oder transnationale Datenströme erzeugen räumliche Spannungen, die weder allein durch nationale Gesetze noch durch technische Standards aufgelöst werden können. Damit wird digitale Transformation nicht als neutraler Modernisierungsprozess, sondern als politisch und räumlich situierte Auseinandersetzung sichtbar.

Auch \textcite{zhangBordersBorderingSovereignty2023} heben hervor, dass digitale Souveränität immer mit Prozessen der Grenzziehung verknüpft ist. Sie zeigen, dass diese digitalen Grenzen keineswegs nur an staatlichen Territorien verlaufen, sondern auf unterschiedlichen Massstabsebenen entstehen: von geopolitischen Regulierungen bis hin zu geschlossenen WhatsApp-Gruppen oder plattforminternen Zugangsbarrieren. Solche digitalen Grenzen strukturieren, wer Zugang zu Informationen und Infrastrukturen erhält, wer ausgeschlossen bleibt und wie digitale Räume angeeignet werden können.

Vor diesem Hintergrund lässt sich digitale Souveränität nicht als ortloses Prinzip begreifen, sondern als Ergebnis konkreter räumlicher Praktiken, Infrastrukturen und Machtverhältnisse. Eine kritische Perspektive versteht darunter zugleich die kollektive Fähigkeit, digitale Infrastrukturen nicht nur zu nutzen, sondern sie auch kritisch zu reflektieren, partizipativ zu gestalten und als Gemeingüter zugänglich zu machen \parencite{baackDataficationEmpowermentHow2015}. 

\vspace{1em}

\gls{opensource}-Praktiken können in diesem Zusammenhang als konkrete Werkzeuge einer relational verstandenen digitalen Souveränität gelesen werden. Ursprünglich aus der Softwareentwicklung stammend, bezeichnen sie die Praxis, Quellcode offen zugänglich zu machen, weiterzugeben und gemeinschaftlich weiterzuentwickeln. Damit durchbrechen sie Logiken exklusiven Eigentums und verlagern technische Gestaltung in kollektive Aushandlungsprozesse \parencite{mathewFeministManifestoResistance2021}. In dieser Perspektive wird Wissen nicht als Ware begriffen, sondern als \emph{commons}, also als geteilte Ressource, die durch Gemeinschaften gepflegt und verändert wird. \textcite{mathewFeministManifestoResistance2021} zeigt, dass gerade geistige Eigentumsregime tief in eurozentrische, patriarchale und kapitalistische Machtverhältnisse eingeschrieben sind und Wissen privatisieren. Ein feministischer und dekolonialer Zugang betont demgegenüber die Notwendigkeit, den öffentlichen Raum von Wissen zurückzugewinnen und alternative Formen kollektiver Wissensproduktion zu etablieren.

Vor diesem Hintergrund argumentieren \textcite{gurumurthyDataBodiesNew2022} aus einer feministischen Perspektive, dass Souveränität über Daten nicht allein durch individuelle Kontrolle oder \enquote{data ownership} gewährleistet werden kann. Am Beispiel von Menstruations-Apps verdeutlichen sie, dass solche Ansätze unzureichend sind, um strukturelle Machtasymmetrien im Datenregime zu adressieren. Stattdessen fordern sie, Daten als \emph{social knowledge commons} zu verstehen und kollektive Praktiken der Kontrolle und Gestaltung in den Vordergrund zu stellen. In dieser Logik erscheint \gls{opensource} nicht nur als technisches Modell, sondern als relationales Prinzip, das auf Kooperation, geteilte Verantwortung und eine feministische Ethik der Datenpraktiken verweist.

Einen verwandten Zugang eröffnet \textcite{baackDataficationEmpowermentHow2015}, der am Beispiel der Open-Data-Bewegung aufzeigt, wie Praktiken und Werte aus der \gls{opensource}-Kultur in demokratische Prozesse übersetzt werden. Indem Daten nicht exklusiv von staatlichen Institutionen interpretiert, sondern offen geteilt werden, wird die Möglichkeit geschaffen, neue Öffentlichkeiten und Formen politischer Teilhabe zu etablieren. \gls{opensource} fungiert hier als Infrastruktur demokratischer Ermächtigung, die sowohl Transparenz herstellt als auch neue Rollen für intermediäre Akteure wie Journalist\genderstern innen oder Civic-Tech-Kollektive eröffnet.

\textcite{wilshireTimeRebootFeminism2024} versteht Offenheit nicht als selbstverständlich inklusives Prinzip: Sie muss aktiv gestaltet und kritisch reflektiert werden, weil offene Infrastrukturen ebenso Ausschlüsse produzieren können wie geschlossene Systeme. Entscheidend ist, wer tatsächlich Zugang erhält, wer von Offenheit profitiert und wessen Perspektiven unsichtbar bleiben. Damit verweist die Verbindung von Data Feminism und digitaler Souveränität auf eine doppelte Aufgabe: Offenheit kann nur dann emanzipatorisch wirken, wenn sie gegen hegemoniale Strukturen verteidigt und mit einer politischen Praxis der Teilhabe verbunden wird. Für diese Arbeit bedeutet das, Offenheit nicht nur als technische Eigenschaft zu verstehen, sondern als normative Orientierung: Indem ich auf offene Software und transparente Infrastrukturen setze, verorte ich mein Projekt bewusst in einer Praxis, die Machtasymmetrien nicht reproduzieren, sondern kritisch hinterfragen und gerechtere Zugänge ermöglichen soll.

\clearpage

% LTeX: language=de-CH

\chapter{Ein eigener Zugang -- methodisch und angewandt} \label{sec:methodik}

In diesem Kapitel positioniere ich den methodischen Zugang meiner Arbeit im Kontext bestehender Ansätze zur Erhebung situativer Daten. Zunächst ordne ich die verwendete Erhebungslogik begrifflich ein und grenze sie gegenüber verwandten Verfahren ab. Danach stelle ich bestehende digitale Erhebungsplattformen vor, die ähnliche Zielsetzungen verfolgen. Die vergleichende Analyse zeigt Gemeinsamkeiten, Unterschiede und Leerstellen auf und dient als Grundlage, um meine eigene Herangehensweise präzise zu positionieren.

Die konkreten technischen und inhaltlichen Umsetzungen -- etwa die Entwicklung der App \gls[noindex]{intermind} (\cref{sec:entwicklung_app}) oder die Gestaltung des Fragebogens (\cref{sec:fragebogenentwicklung}) -- erläutere ich in den folgenden Kapiteln ausführlich.


\section{Situationen erfassen -- Wiederholte Befragung mit ESM, EMA und GEMA}

Die systematische Erhebung von situiertem Wohlbefinden erfordert Methoden, die subjektive Erfahrungen möglichst unmittelbar und kontextspezifisch erfassen. Retrospektive Selbstauskünfte sind hierfür nur begrenzt geeignet, da sie Verzerrungen durch selektive Erinnerung oder nachträgliche Neubewertung unterliegen \parencite[\textit{Recall Bias} \gls{vgl},][]{kahnemanDevelopmentsMeasurementSubjective2006}. Um solche Verzerrungen zu vermeiden, wurde bereits in den 1980er-Jahren die \emph{\glsxtrfull{esm}}-Methode entwickelt. Dieses Verfahren basiert auf der mehrfach wiederholten Erhebung subjektiver Zustände im Alltag -- etwa durch zeitlich zufällig verteilte Aufforderungen an Teilnehmende, ihre momentane Stimmung oder Tätigkeit zu protokollieren \parencite{csikszentmihalyiValidityReliabilityExperienceSampling1987}. Ziel ist es, das Erleben möglichst nah am Zeitpunkt der Erfahrung und im Kontext zu erfassen. Typisch für \glsxtrshort{esm} sind kurze, wiederholte Abfragen zu spezifischen psychologischen Konstrukten, die Verzerrungen minimieren und einen Einblick in die dynamischen Prozesse individuellen Erlebens erlauben.

Während \glsxtrshort[noindex]{esm} ursprünglich primär als psychologisches Messinstrument konzipiert wurde, wurde der Ansatz in den 1990er-Jahren durch die \emph{\glsxtrfull{ema}}-Methode\footnotemark methodologisch erweitert. Mit der \glsxtrshort{ema}-Methode lassen sich zusätzlich explizit physiologische, verhaltensbezogene und weitere kontextuelle Daten erfassen \parencite{shiffmanEcologicalMomentaryAssessment2008}. \glsxtrshort{ema} erlaubt dadurch eine umfassendere Erfassung individueller Zustände und deren Kontextbedingungen. Im Gegensatz zu \glsxtrshort[noindex]{esm} ist \glsxtrshort{ema} zudem methodologisch offener für die Integration verschiedenster Datenquellen und Analyseebenen.

\footnotetext{Der Begriff \textit{ecological} verweist hierbei nicht auf \enquote{natürliche} Umgebungen, sondern auf die Wechselbeziehungen zwischen Lebewesen und ihrer jeweiligen Umwelt -- unabhängig davon, ob diese natürlich, sozial oder technisch geprägt ist.}

Mit der zunehmenden Verbreitung von \gls{gps}-fähigen Endgeräten wurde \glsxtrshort[noindex]{ema} in den 2010er-Jahren durch das Konzept der \emph{\glsxtrfull{gema}}-Methode ergänzt. \glsxtrshort{gema} kombiniert subjektive Momentaufnahmen mit objektiven, räumlich verortbaren Kontextinformationen wie Standort, Wetterbedingungen, Lärmpegel oder Bebauungsstruktur \parencite{kirchnerSpatiotemporalDeterminantsMental2016}. Im Unterschied zu \glsxtrshort{ema} legt \glsxtrshort{gema} damit besonderen Wert auf die räumliche Kontextualisierung der erhobenen Daten. Dabei werden subjektive Erfahrungen nicht nur als zeitlich-situativ, sondern explizit als räumlich-situiert betrachtet. Entscheidend ist hierbei die Möglichkeit, emotionales Erleben in direkten Bezug zum spezifischen räumlich-materiellen Kontext zu setzen und dadurch differenzierte Aussagen über räumliche Einflüsse auf das Erleben zu ermöglichen. \glsxtrshort{gema} erlaubt dadurch eine komplexere Analyse der Wechselwirkungen zwischen individuellen Erfahrungen und räumlicher Umgebung und öffnet die methodologische Perspektive für interdisziplinäre, insbesondere geographische Fragestellungen.

Tatsächlich greifen viele \gls{gema}-Studien geographische Fragestellungen auf, auch wenn sie häufig in gesundheitlichen Forschungskontexten erscheinen. So verknüpfen \textcite{gasikUsingGeographicMomentary2025} Echtzeitangaben zu Sicherheitsempfinden, Stress und Stimmung von Menschen mit \gls{hiv} in New Orleans mit räumlichen Indikatoren wie Gewaltdichte, Alkoholverkaufsstellen oder Brachflächen. \textcite{zhangGeographicEcologicalMomentary2020} untersuchen, wie situative Lärmbelästigung an unterschiedlichen Aufenthaltsorten in Abhängigkeit vom Aktivitätskontext und der täglichen akustischen Belastung wahrgenommen wird. \textcite{zhangTemporalityGeographicContexts2023} analysieren die Wirkung von Umweltfaktoren auf die Stimmung nicht nur in Echtzeit, sondern auch kumulativ und zeitverzögert.

\section{Anknüpfen und Abgrenzen -- Vergleich mit bestehenden Instrumenten}
\label{sec:vergleich_bestehender_instrumente}

Die im Rahmen dieser Arbeit entwickelte App \gls[noindex]{intermind} bewegt sich im Spannungsfeld zweier methodischer Herangehensweisen: der Echtzeiterhebung räumlich kontextualisiertem (Un-)Wohlbefinden (wie bei \gls[noindex]{urbanmind}) und der explizit intersektionalen Analyse subjektiver Raumwahrnehmungen (wie bei \textit{\gls[noindex]{reliefmaps}}). Beide bestehenden Instrumente bilden zentrale Referenzpunkte für die Konzeption des eigenen Ansatzes, da sie jeweils zentrale Teilaspekte adressieren: Während \gls[noindex]{urbanmind} eine räumlich verortete Echtzeiterhebung subjektiven Wohlbefindens umsetzt, fokussiert \textit{\gls[noindex]{reliefmaps}} auf eine reflexive, intersektionale Kartierung räumlicher Erfahrung.

Die Auswahl dieser beiden Erhebungsplattformen erfolgte zum einen aufgrund ihrer inhaltlichen Nähe zum eigenen Untersuchungsinteresse, zum anderen auch aus praktischer Zugänglichkeit: Aktuell ist \gls{urbanmind} eines der wenigen öffentlich zugänglichen \glsxtrshort{gema}-Tools, das in wissenschaftlichen Studien eingesetzt wird.\footnote{Die Dokumentation einer ähnlichen Plattform \textit{\glsxtrfull[noindex]{health}} \parencite{wrayHealthyEnvironmentsActive2025} wurde während der Entstehung dieser Arbeit als Preprint veröffentlicht.} \textit{\gls[noindex]{reliefmaps}} wiederum ist der einzige bekannte Ansatz, der intersektionale Raumwahrnehmungen systematisch operationalisiert und ist durch seine Kombination aus Emotionalität, Raumbezug und \glspl{identitaetsachse} besonders anschlussfähig für das vorliegende Projekt.

Der folgende Vergleich dient dazu, methodische Gemeinsamkeiten und Unterschiede herauszuarbeiten und den eigenen methodischen Zugang klar zu positionieren.


\subsection*{\gls[noindex]{urbanmind}: Ein vielseitige, aber nicht quelloffene Plattform}

\gls{urbanmind}\footnote{\href{https://urbanmind.info/}{urbanmind.info}} ist eine Plattform für \glsxtrshort{gema}-Studien: Sie kombiniert standardisierte Echtzeiterhebungen subjektiven Wohlbefindens mit automatisiert erfassten Geodaten und erlaubt so die kontextsensitive Analyse psychischer Gesundheit im Alltag \parencite{bakolisUrbanMindUsing2018}. Die zugrunde liegende Smartphone-App kann flexibel an unterschiedliche Forschungsfragen angepasst werden.

\gls{urbanmind} wird in mehreren Studien eingesetzt, um Zusammenhänge zwischen Umweltfaktoren und psychischer Gesundheit zu analysieren: So zeigen \textcite{bakolisUrbanMindUsing2018}, dass natürliche Elemente wie Himmel, Wasser oder Grünflächen kurzfristig das Wohlbefinden steigern können, \textcite{bergouMentalHealthBenefits2022} belegen vergleichbare Effekte für Aufenthalte an Flüssen und Kanälen, \textcite{hammoudLonelyCrowdInvestigating2021} identifizieren Zusammenhänge zwischen sozialer Dichte, dem Gefühl sozialer Inklusion und situativer Einsamkeit, und \textcite{hammoudSmartphonebasedEcologicalMomentary2022} finden Hinweise darauf, dass Vögel die psychische Verfassung nachhaltig verbessern können.

Während diese Studien wichtige Beiträge zur Analyse kontextueller Einflüsse auf psychische Gesundheit leisten, bleibt eine explizit intersektionale Perspektive bislang unberücksichtigt. Zwar erlaubt die App die Erfassung zentraler demografischer Merkmale, dieses Potenzial wird in den vorliegenden Auswertungen jedoch nicht genutzt -- obwohl entsprechende Analysen innerhalb der bestehenden Infrastruktur prinzipiell möglich wären.


\begin{figure}[h]
    \centering
    \begin{minipage}[t]{0.38\textwidth}
        \centering
        \includegraphics[width=\textwidth]{Arbeit/Bilder/urban_mind01.jpeg}
        \caption{Screenshot einer typischen Frageseite aus der \gls{urbanmind}-App}
        \label{fig:urban_mind_screenshot_1}
    \end{minipage}
    \hspace{0.1\textwidth}
    \begin{minipage}[t]{0.38\textwidth}
        \centering
        \includegraphics[width=\textwidth]{Arbeit/Bilder/urban_mind_report.jpg}
        \caption{Screenshot eines individuellen Reports aus der \gls{urbanmind}-App}
        \label{fig:urban_mind_report}
    \end{minipage}
\end{figure}

\gls{urbanmind} zeichnet sich durch eine einfache und ansprechend gestaltete Benutzeroberfläche aus, die eine niedrige Einstiegshürde für die Teilnehmenden bietet (siehe \cref{fig:urban_mind_screenshot_1}). Im Mittelpunkt stehen kurze Befragungen, die jeweils etwa drei Minuten dauern und die Teilnehmenden abhängig von der konkreten Studie \gls{bspw} zu ihrem momentanen Wohlbefinden, aktuellen Tätigkeiten sowie ihrer direkten räumlichen und sozialen Umgebung befragen. Diese Befragungen werden in den meisten Studien drei Mal täglich über eine Dauer von zwei Wochen durchgeführt. Teilnehmende werden dazu jeweils mit einer Push-Mittielung benachrichtigt und haben anschliessend jeweils eine Stunde Zeit, um die Befragung abzuschliessen.

Zusätzlich zu den standardisierten Fragebogen-Items erfasst die App kontinuierlich im Hintergrund Standortdaten mittels \gls{gps} sowie optional Gesundheits- und Aktivitätsdaten (\gls{zb} Schrittzahl, zurückgelegte Distanzen), sofern die Teilnehmenden diese Datenerfassung explizit freigeben. Weiter bietet \gls{urbanmind} die Möglichkeit, kurze Audioaufnahmen und Fotos zu teilen. Diese Mediendateien werden nicht nur für wissenschaftliche Analysen, sondern auch für künstlerische Zwecke und Öffentlichkeitsarbeit verwendet \parencite{UrbanMindPrivacy}.

Diese Praxis wirft kritische Fragen hinsichtlich Datenschutz und informierter Einwilligung auf -- insbesondere da besonders sensible Daten wie kontinuierliche Standortverläufe und Gesundheitsinformationen betroffen sind. Hinzu kommt, dass die Teilnehmenden ihre Zustimmung nicht differenziert nach Verwendungszweck (\gls{zb} Forschung, Kunst, Social Media) geben können, sondern pauschal für alle vorgesehenen Nutzungen. Informationen zur tatsächlichen Verwendung der Daten sind zudem nicht durchgängig transparent oder direkt in der App zugänglich, sondern teilweise nur über ergänzende Webseiten auffindbar.

Eine Besonderheit der App sind individuelle Reports, die Teilnehmenden automatisch und übersichtlich Rückmeldungen über ihre Interaktionen mit der Umwelt geben. So wird \gls{bspw} am Ende der Studiendauer dargestellt, bei wie vielen Befragungen die Teilnehmenden in Kontakt mit grünen Elementen waren und wie sich dies auf verschiedene Aspekte des persönlichen Wohlbefindens auswirkte (siehe \Cref{fig:urban_mind_report}). Dies dient sowohl der Reflexion über das eigene Alltagsverhalten als auch der Motivation, längerfristig an der Studie teilzunehmen.

Trotz seiner vielseitigen und benutzerfreundlichen Gestaltung weist \gls{urbanmind} einige Einschränkungen auf: Teilnehmende haben \gls{bspw} keine Möglichkeit, ihre erhobenen Rohdaten direkt zu exportieren, und auch die Löschung persönlicher Daten erfordert den expliziten Kontakt mit dem jeweiligen Forschungsteam. Zudem ist der Quellcode der App nicht öffentlich zugänglich -- eine unabhängige Prüfung oder Weiterentwicklung der technischen Infrastruktur ist somit nicht möglich.

Ich sehe in diesem Mangel an Transparenz und Offenheit eine zentrale Leerstelle im bestehenden Tool-Ökosystem -- sie bildet deshalb einen wesentlichen Ausgangspunkt für die hier entwickelte App \gls[noindex]{intermind}.

\subsection*{\gls[noindex]{reliefmaps}: Reflexive und intersektionale Kartierung retrospektiver Erfahrungen}

Im Unterschied zu \gls{urbanmind} verfolgt \gls{reliefmaps}\footnote{\href{https://reliefmaps.upf.edu/}{reliefmaps.upf.edu}} einen qualitativ-reflexiven Ansatz, der retrospektiv subjektive Erfahrungen intersektional positioniert sichtbar macht. Ursprünglich entwickelt, um Machtstrukturen, subjektive Erfahrungen und Orte relational zu verbinden, werden sie verwendet, um zu zeigen wie Privilegien und Diskriminierungen situativ variieren und durch Räume mitgeprägt werden \parencite{rodo-de-zarateDevelopingGeographiesIntersectionality2014}. Empirische Anwendungen belegen etwa, wie junge lesbische Frauen in öffentlichen Räumen komplexe Aushandlungsprozesse zwischen Identität, Raum und Machtstrukturen erfahren \parencite{rodo-de-zarateYoungLesbiansNegotiating2015}, oder wie Emotionen als räumlich situierte Marker für intersektionale Ungleichheiten genutzt werden können \parencite{rodo-de-zarateIntersectionalitySpatialityEmotions2023}. \textcite{font-casasecaMarginsGeographicalInformation2024} diskutieren \gls{reliefmaps} zudem als methodisches Werkzeug an den Rändern klassischer \gls{gis}-Traditionen, das alternative Formen des Mappings eröffnet.

Zu Beginn des Erhebungsprozesses erstellen Nutzer\genderstern innen einen Avatar auf Basis intersektional relevanter Merkmale wie \gls{gender}, \emph{Sexualität}, \gls{class}, \emph{Herkunft}, \emph{Körperbild} oder \emph{(Dis-)Ability}. Darauf aufbauend reflektieren sie in mehreren Schritten über Erfahrungen in verschiedenen Raumkategorien wie \enquote{öffentliche Räume}, \enquote{Gesundheitseinrichtungen} oder \enquote{virtuelle Räume} (siehe \cref{fig:relief_maps_plus_screenshot_1}). Für jede \glslink[noindex]{identitaetsachse}{Achse} sozialer Positionierung können in einem nächsten Schritt Orte je nach erfahrenem (Un-)Wohlsein als unterdrückend, kontrovers, neutral oder entlastend klassifiziert werden. Ergänzend können Orte direkt auf einer Karte verortet und mit freien Kommentaren sowie Emotionslabels wie \enquote{Angst}, \enquote{Sicherheit} oder \enquote{Empowerment} versehen werden. Diese Funktion fördert eine dichte, kontextualisierte Beschreibung subjektiver Erlebnisse, die sich nicht auf standardisierte Itemskalen reduzieren lässt.

\begin{figure}[htbp]
    \centering
    \includegraphics[width=\textwidth]{Arbeit/Bilder/reliefmap.png}
    \caption{Beispielhafte Ausgabe aus dem \gls{reliefmaps} Tool}
    \label{fig:relief_maps_plus_screenshot_1}
\end{figure}

Ein zentrales methodisches Merkmal von \gls{reliefmaps} ist der Versuch, die emotionale Wirkung sozialer Machtverhältnisse darstellbar zu machen -- ohne diese in eindimensionale Kausalbeziehungen zu überführen. Die Nutzer\genderstern innen bewerten ihre Erfahrungen explizit entlang einzelner \glspl{identitaetsachse}. Gleichzeitig zeigt sich hier eine zentrale methodologische Spannung: Die isolierte Betrachtung einzelner \glslink[noindex]{identitaetsachse}{Diskriminierungsachsen} widerspricht dem Grundgedanken einer \glslink[noindex]{intersektionalitaet}{intersektionaler} Analyse, der gerade auf die Verwobenheit und Gleichzeitigkeit verschiedener Machtverhältnisse verweist. Eine konsequente \glslink[noindex]{intersektionalitaet}{intersektionale} Operationalisierung bleibt damit methodisch herausfordernd.

Einige technische Merkmale von \gls{reliefmaps} sind auch im Hinblick auf die Entwicklung eigener Tools relevant. Die browserbasierte Anwendung erlaubt es Forschenden, eigenständig Projekte zu erstellen und auszuwerten. Allerdings ist der Zugang derzeit stark auf den katalanischen Kontext zugeschnitten: Verfügbare Sprachen sind aktuell nur Katalanisch, Spanisch und Englisch. Da der Quellcode nicht öffentlich zugänglich ist, bleiben Fragen zur Anpassbarkeit, Wiederverwendbarkeit und langfristigen Wartbarkeit offen. Aus methodischer Sicht stellt sich somit die Frage, inwiefern die Software übertragbar ist auf andere sprachliche, kulturelle und geografische Kontexte.

Trotz dieser Einschränkung eröffnet \gls{reliefmaps} wichtige Potenziale: Die bewusste Integration von Reflexivität, die aktive Beteiligung der Nutzer\genderstern innen an der Interpretation ihrer eigenen Erfahrungen sowie die Sichtbarmachung räumlich kontextualisierter Ungleichheiten markieren einen innovativen Zugang für \glslink[noindex]{intersektionalitaet}{intersektionale}, subjektzentrierte Geographien. Die methodische Fundierung des Tools beruht auf einem iterativen Validierungsprozess unter Einbezug feministischer, queerer und dekolonialer Perspektiven \parencite{luizdesouzaSpiralValidationProcess2025}.


\section{Offene Infrastruktur als Gegenentwurf}
\label{sec:offene_infrastruktur_als_gegenentwurf}

Ich verstehe die im Rahmen dieser Arbeit entwickelte App \gls[noindex]{intermind} (\gls{vgl} \cref{sec:entwicklung_app}) als offene, zugängliche und flexibel einsetzbare Plattform für \glsxtrshort[noindex]{ema} und \glsxtrshort[noindex]{gema}-Studien. Ich reagiere damit auf eine zentrale Leerstelle im bestehenden Tool-Ökosystem: Beide hier vorgestellten Anwendungen sind nicht quelloffen und dadurch weder vollständig nachvollziehbar noch unabhängig weiterentwickelbar. Dies betrifft nicht nur technische Details, sondern auch grundlegende Fragen der Datenverwendung, Kontrolle und Zugänglichkeit. Vor dem Hintergrund digitaler Souveränität (\gls[noindex]{vgl} \cref{sec:datafeminism}) stellt \gls[noindex]{intermind} daher bewusst nicht nur einen technischen, sondern auch einen forschungsethischen Gegenentwurf dar.

\gls[noindex]{intermind} versteht sich dabei nicht als methodische Neuerfindung, sondern als infrastrukturelle Ergänzung: Bestehende methodische Ansätze werden aufgegriffen und mit einem Fokus auf Offenheit und Modularität neu zusammengesetzt. Die Offenheit der Infrastruktur ist damit nicht nur technische Eigenschaft, sondern methodischer Anspruch.

Ziel des hier entwickelten Forschungsdesigns ist es, situiertes (Un-)Wohlbefinden nicht nur als individuelle, sondern explizit als kontextuell-räumlich bedingte Erfahrungen wiederholt zu erfassen. Dieses Studiendesign bringt gegenüber querschnittbasierten Verfahren mehrere methodische Vorteile mit sich. Erstens reduziert die wiederholte intraindividuelle Erhebung Verzerrungen durch retrospektive Einschätzungen und erlaubt eine präzisere Erfassung situativer Schwankungen \parencite{randallDevelopmentTrialMobile2013}. Zweitens ermöglicht sie eine Kontrolle individueller Basisniveaus, was insbesondere für intersektionale Analysen relevant ist, die sowohl zwischen als auch innerhalb von Personen Differenzierungen vornehmen. Drittens erlaubt die Kombination von Echtzeitbefragung und intersektionaler Mehrebenenanalyse eine kontextsensitive Modellierung der Beziehungen zwischen affektivem Zustand und Umgebung im Sinne eines relationalen, ökologisch verstandenen Raumbegriffs.


\clearpage

\chapter{«Build your own tools»: Entwicklung der App Intermind}
\label{sec:entwicklung_app}

Im Zuge dieser Arbeit wurde die App \gls{intermind} entwickelt, die als technische Grundlage für pseudonymisierte \gls[noindex]{ema} und \gls[noindex]{gema}-Befragungen dient. Die App und der in dieser Arbeit eingesetzte Fragenkatalog wurden parallel und iterativ konzipiert. Während dieser Abschnitt die technische Entwicklung der App dokumentiert, wird die inhaltliche Gestaltung des Fragebogens im \cref{sec:fragebogenentwicklung} erläutert. 

Der vollständig dokumentierte Quellcode der App ist auf \gls[noindex]{github}\footnote{\href{https://github.com/lbatschelet/intermind}{https://github.com/lbatschelet/intermind}} unter einer \gls{lic:agpl}-Lizenz veröffentlicht.


\section{From Scratch -- Warum eine eigene App?}
\label{sec:entwicklung_app_begruendung}

Um die Fragestellung dieser Arbeit zu bearbeiten, wurde eine Plattform benötigt, welche wiederholte, geolokalisierte und kontextsensitive Erhebungen im Alltag der Teilnehmenden ermöglicht. Naheliegend wäre der Rückgriff auf bestehende und in Forschung eingesetzte Plattformen wie \gls{urbanmind}. Wie in \cref{sec:vergleich_bestehender_instrumente} beschrieben, ist diese App aber nicht quelloffen und daher weder vollständig nachvollziehbar noch eigenständig anpassbar. Insbesondere bei der Erhebung sensibler Daten zu Wohlbefinden, sozialen Positionierungen und erlebter Diskriminierung ist eine transparente, kontrollierbare und sichere Datenverarbeitung jedoch essenziell.

Auch kommerzielle Lösungen wie die Marktforschungsplattform \textit{Avicenna}\footnote{\href{https://avicennaresearch.com/}{https://avicennaresearch.com/}} kommen nicht infrage -- neben hohen Lizenzkosten bieten auch sie nur eingeschränkte Anpassungs- und Kontrollmöglichkeiten und erfüllen zentrale ethische Anforderungen nicht.

Aus dieser Analyse ergibt sich die Notwendigkeit, ein eigene Plattform zu entwickeln, die diesen Anforderungen gerecht wird. Sie soll mobil und einfach nutzbar sein, Antworten im situativen Alltag der Teilnehmenden ermöglichen und Standortdaten automatisch erfassen. Dabei sollen datenschutzrechtliche und technische Hürden möglichst gering gehalten und die Umsetzung im Rahmen dieser Arbeit realisierbar sein.
Gleichzeitig soll sie so flexibel und nachhaltig gestaltet sein, dass Fragenkataloge, Inhalte und Erhebungslogik für zukünftige Forschungsvorhaben problemlos angepasst werden können.

Die Entscheidung zur Entwicklung einer eigenen Erhebungs-Plattform ist nicht nur technisch motiviert, sondern folgt auch einer forschungsethischen Logik: Wie im \cref{sec:datafeminism} dargelegt, sind digitale Infrastrukturen nie neutral, sondern Ausdruck gesellschaftlicher Machtverhältnisse. Eine transparente und kontrollierbare Datenverarbeitung ist insbesondere dann zentral, wenn -- wie im vorliegenden Projekt -- sensible Informationen zu Wohlbefinden, sozialer Zugehörigkeit und Diskriminierung erhoben werden. Die Entscheidung für eine \gls[noindex]{opensource}-Architektur ist dabei Ausdruck eines bewussten Gestaltungswillens im Sinne digitaler Souveränität: Die gesamte Infrastruktur soll nachvollziehbar, anpassbar und kollektiv weiterentwickelbar bleiben, um technologische Gestaltungsmacht nicht an proprietäre Systeme abzugeben, sondern sie partizipativ zurückzugewinnen.

\section{Konzeption und Anforderungen -- Der Weg zur eigenen Infrastruktur}
\label{sec:app_entwicklung_anforderungen}

Auf Basis der beschriebenen Anforderungen wurde zunächst ein detaillierter Anforderungskatalog entwickelt, der als zentraler Leitfaden für die weiteren Schritte der Entwicklung diente. Dieser Katalog wurde iterativ ergänzt, konkretisiert und während des gesamten Entwicklungsprozesses kontinuierlich an methodische und technische Erkenntnisse angepasst. Die Klassifikation der Anforderungen erfolgt orientiert an der in der Softwareentwicklung üblichen Unterscheidung zwischen funktionalen und nicht-funktionalen Anforderungen.

Funktionale Anforderungen definieren konkret, \textit{was} die App leisten muss, und legen somit die notwendigen Funktionen und Abläufe der Anwendung fest. Für diese Anwendung bedeutet dies insbesondere, dass die App den Teilnehmenden täglich mehrere zufällig verteilte Zeitfenster zur Beantwortung von Fragen ermittelt und jeweils zu Beginn dieser Zeiträume \glspl{pushnotification} sendet. Da gängige Webbrowser keine verlässlichen Push-Benachrichtigungen oder zeitgesteuerten Hintergrundprozesse erlauben, schliesst diese Anforderung eine browserbasierte Erhebung aus und führt zur Entscheidung für eine App-basierte Lösung. Die App erfasst bei jeder Befragung automatisiert den aktuellen \gls{gps}-Standort. Um die Erhebung flexibel und bedarfsgerecht zu gestalten, unterstützt sie verschiedene Fragetypen -- darunter Single-Choice, Multiple-Choice, Skalen-basierte Fragen (Slider) sowie Freitextfelder. Im Sinne der Selbstbestimmung über die eigenen Daten ist es funktional zwingend vorgesehen, dass Teilnehmende sämtliche mit ihrem Gerät verknüpften Daten eigenständig und dauerhaft löschen können. Die Teilnahme erfolgt vollständig pseudonym, ohne dass eine Registrierung oder die Angabe personenbezogener Daten erforderlich ist. Darüber hinaus muss die App auf \gls{android}- und \gls{ios}-Geräten lauffähig sein, in Deutsch, Englisch und Französisch verfügbar sein und die Möglichkeit zur Erweiterung um weitere Sprachen bieten. Eine ursprünglich geplante Offlinefähigkeit wurde im Verlauf der Entwicklung verworfen, da sie zu Inkompatibilitäten bei der Aktualisierung des Fragenkatalogs geführt hätte.

Nicht-funktionale Anforderungen legen fest, \textit{wie} die oben beschriebenen Funktionen umgesetzt werden sollen, und beschreiben qualitative Merkmale wie Sicherheit, Benutzerfreundlichkeit oder technische Nachvollziehbarkeit. Zu den zentralen nicht-funktionalen Anforderungen zählen Datenschutz, Datensicherheit und technische Qualität. Sämtliche Datenverarbeitungsprozesse müssen im Einklang mit dem Schweizer Datenschutzgesetz (\glsxtrshort{dsg}) sowie der Europäischen Datenschutzgrundverordnung (\glsxtrshort{dsgvo}) erfolgen. Darüber hinaus ist sicherzustellen, dass alle Datenübertragungen verschlüsselt erfolgen und keine Dritten Zugriff auf die gespeicherten Daten erhalten. Diese Ausgestaltung folgt nicht nur rechtlichen Vorgaben, sondern knüpft auch an die im \cref{sec:datafeminism} entwickelten Prinzipien einer digitalen Souveränität an, die Transparenz, Kontrolle und Selbstbestimmung in den Mittelpunkt stellt. Eine offene, modulare und nachvollziehbare Codebasis soll gewährleisten, dass Anpassungen und Erweiterungen des Systems durch andere Forschende mit minimalem Aufwand möglich sind. Dies wurde durch die Veröffentlichung der App als \gls{opensource}-Projekt auf \gls{github} umgesetzt.

Zur systematischen Umsetzung der Anforderungen wird ein iterativer Entwicklungsprozess auf Basis von \glspl{githubissue} genutzt, in dem jede funktionale und nicht-funktionale Anforderung als eigenes \glslink{githubissue}{Issue} dokumentiert und mit einem Meilenstein versehen ist, der den geplanten Umsetzungszeitpunkt markiert. Diese Meilensteine orientieren sich an vier Entwicklungsstufen: Als \textit{core \glsxtrfull{mvp}} wird die minimal funktionsfähige Version der App bezeichnet, die alle für die Durchführung der Studie zwingend notwendigen Funktionen enthält, wie etwa die zeitgesteuerte Versendung von \glspl{pushnotification}, die Erfassung des \gls{gps}-Standorts oder die Bereitstellung zentraler Fragetypen. Das \textit{extended \gls{mvp}} umfasst zusätzliche Funktionen, die den Erhebungsprozess verbessern, für die Beantwortung der Forschungsfragen jedoch nicht zwingend erforderlich sind, beispielsweise die Unterstützung mehrerer Sprachen oder zusätzliche Fragetypen. Der Meilenstein \textit{app store release} umfasst alle Aufgaben, die für die Veröffentlichung in App-Stores erforderlich sind, jedoch keinen direkten Einfluss auf die eigentliche Datenerhebung oder Kernfunktionen der App haben. Dazu zählen begleitende Arbeiten wie die Erstellung einer Projektwebsite mit Datenschutzrichtlinie, die Bereitstellung der für die App-Store-Einreichung notwendigen Assets, die Einrichtung einer kontinuierlichen Integrations- und Auslieferungspipeline (\gls{cicd}) sowie die Durchführung des formalen Prüf- und Freigabeprozesses der App-Stores. Unter \textit{future enhancements} werden schliesslich langfristig geplante Erweiterungen verstanden, die den Funktionsumfang der App über die Anforderungen der vorliegenden Arbeit hinaus erweitern. Derzeit stehen hier vor allem eine Offlinefähigkeit der App sowie die Möglichkeit einer direkten Auswertung der erhobenen Daten innerhalb der App auf der Liste, wobei noch offen ist, ob und in welchem Umfang diese Funktionen umgesetzt werden. Die Priorisierung innerhalb dieser Kategorien orientiert sich an den Forschungszielen, den rechtlichen Vorgaben, der technischen Machbarkeit sowie den in \cref{sec:datafeminism} ausgeführten Prinzipien, wobei Änderungen am Funktionsumfang während der Entwicklung fortlaufend in den entsprechenden \glslink{githubissue}{Issues} dokumentiert werden.


\section{Technische Umsetzung -- Prinzipien, Praktiken und Kompromisse}
\label{sec:app_entwicklung_technische_umsetzung}

Die technische Umsetzung folgt etablierten Prinzipien der Softwareentwicklung, insbesondere \textit{Privacy by Design} \parencite{cavoukianPrivacyDesign72009} und den Gestaltungsprinzipien von \gls{solid} \parencite{martinCleanArchitectureCraftsmans2018}. Ziel ist eine modulare, wartbare und erweiterbare Architektur, die funktionale Anforderungen effizient umsetzt und nicht-funktionale Anforderungen -- insbesondere Datenschutz und Sicherheit -- von Beginn an integriert. Dabei wird eine klare Trennung zwischen Anwendungslogik, Datenhaltung und Benutzeroberfläche konsequent umgesetzt, um spätere Anpassungen und Erweiterungen mit minimalem Eingriff in bestehende Komponenten zu ermöglichen.

Für die Entwicklung der mobilen Anwendung wurde \gls{reactnative} in Kombination mit \gls{expo} gewählt. \gls{reactnative} ist ein von \gls{meta} entwickeltes, \gls{opensource} \gls{framework}, das die Entwicklung plattformübergreifender Anwendungen mit einer einzigen Codebasis ermöglicht. Dadurch können \gls{ios}- und \gls{android}-Versionen parallel gepflegt werden, was den Entwicklungs- und Wartungsaufwand erheblich reduziert. Obwohl React Native ursprünglich von einem grossen Technologiekonzern stammt, erfolgt in diesem Projekt keinerlei Datenaustausch mit \gls{meta}, da ausschliesslich das in der Entwicklungsumgebung installierte Framework verwendet wird, das weder auf den Endgeräten der Teilnehmenden noch auf externen Servern von \gls{meta} ausgeführt wird.

\gls{expo} ergänzt React Native um eine ebenfalls \gls{opensource} integrierte Entwicklungsumgebung mit Werkzeugen für Build, Test und Veröffentlichung. Dies erlaubt es, zentrale Infrastrukturaufgaben ohne eigenes \gls{devops}-Team effizient umzusetzen. Insbesondere die Möglichkeit, native Funktionen wie Push-Benachrichtigungen, Kamera- oder Standortzugriff über ein einheitliches API zu nutzen, beschleunigt die Umsetzung und reduziert die Komplexität der Codebasis. 

Als serverseitige Infrastruktur kommt \gls{supabase} zum Einsatz -- ein \gls{opensource} \gls{backend}-as-a-Service auf Basis von \gls{postgresql}, das Authentifizierung, Autorisierung, Datenspeicherung und Schnittstellenbereitstellung integriert. Die Entscheidung für Supabase erfolgte bewusst gegen den Einsatz von \gls{firebase}, das als De-facto-Standard für mobile Anwendungen gilt und in vielen Bereichen eine einfachere Implementierung ermöglicht hätte. Firebase ist jedoch ein proprietärer Dienst von \gls{google}, der zentrale Kontrolle über die Infrastruktur ausübt, den Serverstandort nicht frei wählen lässt und potenziell die Datenhoheit der Forschenden einschränkt. Wie in \cref{sec:datafeminism} ausgeführt, stehen solche zentralistischen Strukturen im Widerspruch zu Prinzipien digitaler Souveränität. Supabase ermöglicht hingegen, den Standort des Servers (hier: Schweiz) festzulegen und bietet die Option eines vollständig selbstverwalteten und gehosteten Betriebs. Neben der offenen Lizenz und der \gls{tech:sql}-basierten Datenstruktur war auch die Möglichkeit eines kostenlosen Hostings für kleine Projekte ausschlaggebend, wodurch der Betrieb ohne zusätzliche Infrastrukturkosten möglich ist. Die Wahl dieser Toolchain stellt damit einen pragmatischen Kompromiss dar: Sie bietet die notwendige technische Leistungsfähigkeit und Flexibilität, ohne die Kontrolle über Daten an externe Plattformanbieter abzugeben.

\begin{figure}[h]
    \centering
    \begin{minipage}[t]{0.38\textwidth}
        \centering
        \includegraphics[width=\textwidth]{Arbeit/Bilder/printscreens/startscreen.jpeg}
        \caption{Startbildschirm der App \gls[noindex]{intermind}}
        \label{fig:startscreen}
    \end{minipage}
    \hspace{0.1\textwidth}
    \begin{minipage}[t]{0.38\textwidth}
        \centering
        \includegraphics[width=\textwidth]{Arbeit/Bilder/printscreens/welcome.jpeg}
        \caption{Begrüssungstext der App \gls[noindex]{intermind}}
        \label{fig:welcome}
    \end{minipage}
\end{figure}

Der Quellcode folgt einer komponentenbasierten Struktur, in der jede Funktion klar abgegrenzte Verantwortlichkeiten besitzt. Diese Struktur erleichtert nicht nur die Wiederverwendung bestehender Module, sondern unterstützt auch die Adaption der Anwendung für andere Forschungsprojekte mit ähnlichem methodischen Aufbau. Die konkreten Fragebögen (\gls[noindex]{vgl} \cref{sec:fragebogenentwicklung}) werden nicht im Quellcode gespeichert, sondern als \gls{json}-Konfigurationsdateien in der \gls{tech:db} hinterlegt. Die App lädt diese Inhalte dynamisch beim Start oder bei Bedarf nach, wodurch Änderungen am Fragenkatalog ohne App-Update möglich sind. Die Entscheidung für serverseitige Speicherung erhöht die Flexibilität, birgt jedoch den Nachteil, dass eine aktive Internetverbindung erforderlich ist. Auf eine vollständige Offlinefähigkeit wird bewusst verzichtet, um Inkonsistenzen zwischen verschiedenen App-Versionen zu vermeiden und stets aktuelle Inhalte bereitzustellen.

Die datenschutzbezogene Umsetzung basiert auf einer strikten Pseudonymisierung. Beim ersten Start generiert die App automatisch eine gerätegebundene \gls{uuid}, die für alle weiteren Interaktionen verwendet wird. Aus Sicht des Systems existieren damit keine individuellen Nutzer\genderstern innen, sondern ausschliesslich Geräte-IDs. Personenbezogene Daten wie Name, Telefonnummer oder E-Mail-Adresse werden nicht erhoben. Standortdaten werden ausschliesslich zum Zeitpunkt einer beantworteten Befragung erfasst. Die Löschung aller mit einer \gls{uuid} verknüpften Datensätze kann jederzeit direkt in der App ausgelöst werden und entfernt sämtliche Einträge aus der \gls{tech:db}.

Der Zugriffsschutz wird durch eine Zugriffskontrolle auf Zeilenebene (\gls{rls}) in der \gls{postgresql}-\gls{tech:db} realisiert. Jede Anfrage an den Server ist an die jeweilige \gls{uuid} gebunden; Abfragen liefern nur Datensätze, die mit dieser ID verknüpft sind. Alle Datenübertragungen zwischen App und Server erfolgen verschlüsselt über authentifizierte Schnittstellen. Die Serverinfrastruktur befindet sich physisch in der Schweiz und unterliegt damit dem Schweizer Datenschutzgesetz (\gls{dsg}); zusätzlich werden die Vorgaben der Europäischen Datenschutzgrundverordnung (\gls{dsgvo}) eingehalten. Die vollständigen Regelungen sind in einer öffentlich zugänglichen Datenschutzrichtlinie dokumentiert, die in der App sowie auf der Projektwebseite\footnote{\href{https://intermind.ch/privacy-policy.html}{https://intermind.ch/privacy-policy.html}} verfügbar ist.

\begin{figure}[h]
    \centering
    \begin{minipage}[t]{0.38\textwidth}
        \centering
        \includegraphics[width=\textwidth]{Arbeit/Bilder/printscreens/beschaeftigung.jpeg}
        \caption{Multiple-Choice-Frage zur aktuellen Beschäftigung}
        \label{fig:beschaeftigung}
    \end{minipage}
    \hspace{0.1\textwidth}
    \begin{minipage}[t]{0.38\textwidth}
        \centering
        \includegraphics[width=\textwidth]{Arbeit/Bilder/printscreens/zugehoerigkeit.jpeg}
        \caption{Slider-Frage zur sozialen Zugehörigkeit}
        \label{fig:zugehoerigkeit}
    \end{minipage}
\end{figure}

Die Befragungslogik berechnet nach der ersten Teilnahme täglich drei zufällige Befragungszeitpunkte, die innerhalb fester Tagesabschnitte (Morgen, Mittag/Nachmittag, Abend) ausgewählt werden. Diese Zeitpunkte werden lokal auf dem Gerät gespeichert. Zwischen zwei Befragungen wird ein Mindestabstand von zwei Stunden eingehalten, gerechnet zwischen dem Ende des vorigen und dem Beginn des nächsten Befragungsfensters, um zu vermeiden, dass Teilnehmende bei kurzfristiger Nichtverfügbarkeit mehrere Erhebungen unmittelbar hintereinander verpassen. Zum Start eines Zeitfensters wird eine Push-Benachrichtigung versendet; der Fragebogen kann innerhalb einer Stunde beantwortet werden, danach verfällt der Slot.

Die Entscheidung für dieses Zeitplanmodell orientiert sich am Design der \textit{Urban Mind}-App \parencite{bakolisUrbanMindUsing2018}, das sich in der Praxis als gut umsetzbar erwiesen hat. Die Kombination aus zufälliger Platzierung der Startzeiten innerhalb fest definierter Tagesfenster und einer begrenzten Bearbeitungsdauer ermöglicht es, Antworten zu unterschiedlichen Zeitpunkten des Tages zu erfassen und damit Variabilität im Tagesablauf der Teilnehmenden abzubilden. Gleichzeitig wird vermieden, dass Befragungen immer zu denselben Uhrzeiten stattfinden, was potenzielle Antwortmuster verzerren könnte.

Die Eckzeiten der drei Hauptzeitfenster sind als Variablen in der Anwendung hinterlegt und können für andere Studien oder Fragebogendesigns angepasst werden. Auf diese Weise lässt sich der Befragungsrhythmus flexibel anpassen, beispielsweise indem Tagesfenster auf Grundlage individueller Angaben zu Aufsteh- und Schlafenszeiten definiert werden. Eine solche Erweiterung würde auch nicht-normative Tagesrhythmen berücksichtigen und könnte die Erreichbarkeit der Teilnehmenden weiter verbessern.

\begin{figure}[h]
    \centering
    \begin{minipage}[t]{0.38\textwidth}
        \centering
        \includegraphics[width=\textwidth]{Arbeit/Bilder/printscreens/fragen_zu_dir.jpeg}
        \caption{Überleitungsbildschirm zu den einmaligen Fragen}
        \label{fig:ueberleitungsbildschirm}
    \end{minipage}
    \hspace{0.1\textwidth}
    \begin{minipage}[t]{0.38\textwidth}
        \centering
        \includegraphics[width=\textwidth]{Arbeit/Bilder/printscreens/offen_unwohl.jpeg}
        \caption{Offene Textfrage zu weiteren Gründen für Unwohlsein an diesem Ort}
        \label{fig:offene_textfrage}
    \end{minipage}
\end{figure}

Die Benutzeroberfläche ist bewusst reduziert und funktional gestaltet, um eine intuitive Bedienung zu ermöglichen und die Fragen möglichst neutral darzustellen \parencite{rogersInteractionDesignHumancomputer2023}. Die App umfasst drei Hauptbereiche: den Startbildschirm (\cref{fig:startscreen}), der standardmässig den nächstmöglichen Befragungszeitpunkt prominent anzeigt und -- sofern aktuell eine Befragung verfügbar ist -- direkt einen „Umfrage starten“-Button einblendet; den Fragebogenbereich (\cref{fig:welcome,fig:beschaeftigung,fig:zugehoerigkeit,fig:ueberleitungsbildschirm,fig:offene_textfrage}), der sowohl einleitende und überleitende Texte als auch die einzelnen Fragen in einem klar strukturierten Layout präsentiert; sowie einen Informations- und Einstellungsbereich mit Hinweisen zum Datenschutz und zur Studie.

Grafiken werden ausschliesslich auf Einleitungs-, Überleitungs- und Informationsbildschirmen eingesetzt, nicht jedoch während der eigentlichen Befragung. Diese bewusste Trennung soll sicherstellen, dass die Beantwortung der Fragen nicht durch Designelemente beeinflusst wird. Für diese visuellen Elemente kommen ausschliesslich \gls{opensource}-Vektorgrafiken von Katerina Limpitsouni\footnote{\href{https://undraw.co/}{undraw.co/}} zum Einsatz, die thematisch passend, aber stilistisch neutral gehalten sind.

Die Farbpalette ist dezent gewählt, um Barrierefreiheit zu fördern und gute Lesbarkeit unter verschiedenen Lichtbedingungen sicherzustellen. Die Navigation ist linear aufgebaut: Nach Abschluss einer Befragung kehren die Nutzenden automatisch zum Startbildschirm zurück, wodurch der Fokus klar auf den nächsten Befragungszeitpunkt gelenkt wird. Komplexe Menüs oder verschachtelte Navigationsebenen werden vermieden, um die Nutzung auch für Personen mit geringer technischer Erfahrung zu erleichtern.

\section{Von der Simulation zum Alltagstest -- Feldtest und Feinschliff}
\label{sec:app_entwicklung_feldtest}

Zur Überprüfung der technischen Funktionsfähigkeit wurde ein zweistufiges Testverfahren durchgeführt: fortlaufende Tests während der Entwicklung sowie ein abschliessender interner Pretest. Auf automatisierte Tests wurde verzichtet, da deren Relevanz zu Beginn des Projekts unterschätzt und eine nachträgliche Integration als zu aufwändig eingeschätzt wurde. Stattdessen kam ein manueller, iterativer Ansatz zum Einsatz, bei dem die App regelmässig in \glspl{emulator} unterschiedlicher Bildschirmgrössen und auf physischen Geräten geprüft wurde. Die modulare Struktur der Codebasis erleichterte dabei die gezielte Überprüfung einzelner Komponenten. Im Mittelpunkt standen die dynamische Verarbeitung des Fragenkatalogs, die Datenübertragung an das \gls{supabase}-\gls{backend}, das Verhalten bei instabiler Internetverbindung sowie die Funktionsweise der lokalen Push-Benachrichtigungen.

Der anschliessende interne Pretest wurde mit vier Personen durchgeführt, die über die offiziellen Plattformen (\gls{testflight} und \gls{googleplayconsole}) Zugang zur App erhielten und diese über zwei Wochen im Alltag nutzten. Ziel war es, zentrale Funktionen unter realen Bedingungen zu überprüfen, insbesondere das Verhalten beim ersten App-Start, die Stabilität der Datenerfassung und die Darstellung auf unterschiedlichen Geräten. Rückmeldungen zur Bedienbarkeit wurden laufend dokumentiert.

Aus den Testergebnissen ergaben sich mehrere Anpassungen. Die Logik zur Planung der Slots und Benachrichtigungen wurde grundlegend überarbeitet: Anstelle von Hintergrundprozessen werden nun sämtliche Befragungszeitpunkte direkt nach Abschluss der ersten Befragung berechnet und lokal gespeichert, wodurch die Abhängigkeit von Betriebssystemprozessen entfällt. Zudem wurden verschiedene Anpassungen an der Benutzeroberfläche umgesetzt, etwa zur optimierten Darstellung auf kleineren Bildschirmen und zur besseren Lesbarkeit von Slider-Beschriftungen. Diese Änderungen verbesserten die visuelle Konsistenz und Zuverlässigkeit der App auf unterschiedlichen Endgeräten.

\section{App-Veröffentlichung -- Prozesse, Plattformen, Abhängigkeiten}

Um die entwickelte App für die Datenerhebung bereitzustellen, war eine Veröffentlichung über die offiziellen Distributionsplattformen von \gls{apple} (\gls{ios}) und \gls{google} (\gls{android}) vorgesehen. Beide Anbieter stellen unterschiedliche technische, administrative und finanzielle Anforderungen, die den Veröffentlichungsprozess beeinflussten.

Für den \gls{apple} App Store war der Erwerb einer kostenpflichtigen Entwicklerlizenz erforderlich (CHF 100 pro Jahr). Bereits das Testen einer App auf einem physischen \gls{ios}-Gerät setzt ein solches Entwicklerkonto voraus; ohne Lizenz ist die Ausführung nur in einem \gls{emulator} möglich. Nach Einrichtung des Kontos wurde die App über das \gls{apple} Developer Portal eingereicht und durchlief den obligatorischen Prüfprozess. Eine Veröffentlichung im regulären App Store wurde zunächst abgelehnt, mit der Begründung, die App biete zu wenig inhaltlichen Mehrwert. Zum Zeitpunkt des Abschlusses dieser Arbeit war der Fall noch nicht abschliessend geklärt. Parallel konnte die App über die \gls{apple}-Plattform \gls{testflight} für öffentliche Beta-Tests bereitgestellt werden, sodass Teilnehmende über einen Einladungslink Zugriff erhielten.

\gls{google} erhebt für die Veröffentlichung im Play Store keine wiederkehrenden Gebühren, verlangt jedoch vor einer offenen Betaversion einen geschlossenen Test mit mindestens 20 Personen über zwei Wochen. Die Verwaltung erfolgt über die \gls{googleplayconsole}. Da diese Anforderung im Projektzeitrahmen nicht durch eigene Rekrutierung erfüllbar war, wurde ein externer Testdienst beauftragt (Kosten: CHF 30). Nach Abschluss des Tests und der formalen Prüfung wurde die App als offene Beta im Play Store veröffentlicht und war damit öffentlich verfügbar.

Beide Plattformen setzen zudem eine öffentlich zugängliche Datenschutzrichtlinie voraus. Hierfür musste eine eigenständige Projektwebseite\footnote{\href{https://intermind.ch/privacy-policy.html?lang=de}{intermind.ch/privacy-policy}} eingerichtet werden, auf der die vollständige Erklärung abrufbar ist. Obwohl inhaltlich bereits eine Datenschutzdokumentation vorlag, erwies sich die formale Umsetzung als zeitaufwändiger als erwartet: Neben der Erstellung einer mobilfreundlichen HTML-Version mussten die Richtlinien in einer klar strukturierten, rechtlich konsistenten Form bereitgestellt und über eine dauerhaft erreichbare URL zugänglich gemacht werden. Die einmaligen Kosten für die Domainregistrierung beliefen sich auf CHF 10; für das Hosting konnte auf bestehende Infrastruktur zurückgegriffen werden.

\section{Struktur, Qualitätssicherung und Optimierungspotenzial}

Die Entwicklung von \gls{intermind} erfolgte in \gls{typescript} unter Verwendung von \gls{reactnative} und \gls{expo}. Der komponentenbasierte Ansatz in Kombination mit den \gls{solid}-Prinzipien ermöglichte eine nachvollziehbare Strukturierung der Anwendung und erleichterte gezielte Anpassungen im Entwicklungsverlauf.

Rückblickend zeigte sich jedoch, dass eine von Beginn an systematischere Auseinandersetzung mit der Softwarearchitektur von Vorteil gewesen wäre. Zwar wurde eine modulare Struktur umgesetzt, viele Designentscheidungen wurden jedoch situativ getroffen und nicht regelmässig im Sinne eines Gesamtkonzepts überprüft. Ein methodisch enger geführter Architekturprozess hätte hier zu klareren Abhängigkeiten und stabileren Schnittstellen geführt.

Die Anwendung von Methoden wie \emph{Test-Driven Development} hätte diesen Prozess zusätzlich unterstützt, indem Schnittstellen und Verantwortlichkeiten bereits in frühen Entwicklungsphasen festgelegt worden wären. Auch automatisierte Tests und eine kontinuierliche Codeanalyse hätten dazu beigetragen, Fehler frühzeitig zu erkennen und die langfristige Wartbarkeit zu erhöhen. Während viele kleinere Schwächen pragmatisch behoben wurden, hätte ein strukturierteres Qualitätsmanagement den späteren \gls{refactoring}-Aufwand verringert.

Ein weiteres Optimierungspotenzial liegt in der Gestaltung des Interfaces zur Datenbank. Derzeit erfolgt der Datenaustausch überwiegend über verschachtelte \gls{json}-Strings, teils aus pragmatischen Gründen, um serverseitige Verarbeitung zu vermeiden. Eine stärkere Modularisierung und Entkopplung dieser Schnittstelle von der restlichen Anwendungslogik würde die Lesbarkeit verbessern, Fehlerquellen reduzieren und künftige Anpassungen -- etwa bei der Erweiterung des Datenmodells -- erleichtern.

In dieser Hinsicht weist das Projekt Parallelen zu vielen \gls{opensource}-Entwicklungen auf: Es wurde aus einem konkreten Bedarf heraus realisiert, ist funktionsfähig und dokumentiert, jedoch nicht in allen Teilen optimal strukturiert. Die Veröffentlichung des Quellcodes eröffnet allerdings auch die Möglichkeit, dass andere Entwickler\genderstern innen auf der bestehenden Basis aufbauen, Verbesserungsvorschläge einbringen oder Erweiterungen umsetzen können.


\clearpage

\chapter{Kontextspezifisch und alltagstauglich -- Entwicklung des Fragebogens}
\label{sec:fragebogenentwicklung}

Zentrales methodisches Instrument dieser Arbeit ist ein Fragebogen, der erfasst, \emph{wie räumliche Umgebungen das momentane (Un-)Wohlbefinden intersektional positionierter Personen im Alltag beeinflussen}. Die Entwicklung des Fragebogens war unabhängig von der technischen Umsetzung in der App (\gls[noindex]{vgl} \cref{sec:entwicklung_app}) konzipiert und diente zugleich dazu, deren Flexibilität und Praxistauglichkeit zu prüfen.

Kernherausforderung war es, zwei Aspekte zu verbinden: Zum einen sollten grundlegende Merkmale zur Charakterisierung der Stichprobe erhoben werden (Baseline-Modul), zum anderen das situative, affektive (Un-)Wohlbefinden im unmittelbaren räumlichen und sozialen Kontext (\gls{ema}-Modul). Die Befragung sollte dabei so kurz wie möglich bleiben, um Akzeptanz und Teilnahmebereitschaft zu sichern. Als Zielvorgaben wurden eine maximale Dauer von zehn Minuten für die Baseline und drei Minuten für die wiederholten situativen Erhebungen festgelegt. Ergänzend wurde der Fragebogen mehrsprachig in Deutsch, Englisch und Französisch umgesetzt, um den Zugang für eine breite Teilnehmendengruppe zu ermöglichen.

Die Aufteilung in ein einmaliges Baseline-Modul und wiederholte situative Erhebungen folgt direkt aus den methodischen Anforderungen der Forschungsfrage: Die Baseline dient der Charakterisierung der Stichprobe für differenzierte intersektionale Analysen, während die situativen Fragen den eigentlichen Kern der Datenerhebung bilden, indem sie (Un-)Wohlbefinden in konkreten Alltagskontexten erfassen.

Der vollständige Fragebogen ist in \cref{app:appendix_fragebogen} zu finden.

\section{Kontext schaffen -- Einmalige Eingangsbefragung}

Die einmalige Baseline-Erhebung (siehe \cref{tab:baseline-fragen}) zielte darauf ab, die sozialen Positionierungen der Teilnehmenden möglichst differenziert zu erfassen. Erhoben wurden Merkmale wie Alter, \gls{gender}, sexuelle Orientierung, Behinderung sowie soziale Klasse (\gls{class}) \parencite{bauerIntersectionalityQuantitativeResearch2021}. 

Die Erfassung von \gls[noindex]{race} erwies sich als methodisch anspruchsvoll. Im europäischen Kontext existieren kaum etablierte Kategorien, die rassifizierte Zugehörigkeiten erfassen, ohne problematische koloniale oder biologistische Zuschreibungen zu reproduzieren \parencite[\gls{vgl}][]{roigIntersectionalityEuropeDepoliticized2018}. Anders als in der US-amerikanischen Tradition, in der standardisierte Selbstkategorisierungen verbreitet sind, fehlen im hiesigen Kontext praktikable, breit akzeptierte Formate für quantitative Erhebungen. Aus diesem Grund wurde im Fragebogen lediglich erfasst, ob Teilnehmende aktuell in einem anderen Land leben als in jenem, in dem sie geboren wurden.

Auch die Erfassung von \gls{class} stellte methodische Anforderungen. Sie erfolgte über eine Kombination mehrerer sozioökonomischer Indikatoren: höchster Bildungsabschluss, aktuelle Beschäftigungssituation, Haushaltseinkommen sowie Anzahl der Haushaltsmitglieder und deren Einkommensbeitrag. Auf klassische Schemata wie \Acrfull{egp} oder \acrfull{esec} wurde verzichtet, da deren Operationalisierung detailliertere Daten zu standardisierten Berufen und sozialstrukturellen Kategorien erfordert hätte, was im Rahmen dieser Erhebung nicht praktikabel war \parencite{bihagenSocialClassEmployment2010}. Stattdessen wurde eine pragmatische, mehrdimensionale Annäherung gewählt, die zentrale Aspekte sozialer Lage abbildet, ohne den Fragebogen unnötig zu verlängern.

Zur Erfassung bereits erfahrener Diskriminierungen wurde ergänzend eine Multiple-Choice-Frage eingesetzt, die sowohl das Vorhandensein als auch die Art der Diskriminierung abfragt. Die Auswahl dieser Merkmale beruhte auf einer pragmatischen Abwägung zwischen analytischer Relevanz, praktischer Umsetzbarkeit und dem Ziel, die Befragung kurz und zugänglich zu halten.

\section{Vom Ort zur Emotion -- situativ befragen}

Der situative Teil des Fragebogens (siehe \cref{tab:wiederholte-fragen}) erfasste die unmittelbare räumliche und soziale Umgebung der Befragten, um deren Einfluss auf das momentane affektive Wohlbefinden abzubilden. Zunächst wurde zwischen Innen- und Aussenaufenthalt unterschieden, gefolgt von einer genaueren Ortskategorisierung (\gls[noindex]{zb} Zuhause, Arbeitsplatz, Café, Park, öffentlicher Verkehr). Weitere erfasste Merkmale waren die Geräuschkulisse, Sichtbarkeit von Pflanzen oder Bäumen, Lebhaftigkeit sowie die subjektiv wahrgenommene Qualität des Ortes. Die soziale Umgebung wurde durch Angaben zu anwesenden Personen und deren Beziehung zu den Befragten beschrieben.

Die Gestaltung dieser Items orientierte sich an der Urban-Mind-Studie \parencite{bakolisUrbanMindUsing2018}, wurde jedoch in kompakter Form umgesetzt. Längere standardisierte Skalen zur Umgebungsqualität (\gls[noindex]{zb} \acrfull{peqi} \parencite{bonaiutoPerceivedResidentialEnvironment2015}, \acrfull{news} \parencite{saelensNeighborhoodEnvironmentWalkability2018}) erwiesen sich aufgrund ihrer Länge und Komplexität als ungeeignet für wiederholte Erhebungen. Die kompakte Umsetzung stellt somit einen bewussten methodischen Kompromiss dar.

Nach aktuellem Forschungsstand existiert kein standardisiertes und breit eingesetztes Instrument zur Erfassung \emph{situativen} affektiven Wohlbefindens, das für mehrfache Erhebungen pro Tag konzipiert ist. Die gängigen Skalen -- etwa \gls{panas} \parencite{yountMeasuringMoodComparison2023}, WHO-5 \parencite{toppWHO5WellBeingIndex2015} oder \gls{wemwbs} \parencite{tennantWarwickEdinburghMentalWellbeing2007} -- stammen überwiegend aus der psychologischen Gesundheitsforschung und sind auf mittlere bis längere Zeiträume (\gls[noindex]{zb} die letzten zwei Wochen) ausgelegt. Sie sind in Umfang und Formulierung nicht auf hochfrequente Erhebungen zugeschnitten und würden den zeitlichen Rahmen von wenigen Minuten pro Befragung deutlich überschreiten.

Vor diesem Hintergrund wurde ein eigener, stark reduzierter Item-Satz entwickelt, um zentrale Dimensionen des Wohlbefindens situativ abbilden zu können. Ausgewählt wurden fünf Dimensionen: generelles Wohlbefinden, Zufriedenheit, Anspannung, Energie und Zugehörigkeit. Die Antworten wurden über lineare Slider-Skalen erfasst, um eine schnelle und intuitive Bearbeitung zu ermöglichen. 

Ein zentrales Merkmal des Moduls war die Einbindung intersektionaler Perspektiven auf situativer Ebene. Ziel war es, nicht nur strukturelle Positionierungen (wie im Baseline-Modul), sondern auch deren situative Wechselwirkungen mit Raum und sozialer Wahrnehmung zu erfassen. Zu diesem Zweck wurden zwei Items entwickelt, die abfragten, ob das aktuelle Zugehörigkeits- oder Fremdheitsgefühl am Ort mit der eigenen gesellschaftlichen Positionierung zusammenhängt, sowie in welchen Merkmalen sich die Befragten im Vergleich zu Anwesenden als zugehörig oder nicht zugehörig empfanden.

Die Entwicklung dieser Items orientierte sich inhaltlich an den Überlegungen von \textcite{rodo-de-zarateIntersectionalitySpatialityEmotions2023} zur räumlichen Dimension von Emotionen und deren Rolle bei der (Re-)Produktion intersektionaler Ungleichheiten. Insbesondere die von Rodó-de-Zárate vorgeschlagene Differenzierung von (Un-)Wohlbefinden in Relation zu Machtgeometrien diente als konzeptioneller Ausgangspunkt. Mangels eines standardisierten, auf situative Mehrfacherhebungen zugeschnittenen Instruments erfolgte die konkrete Formulierung jedoch in einem pragmatischen, explorativen Prozess, mit dem Ziel, die Fragen in wenigen Sekunden beantworten zu können.

Ergänzend boten zwei offene Fragen Raum für die Benennung weiterer kontextgebundener Gründe für situatives (Un-)Wohlbefinden. Diese qualitativen Elemente ermöglichen es, affektive und kontextuelle Faktoren sichtbar zu machen, die durch geschlossene Fragen nicht erfasst werden können, und verhindern so eine Reduktion komplexer Ungleichheitsverhältnisse auf rein numerische Merkmale.

\section{Klar, verständlich, iterativ -- Der Weg zum finalen Fragebogen}

Die sprachliche Gestaltung der Fragebogen-Items stellte im Entwicklungsprozess eine zentrale methodische Herausforderung dar. Ziel war es, die Befragung möglichst zugänglich, verständlich und gleichzeitig inhaltlich präzise zu gestalten. Da die Erhebung explizit auf eine intersektionale Analyse abzielt, wurde besonderer Wert darauf gelegt, die sprachliche Zugänglichkeit möglichst breit zu gewährleisten. Der Fragebogen wurde daher mehrsprachig konzipiert und auf Deutsch, Englisch sowie Französisch umgesetzt. Weitere Sprachversionen wären aus Sicht der Zugänglichkeit sinnvoll gewesen, erforderten jedoch einen hohen Übersetzungs- und Abstimmungsaufwand, um inhaltliche Konsistenz zu sichern.

Ein bewusst gewählter Bestandteil der Konzeption war eine direkte, adressierende Sprache in der \enquote{Du}-Form. Sie sollte einen niederschwelligen Zugang fördern und hierarchische Distanz zwischen Forschenden und Teilnehmenden verringern. Gleichzeitig mussten komplexe Konzepte so operationalisiert werden, dass sie in alltagsnaher, schnell erfassbarer Form vermittelt werden konnten. So wurde das Konzept der \gls{intersektionalitaet} im Einführungsteil erläutert, in den eigentlichen Items jedoch vermieden, um unnötige Barrieren zu verhindern. Stattdessen kamen allgemeinere Formulierungen wie \enquote{persönliche Merkmale} zum Einsatz.

Besondere Aufmerksamkeit erforderte die Übersetzung und Anpassung zentraler Begriffe zwischen den Sprachversionen. Im Fall von \gls[noindex]{race} stellte sich insbesondere im deutschsprachigen Kontext die Frage nach geeigneten Begrifflichkeiten, da etablierte Termini entweder ungebräuchlich, problematisch oder unpräzise sind \parencite[\gls{vgl}][]{roigIntersectionalityEuropeDepoliticized2018}. Auch bei affektiven Zustandsbeschreibungen wurden die Formulierungen nicht wörtlich, sondern sinngemäß übertragen und kulturelle Unterschiede in der Wortverwendung berücksichtigt.

Der Übersetzungsprozess war Teil eines iterativen Entwicklungsablaufs, der auf Literaturrecherche, Rückmeldungen aus der Testphase der App (siehe \cref{sec:app_entwicklung_feldtest}) und Abstimmungen mit der betreuenden Dozentin basierte. Mehrere Überarbeitungsrunden führten zu sprachlichen und strukturellen Anpassungen, die sowohl die Verständlichkeit als auch die Anschlussfähigkeit der Items verbesserten. Ein durchgängiges Kriterium war dabei, den zeitlichen und kognitiven Aufwand für Teilnehmende gering zu halten, ohne zentrale Aspekte der Forschungsfrage zu vernachlässigen.



\clearpage

\chapter{Pilotstudie}
\label{sec:pilotstudie}

Dieses Kapitel prüft, ob das in dieser Arbeit entwickelte Erhebungsinstrument und der dazugehörige Fragebogen (\gls[noindex]{vgl} \cref{sec:entwicklung_app,sec:fragebogenentwicklung}) geeignet ist, Daten zu generieren, die sich für eine intersektional-quantitative Analyse nutzen lassen.

Als Testfall dient die folgende Überprüfungsfrage:
\begin{quote}
\emph{Wie beeinflussen räumliche Umgebungen das momentane Wohlbefinden intersektional positionierter Personen im Alltag?}
\end{quote}

Die Frage ist bewusst allgemein formuliert, da sie in dieser Pilotstudie nicht vollständig beantwortet, sondern methodisch erprobt wird. Ziel ist es zu untersuchen, ob die erhobenen Daten eine statistische Auswertung grundsätzlich zulassen und welche praktischen, technischen und konzeptionellen Herausforderungen dabei sichtbar werden.

Sämtlicher Code zur Analyse ist im \gls[noindex]{github}-Repository\footnote{\href{https://github.com/lbatschelet/Designing-InterMind}{https://github.com/lbatschelet/Designing-InterMind}} dieser Arbeit verfügbar.

\section{Stichprobe}

Die Datenerhebung fand im Rahmen der einführenden Exkursion Recht auf Stadt im ersten Studienjahr des Bachelorstudiengangs Geographie an der Universität Bern im Mai 2025 statt. Zu Beginn jedes der insgesamt vier Exkursionstage erfolgte eine Einladung zur freiwilligen Teilnahme an der Studie -- beim ersten Termin von mir persönlich, an den folgenden Terminen durch die Exkursionsleitenden. Für jede teilnehmende Person begann die Erhebungsphase mit einer einmaligen Baseline-Befragung und dauerte ab diesem Zeitpunkt sieben Tage.

\subsection*{Demographische Daten aus der Baseline Befragung}

Insgesamt wurden rund \num{80} Personen zur Teilnahme eingeladen. \num{24} davon schlossen mindestens die einmalige Baseline-Befragung vollständig ab und wurden in die Stichprobe aufgenommen. \num{8} begonnene, aber nicht abgeschlossene Baseline-Befragungen wurden ausgeschlossen. Ebenfalls ausgeschlossen wurden \num{6} während der Erhebungsphase begonnene, aber nicht abgeschlossene Momentaufnahmen. Die endgültige Stichprobe umfasst somit \num{24} Personen. \cref{tab:kreuztabelle_abs} zeigt die Verteilung von sozialem Geschlecht und Altersgruppe.

\begin{longtable}{lcccS}
    \caption{Kreuztabelle: Soziales Geschlecht und Altersgruppe (absolute Häufigkeiten)}
    \label{tab:kreuztabelle_abs}\\
    
    \toprule
    \textbf{Geschlecht} & 16--25 & 26--35 & Keine Angabe & \multicolumn{1}{c}{Gesamt} \\
    \midrule
    \endfirsthead
    
    \toprule
    \textbf{Geschlecht} & 16--25 & 26--35 & Keine Angabe & \multicolumn{1}{c}{Gesamt} \\
    \midrule
    \endhead
    
    \midrule
    \multicolumn{5}{r}{\textit{Fortsetzung auf der nächsten Seite}} \\
    \endfoot
    
    \bottomrule
    \endlastfoot
    
    Mann & 12 & 2 & 1 & 15 \\
    Frau &  8 & 1 & 0 &  9 \\
    \midrule
    \textbf{Gesamt} & 20 & 3 & 1 & 24 \\
    \end{longtable}
    
  

Die Mehrheit der Teilnehmenden verfügt über eine \emph{Matura oder ein gleichwertiges Abschlusszeugnis} (\num{22}; \SI{92}{\percent}), zwei Personen (\SI{8}{\percent}) besitzen einen Hochschulabschluss. Der überwiegende Teil ist als \emph{Student\genderstern in oder Schüler\genderstern in} erwerbstätig (\num{21}; \SI{88}{\percent}), drei Personen (\SI{12}{\percent}) sind angestellt. Die grosse Mehrheit wurde im gleichen Land geboren, in dem sie derzeit lebt (\num{16}; \SI{68}{\percent}), \num{7} Personen (\SI{28}{\percent}) nicht; eine Person (\SI{4}{\percent}) machte keine Angabe.  
Alle Personen gaben keine vorhandene Behinderung an (\num{24}; \SI{100}{\percent}).

Bezüglich der sexuellen Orientierung gaben \num{17} Personen (\SI{68}{\percent}) \emph{hetero} an, jeweils drei (\SI{12}{\percent}) \emph{homosexuell} oder \emph{bisexuell}, und eine Person (\SI{4}{\percent}) \emph{queer}. 

Beim gruppierten Äquivalenzeinkommen entfallen \num{8} Personen (\SI{32}{\percent}) auf die Kategorie \emph{Sehr niedrig}, \num{6} (\SI{24}{\percent}) machten keine Angabe, \num{5} (\SI{20}{\percent}) gehören zur Kategorie \emph{Hoch}, \num{4} (\SI{16}{\percent}) zu \emph{Niedrig} und \num{1} (\SI{4}{\percent}) zu \emph{Sehr hoch}.

Die hier gewählte Darstellung trennt die einzelnen Merkmale bewusst auf, um die Zusammensetzung der Stichprobe transparent zu machen. Methodisch betrachtet widerspricht diese Entzerrung jedoch einem intersektionalen Ansatz, da \glspl[noindex]{identitaetsachse} in isolierte Kategorien zerlegt werden. Die vollständige Übersicht über die Angaben aus der Baseline Befragung ist im \cref{app:appendix_demographics}.

\subsection*{Momentaufnahmen}

Insgesamt wurden \num{106} vollständig abgeschlossene Momentaufnahmen erhoben.  
Die Verteilung der Anzahl abgeschlossener Momentaufnahmen pro Person ist in \cref{fig:survey_counts} dargestellt.

Die Verteilungen der während der Momentaufnahmen angegebenen Tätigkeiten und Aufenthaltsortkategorien sind in \cref{fig:survey_activities} bzw. \cref{fig:survey_locations} dargestellt.  
Innenräume ($n = \num{54}; \SI{51}{\percent}$) und Aussenräume ($n = \num{52}; \SI{49}{\percent}$) waren nahezu gleich häufig vertreten.

Das soziale Umfeld variierte: Etwa ein Drittel der Momentaufnahmen wurde allein durchgeführt ($n = \num{37}; \SI{35}{\percent}$), ein weiteres Drittel in Gegenwart von Freund\genderstern innen ($n = \num{28}; \SI{26}{\percent}$). Weitere Angaben betrafen die Anwesenheit von Fremden ($n = \num{10}; \SI{9}{\percent}$), Kolleg\genderstern innen ($n = \num{8}; \SI{8}{\percent}$) oder Kombinationen dieser Gruppen (\gls[noindex]{vgl} \cref{app:people_table}).

\begin{figure}[h]
    \centering
    \includegraphics[width=8cm]{Analyse/Plots/survey_counts.pdf}
    \caption{Verteilung der Anzahl abgeschlossener Momentaufnahmen pro Person}
    \label{fig:survey_counts}
\end{figure}

\begin{figure}[h]
    \centering
    \includegraphics[width=10cm]{Analyse/Plots/cat_dist_activity.pdf}
    \caption{Tätigkeit während der Momentaufnahme}
    \label{fig:survey_activities}
\end{figure}

\begin{figure}[h]
    \centering
    \includegraphics[width=10cm]{Analyse/Plots/cat_dist_location_category.pdf}
    \caption{Aufenthaltsortkategorie während der Momentaufnahme}
    \label{fig:survey_locations}
\end{figure}


\section{Quantitativ-intersektional analysieren -- Ein Widerspruch?}

Wie im theoretischen Rahmen zu Intersektionalität (\cref{sec:theoretischer_rahmen}) dargelegt, besteht eine grundlegende Spannung zwischen den theoretischen Ansprüchen intersektionaler Forschung und den Anforderungen quantitativer Analyseverfahren. Während Intersektionalität auf die komplexe, relationale und kontextabhängige Überlagerung sozialer Kategorien abzielt, verlangen statistische Modelle in der Regel klar definierte, operationalisierte Variablen. Damit einher geht die Gefahr, fluid-dynamische Identitäten in starre Kategorien zu übersetzen und deren soziale Konstruiertheit zu verschleiern \parencite{hancockWhenMultiplicationDoesnt2007, bowlegInvitedReflectionQuantifying2016}. Hinzu kommt, dass viele herkömmliche Verfahren additive oder eindimensionale Effekte modellieren, wodurch genau jene Interdependenzen und Wechselwirkungen nivelliert werden, die intersektionale Ansätze sichtbar machen wollen \parencite{scottIntersectionalityQuantitativeMethods2017}.

Diese methodische Spannung ist nicht nur ein technisches Problem, sondern berührt den Kern intersektionaler Forschung: Die Gefahr, sozial konstruierte Kategorien wie feste, unveränderliche Eigenschaften zu behandeln, steht im Widerspruch zu ihrem theoretischen Verständnis als historisch, räumlich und sozial wandelbare Konstrukte. Jede quantitative Operationalisierung muss daher reflexiv mit diesen Grenzen umgehen und das Risiko methodischer Vereinfachungen offenlegen \parencite{rodo-de-zarateDevelopingGeographiesIntersectionality2014, websterCenteringSocialtechnicalRelations2021}.

Vor diesem Hintergrund wird in dieser Pilotanalyse ein \glsxtrfull{maihda}\footnote{In der Literatur finden sich sowohl die Bezeichnungen \emph{\glsxtrfull{maihda}} als auch \emph{\glsxtrfull{i-maihda}} \parencite{evansTutorialConductingIntersectional2024}. Beide Begriffe beziehen sich auf das gleiche statistische Verfahren; die Bezeichnung mit vorangestelltem „I“ hebt den Bezug zu intersektionaler Theorie explizit hervor. In dieser Arbeit wird \emph{\gls{i-maihda}} verwendet, um diesen theoretischen Rahmen klar zu signalisieren.} eingesetzt. \gls{i-maihda} ist ein flexibles, mehrstufiges Analysemodell, das Daten in Gruppen („\glspl{stratum}“) verschachtelt, die sich aus der Kombination mehrerer sozialer Merkmale ergeben. Jede Person gehört genau zu einem solchen sozialen \gls{stratum}. Innerhalb eines sozialen \gls{stratum} können sich die Werte der untersuchten Variablen (\gls[noindex]{zb} Wohlbefinden) zwischen Personen unterscheiden, während sich gleichzeitig Unterschiede zwischen den sozialen \glspl{stratum} selbst zeigen.

Aus statistischer Sicht handelt es sich um ein hierarchisches Modell, das mindestens zwei Ebenen umfasst: \textit{Level 1} sind die einzelnen Beobachtungen, \textit{Level 2} die sozialen \glspl{stratum}. \gls{i-maihda} schätzt, wie sich die Gesamtvarianz -- also die Streuung der Messwerte im gesamten Datensatz -- auf unterschiedliche Ebenen verteilt. Dabei wird getrennt zwischen Varianz, die zwischen den sozialen \glspl{stratum} liegt, und Varianz, die innerhalb der sozialen \glspl{stratum} entsteht. Diese Zerlegung erlaubt es zu erkennen, in welchem Ausmass die Kombination sozialer Merkmale systematische Unterschiede im Outcome erklärt und wie viel der Unterschiede auf individuelle oder situative Faktoren zurückzuführen ist. In grossen Datensätzen ermöglicht dieser Ansatz die Modellierung komplexer \glspl{stratum} mit zahlreichen kombinierten Merkmalen, da pro sozielem \gls{stratum} genügend Beobachtungen vorliegen, um stabile und präzise Schätzungen zu erhalten.

Der zentrale Vorteil von \gls{i-maihda} gegenüber klassischen Regressionsmodellen liegt darin, dass nicht nur einzelne Haupteffekte und ausgewählte Interaktionsterme berücksichtigt werden, sondern jede Merkmalskombination als eigenständige Analyseeinheit behandelt wird \parencite{hancockWhenMultiplicationDoesnt2007, bowlegInvitedReflectionQuantifying2016}. Zudem ermöglicht \gls{i-maihda} die Berechnung der sogenannten „diskriminatorischen Genauigkeit“ -- ein Mass dafür, wie trennscharf die gewählten sozialen \glspl{stratum} das Outcome im jeweiligen Kontext erklären \parencite{evansTutorialConductingIntersectional2024}.

Trotz dieser Stärken ist \gls{i-maihda} kein ursprünglich intersektionales Verfahren. Es ist aus der epidemiologischen Mehrebenenanalyse hervorgegangen und wurde nicht primär entwickelt, um intersektionale Theorien oder Machtverhältnisse theoretisch zu adressieren. Seine intersektionale Anschlussfähigkeit entsteht erst durch eine bewusste, theoriegeleitete Auswahl der Merkmale, eine reflektierte Modellierung und die Einbettung der Ergebnisse in einen sozialen und politischen Kontext \parencite{grossModellingIntersectionalityQuantitative2023}. In diesem Sinne kann \gls{i-maihda} helfen, die eingangs skizzierte Spannung zwischen theoretischem Anspruch und quantitativer Operationalisierung zu verringern -- sie jedoch nicht vollständig auflösen.

\section{Versuch einer Analyse}

Ziel dieser Pilotanalyse ist es, zu prüfen, ob und in welchem Ausmass sich Unterschiede im situativen Wohlbefinden durch die Kombination mehrerer sozialer \glspl{stratum} und durch situative Kontextfaktoren erklären lassen. Dabei wird ein mehrstufiges Analyseverfahren eingesetzt, das die Messwerte auf verschiedenen Ebenen der Datenhierarchie modelliert. Da die Datengrundlage klein und unbalanciert ist -- viele \glspl{stratum} enthalten nur eine Person und die Zahl der Wiederholungsmessungen pro Person ist gering -- dienen die folgenden Schritte primär der methodischen Illustration und nicht der inhaltlich belastbaren Beantwortung der Forschungsfrage.

Als abhängige Variable wird ein \emph{Wohlbefindensindex} verwendet, der aus fünf Einzelfragen gebildet wurde: \emph{Generelles Wohlbefinden}, \emph{Zufriedenheit}, \emph{Anspannung}, \emph{Energie} und \emph{Zugehörigkeit} (\gls{vgl} \cref{tab:moments}). Alle Items sind auf einen Wertebereich von $0$ bis $1$ skaliert, wobei höhere Werte stets ein positiveres Befinden darstellen. Durch die anschliessende Aggregation mittels des geometrischen Mittels wird vermieden, dass ein sehr hoher Wert in einer Dimension einen niedrigen Wert in einer anderen vollständig ausgleichen kann; zugleich wird der Einfluss einzelner Ausreisser reduziert.

Die zeitinvarianten erklärenden Variablen sind die vier Achsen \emph{\gls[noindex]{gender}}, \emph{Altersgruppe}, \emph{sexuelle Orientierung} und \emph{Äquivalenzeinkommensgruppe} (\gls{vgl} \cref{tab:soziodemografie_gesamt}). Die eindeutige Kombination dieser Merkmale definiert ein soziales \gls{stratum}. Damit gehört jede Person genau zu einem solchen \gls{stratum}.

Die zeitvariablen Kontextmerkmale beziehen sich auf die jeweilige Situation der Momentaufnahme und umfassen Aufenthaltsort (Innen- oder Aussenraum, spezifische Ortskategorie), Anwesenheit und Art der Beziehung zu anderen Personen, Hauptaktivität, Mehrheitsvergleich sowie vier metrische Bewertungen der Umgebung: wahrgenommene Lautstärke, sichtbare Natur, Lebhaftigkeit und empfundene Angenehmheit (\gls{vgl} \cref{tab:moments}).

Zur Modellierung werden kategoriale Variablen als Dummy-Variablen kodiert, wobei jeweils eine Referenzkategorie entfällt, um die statistische Identifizierbarkeit sicherzustellen. Die vier metrischen Umweltbewertungen werden \emph{person-mean}-zentriert, indem von jeder Beobachtung der individuelle Durchschnittswert der jeweiligen Person abgezogen wird. Ein positiver Wert zeigt an, dass eine Situation lauter, naturreicher, lebhafter oder angenehmer erlebt wird als für diese Person gewöhnlich. Dieses Vorgehen trennt kurzfristige Schwankungen innerhalb einer Person von stabilen Unterschieden zwischen Personen.

\subsection*{Modellbildung}

Das erste Modell ($M0_{3L}$) dient dazu, die Gesamtvarianz des Wohlbefindens auf die verschiedenen Ebenen zu zerlegen. Die Ebenen sind:

\begin{enumerate}
    \item \textbf{Level~1:} einzelne Momentaufnahmen,
    \item \textbf{Level~2:} Personen,
    \item \textbf{Level~3:} soziale \glspl{stratum}.
\end{enumerate}

\begin{longtable}{llll ccc}
    \caption{Übersicht über soziale \glspl[noindex]{stratum}}\label{tab:strata-uebersicht}\\
    \toprule
    Geschl. & Alter & Sex. Orient. & Äquiv.-Eink. & Pers. & Befr. & Befr./Pers.\\
    \midrule
    \endfirsthead
    
    \multicolumn{7}{c}{{Tabelle \thetable{} -- Fortsetzung}} \\
    \toprule
    Geschl. & Alter & Sex. Orient. & Äquiv.-Eink. & Pers. & Befr. & Befr./Pers.\\
    \midrule
    \endhead
    
    \midrule
    \multicolumn{7}{r}{Fortsetzung auf der nächsten Seite}\\
    \endfoot
    
    \bottomrule
    \endlastfoot
    
    weiblich    & 16 -- 25    & heterosexuell & Hoch           & 3 & 13 & 4.33 \\
    männlich    & 16 -- 25    & heterosexuell & Sehr niedrig   & 3 &  9 & 3.00 \\
    männlich    & 16 -- 25    & heterosexuell & \textemdash    & 2 &  9 & 4.50 \\
    weiblich    & 16 -- 25    & heterosexuell & \textemdash    & 2 &  8 & 4.00 \\
    weiblich    & 16 -- 25    & bisexuell     & \textemdash    & 1 & 12 & 12.00\\
    männlich    & 16 -- 25    & heterosexuell & Sehr hoch      & 1 &  9 & 9.00 \\
    männlich    & 16 -- 25    & heterosexuell & Niedrig        & 1 &  8 & 8.00 \\
    männlich    & 16 -- 25    & homosexuell   & Niedrig        & 1 &  7 & 7.00 \\
    weiblich    & 16 -- 25    & heterosexuell & Sehr niedrig   & 1 &  6 & 6.00 \\
    weiblich    & 26 -- 35    & heterosexuell & Sehr niedrig   & 1 &  5 & 5.00 \\
    männlich    & 26 -- 35    & heterosexuell & Sehr hoch      & 1 &  4 & 4.00 \\
    männlich    & 16 -- 25    & homosexuell   & Hoch           & 1 &  3 & 3.00 \\
    männlich    & 16 -- 25    & heterosexuell & Hoch           & 1 &  3 & 3.00 \\
    männlich    & 16 -- 25    & homosexuell   & \textemdash    & 1 &  3 & 3.00 \\
    männlich    & 16 -- 25    & bisexuell     & Sehr niedrig   & 1 &  2 & 2.00 \\
    weiblich    & 16 -- 25    & queer         & Sehr niedrig   & 1 &  2 & 2.00 \\
    \textemdash & \textemdash & \textemdash   & \textemdash    & 1 &  2 & 2.00 \\
    männlich    & 26 -- 35    & bisexuell     & Sehr niedrig   & 1 &  1 & 1.00 \\
    
\end{longtable}

Die Schätzung ergibt, dass rund \SI{8.9}{\percent} der Gesamtvarianz zwischen den \glspl{stratum} liegt, während auf der Personenebene keine eigenständige Varianz feststellbar ist. Mit anderen Worten: Innerhalb desselben \glspl{stratum} unterscheiden sich die mittleren Wohlbefindenswerte der einzelnen Personen in diesen Daten nicht systematisch. Der Grossteil der Varianz ($\approx$\SI{91.1}{\percent}) entfällt auf kurzfristige Schwankungen zwischen verschiedenen Momentaufnahmen derselben Person.

Diese fehlende Varianz auf der Personenebene ist eine direkte Folge der Datenstruktur: Viele \glspl{stratum} bestehen nur aus einer einzelnen Person, und auch bei den übrigen \glspl{stratum} liegt nur eine geringe Zahl an Wiederholungsmessungen pro Person vor. Unter diesen Bedingungen kann das Modell keine stabilen Unterschiede zwischen Personen desselben \gls{stratum} identifizieren. Eine dreistufige Modellierung ist daher hier nicht sinnvoll; die folgenden Schritte basieren auf einer reduzierten zweistufigen Struktur:

\begin{itemize}
    \item \textbf{Level~1:} Momentaufnahmen,
    \item \textbf{Level~2:} \glspl{stratum}.
\end{itemize}

Auch das zweistufige Nullmodell $(M0_{2L})$ ergibt für die sozialen \glspl{stratum} einen \gls{icc} von $\approx$\SI{8.9}{\percent}. Damit lassen sich knapp neun Prozent der Unterschiede im situativen Wohlbefinden auf systematische Differenzen zwischen den sozialen \glspl{stratum} zurückführen.

Im nächsten Schritt $(M1_{2L})$ werden die vier Achsen (\gls[noindex]{gender}, Altersgruppe, sexuelle Orientierung, Äquivalenzeinkommen) als additive Haupteffekte aufgenommen, um den durch Einzeleffekte erklärbaren Anteil dieser Unterschiede zu bestimmen -- ohne Wechselwirkungen zu berücksichtigen. Dieses bewusste \enquote{Auseinanderlegen} verdeutlicht die Spannung zwischen intersektionaler Theorie und quantitativer Analyse: Um den spezifischen Effekt einer Merkmalskombination zu isolieren, werden zunächst die erwarteten Einzeleffekte herausgerechnet. Das verbleibende Mass wird als \emph{intersektionaler Überschuss} bezeichnet.

In diesem Modell sinkt die geschätzte Varianz zwischen den sozialen \glspl{stratum} von \num{0.003024} auf \num{0.001110}, was einer \gls{pev} von rund \SI{63}{\percent} entspricht. Somit lassen sich etwa zwei Drittel der gruppenbezogenen Unterschiede durch die additiven Effekte der vier Achsen erklären; der verbleibende Anteil von rund einem Drittel beruht ausschliesslich auf deren spezifischer Kombination und stellt den intersektionalen Überschuss dar.

Im dritten Modell $(M2_{2L})$ werden zusätzlich die situativen Kontextvariablen aufgenommen. Die geschätzte Varianz zwischen den sozialen \glspl{stratum} sinkt dadurch nahezu auf Null, und auch die verbleibende Restvarianz reduziert sich deutlich. Relativ zum Nullmodell entspricht dies einer erklärten zwischenstratalen Varianz von etwa \SI{99.9}{\percent} sowie einer erklärten Restvarianz von etwa \SI{99.3}{\percent}. Damit lassen sich die Unterschiede im Wohlbefinden zwischen den sozialen \glspl{stratum} in dieser Stichprobe nahezu vollständig durch die Kombination aus Einzelachsen und situativen Kontextfaktoren erklären. Aufgrund der kleinen und unbalancierten Stichprobe ist dieses Ergebnis jedoch mit Vorsicht zu interpretieren.


\subsection*{Analyse variierender Umwelteinflüsse zwischen sozialen Strata}

Die Forschungsfrage dieses Kapitels zielt darauf, zu verstehen, inwieweit sich situative Umweltfaktoren unterschiedlich auf das Wohlbefinden verschiedener sozialer \glspl{stratum} auswirken. Während die bisherigen Modelle lediglich Mittelwertsunterschiede zwischen den \glspl{stratum} abbildeten (\emph{Random Intercepts}), wird dieser Schritt um \emph{Random Slopes} erweitert: Dadurch lässt sich modellieren, ob und wie stark sich die Wirkung einzelner Kontextfaktoren systematisch zwischen den sozialen \glspl{stratum} unterscheidet.

Methodisch eröffnet dieser Ansatz die Möglichkeit, \gls[noindex]{ema}- und \gls[noindex]{gema}-Daten so auszuwerten, dass nicht nur konstante Gruppenunterschiede, sondern auch unterschiedliche Sensitivitäten gegenüber situativen Einflüssen sichtbar werden. In dieser Pilotanalyse dient er der Illustration des methodischen Potenzials; inhaltliche Schlussfolgerungen sind aufgrund der kleinen und unbalancierten Stichprobe nicht möglich.

Für jede der vier metrischen Umweltbewertungen (\emph{Lautstärke}, \emph{sichtbare Natur}, \emph{Lebhaftigkeit} und \emph{empfundende Angenehmheit}) wurde ein separates Mehrebenenmodell mit zufälligen Steigungen geschätzt. Die Umweltvariablen wurden vorab \emph{person-mean}-zentriert und standardisiert, sodass die Koeffizienten als Veränderung des Wohlbefindens pro Anstieg um eine Standardabweichung gegenüber dem individuellen Mittelwert interpretiert werden können. Jedes Modell enthielt die vier sozialen Achsen als feste Effekte, während für die jeweilige Umweltvariable eine variierende Steigung (\emph{Random Slope}) pro \gls{stratum} geschätzt wurde. Aus den Modellen wurden stratum-spezifische Steigungen mit 95\%-Konfidenzintervallen extrahiert und in Tabelle~\ref{tab:effekte-pro-stratum} zusammengeführt.

\begin{ThreePartTable}
    \begin{TableNotes}[flushleft]
        \item $\Delta$ Wohlbefindensindex pro Anstieg der erklärenden Variable um eine Standardabweichung.
        \item \textbf{Fett} = Effekt ist statistisch signifikant (95\%-Konfidenzintervall schliesst den Wert 0 aus).
        \item (\textemdash = unbekannt)
    \end{TableNotes}
      
    \begin{longtable}{llllr rrrr}
        \caption{Effekte pro Stratum} 
        \label{tab:effekte-pro-stratum} \\
        \toprule
        Geschl. & Alter & Sex. Orient. & Äquiv.-Eink. & Befr. & Lärm & Natur & Lebhaftigkeit & Angenehmkeit \\
        \midrule
        \endfirsthead
        
        \multicolumn{9}{c}{{\bfseries Tabelle \thetable\ -- Fortsetzung}} \\
        \toprule
        Geschl. & Alter & Sex. Orient. & Äquiv.-Eink. & Befr. & Lärm & Natur & Lebhaftigkeit & Angenehmkeit \\
        \midrule
        \endhead
        
        \midrule
        \multicolumn{9}{r}{{Fortsetzung auf der nächsten Seite}} \\
        \endfoot
        
        \bottomrule
        \insertTableNotes
        \endlastfoot
        
        Frau & 16 -- 25 & heterosexuell & hoch         & 13 & \textbf{0.04} & 0.04          & 0.04          & 0.04          \\
        Frau & 16 -- 25 & bisexuell     & ---          & 12 & \textbf{0.04} & 0.05          & \textbf{0.07} & \textbf{0.07} \\
        Mann & 16 -- 25 & heterosexuell & ---          &  9 & \textbf{0.05} & \textbf{0.07} & 0.06          & 0.05          \\
        Mann & 16 -- 25 & heterosexuell & sehr hoch    &  9 & \textbf{0.04} & 0.01          & 0.01          & 0.03          \\
        Mann & 16 -- 25 & heterosexuell & sehr niedrig &  9 & \textbf{0.04} & \textbf{0.07} & 0.03          & 0.03          \\
        Mann & 16 -- 25 & heterosexuell & niedrig      &  8 & \textbf{0.04} & 0.03          & 0.00          & 0.02          \\
        Frau & 16 -- 25 & heterosexuell & ---          &  8 & \textbf{0.04} & 0.02          & 0.04          & 0.03          \\
        Mann & 16 -- 25 & homosexuell   & niedrig      &  7 & \textbf{0.05} & 0.07          & 0.05          & 0.05          \\
        Frau & 16 -- 25 & heterosexuell & sehr niedrig &  6 & \textbf{0.04} & 0.05          & 0.04          & 0.04          \\
        Frau & 26 -- 35 & heterosexuell & sehr niedrig &  5 & \textbf{0.04} & 0.04          & 0.03          & 0.04          \\
        Mann & 26 -- 35 & heterosexuell & sehr hoch    &  4 & \textbf{0.04} & 0.05          & 0.04          & 0.05          \\
        Mann & 16 -- 25 & homosexuell   & hoch         &  3 & \textbf{0.04} & 0.02          & 0.02          & 0.02          \\
        Mann & 16 -- 25 & heterosexuell & hoch         &  3 & \textbf{0.04} & 0.04          & 0.03          & 0.03          \\
        Mann & 16 -- 25 & homosexuell   & ---          &  3 & \textbf{0.04} & \textbf{0.07} & 0.03          & 0.02          \\
        Mann & 16 -- 25 & bisexuell     & sehr niedrig &  2 & \textbf{0.04} & 0.03          & 0.02          & 0.02          \\
        Frau & 16 -- 25 & queer         & sehr niedrig &  2 & \textbf{0.04} & 0.04          & 0.03          & 0.04          \\
    
    \end{longtable}
\end{ThreePartTable}

Einzelne Effektschätzungen waren statistisch signifikant (95\%-Konfidenzintervall schliesst 0 aus). Angesichts der geringen Fallzahlen pro Stratum und der unbalancierten Stichprobe sind diese Resultate jedoch nicht inhaltlich belastbar. Die Auswertung ist hier vor allem als Demonstration zu verstehen, wie heterogene Umwelteffekte entlang sozialer \glspl{stratum} modelliert und dargestellt werden können.

\subsection*{Methodische Beobachtungen und Implikationen}

Die Ergebnisse zeigen, dass sich mit den hier erhobenen \gls[noindex]{ema}- und \gls[noindex]{gema}-Daten grundsätzlich auch stratum-spezifische Umwelteffekte modellieren lassen. Die Umsetzung der \emph{Random-Slopes}-Modelle verlief technisch problemlos; die Modellschätzungen lieferten für jedes \gls{stratum} interpretierbare Effektkoeffizienten und Konfidenzintervalle. 

Gleichzeitig verdeutlicht die Analyse die hohen Anforderungen an die Datengrundlage: Die geringe Fallzahl pro \gls{stratum} führt zu breiten Konfidenzintervallen und erschwert die Trennung systematischer Unterschiede von zufälligen Schwankungen. Die unbalancierte Stichprobenstruktur bewirkt zudem, dass einige Strata vollständig von einzelnen Personen repräsentiert werden, wodurch Schätzungen stark durch individuelle Ausprägungen beeinflusst werden können. 

Die grossen Varianzanteile auf der Ebene der Momentaufnahmen deuten darauf hin, dass kurzfristige, situative Schwankungen einen erheblichen Einfluss auf das Wohlbefinden haben können -- unabhängig von stabilen Gruppenunterschieden. Methodisch unterstreicht dies den Mehrwert von \gls[noindex]{ema}- und \gls[noindex]{gema}-Designs, die solche intraindividuellen Veränderungen erfassen. 

Diese Beobachtungen legen nahe, dass das Verfahren für grössere, stärker ausbalancierte Datensätze geeignet ist, um vergleichbare Forschungsfragen belastbar zu beantworten. In der vorliegenden Pilotstudie dient es vor allem als Machbarkeitsnachweis und zur Identifikation von Herausforderungen, die bei einer weiteren Untersuchung gezielt adressiert werden können.

Darüber hinaus wurde in der Pilotanalyse bewusst darauf verzichtet, die tatsächlichen Standortdaten in die Modellierung einzubeziehen. Angesichts der kleinen Stichprobe hätte dies nur einen geringen inhaltlichen Mehrwert gebracht und potenziell Rückschlüsse auf \gls{bspw} den Wohnort einzelner Teilnehmenden ermöglicht. Sämtliche Standortdaten wurden vor der Veröffentlichung aus dem Datensatz entfernt. Für künftige Erhebungen mit grösserer Fallzahl bietet sich jedoch die Verknüpfung mit hochaufgelösten Kontextinformationen an, beispielsweise mit lokalen Hitzedaten \parencite[siehe][]{burgerModellingSpatialPattern2021} oder präziseren Geofences, um spezifische räumliche Kontexte gezielt zu analysieren.



\clearpage

% LTeX: language=de-CH
\chapter{Diskussion} \label{sec:diskussion}

Im letzten Kapitel dieser Arbeit diskutiere ich die zentralen Ergebnisse, Entscheidungen und Spannungen, die den Forschungsprozess geprägt haben. Dabei gehe ich sowohl auf methodische Potenziale und Grenzen als auch auf theoretische und infrastrukturelle Fragen ein. Mein Ziel ist es, die Arbeit nicht nur im engeren Forschungskontext zu verorten, sondern auch kritisch zu reflektieren, welche Punkte offen bleiben und wie zukünftige Arbeiten darauf aufbauen können.

\section{Einordnung der Arbeit im Forschungsfeld}

Ausgangspunkt dieser Arbeit ist die Frage, wie sich der Einfluss räumlicher Umgebungen auf das situierte (Un-)Wohlbefinden \glslink{intersektionalitaet}{intersektional} positionierter Personen erfassen und analysieren lässt. Eine abschliessende Beantwortung dieser komplexen Leitfrage kann im Rahmen dieser Bachelorarbeit nicht geleistet werden. Stattdessen unternehme ich einen ersten Schritt, ein Forschungsdesign zu entwickeln, das einen Beitrag zu dieser übergeordneten Fragestellung leisten kann.

Im Verlauf der Arbeit zeige ich auf, dass ein geeigneter Erhebungsansatz weit mehr umfasst als die Auswahl methodischer Verfahren. Transparenz, Nachvollziehbarkeit und eine kritische Reflexion der zugrunde liegenden Infrastruktur sind ebenso zentral wie die konkrete Gestaltung der Erhebung. Methodisch erweist sich die Verbindung von \gls{ema}/\gls{gema}-Methoden mit einer intersektionalen Mehrebenenanalyse als vielversprechend, auch wenn Spannungen in der Übersetzung intersektionaler Theorie in quantitative Verfahren bestehen bleiben. Das hier entwickelte Forschungsdesign kann somit als ein möglicher, wenn auch nicht spannungsfreier Weg verstanden werden, situiertes (Un-)Wohlbefinden in seiner sozialen und räumlichen Einbettung quantitativ zu erfassen.

Mit der Entwicklung und Veröffentlichung einer eigenen \gls{opensource}-Erhebungsplattform versuche ich, Prinzipien wie Nachvollziehbarkeit, Offenheit und Reflexivität praktisch umzusetzen. Auch wenn \gls{intermind} in seiner jetzigen Form nur begrenzt anschlussfähig ist, verdeutlicht das Projekt, dass wissenschaftliche Werkzeuge nicht zwingend an proprietäre Systeme gebunden sein müssen, sondern auch im Sinne einer offenen, gemeinschaftsorientierten Infrastruktur realisierbar sind.

In \cref{sec:pilotstudie} zeige ich, dass die erhobenen Pilotdaten für eine \glslink{intersektionalitaet}{intersektionale} Mehrebenenanalyse nur eingeschränkt geeignet sind. Zwar lassen sich Modellierungen prinzipiell durchführen, die kleine und sehr homogene Stichprobe verhindert jedoch belastbare Ergebnisse. Manche Strata sind so klein, dass eine intrapersonelle Ebene nicht abgebildet werden kann. Hinzu kommt, dass die inhaltlich nicht konsequent theoriegeleitete Entwicklung des Fragebogens, die Datengrundlage zusätzlich schwächt. Entsprechend ist die Überprüfung der Eignung der Analysemethode mit den Daten aus der Pilotstudie nur teilweise möglich.

Ich sehe den Beitrag dieser Arbeit nicht primär in belastbaren empirischen Resultaten, sondern in der Entwicklung und Erprobung eines durchgängig kritisch ausgerichteten Forschungsdesigns, das Theorie, Methodenwahl, Erhebungsinstrument, Fragebogen und Analyse als zusammenhängendes Ganzes verbindet. Dieser Ansatz macht Spannungen sichtbar -- zwischen Transparenz und Praktikabilität, zwischen Offenheit und Anschlussfähigkeit, zwischen theoretischer Schärfe und empirischer Umsetzbarkeit. Gerade dadurch wird deutlich, dass eine \glslink{intersektionalitaet}{intersektionale} Betrachtung von situiertem (Un-)Wohlbefinden nicht allein auf der Ebene von Items und Skalen beantwortet werden kann, sondern auch grundlegende infrastrukturelle und methodologische Entscheidungen umfasst. In dieser doppelten Hinsicht -- als methodisches Forschungsdesign wie auch als theoretisch-praktische Rahmung -- versteht sich die Arbeit als Beitrag zu einer kritisch-feministischen Geographie, die methodische, digitale und intersektionale Fragen nicht getrennt behandelt, sondern gemeinsam denkt -- auch wenn dies gezwungenermassen Brüche und Unvollständigkeiten mit sich bringt.

Meine Motivation für diese Arbeit ist auch in einer persönlichen Haltung begründet. Ich sehe in der Konzentration digitaler Infrastrukturen bei einigen wenigen Konzernen ein gesamtgesellschaftliches Problem, das Abhängigkeiten und Intransparenzen schafft und Ungleichheiten verstärkt. Gleichzeitig nutze ich selbst täglich Geräte und Dienste dieser Konzerne. Diese Verflechtung von Kritik und Abhängigkeit ist sinnbildlich für meinen Zugang: Sie macht deutlich, wie schwer es ist, konsequent offene Infrastrukturen umzusetzen, und erklärt zugleich, warum mir dieses Anliegen wichtig ist. Offenheit und Nachvollziehbarkeit im Sinne digitaler Souveränität sind nicht nur abstrakte Prinzipien, sondern eine Antwort auf diese Spannung. Diese Spannung zwischen wissenschaftlichem Anspruch und praktischer Machbarkeit zieht sich durch die gesamte Arbeit -- sie erklärt, warum an einigen Stellen Konzepte nur skizziert, Verfahren nicht voll validiert oder Entscheidungen situativ getroffen wurden. Diese Brüche verstehe ich als typische Dynamik eines geographisch-interdisziplinären Arbeitens, bei dem wissenschaftlicher Anspruch und praktische Machbarkeit ständig neu austariert werden müssen.

\section{Potenziale und Grenzen von \textit{InterMind}}

Das im Rahmen dieser Arbeit entwickelte Erhebungsplattform \gls{intermind} erweist sich einerseits als funktionales, transparentes und datenschutzfreundliches Werkzeug, und zeigt andererseits auch klare Grenzen auf. Ein zentrales Potenzial liegt in der Modularität und Anpassbarkeit: Das Grundgerüst von \gls{intermind} kann mit überschaubarem Aufwand für andere Fragebögen genutzt und durch zusätzliche Komponenten erweitert werden. Damit entsteht ein flexibles System, das nicht auf ein einzelnes Forschungsszenario beschränkt ist, sondern prinzipiell in ganz unterschiedlichen Kontexten eingesetzt werden kann. Die Entscheidung, den Code offen zu veröffentlichen, stärkt diese Anschlussfähigkeit zusätzlich und macht den Entwicklungsprozess nachvollziehbar.

Gleichzeitig wird deutlich, dass die Plattform für bestimmte Forschungsszenarien noch ergänzt werden müsste. Funktionen wie eine Offline-Nutzung, standortbasierte Trigger oder Echtzeitauswertungen sind bislang nicht umgesetzt. In grösser angelegten Studien wäre es zudem sinnvoll, die Architektur um zusätzliche serverseitige Module zu erweitern, die eine engere Steuerung der Erhebungen erlauben. Solche Erweiterungen sind weniger prinzipielle Grenzen des Systems als vielmehr Ausdruck des Umfangs, der im Rahmen einer Bachelorarbeit realisierbar ist. Gleichzeitig verweisen diese Überlegungen auf grundlegende Spannungen, die jede Weiterentwicklung dieser Plattform mit sich bringt: Mehr Funktionalität bedeutet zunächst, dass Teilnehmende bereit sein müssen, zusätzliche Daten zu teilen -- und damit auch Kontrolle darüber abzugeben, wie diese Daten weiterverarbeitet werden. Gerade hier liegt ein kritischer Punkt: Daten, die im Rahmen einer Wissenschaftlichen Studie erhoben werden, sind immer mit Erwartungen und einem impliziten Vertrauensvorschuss verbunden, der wissenschaftlich legitimiert und geschützt werden muss.

Auch auf der Seite der Forschenden entstehen durch die Erhebung zusätzlicher Daten erhöhte Anforderungen. Mehr Daten bedeuten nicht automatisch mehr Erkenntnis, sondern bergen das Risiko, sensible oder besonders schutzbedürftige Bereiche zu berühren, die über das eigentliche Forschungsinteresse hinausgehen. Die Pilotstudie macht dies deutlich: So habe ich \gls{bspw} präzise Standortdaten erhoben, welche ich in der anschliessenden Analyse nicht verwende. Auch wenn diese Daten technisch sicher gespeichert sind, wurden sie mir damit ohne klaren Erkenntnisgewinn anvertraut. Der in der Befragung implizit suggerierte Nutzen kann so nicht eingelöst werden. Rückblickend wäre es notwendig, gegenüber den Teilnehmenden transparenter zu kommunizieren, dass es sich hier in erster Linie um eine explorative Studie handelt. Dieses Missverhältnis verdeutlicht, wie wichtig es ist, bereits vor Beginn einer Erhebung kritisch zu reflektieren, welche Daten tatsächlich benötigt werden -- und wie eng technische Gestaltung, methodische Entscheidungen und ethische Verantwortung miteinander verflochten sind.

\vspace{1em}

Im Entwicklungsprozess zeigt sich besonders deutlich das Spannungsfeld zwischen der offenen Logik von \gls{opensource}-Software und den geschlossenen Ökosystemen grosser Plattformbetreiber. Zwar steht der Quellcode von \gls{intermind} öffentlich zur Verfügung, die Distribution über App-Stores bleibt jedoch an intransparente und kommerziell geprägte Verfahren gebunden. Die Veröffentlichung im Apple App Store scheitert schliesslich an einer nur schwer nachvollziehbaren und intransparenten Ablehnung, während die Bereitstellung im Google Play Store zusätzliche Gebühren und aufwändige Prüfprozesse erfordert. Hinzu kommt, dass die Entwicklung der App selbst zwar aufwendig ist, aber der zeitliche und organisatorische Aufwand rund um die Veröffentlichung in den App Stores deutlich grösser ausfällt als erwartet – von Datenschutzrichtlinien und benötigten Webseiten bis hin zu sich ständig verändernden Store-Vorgaben und geforderten Updates. Rückblickend zeigt sich, dass dieser Teil des Projekts wesentlich mehr Ressourcen gebunden hat als die eigentliche Programmierung der App. Während ich den reinen Coding-Aufwand aufgrund meiner bisherigen Erfahrung im Vorfeld relativ gut einschätzen konnte, habe ich die zeitintensiven Prozesse der Distribution massiv unterschätzt.

Besonders deutlich wurde dies im Vorfeld der Pilotstudie: Da ich die einmalige Gelegenheit hatte, die App im Rahmen der Exkursion \enquote{Recht auf Stadt} mit einer ganzen Studierendengruppe zu testen, war der Termin klar vorgegeben und nicht verschiebbar. Damit die Erhebung durchgeführt werden konnte, musste die App rechtzeitig auf beiden Betriebssystemen verfügbar sein. In der unmittelbaren Vorbereitungsphase konzentrierte sich die Arbeit daher unerwartet stark auf die Veröffentlichung in den Stores, was erhebliche zeitliche und organisatorische Ressourcen band. Dieser Aufwand limitierte direkt die Zeit, die für eine sorgfältigere theoretische Fundierung und Ausarbeitung des Fragebogens zur Verfügung stand.

Viele im Prozess getroffene technische Entscheidungen reflektiere ich in dieser Arbeit nicht. Dadurch verlieren sie an Sichtbarkeit und Transparenz, obwohl sie methodisch wie epistemisch bedeutsam sind. Dass solche Entscheidungen im Text unsichtbar bleiben, macht zugleich ein grundlegendes Spannungsfeld sichtbar: Kritisch-sozialwissenschaftliche Ansprüche zielen auf Transparenz und Reflexion, während technische Erfordernisse oft pragmatische und situative Entscheidungen verlangen. Künftig wäre es wichtig, Wege zu finden, auch diese technischen Entscheidungen methodisch sichtbar zu machen, sei es durch begleitende Reflexion oder durch eine engere Verzahnung von Entwicklung und Dokumentation.

\section{Offenheit und Copyleft}

Die Veröffentlichung des Quellcodes schafft eine produktive Ambivalenz. Sie ermöglicht eine einfache Nachnutzung und eröffnet die Möglichkeit, dass andere auf der bestehenden Arbeit aufbauen. Gleichzeitig bedeutet sie, sich mit einer Arbeit sichtbar zu machen, die im Rahmen einer Bachelorarbeit entstanden ist und nicht den Anspruch auf Perfektion erheben kann. Anders als in der wissenschaftlichen Publikationspraxis gibt es hier keine formalisierte Qualitätssicherung -- der Code steht so zur Verfügung, wie er entwickelt wurde. Damit wird er auch beurteilbar, kritisierbar und in seiner Prozesshaftigkeit sichtbar. Diese Offenlegung erfordert eine Bereitschaft zur Exponierung, macht aber auch den Kern von \gls{opensource} aus: Der Wert liegt weniger in einem perfekten Endprodukt als in der Transparenz des Entwicklungsprozesses und in der Möglichkeit, dass andere darauf aufbauen.

Ein wesentlicher Teil dieser Offenlegung betrifft die Wahl der Lizenz. Sowohl \gls{intermind} als auch die vorliegende Arbeit sind unter \gls{lic:copyleft}-Lizenzen veröffentlicht (\gls{lic:agpl}~3.0 für den Code, \gls{lic:cc-by-sa}~4.0 für den Text). Copyleft-Lizenzen unterscheiden sich von anderen offenen Lizenzen darin, dass sie Offenheit nicht nur erlauben, sondern auch verpflichtend machen: Wer auf dieser Grundlage weiterentwickelt, muss seine eigenen Ableitungen wiederum unter einer offenen Lizenz zugänglich machen. Damit wird Offenheit als wechselseitiges Prinzip gedacht.

Auffällig ist dabei, dass gerade in der Research Software Engineering Community \gls{lic:copyleft}-Lizenzen besonders umstritten sind. Während sie im weiteren Open-Source-Kontext seit Jahrzehnten etabliert sind, dominieren in der Wissenschaft permissive Lizenzen wie MIT oder BSD \parencite{sethiWhyEarthAre2020}. Diese gelten als anschlussfähig, weil sie auch die Einbindung in proprietäre Kontexte ermöglichen. Die dahinterliegende Logik ist für mich jedoch nur begrenzt nachvollziehbar: Es bleibt unklar, warum es ein wissenschaftliches Ziel sein sollte, dass Forschungscode ohne wechselseitige Verpflichtung in kommerzielle Produkte integriert werden kann. Ich schliesse mich hier \textcite{sethiWhyEarthAre2020} an, der betont, dass diese ablehnende Haltung auf eine verkürzte Debatte verweist: Sie blendet die Frage aus, wie Offenheit nicht bloss als individuelle Grosszügigkeit verstanden werden kann, sondern als wissenschaftlich-kollektives Prinzip geteilter Verantwortung.

Mit dieser Arbeit wird ein Gegenentwurf zu bestehenden wissenschaftlichen Infrastrukturen erprobt: Nebst dem Quellcode ist auch die Arbeit selbst unter einer \gls{lic:copyleft}-Lizenz veröffentlicht. Damit wird zugleich auf eine Leerstelle aufmerksam gemacht: In wissenschaftlichen Publikationen sind \gls{lic:copyleft}-Lizenzen bislang praktisch inexistent. Gerade deshalb ist die Entscheidung bewusst als Intervention zu lesen, die die Spannungen in der aktuellen Praxis sichtbar macht.

\section{Methodische Lehren und Ausblick}

Die Verbindung von \gls{ema}/\gls{gema}-Methoden mit intersektionalen Mehrebenenmodellen (\gls{i-maihda}) setze ich als Zugang ein, der situiertes (Un-)Wohlbefinden im Verhältnis zu sozialen Positionierungen analysierbar macht. Alltagsräume verstehe ich machtkritisch als Gefüge, die soziale Positionierungen und ihre Überschneidungen (Intersektionen) zugleich widerspiegeln \emph{und} mit hervorbringen; sie sind nicht neutrale Kulissen, sondern an der Herstellung und Reproduktion von Machtverhältnissen beteiligt. 
Eine solche Quantifizierung muss aus einer machtkritischen Persektive immer mit qualitativen Verfahren verbunden werden: Die Modelle können verorten, \emph{wo} und \emph{für wen} Unterschiede auftreten, während qualitative Verfahren klären, \emph{wie} und \emph{warum} sie entstehen. Damit das möglich it muss aber während dem ganzen Prozess immer wieder kritisch hinterfragt werden, welche Elemente oder Ebenen tatsächlich relevant sind und welche dazugehörigen qualitativen Aspekte berücksichtigt werden müssen. Ich argumentiere daher explizit für einen Mixed\-Methods\-Ansatz.

Die Umsetzung in dieser Arbeit zeigt zugleich klare Grenzen. Ohne Pretests und mit nur teilweise validierten Skalen bleiben einzelne Items unscharf. Mit einer kleinen und relativ homogenen Stichprobe lassen sich die Potenziale von \gls{i-maihda} nur begrenzt zeigen: Strata sind unterbesetzt, Varianz innerhalb von Personen ist kaum abbildbar, robuste Effektschätzungen sind nicht realistisch. Ich lese diese Punkte als methodische Lehren: eine Studie die diesen hier entwickelten Ansatz umsetzt braucht ausreichend Teilnehmende, eine heterogenere Gesamtstichprobe und inhaltlich besser entwickelte, geprüfte Items, damit der Ansatz sein Potenzial entfaltet.

Perspektivisch ist eine Erweiterung auf \gls{gema} im engeren Sinn anzudenken: die Einbindung externer Kontextdaten (\gls{zb} hochaufgelöster Stadthitzedaten wie im \enquote{Bernometer} \parencite[siehe][]{burgerModellingSpatialPattern2021} oder punktueller Messungen der Umgebungslautstärke). Aus einer feministischen und intersektionalen Perspektive gehen damit aber Fragen der Messpolitik einher: Welche Wirklichkeiten mache ich durch Sensorik überhaupt sichtbar -- und welche blende ich aus? Welche Kategorien schreibe ich damit fest? Wer gewinnt Erkenntnis? Gibt es Risiken?

Entscheidend für eine erfolgreiche Studie mit dem in dieser Arbeit kombinierten Ansatz ist eine hohe Teilnahmestabilität, also eine grosse Anzahl an Erhebungen pro Person. Um dies zu erreichen, sind verschiedene Ansätze denkbar, die ausprobiert und wahrscheinlich auch kombiniert werden müssen. Einerseits ist die Forschungsfrage der durchgeführten Pilotstudie sehr allgemein gehalten. Eine stärkere lokale Einbettung, etwa in einem Quartier, und die Einbindung der Menschen vor Ort könnte die Relevanz für die Teilnehmenden erhöhen und sich dadurch positiv auf die Teilnahmebereitschaft auswirken. Ebenso denkbar ist eine aufbereitete Rückmeldung der eigenen Erhebungen für Teilnehmende, wie sie in \gls{urbanmind} umgesetzt ist.

Zusammenfassend zeige ich in dieser Bachelorarbeit, dass die Erhebung und Analyse von intersektionalem (Un-)Wohlbefinden im Stadtraum nicht nur eine methodische, sondern auch eine infrastrukturelle und politische Herausforderung darstellt. Mit \textit{InterMind} habe ich einen ersten, prototypischen Ansatz entwickelt, der digitale Offenheit, methodische Reflexivität und kritische Perspektiven miteinander verbindet. Auch wenn die empirische Basis begrenzt bleibt und zahlreiche Weiterentwicklungen notwendig sind, liegt der Beitrag dieser Arbeit darin, ein Forschungsfeld zu skizzieren und ein Werkzeug bereitzustellen, das künftige Arbeiten aufgreifen und weiterführen können. Damit versteht sich die Arbeit weniger als abgeschlossene Antwort, sondern als Einladung, die hier angestossenen Fragen weiterzudenken, kritisch zu vertiefen und in zukünftigen Arbeiten weiterzuentwickeln.

% Unbedingt visualisierung diskutieren als limitation dieser arbeit und grosse herausforderung kommender arbeiten. Emotional mapping etc.


\clearpage


% ------------------ Glossar
%TC:ignore
\clearpage
\printglossary[label=sec:glossary]



% ----------------- Bibliographie ------------------
\clearpage
\phantomsection
\printbibliography[heading=bibintoc, title=Literaturverzeichnis]

\clearpage

\phantomsection
\section*{Hinweis für den Einsatz von künstlicher Intelligenz}

Dieses Arbeit wurde punktuell mit KI-basierten Tools überarbeitet. Zur sprachlichen Optimierung kamen DeepL Write und LanguageTool zum Einsatz, während ChatGPT von OpenAI genutzt wurde, um Feedback zur Verständlichkeit und Struktur der Arbeit zu erhalten. Im Rahmen des Coding-Prozesses wurden zudem ChatGPT (OpenAI) und Claude (Anthropic) unterstützend verwendet. Es wurde jedoch keine KI zur Erstellung von Originalinhalten eingesetzt.


\clearpage

\phantomsection
\section*{Selbstständigkeitserklärung}

Ich erkläre hiermit, dass ich diese Arbeit selbstständig verfasst und keine anderen als die angegebenen Quellen benutzt habe. Alle Stellen, die wörtlich oder sinngemäss aus Quellen entnommen wurden, habe ich als solche gekennzeichnet. Mir ist bekannt, dass andernfalls der Senat gemäss Artikel 36 Absatz 1 Buchstabe r des Gesetzes vom 5. September 1996 über die Universität zum Entzug des aufgrund dieser Arbeit verliehenen Titels berechtigt ist.

Für die Zwecke der Begutachtung und der Überprüfung der Einhaltung der Selbständigkeitserklärung bzw. der Reglemente betreffend Plagiate erteile ich der Universität Bern das Recht, die dazu erforderlichen Personendaten zu bearbeiten und Nutzungshandlungen vorzunehmen, insbesondere die schriftliche Arbeit zu vervielfältigen und dauerhaft in einer Datenbank zu speichern sowie diese zur Überprüfung von Arbeiten Dritter zu verwenden oder hierzu zur Verfügung zu stellen.

\vspace{8cm}

\noindent Bern, \today

\vspace{2cm}

\noindent\rule{6cm}{0.4pt} \\
\noindent Lukas Batschelet




\clearpage
\pagenumbering{alph}
   
   \begin{appendices}

\bookmarksetup{startatroot=false}   % Alle folgenden Überschriften ↓ im pdf eine Ebene tiefer

\addtocontents{toc}{\protect\setcounter{tocdepth}{-1}}



\chapter{Fragebogen}
\label{app:appendix_fragebogen}

Deutsche Version des Fragebogens. Die übersetzten Versionen auf Englisch und Französisch können im \gls[noindex]{github}-Repository\footnote{\href{https://raw.githubusercontent.com/lbatschelet/Designing-InterMind/main/Questionnaire.xlsx}{https://raw.githubusercontent.com/lbatschelet/Designing-InterMind/main/Questionnaire.xlsx}} der Arbeit heruntergeladen werden.


\section*{Hallo!}
Schön bist Du hier!

In dieser App wirst Du eine Woche lang drei Mal am Tag kurze Fragen zu Deinem aktuellen Wohlbefinden und zu Deiner Umgebung beantworten.

Deine Antworten helfen uns dabei, besser zu verstehen, wie Menschen verschiedene Orte erleben – und wie diese Erfahrungen mit unterschiedlichen Lebenssituationen zusammenhängen.

\hrulefill

\section*{Worum geht es in dieser Studie?}
Wie wir uns an einem Ort fühlen, hängt stark von unserer Umgebung ab. Manche Orte wirken beruhigend, vertraut oder einladend. Andere lassen uns unruhig werden, ausgegrenzt erscheinen oder fehl am Platz fühlen.

Solche Erfahrungen sind jedoch nicht für alle Menschen gleich. Sie können davon abhängen, wie wir an einem Ort wahrgenommen und behandelt werden – z.\,B. aufgrund von Geschlecht, Herkunft, Sprache, Aussehen oder anderen Merkmalen, die unsere gesellschaftliche Position prägen.

\subsection*{Was meinen wir mit Wohlbefinden?}
Wohlbefinden kann vieles bedeuten. Manchmal geht es dabei um etwas Langfristiges – etwa, wie zufrieden wir mit unserem Leben insgesamt sind, wie gesund wir uns fühlen oder ob wir uns sicher und unterstützt fühlen.

In dieser Studie interessiert uns jedoch vor allem das \textbf{momentane Wohlbefinden}: Wie geht es Dir \emph{jetzt gerade}, an diesem Ort, in dieser Situation?  
Wohlbefinden umfasst sowohl \textbf{körperliche} Aspekte (z.\,B. Müdigkeit, Wärme, Ruhe) als auch \textbf{psychische} Empfindungen (z.\,B. Zufriedenheit, Sicherheit, Zugehörigkeit).

\subsection*{Wer führt die Studie durch?}
Diese Bachelorarbeit wird am Geographischen Institut der Universität Bern von Lukas Batschelet durchgeführt und von Prof.\ Dr.\ Carolin Schurr sowie Dr.\ Moritz Gubler betreut.

\subsection*{Was ist das Ziel dieser Studie?}
Wir untersuchen, wie sich verschiedene Merkmale – einzeln oder kombiniert – auf das momentane Wohlbefinden auswirken.

\subsection*{Wie läuft die Teilnahme ab?}
Die Studie dauert eine Woche; in dieser Zeit erhältst Du dreimal täglich eine Kurzbefragung auf Deinem Smartphone:

\begin{itemize}\setlength{\itemsep}{0pt}
  \item Ort, an dem Du Dich befindest
  \item Deine Tätigkeit dort
  \item Dein aktuelles Befinden
  \item Gefühl der Zugehörigkeit oder Fremdheit
\end{itemize}

Jede Befragung ist eine Stunde lang verfügbar; verpasste Befragungen kannst Du einfach überspringen.

\hrulefill

\section*{Einwilligung zur Teilnahme}
Bevor Du mit der Befragung startest, bitten wir Dich um Deine Zustimmung zur Teilnahme.  
Die Teilnahme ist freiwillig; einzelne Fragen können übersprungen und die Teilnahme jederzeit beendet werden. In den App-Einstellungen kannst Du Deine Daten nachträglich vollständig löschen.

\subsection*{Welche Daten werden erhoben?}
\begin{itemize}\setlength{\itemsep}{0pt}
  \item Angaben zu Deiner Person (z.\,B. Alter, Geschlecht, Bildung)
  \item Antworten zu Deinem aktuellen Befinden und Aufenthaltsort
  \item Standortdaten (sofern freigegeben)
\end{itemize}

\subsection*{Wie gehen wir mit Deinen Daten um?}
\begin{itemize}\setlength{\itemsep}{0pt}
  \item Keine Speicherung von Namen, E-Mail-Adressen o.\,ä.
  \item Anonymisierte Speicherung auf einem gesicherten Server in der Schweiz
  \item Keine Bewegungsprofile oder dauerhafte Standortverläufe
  \item Nutzung ausschliesslich für wissenschaftliche Zwecke, keine Weitergabe an Dritte
\end{itemize}

Mit \enquote{Ich stimme zu} bestätigst Du, dass Du die Informationen verstanden hast und freiwillig teilnimmst. Weitere Details findest Du in unserer \href{https://intermind.ch/privacy-policy.html}{Datenschutzrichtlinie}.

\hrulefill

\section*{Benachrichtigungen}
Damit Du keine Befragung verpasst, senden wir Dir Benachrichtigungen, sobald ein neues Umfrageslot startet (jeweils eine Stunde Antwortzeit).  
Du kannst die Benachrichtigungen in den Geräteeinstellungen abschalten – dann besteht jedoch die Gefahr, Befragungen zu verpassen.  
Wir empfehlen, sie eingeschaltet zu lassen, um möglichst viele unterschiedliche Situationen zu erfassen.

\hrulefill

\section*{Standort}
Um räumliche Muster zu erkennen, bitten wir Dich, die Standortfreigabe zu erlauben. So können wir z.\,B. unterscheiden, ob Erleben an belebten Plätzen anders ist als in ruhigen Gegenden – ohne Deinen Namen oder exakte Adressen zu kennen.

Standortdaten werden ausschliesslich anonymisiert gespeichert und nicht dauerhaft verfolgt.  
Du kannst die Standortfreigabe jederzeit in den Einstellungen Deines Geräts deaktivieren.


\begin{landscape}
\tiny
\begin{longtable}{p{1.2cm} p{5.8cm} *{11}{p{1cm}}}
    \caption{Einmalige Baseline-Fragen}\label{tab:baseline-fragen} \\
    \toprule
    \textbf{Fragetyp} & \textbf{Frage} & \textbf{Option 1} & \textbf{Option 2} & \textbf{Option 3} & \textbf{Option 4} & \textbf{Option 5} & \textbf{Option 6} & \textbf{Option 7} & \textbf{Option 8} & \textbf{Option 9} & \textbf{Option 10} & \textbf{Option 11} \\
    \midrule
    \endfirsthead
    
    \multicolumn{13}{l}{\textit{Fortsetzung der Tabelle auf der nächsten Seite}} \\
    \toprule
    \textbf{Fragetyp} & \textbf{Frage} & \textbf{Option 1} & \textbf{Option 2} & \textbf{Option 3} & \textbf{Option 4} & \textbf{Option 5} & \textbf{Option 6} & \textbf{Option 7} & \textbf{Option 8} & \textbf{Option 9} & \textbf{Option 10} & \textbf{Option 11} \\
    \midrule
    \endhead
    
    \midrule
    \multicolumn{13}{r}{\textit{Fortsetzung auf der nächsten Seite}} \\
    \endfoot
    
    \bottomrule
    \endlastfoot

Info & Bevor wir mit den täglichen Befragungen starten, stellen wir Dir einmalig einige Fragen zu Dir selbst – zum Beispiel zu deinem Alter, Geschlecht, deiner Ausbildung und deiner Lebenssituation. Du kannst jede Frage überspringen, wenn Du sie nicht beantworten möchtest. &  &  &  &  &  &  &  &  &  &  & \\
\midrule
Single-Choice & In welcher Altersgruppe befindest Du dich? & Unter 16 & 16-25 & 26-35 & 36-45 & 46-55 & 56-65 & 66-75 & 75+ &  &  & \\
\midrule
Single-Choice & Welches Geschlecht wurde Dir bei der Geburt zugewiesen? & Weiblich & Männlich & Inter / Variante der Geschlechtsentwicklung &  &  &  &  &  &  &  & \\
\midrule
Single-Choice & Mit welcher Geschlechtsidentität identifizierst Du dich? & Weiblich & Männlich & Nicht-binär / genderqueer & Trans Frau & Trans Mann & Agender & Intersex & Andere &  &  & \\
\midrule
Single-Choice & Mit welchen Begriffen würdest du Deine sexuelle Orientierung beschreiben? & Heterosexuell & Homosexuell & Bisexuell & Pansexuell & Asexuell & Queer & Andere &  &  &  & \\
\midrule
Single-Choice & Was ist Dein höchster Bildungsabschluss? & Noch kein Abschluss & Obligatorische Schulzeit (z. B. Sek I) & Berufsausbildung (EFZ / EBA) & Matura / FMS / HMS / etc. & Fachhochschule (FH) oder Höhere Fachschule (HF) & Universität / ETH &  &  &  &  & \\
\midrule
Single-Choice & Wie viele Personen leben in Deinem Haushalt (einschliesslich Dir selbst)? & 1 (lebe allein) & 2 & 3 & 4 & 5 & 6 & 7 & 8 & 9 & 10 oder mehr & \\
\midrule
Single-Choice & Wie viele Personen in Deinem Haushalt tragen (einschliesslich dir selbst) zum gemeinsamen Einkommen bei? & 1 Person (nur ich) & 2 Personen & 3 Personen & 4 Personen & 5 Personen & 6 Personen & 7 Personen & 8 Personen & 9 Personen & 10 oder mehr & \\
\midrule
Single-Choice & Wie hoch ist ungefähr Euer gemeinsames monatliches Haushaltseinkommen (nach Abzug von Steuern)? & Unter CHF 1500 & CHF 1500–3000 & CHF 3000–4500 & CHF 4500–6000 & CHF 6000–7500 & CHF 7500–10'000 & Mehr als CHF 10'000 & Weiss nicht &  &  & \\
\midrule
Multiple-Choice & Wie ist Deine derzeitige berufliche oder schulische Situation? & Schüler*in / Student*in & Angestellt & Selbstständig & Pensioniert & Arbeitslos &  &  &  &  &  & \\
\midrule
Single-Choice & Hast Du eine körperliche oder psychische Beeinträchtigung, chronische Erkrankung oder andere gesundheitliche Einschränkung, die Deinen Alltag beeinflusst? & Ja & Nein &  &  &  &  &  &  &  &  & \\
\midrule
Single-Choice & Lebst Du in einem anderen Land, als in welchem du geboren wurdest? & Ja & Nein &  &  &  &  &  &  &  &  &  \\
\midrule
Multiple-Choice & Hast Du im Alltag schon Diskriminierung aufgrund persönlicher Merkmale erlebt? & Ja, wegen meines Geschlechts & Ja, wegen meines Alters & Ja, wegen meiner Herkunft & Ja, wegen meiner Hautfarbe oder meines Aussehens & Ja, wegen meiner Sprache oder meines Akzents & Ja, wegen meiner sozialen oder finanziellen Situation & Ja, wegen meiner Kleidung oder meines Stils & Ja, wegen meiner sexuellen Orientierung & Ja, wegen meines Gesundheitszustands oder einer Behinderung & Ja, aus einem anderen Grund & Nein \\
\midrule
Info & Als Nächstes stellen wir Dir einige Fragen dazu, wo Du gerade bist, was Du machst und wie Deine Umgebung aussieht. &  &  &  &  &  &  &  &  &  &  & \\
\bottomrule
\end{longtable}
\normalsize


\tiny
\begin{longtable}{p{1.2cm} p{3.8cm} *{13}{p{1cm}}}
    \caption{Wiederholte Fragen zum aktuellen Befinden und der unmittelbaren Umgebung}\label{tab:wiederholte-fragen} \\
    \textbf{Fragetyp} & \textbf{Frage} & \textbf{Option 1} & \textbf{Option 2} & \textbf{Option 3} & \textbf{Option 4} & \textbf{Option 5} & \textbf{Option 6} & \textbf{Option 7} & \textbf{Option 8} & \textbf{Option 9} & \textbf{Option 10} & \textbf{Option 11} & \textbf{Option 12} & \textbf{Option 13} \\
    \midrule
    \endfirsthead
    
    \multicolumn{15}{l}{\textit{Fortsetzung der Tabelle auf der nächsten Seite}} \\
    \toprule
    \textbf{Fragetyp} & \textbf{Frage} & \textbf{Option 1} & \textbf{Option 2} & \textbf{Option 3} & \textbf{Option 4} & \textbf{Option 5} & \textbf{Option 6} & \textbf{Option 7} & \textbf{Option 8} & \textbf{Option 9} & \textbf{Option 10} & \textbf{Option 11} & \textbf{Option 12} & \textbf{Option 13} \\
    \midrule
    \endhead
    
    \midrule
    \multicolumn{15}{r}{\textit{Fortsetzung auf der nächsten Seite}} \\
    \endfoot
    
    \bottomrule
    \endlastfoot
Single-Choice & Bist Du drinnen oder draussen? & Drinnen & Draussen &  &  &  &  &  &  &  &  &  &  & \\
\midrule
Single-Choice & Wo genau befindest Du dich? & Zuhause & Bei jemand anderem zuhause & Arbeitsplatz & Schule / Universität & Einkaufen oder Dienstleistungen & Café / Restaurant / Bar & Freizeit- oder Sporteinrichtung & Park oder Grünfläche & Kultureller oder religiöser Ort & Gesundheitseinrichtung / Therapie & Unterwegs (zu Fuss, Fahrrad, Auto) & Öffentlicher Verkehr & Anderer Ort \\
\midrule
Multiple-Choice & Mit wem bist Du gerade zusammen? & Niemand & Partner*in & Kinder & Familie & Freund*innen & Arbeitskolleg*innen & Bekannte & Tiere/Haustiere & Fremde & Andere &  &  & \\
\midrule
Multiple-Choice & Was machst Du gerade hauptsächlich? & Freizeit oder Entspannung & Unterwegs sein oder pendeln & Arbeiten oder studieren & Einkaufen oder Besorgungen & Haushalt oder Aufräumen & Kochen oder Essen & Betreuungspflichten & Soziale Aktivitäten & Mediennutzung & Ausruhen oder schlafen & Sonstiges &  & \\
\midrule
Slider & Wie nimmst Du die Geräuschkulisse an diesem Ort wahr? & Sehr laut & Sehr leise &  &  &  &  &  &  &  &  &  &  & \\
\midrule
Slider & Wie viel Natur ist an diesem Ort sichtbar? & Keine Natur & Viel Natur &  &  &  &  &  &  &  &  &  &  & \\
\midrule
Slider & Wie lebhaft oder ruhig wirkt der Ort? & Lebhaft & Ruhig &  &  &  &  &  &  &  &  &  &  & \\
\midrule
Slider & Wie angenehm empfindest Du den Ort insgesamt? & Unangenehm & Angenehm &  &  &  &  &  &  &  &  &  &  & \\
\midrule
Slider & Zum Schluss noch einige Fragen zu Deinem aktuellen Wohlbefinden. &  &  &  &  &  &  &  &  &  &  &  &  & \\
\midrule
Slider & Wie fühlst Du dich gerade insgesamt? & Sehr unwohl & Sehr wohl &  &  &  &  &  &  &  &  &  &  & \\
\midrule
Slider & Ganz allgemein - wie zufrieden fühlst Du dich im Moment? & Sehr unzufrieden & Sehr zufrieden &  &  &  &  &  &  &  &  &  &  & \\
\midrule
Slider & Wie angespannt oder entspannt fühlst Du dich? & Sehr angespannt & Sehr entspannt &  &  &  &  &  &  &  &  &  &  & \\
\midrule
Slider & Wie wach fühlst Du dich im Moment? & Sehr müde & Sehr wach &  &  &  &  &  &  &  &  &  &  & \\
\midrule
Slider & Wie zugehörig oder fremd fühlst Du dich an diesem Ort? & Sehr fremd & Sehr zugehörig &  &  &  &  &  &  &  &  &  &  & \\
\midrule
Multiple-Choice & Glaubst Du, dass dein Gefühl von Zugehörigkeit oder Fremdheit an diesem Ort damit zu tun hat, wie du als Person wahrgenommen wirst? & Ja, wegen meines Geschlechts & Ja, wegen meines Alters & Ja, wegen meiner Herkunft & Ja, wegen meiner Hautfarbe oder meines Aussehens & Ja, wegen meiner Sprache oder meines Akzents & Ja, wegen meiner sozialen oder finanziellen Situation & Ja, wegen meiner Kleidung oder meines Stils & Ja, wegen meiner sexuellen Orientierung & Ja, wegen meines Gesundheitszustands oder einer Behinderung & Ja, aus einem anderen Grund & Nein &  & \\
\midrule
Multiple-Choice & Verglichen mit den anderen Personen hier: Bei welchen Merkmalen fühlst Du dich der Mehrheit zugehörig? & In meinem Geschlecht & In meinem Alter & In meiner Herkunft & In meiner Hautfarbe oder meines Aussehens & In meiner Sprache oder Akzents & In meiner sozialen oder finanziellen Situation & In meiner Kleidung oder meinem Stil & In meiner sexuellen Orientierung & In meinem Gesundheitszustand oder einer Behinderung & Ich bin allein hier &  &  & \\
\midrule
Offene Frage & Gibt es andere Dinge die dazu führen, dass Du dich hier weniger wohl oder unwohl fühlst? &  &  &  &  &  &  &  &  &  &  &  &  & \\
\midrule
Offene Frage & Gibt es andere Dinge die dazu führen, dass Du dich hier wohler fühlst? &  &  &  &  &  &  &  &  &  &  &  &  & \\
\bottomrule
\end{longtable}
\normalsize

    
\end{landscape}




\chapter{Stichprobe}
\section{Soziodemografische Merkmale der Stichprobe}
\label{app:appendix_demographics}
\footnotesize
\begin{longtable}{p{5.5cm}p{5.5cm}rr}
    \caption{Übersicht über die Verteilung zentraler soziodemografischer Merkmale und Erfahrungen}
    \label{tab:soziodemografie_gesamt}\\
    \toprule
    Frage & Kategorie & Anzahl & Prozent \\
    \midrule
    \endfirsthead

    \multicolumn{4}{c}{{\bfseries Tabelle \thetable{} -- Fortsetzung}} \\
    \toprule
    Frage & Kategorie & Anzahl & Prozent \\
    \midrule
    \endhead
    
    \midrule
    \multicolumn{4}{r}{Fortsetzung auf der nächsten Seite}\\
    \endfoot
    
    \bottomrule
    \endlastfoot

    In welcher Altersgruppe befindest Du dich? & 16 – 25 & 20 & 80.0 \\*
     & 26 – 35 & 3 & 12.0 \\*
     & 56 – 65 & 1 & 4.0 \\*
     & Keine Angabe & 1 & 4.0 \\
    \midrule
    \addlinespace
    Welches Geschlecht wurde Dir bei der Geburt zugewiesen? & Männlich & 16 & 64.0 \\*
     & Weiblich & 8 & 32.0 \\*
     & Keine Angabe & 1 & 4.0 \\
    \midrule
    \addlinespace
    Mit welcher Geschlechtsidentität identifizierst Du dich? & Mann & 15 & 60.0 \\*
     & Frau & 9 & 36.0 \\*
     & Trans Mann & 1 & 4.0 \\
    \midrule
    \addlinespace
    Mit welchen Begriffen würdest du Deine sexuelle Orientierung beschreiben? & Heterosexuell & 17 & 68.0 \\*
     & Bisexuell & 3 & 12.0 \\*
     & Homosexual & 3 & 12.0 \\*
     & Queer & 1 & 4.0 \\*
     & Asexuell & 1 & 4.0 \\
    \midrule
    \addlinespace
    Was ist Dein höchster Bildungsabschluss? & Matura / Äquivalent & 23 & 92.0 \\*
     & Universitätsabschluss & 2 & 8.0 \\
    \midrule
    \addlinespace
    Wie ist Deine derzeitige berufliche oder schulische Situation? & Student\genderstern in / Schüler\genderstern in & 22 & 88.0 \\*
    & Angestellt & 3 & 12.0 \\
    \midrule
    \addlinespace
    Wie hoch ist ungefähr Euer gemeinsames monatliches Haushaltseinkommen (nach Abzug von Steuern)? & < CHF 1 500 & 7 & 28.0 \\*
     & CHF 1 500 – 3 000 & 2 & 8.0 \\*
     & CHF 3 000 – 4 500 & 2 & 8.0 \\*
     & CHF 6 000 – 7 500 & 2 & 8.0 \\*
     & CHF 7 500 – 10 000 & 1 & 4.0 \\*
     & > CHF 10 000 & 5 & 20.0 \\*
     & Nicht bekannt / bevorzugt nicht anzugeben & 6 & 24.0 \\
    \midrule
    \addlinespace
    Wie viele Personen leben in Deinem Haushalt (einschliesslich Dir selbst)? & 1 & 2 & 8.0 \\*
     & 2 & 3 & 12.0 \\*
     & 3 & 10 & 40.0 \\*
     & 4 & 6 & 24.0 \\*
     & 5 & 1 & 4.0 \\*
     & 6 & 2 & 8.0 \\*
     & 9 & 1 & 4.0 \\
    \midrule
    \addlinespace
    Wie viele Personen in Deinem Haushalt tragen (einschliesslich dir selbst) zum gemeinsamen Einkommen bei? & 1 & 6 & 24.0 \\*
     & 2 & 13 & 52.0 \\*
     & 3 & 4 & 16.0 \\*
     & 5 & 1 & 4.0 \\*
     & 6 & 1 & 4.0 \\
     \midrule
    \addlinespace
    Berechnetes Äquivalenz-Einkommen \parencite[nach][]{bundesamtfuerstatistikVerteilungVerfuegbarenAequivalenzeinkommens2025} & Armutsgefährdet & 8 & 32.0 \\*
     & Tief & 4 & 16.0 \\*
     & Mittel & 5 & 20.0 \\*
     & Hoch & 2 & 8.0 \\*
     & Unbekannt & 6 & 24.0 \\
     \midrule
    \addlinespace
    Hast Du eine körperliche oder psychische Beeinträchtigung, chronische Erkrankung oder andere gesundheitliche Einschränkung, die Deinen Alltag beeinflusst? & Nein & 25 & 100.0 \\
    \midrule
    \addlinespace
    Lebst Du in einem anderen Land, als in welchem du geboren wurdest? & Nein & 17 & 68.0 \\*
     & Ja & 7 & 28.0 \\*
     & Keine Angabe & 1 & 4.0 \\
     \midrule
    \addlinespace
    Hast Du im Alltag schon Diskriminierung aufgrund persönlicher Merkmale erlebt? & Ja, wegen meines Geschlechts & 4 & 16.0 \\*
     & Ja, wegen meiner Sprache oder meines Akzents & 4 & 16.0 \\*
     & Ja, wegen meiner Herkunft & 4 & 16.0 \\*
     & Ja, wegen meiner sexuellen Orientierung & 3 & 12.0 \\*
     & Ja, wegen meiner Kleidung oder meines Stils & 2 & 8.0 \\*
     & Ja, wegen meiner sozialen oder finanziellen Situation & 1 & 4.0 \\*
     & Ja, wegen meiner Hautfarbe oder meines Aussehens & 1 & 4.0 \\*
     & Ja, wegen meines Alters & 0 & 0.0 \\*
     & Ja, wegen meines Gesundheitszustands oder einer Behinderung & 0 & 0.0 \\*
     & Ja, aus einem anderen Grund & 0 & 0.0 \\*
     & Nein & 12 & 48.0 \\*
     & Keine Angabe & 0 & 0.0 \\
     \midrule
    \addlinespace
    Anzahl unterschiedlicher erlebter Diskriminierungsarten pro Person & 0 & 12 & 48.0 \\*
     & 1 & 8 & 32.0 \\*
     & 2 & 4 & 16.0 \\*
     & 3 & 1 & 4.0 \\
     \bottomrule
\end{longtable}
\normalsize


\clearpage
\section{Beschreibung der erfassten Momentaufnahmen}
\label{app:appendix_moments}

\footnotesize
\begin{longtable}{p{5.5cm}p{5.5cm}rr}
    \caption{Antworten auf die Fragen zu den Momentaufnahmen}
    \label{tab:moments}\\
    \toprule
    Frage & Kategorie & Anzahl & Prozent \\
    \midrule
    \endfirsthead

    \multicolumn{4}{c}{{\bfseries Tabelle \thetable{} -- Fortsetzung}} \\
    \toprule
    Frage & Kategorie & Anzahl & Prozent \\
    \midrule
    \endhead
    
    \midrule
    \multicolumn{4}{r}{Fortsetzung auf der nächsten Seite}\\
    \endfoot
    
    \bottomrule
    \endlastfoot

    Was machst Du gerade hauptsächlich? & Arbeiten oder studieren & 53 & 50.0 \\*
     & Freizeit oder Entspannung & 27 & 25.5 \\*
     & Unterwegs sein oder pendeln & 12 & 11.3 \\*
     & Kochen oder Essen & 8 & 7.5 \\*
     & Mediennutzung & 8 & 7.5 \\*
     & Soziale Aktivitäten & 7 & 6.6 \\*
     & Haushalt oder Aufräumen & 2 & 1.9 \\*
     & Ruhen / Schlafen & 2 & 1.9 \\*
     & Einkaufen oder Besorgungen & 2 & 1.9 \\*
     & Betreuungspflichten & 0 & 0.0 \\*
     & Sonstiges & 1 & 0.9 \\
    \midrule
    \addlinespace
    Bist Du drinnen oder draussen? & Drinnen & 54 & 50.9 \\*
     & Draussen & 52 & 49.1 \\
    \midrule
    \addlinespace
    Wo genau befindest Du dich? & Schule oder Universität & 38 & 35.8 \\*
     & Zuhause & 29 & 27.4 \\*
     & Unterwegs (zu Fuss, Fahrrad, Auto) & 12 & 11.3 \\*
     & Öffentlicher Verkehr & 8 & 7.5 \\*
     & Bei jemand anderem zuhause & 6 & 5.7 \\*
     & Arbeitsplatz & 5 & 4.7 \\*
     & Park oder Grünfläche & 5 & 4.7 \\*
     & Einkaufen oder Dienstleistungen & 2 & 1.9 \\*
     & Freizeit- oder Sporteinrichtung & 1 & 0.9 \\*
     & Café / Restaurant / Bar & 0 & 0.0 \\*
     & Kultureller oder religiöser Ort & 0 & 0.0 \\*
     & Gesundheitseinrichtung / Therapie & 0 & 0.0 \\*
     & Anderer Ort & 2 & 1.9 \\
     \midrule
    \addlinespace
    Mit wem bist Du gerade zusammen? & Freund\genderstern innen & 40 & 37.7 \\*
     & Allein & 38 & 35.8 \\*
     & Arbeitskolleg\genderstern innen & 17 & 16.0 \\*
     & Fremde & 17 & 16.0 \\*
     & Familie & 4 & 3.8 \\*
     & Bekannte & 3 & 2.8 \\*
     & Partner\genderstern in & 2 & 1.9 \\*
     & Tiere und Haustiere & 0 & 0.0 \\*
     & Kinder & 0 & 0.0 \\*
     & Andere & 2 & 1.9 \\
     \midrule
     \addlinespace
     Glaubst Du, dass dein Gefühl von Zugehörigkeit oder Fremdheit an diesem Ort damit zu tun hat, wie du als Person wahrgenommen wirst? & Nein & 58 & 54.7 \\*
     & Ja, wegen meines Alters & 17 & 16.0 \\*
     & Ja, wegen meiner Sprache oder meines Akzents & 17 & 16.0 \\*
     & Ja, wegen meiner sozialen oder finanziellen Situation & 15 & 14.2 \\*
     & Ja, wegen meiner Kleidung oder meines Stils & 13 & 12.3 \\*
     & Ja, wegen meiner Herkunft & 12 & 11.3 \\*
     & Ja, aus einem anderen Grund & 10 & 9.4 \\*
     & Ja, wegen meiner Hautfarbe oder meines Aussehens & 10 & 9.4 \\*
     & Ja, wegen meines Geschlechts & 9 & 8.5 \\*
     & Ja, wegen meines Gesundheitszustands oder einer Behinderung & 7 & 6.6 \\*
     & Ja, wegen meiner sexuellen Orientierung & 2 & 1.9 \\
     \midrule
     \addlinespace
     Verglichen mit den anderen Personen hier: Bei welchen Merkmalen fühlst Du dich der Mehrheit zugehörig? & In meinem Alter & 50 & 47.2 \\*
     & In meiner Sprache oder meines Akzents & 49 & 46.2 \\*
     & In meiner Hautfarbe oder meines Aussehens & 48 & 45.3 \\*
     & In meinem Gesundheitszustand oder einer Behinderung & 41 & 38.7 \\*
     & In meiner Herkunft & 37 & 34.9 \\*
     & In meiner sozialen oder finanziellen Situation & 37 & 34.9 \\*
     & In meiner Kleidung oder meines Stils & 34 & 32.1 \\*
     & In meinem Geschlecht & 27 & 25.5 \\*
     & In meiner sexuellen Orientierung & 22 & 20.8 \\*
     & Ich bin alleine hier & 22 & 20.8 \\
     \bottomrule
\end{longtable}
\normalsize


\footnotesize
\begin{longtable}{p{5.5cm}p{9.5cm}}
    \caption{Antworten auf Freitextfragen}
    \label{tab:freitext}\\
    \toprule
    Frage & Antwort \\
    \midrule
    \endfirsthead

    \multicolumn{2}{c}{{\bfseries Tabelle \thetable{} -- Fortsetzung}} \\
    \toprule
    Frage & Antwort \\
    \midrule
    \endhead
    
    \midrule
    \multicolumn{2}{r}{Fortsetzung auf der nächsten Seite}\\
    \endfoot
    
    \bottomrule
    \endlastfoot

    Gibt es andere Dinge die dazu führen, dass Du dich hier weniger wohl oder unwohl fühlst? & heat \\*
     & Everyone is doing the same, so it kind of feels like being at the right place \\*
     & The contact with strangers \\*
     & Bed \\*
     & health issues \\*
     & no natural sunlight room without windows no fresh air \\*
     & a lot of people - personal space \\*
     & No \\*
     & / \\*
     & no \\*
     & Not really \\
    \midrule
    \addlinespace
    Gibt es andere Dinge die dazu führen, dass Du dich hier wohler fühlst? & place i know and is mine i have control over it \\*
     & know this place and can do what i want \\*
     & my room and cozy for the night \\*
     & pets \\*
     & spending time with family pets \\*
     & I am not by myself \\*
     & Less noise from construction works \\
    \bottomrule
\end{longtable}
\normalsize

\cleardoublepage

\subsection*{Histogramme der Slider-Items}
\label{app:slider_hists}

\begin{figure}[htb]
    \centering
    \includegraphics[width=7.5cm]{Analyse/Plots/sliders/environment_noise_hist.pdf}
    \caption{Histogramm der Wahrnehmung der Lautstärke}
    \label{fig:slider_hists_environment_noise}
\end{figure}

\begin{figure}[htb]
    \centering
    \includegraphics[width=7.5cm]{Analyse/Plots/sliders/environment_nature_hist.pdf}
    \caption{Histogramm der Wahrnehmung der Natur}
    \label{fig:slider_hists_environment_nature}
\end{figure}

\begin{figure}[htb]
    \centering
    \includegraphics[width=7.5cm]{Analyse/Plots/sliders/environment_lively_hist.pdf}
    \caption{Histogramm der Wahrnehmung der Lebhaftigkeit}
    \label{fig:slider_hists_environment_lively}
\end{figure}

\begin{figure}[htb]
    \centering
    \includegraphics[width=7.5cm]{Analyse/Plots/sliders/environmen_pleasure_hist.pdf}
    \caption{Histogramm der Wahrnehmung der Angenehmeit}
    \label{fig:slider_hists_environmen_pleasure}
\end{figure}

\begin{figure}[htb]
    \centering
    \includegraphics[width=7.5cm]{Analyse/Plots/sliders/general_wellbeing_hist.pdf}
    \caption{Histogramm der Wahrnehmung des generellen Wohlbefindens}
    \label{fig:slider_hists_general_wellbeing}
\end{figure}

\begin{figure}[htb]
    \centering
    \includegraphics[width=7.5cm]{Analyse/Plots/sliders/content_hist.pdf}
    \caption{Histogramm der Wahrnehmung des aktuellen Wohlbefindens}
    \label{fig:slider_hists_content}
\end{figure}

\begin{figure}[htb]
    \centering
    \includegraphics[width=7.5cm]{Analyse/Plots/sliders/tense_relaxed_hist.pdf}
    \caption{Histogramm der Wahrnehmung der Anspannung}
    \label{fig:slider_hists_tense_relaxed}
\end{figure}

\begin{figure}[htb]
    \centering
    \includegraphics[width=7.5cm]{Analyse/Plots/sliders/awake_hist.pdf}
    \caption{Histogramm der Wahrnehmung der Energie}
    \label{fig:slider_hists_awake}
\end{figure}

\begin{figure}[htb]
    \centering
    \includegraphics[width=7.5cm]{Analyse/Plots/sliders/sense_of_belonging_hist.pdf}
    \caption{Histogramm der Wahrnehmung der sozialen Zugehörigkeit}
    \label{fig:slider_hists_sense_of_belonging}
\end{figure}






\addtocontents{toc}{\protect\setcounter{tocdepth}{2}}

\end{appendices}
%TC:endignore
\end{document}