\documentclass{template}
\usepackage{array}
\usepackage[nswissgerman]{babel}
\usepackage{graphicx}
\usepackage{float}
\usepackage{csquotes}




\SetTitle{Intersektionales Wohlbefinden im Stadtraum: \\
Konzeption und Umsetzung einer App zur räumlichen Erfassung von Wohlbefinden}
\SetAuthor{\textbf{Bachelorarbeit}\\
Lukas Batschelet, \href{mailto:lukas.batschelet@students.unibe.ch}{lukas.batschelet@students.unibe.ch} \\ 
Betreuung: Prof. Dr. Carolin Schurr und Dr. Moritz Gubler\\
Geographisches Institut Universität Bern}
\SetDate{\today}

\loadglsentries{Arbeit/glossary}
\makeglossaries


\AddBibFile{BA_Urban Mental Health.bib}




% Hinführung, wieso ist das wichtig? Klimagerechtigkeit, warum muss man das untersuchen?
% Erwarteter Beitrag -> Bernometer wohlfülmap, erste visualisierung.
% MA Daniela Kognitive Karten von Subjektives Wohlbefinden. Eine Karte anhängen, etwa so könnte eine Karte aussehen.

% Mapping Intersectional Wellbeing in the urban
% 

\begin{document}
\maketitle

\newpage

\tableofcontents


\newpage

\section{Einleitung}„Städte sind für alle da“ – diese Vorstellung urbaner Gleichheit wird oft in Leitbildern und Planungsstrategien bemüht. Sie knüpft an Debatten an, die auch durch den Anspruch auf das \emph{Recht auf Stadt} nach \textcite{lefebvreDroitVille1967} inspiriert sind, auch wenn Lefebvres ursprüngliches Konzept weitaus radikaler war und eine grundlegende Transformation urbanen Lebens forderte. Unabhängig von der theoretischen Tiefe dieser Forderung stellt sich jedoch die Frage: Wie erleben Menschen den urbanen Raum tatsächlich? Und wie beeinflusst ihre soziale Position – etwa hinsichtlich Alter, Geschlecht, Herkunft oder Gesundheit – ihr momentanes Wohlbefinden in bestimmten Umgebungen? Diese Fragen stehen im Zentrum der vorliegenden Bachelorarbeit, die sich der intersektionalen Analyse des unmittelbaren Wohlbefindens in alltäglichen Lebensräumen widmet.

Methoden zur Erfassung momentaner psychischer Zustände und Erfahrungen im Alltag, wie die \gls{esm} und insbesondere das \gls{ema}, wurden bereits in den 1990er Jahren konzipiert, vor allem in der Psychologie \parencite[vgl.][]{stoneEcologicalMomentaryAssessment1994, shiffmanEcologicalMomentaryAssessment2008}. Sie zielten darauf ab, kontextbezogene Daten zu erheben und Nachteile rein retrospektiver Ansätze zu überwinden \parencite{kahnemanDevelopmentsMeasurementSubjective2006}. Das volle Potenzial dieser Methoden, insbesondere für eine unmittelbare, georeferenzierte Datenerhebung in Echtzeit, entfaltete sich jedoch erst mit der Verbreitung von Smartphones. An der Schnittstelle von Stadtplanung und Psychologie wurden zudem Ansätze zur räumlich expliziten Erfassung von Alltagserfahrungen entwickelt, wie etwa das \gls{gema} \parencite[vgl.][]{kirchnerSpatiotemporalDeterminantsMental2016}. Seit etwa Mitte der 2010er Jahre ist eine deutliche Zunahme an Studien zu beobachten, welche die durch Smartphones erweiterten \gls{ema}/\gls{gema}-Möglichkeiten nutzen, um den Zusammenhang zwischen spezifischen räumlichen Umgebungen und psychischer Gesundheit bzw. Wohlbefinden detailliert zu untersuchen. Ein Beispiel hierfür ist das grossangelegte Projekt Urban Mind: Die Arbeiten von \textcite{bakolisUrbanMindUsing2018}, \textcite{bergouMentalHealthBenefits2022} und \textcite{hammoudSmartphonebasedEcologicalMomentary2024} nutzen diesen Ansatz bzw. dessen Methodik, um insbesondere den Einfluss von Grün- und Stadträumen auf die psychische Gesundheit zu analysieren. Diese Studien prägen den aktuellen Forschungstrend, situative affektive Reaktionen systematisch in Bezug auf räumliche Kontexte zu untersuchen.

Parallel dazu existiert eine umfangreiche Forschungsliteratur zur Intersektionalität und deren räumlichen Implikationen, massgeblich geprägt durch feministische und kritische Perspektiven \parencite[vgl.][]{crenshawMappingMarginsIntersectionality1991, rodo-de-zarateDevelopingGeographiesIntersectionality2014, rodo-de-zarateYoungLesbiansNegotiating2015, rodo-de-zarateIntersectionalityFeministGeographies2018}. Diese Arbeiten verdeutlichen, wie unterschiedliche soziale Kategorien wie Geschlecht, Klasse oder ethnische Zugehörigkeit in räumlichen Kontexten miteinander verwoben sind und Ungleichheiten erzeugen oder verstärken können. Insbesondere methodische Innovationen wie die Relief Maps \parencite{rodo-de-zarateDevelopingGeographiesIntersectionality2014} erlauben eine Visualisierung und Analyse dieser komplexen Wechselwirkungen.

Diese Arbeit verbindet die beiden Perspektiven: Sie nutzt die methodischen Möglichkeiten der smartphone-basierten Echtzeit-Datenerfassung, wie sie in der \gls{esm}/\gls{ema}-Forschung etabliert wurden, verknüpft diese jedoch explizit mit der intersektionalen Ungleichheitsanalyse. Der Fokus verschiebt sich dabei von einer rein 'ökologischen' Betrachtung oder einer engen Definition von 'psychischer Gesundheit' hin zu einer Untersuchung des \emph{situativen affektiven Wohlbefindens} in vielfältigen alltäglichen Umgebungen. Es wird untersucht, wie sich intersektionale Positionierungen konkret auf dieses situative Wohlbefinden auswirken. Ziel dieser Bachelorarbeit ist es, im Rahmen einer explorativen Pilotstudie das Potenzial dieser methodischen Adaption und Verknüpfung auszuloten: Es soll geprüft werden, ob und wie dieser Ansatz an der Schnittstelle von feministischer Sozial- und Kulturgeographie, Intersektionalitätsforschung und der Analyse digital erhobener Alltagsdaten erste Einblicke und Hypothesen generieren kann.

Die persönliche Motivation für diese Arbeit ergibt sich aus dem Wunsch, mit eigens entwickelten digitalen Werkzeugen neue Einblicke in Fragen sozialer Gerechtigkeit und Wohlbefinden im Alltag zu ermöglichen. Perspektivisch könnte der hier erprobte methodische Ansatz in weiterführenden Arbeiten dazu dienen, sozialräumliche Fragestellungen mit Themen wie Klimaanpassung oder -mitigation zu verbinden. Eine solche Verknüpfung könnte beispielsweise für den Berner Kontext relevant sein, etwa für die Forschung zur Stadthitze \parencite[vgl.][]{burgerModellingSpatialPattern2021} und für Projekte wie dem \emph{Bernometer}\footnote{\href{https://bernometer.unibe.ch/bernometer/}{bernometer.unibe.ch}}, die mit detaillierten raumbezogenen Daten zum Wohlbefinden weiter ausgebaut werden könnten.Im Fokus steht dabei folgende Forschungsfrage:

\begin{quote}
\emph{Wie beeinflussen räumliche Umgebungen das momentane Wohlbefinden intersektional positionierter Personen im Alltag?}
\end{quote}

Dabei geht es explizit nicht um langfristige subjektive Wohlbefindenswerte, sondern um die im Alltag erlebten situativen, affektiven Reaktionen. Ziel der Analyse ist es, aus einer intersektionalen Perspektive zu untersuchen, unter welchen Bedingungen und an welchen Orten sich Menschen zugehörig oder fremd fühlen. Es soll also ergründet werden, wie soziale Positionierungen und räumliche Kontexte zusammenwirken und das momentane Gefühl der (Nicht-)Zugehörigkeit beeinflussen. Als analytischer Ansatz zur quantitativen Untersuchung der zugrundeliegenden intersektionalen Muster dient \gls{maihda} nach \textcite{grossModellingIntersectionalityQuantitative2023}.

Zur Erhebung der für diese Arbeit notwendigen Daten wurde das digitale Werkzeug \emph{InterMind}\footnote{\href{https://intermind.ch/app}{intermind.ch/app}} entwickelt. Diese App ermöglicht es, Teilnehmende über einen festgelegten Zeitraum hinweg wiederholt zu befragen und ihre Antworten zusammen mit georeferenzierten Informationen in Echtzeit zu erfassen und anonymisiert zu speichern. Für die vorliegende Untersuchung werden im Rahmen einer Pilotstudie mit Studierenden der Universität Bern erste explorative Daten gesammelt. Die App selbst wurde bewusst Open-Source entwickelt, um eine flexible Anpassung an ähnliche Forschungskontexte zu ermöglichen und potenziell eine nachhaltige Infrastruktur für kontextualisierte Alltagsdaten zu bieten.


Der Aufbau der Arbeit gliedert sich wie folgt: Kapitel 2 erörtert den theoretischen Rahmen hinsichtlich Intersektionalität, Well-Being-Konzepten sowie \gls{ema}-Methoden. Kapitel 3 beschreibt die technische und methodische Umsetzung, insbesondere die Entwicklung der App, das Studiendesign sowie den analytischen Zugriff mittels des \gls{maihda}-Ansatzes. Kapitel 4 präsentiert zentrale empirische Ergebnisse zur Beziehung von Wohlbefinden, Raum und intersektionaler Positionierung. Abschliessend werden diese Ergebnisse in Kapitel 5 diskutiert, kontextualisiert und Implikationen für weitere Forschung aufgezeigt.

Diese Arbeit versteht sich als explorativer Beitrag, der methodische Innovationen mit gesellschaftlich relevanten Fragestellungen verbindet. Sie erhebt nicht den Anspruch auf allgemeine Repräsentativität, sondern zielt darauf ab, erste Hypothesen und methodische Potenziale für zukünftige intersektionale Analysen des momentanen Wohlbefindens in alltäglichen Lebensräumen aufzuzeigen.

\section{Arbeitsdefinitionen}

\subsection{Well-Being}

In dieser Arbeit wird unter Well-Being das subjektive psychische Erleben von Personen verstanden, welches kognitive und affektive Komponenten umfasst und durch soziale sowie räumliche Kontexte beeinflusst werden kann. Die Definition orientiert sich an einem subjektiven, multidimensionalen Verständnis von Wohlbefinden, das in der Literatur auch unter dem Begriff \glqq subjective well-being\grqq{} (SWB) diskutiert wird \parencite{dienerSubjectiveWellBeingGeneral2009, kahnemanDevelopmentsMeasurementSubjective2006}. Dabei wird zwischen langfristiger Lebenszufriedenheit (life satisfaction) und kurzfristigeren affektiven Zuständen wie Zufriedenheit, Stress oder Sicherheit unterschieden \parencite{schwanenWellBeingContextEveryday2014}.

Well-Being wird hier nicht als stabile Persönlichkeitseigenschaft aufgefasst, sondern als dynamisches, kontextabhängiges Erleben, das mit alltäglichen Aktivitäten, sozialen Kontakten und räumlichen Situationen in Verbindung steht. Die Analyse stützt sich dabei nicht auf ecological momentary assessment (EMA) oder realzeitliche digitale Erhebungstechniken, sondern auf zeitpunktbezogene Selbstauskünfte, die auf spezifische Situationen Bezug nehmen. Damit wird ein Ansatz verfolgt, der zwar kontextsensitive Daten erlaubt, aber methodisch niedrigschwellig bleibt und bewusst auf invasive oder ortsabhängige Datenerhebung verzichtet.

Die Gestaltung der Skalen und Frageformulierungen orientiert sich an etablierten Operationalisierungen von subjektivem Wohlbefinden, ohne eine standardisierte Skala vollständig zu übernehmen. Der verwendete Ansatz umfasst sowohl affektive Zustände (z.B. \glqq angstfrei\grqq, \glqq inkludiert\grqq, \glqq körperlich wohl\grqq) als auch soziale und physische Kontextwahrnehmungen, die in der Forschung als Prädiktoren für situatives Wohlbefinden identifiziert wurden \parencite{bautistaWhatWellbeingScoping2023, schwanenWellBeingContextEveryday2014}.

\subsection{Intersektionalität}

Intersektionalität bezeichnet in dieser Arbeit die gleichzeitige Wirksamkeit und gegenseitige Durchdringung mehrerer sozialer Positionierungen – etwa Geschlecht, ethnische Zugehörigkeit, Behinderung, sexuelle Orientierung, soziale Klasse – in der Erfahrung von Individuen. Der Begriff geht zurück auf die Arbeiten von Crenshaw und wurde im Kontext der Black Feminist Theory entwickelt, um strukturelle Ungleichheiten sichtbar zu machen, die aus dem Zusammenspiel mehrerer Diskriminierungsachsen entstehen \parencite{crenshawMappingMarginsIntersectionality1991, bauerIntersectionalityQuantitativeResearch2021}.

In quantitativen Designs bedeutet ein intersektionaler Zugang, dass soziale Positionen nicht als additive Einzelkategorien betrachtet werden dürfen (z. B. \glqq Frau\grqq{} plus \glqq migrantisch\grqq{} plus \glqq behindert\grqq{}), sondern dass ihre Kombination eine eigene soziale Position mit spezifischen Effekten bildet. Diese Positionen können nicht durch die blosse Addition einzelner Effekte erfasst werden, sondern müssen als ko-konstituierte Konfigurationen begriffen werden \parencite{hancockWhenMultiplicationDoesnt2007, bauerIntersectionalityQuantitativeResearch2021}.

Das hier entwickelte Fragebogendesign folgt diesem Verständnis, indem es zwar einzelne Dimensionen sozialer Positionierung erfasst (z. B. Geschlecht, Behinderung, Ethnizität), deren Wirkungen jedoch nicht isoliert, sondern in Kombination analysiert werden. Es wird bewusst darauf verzichtet, nach individuellen Einschätzungen des Einflusses einzelner Merkmale auf das Wohlbefinden zu fragen. Stattdessen werden im Analyseprozess durch statistische Verfahren (z. B. Interaktionstermen, Mehrebenenmodelle) differenzierte Zusammenhänge zwischen kombinatorischen Gruppenpositionen und situativem Wohlbefinden untersucht \parencite{bauerIntersectionalityQuantitativeResearch2021}.

Der Fokus liegt dabei auf struktureller Ungleichheit und der Kontextabhängigkeit sozialer Positionen. Das bedeutet auch, dass die Wirkung intersektionaler Positionen nicht in allen Kontexten gleich sein muss. Die Interaktion mit räumlichen und sozialen Bedingungen – etwa Exklusion im öffentlichen Raum – ist zentraler Bestandteil der Analyse \parencite{rodo-de-zarateDevelopingGeographiesIntersectionality2014}.

\subsection{Zusammenführung}

Die hier vorgeschlagenen Definitionen bilden die konzeptionelle Grundlage für die Entwicklung eines Erhebungsinstruments, das sowohl subjektives Wohlbefinden als auch soziale Positionierung erfasst, ohne die Komplexität intersektionaler Erfahrungen zu reduzieren. Die Entscheidung, auf eine realzeitliche Erhebung (z. B. mittels EMA) zu verzichten, folgt dabei nicht nur aus methodischen Überlegungen, sondern auch aus dem Ziel, möglichst barrierearme, partizipative und datenschutzfreundliche Forschung zu ermöglichen.


\section{Quantitative Erfassung von moment­anem Wohlbefinden: Methoden und empirische Befunde}

Die quantitative Erfassung momentaner emotionaler Zustände im Alltag hat in den letzten zwei Jahrzehnten zunehmend Bedeutung erlangt. Anders als klassische retrospektive Bewertungen („Lebenszufriedenheit“) erfassen solche Echtzeitmethoden flüchtige affektive Zustände unmittelbar im Alltag und minimieren dadurch Erinnerungs- und Bewertungsverzerrungen \citep{kahnemanDevelopmentsMeasurementSubjective2006a}. Im Folgenden werden zentrale methodische Zugänge vorgestellt und deren Anwendung anhand empirischer Studien erläutert.

\paragraph{Experience Sampling Method (ESM) und Ecological Momentary Assessment (EMA).}
Die am weitesten verbreitete Methode zur Echtzeiterfassung ist das Experience Sampling (ESM), häufig auch Ecological Momentary Assessment (EMA) genannt. Hierbei beantworten Teilnehmende mehrmals täglich kurze, standardisierte Fragen zu ihrem aktuellen emotionalen Zustand. In einer umfangreichen Studie von \citet{birenboimInfluenceUrbanEnvironments2018} berichteten 91 Teilnehmende über einen Zeitraum von acht Monaten hinweg per Smartphone-App viermal täglich ihr momentanes Wohlbefinden in den Dimensionen Sicherheit, Glück, Komfort und Ärger. Methodisch typisch für EMA sind einfache, numerische Skalen oder Kurzitems zur Erfassung emotionaler Zustände. Ähnlich arbeiteten \citet{chenPerceivedUrbanEnvironment2025}, die mittels einer App in Japan Teilnehmer aufforderten, ortsbezogen (z.B. in Cafés oder Parks) Befragungen auszulösen, um situative Einflüsse auf momentanes Wohlbefinden zu erfassen.

\paragraph{Geographically Explicit Ecological Momentary Assessment (GEMA).}
GEMA erweitert die EMA-Methode durch das automatische Erfassen räumlicher Kontextdaten mittels GPS-Tracking. Kirchner und Shiffman (2016) erläutern, dass durch die Integration räumlicher Informationen eine detaillierte Modellierung möglich ist, bei der Umwelteinflüsse auf momentanes Wohlbefinden unmittelbar mit subjektiven Daten verknüpft werden können. Mascherek et al. (2025) nutzen diese Methode, um den Einfluss urbaner Grünräume, Wetterbedingungen und sozialer Interaktionen auf das Wohlbefinden in drei deutschen Städten zu untersuchen. Sie zeigen, dass dynamische Strukturgleichungsmodelle (DSEM) besonders geeignet sind, um solche komplex verschachtelten Datensätze auszuwerten, da sie zeitliche und räumliche Abhängigkeiten gleichzeitig berücksichtigen können.

\paragraph{Day Reconstruction Method (DRM).}
Die Day Reconstruction Method (DRM) bietet eine retrospektive Alternative zu EMA. Hier rekonstruieren Befragte ihren vorherigen Tag in Episoden und geben für jede Episode ihr affektives Befinden retrospektiv an. Obwohl die DRM retrospektiv ist, zeigen Kahneman und Krueger (2006), dass diese Methode Ergebnisse liefert, die stark mit denen von Echtzeit-Erhebungen übereinstimmen, aber mit deutlich geringerem Aufwand verbunden sind.

\paragraph{Sensorbasierte Kombinationen und Biodiversitätsindikatoren.}
Ein neuerer methodischer Trend kombiniert EMA mit sensorischen oder umweltbezogenen Zusatzdaten. \citet{hammoudSmartphonebasedEcologicalMomentary2024} verwendeten etwa Biodiversitätsindikatoren in ihrer Studie, um den Einfluss der Vielfalt natürlicher Elemente auf momentanes Wohlbefinden zu messen. Ihre Ergebnisse deuten darauf hin, dass eine höhere Artenvielfalt, wie etwa verschiedene Pflanzenarten und Vogelarten, das Wohlbefinden signifikant positiv beeinflussen.

\paragraph{Verwendete Messinstrumente.}
Die Auswahl geeigneter Messinstrumente ist entscheidend für die Validität momentaner Wohlbefindensmessungen. Laut einer umfassenden Übersicht von \citet{cookeMeasuringWellBeingReview2016} existieren zahlreiche Instrumente mit unterschiedlicher theoretischer Grundlage und unterschiedlicher Skalenlänge. Für EMA-Studien sind besonders ultrakurze Skalen oder Kurzformen bekannter Instrumente wie PANAS gebräuchlich, da sie die Belastung der Befragten minimieren.

\paragraph{Statistische Auswertungsmethoden.}
Zur statistischen Auswertung der durch EMA oder GEMA gewonnenen Daten kommen häufig hierarchische lineare Modelle (Multilevel Models) oder dynamische Strukturgleichungsmodelle (DSEM) zum Einsatz. Solche Methoden erlauben, intra- und interindividuelle Varianzen gleichzeitig zu modellieren und bieten zudem die Möglichkeit, komplexe zeitliche Abhängigkeiten und Interaktionen zwischen Variablen zu erfassen \citep{mascherekMeadowsAsphaltRoad2025, hammoudSmartphonebasedEcologicalMomentary2024, chenPerceivedUrbanEnvironment2025}.

\paragraph{Fazit und Forschungslücken.}
Insgesamt zeigen aktuelle methodische Ansätze, dass situative und räumliche Kontextvariablen oft mehr zur Erklärung momentanen Wohlbefindens beitragen als stabile individuelle Faktoren. Dennoch verbleiben methodische Herausforderungen, etwa hinsichtlich der optimalen Messfrequenz, des Datenschutzes bei GPS-Daten und der Generalisierbarkeit der Befunde. Zukünftige Forschungen könnten daher insbesondere auf diversere Stichproben und die langfristige Integration sensorischer Daten fokussieren, um die Robustheit und Übertragbarkeit der bisherigen Ergebnisse weiter zu prüfen.


\chapter{Kontextspezifisch und alltagstauglich -- Entwicklung des Fragebogens}
\label{sec:fragebogenentwicklung}

Zentrales methodisches Instrument dieser Arbeit ist ein Fragebogen, der erfasst, \emph{wie räumliche Umgebungen das momentane (Un-)Wohlbefinden intersektional positionierter Personen im Alltag beeinflussen}. Die Entwicklung des Fragebogens war unabhängig von der technischen Umsetzung in der App (\gls[noindex]{vgl} \cref{sec:entwicklung_app}) konzipiert und diente zugleich dazu, deren Flexibilität und Praxistauglichkeit zu prüfen.

Kernherausforderung war es, zwei Aspekte zu verbinden: Zum einen sollten grundlegende Merkmale zur Charakterisierung der Stichprobe erhoben werden (Baseline-Modul), zum anderen das situative, affektive (Un-)Wohlbefinden im unmittelbaren räumlichen und sozialen Kontext (\gls{ema}-Modul). Die Befragung sollte dabei so kurz wie möglich bleiben, um Akzeptanz und Teilnahmebereitschaft zu sichern. Als Zielvorgaben wurden eine maximale Dauer von zehn Minuten für die Baseline und drei Minuten für die wiederholten situativen Erhebungen festgelegt. Ergänzend wurde der Fragebogen mehrsprachig in Deutsch, Englisch und Französisch umgesetzt, um den Zugang für eine breite Teilnehmendengruppe zu ermöglichen.

Die Aufteilung in ein einmaliges Baseline-Modul und wiederholte situative Erhebungen folgt direkt aus den methodischen Anforderungen der Forschungsfrage: Die Baseline dient der Charakterisierung der Stichprobe für differenzierte intersektionale Analysen, während die situativen Fragen den eigentlichen Kern der Datenerhebung bilden, indem sie (Un-)Wohlbefinden in konkreten Alltagskontexten erfassen.

Der vollständige Fragebogen ist in \cref{app:appendix_fragebogen} zu finden.

\section{Kontext schaffen -- Einmalige Eingangsbefragung}

Die einmalige Baseline-Erhebung (siehe \cref{tab:baseline-fragen}) zielte darauf ab, die sozialen Positionierungen der Teilnehmenden möglichst differenziert zu erfassen. Erhoben wurden Merkmale wie Alter, \gls{gender}, sexuelle Orientierung, Behinderung sowie soziale Klasse (\gls{class}) \parencite{bauerIntersectionalityQuantitativeResearch2021}. 

Die Erfassung von \gls[noindex]{race} erwies sich als methodisch anspruchsvoll. Im europäischen Kontext existieren kaum etablierte Kategorien, die rassifizierte Zugehörigkeiten erfassen, ohne problematische koloniale oder biologistische Zuschreibungen zu reproduzieren \parencite[\gls{vgl}][]{roigIntersectionalityEuropeDepoliticized2018}. Anders als in der US-amerikanischen Tradition, in der standardisierte Selbstkategorisierungen verbreitet sind, fehlen im hiesigen Kontext praktikable, breit akzeptierte Formate für quantitative Erhebungen. Aus diesem Grund wurde im Fragebogen lediglich erfasst, ob Teilnehmende aktuell in einem anderen Land leben als in jenem, in dem sie geboren wurden.

Auch die Erfassung von \gls{class} stellte methodische Anforderungen. Sie erfolgte über eine Kombination mehrerer sozioökonomischer Indikatoren: höchster Bildungsabschluss, aktuelle Beschäftigungssituation, Haushaltseinkommen sowie Anzahl der Haushaltsmitglieder und deren Einkommensbeitrag. Auf klassische Schemata wie \Acrfull{egp} oder \acrfull{esec} wurde verzichtet, da deren Operationalisierung detailliertere Daten zu standardisierten Berufen und sozialstrukturellen Kategorien erfordert hätte, was im Rahmen dieser Erhebung nicht praktikabel war \parencite{bihagenSocialClassEmployment2010}. Stattdessen wurde eine pragmatische, mehrdimensionale Annäherung gewählt, die zentrale Aspekte sozialer Lage abbildet, ohne den Fragebogen unnötig zu verlängern.

Zur Erfassung bereits erfahrener Diskriminierungen wurde ergänzend eine Multiple-Choice-Frage eingesetzt, die sowohl das Vorhandensein als auch die Art der Diskriminierung abfragt. Die Auswahl dieser Merkmale beruhte auf einer pragmatischen Abwägung zwischen analytischer Relevanz, praktischer Umsetzbarkeit und dem Ziel, die Befragung kurz und zugänglich zu halten.

\section{Vom Ort zur Emotion -- situativ befragen}

Der situative Teil des Fragebogens (siehe \cref{tab:wiederholte-fragen}) erfasste die unmittelbare räumliche und soziale Umgebung der Befragten, um deren Einfluss auf das momentane affektive Wohlbefinden abzubilden. Zunächst wurde zwischen Innen- und Aussenaufenthalt unterschieden, gefolgt von einer genaueren Ortskategorisierung (\gls[noindex]{zb} Zuhause, Arbeitsplatz, Café, Park, öffentlicher Verkehr). Weitere erfasste Merkmale waren die Geräuschkulisse, Sichtbarkeit von Pflanzen oder Bäumen, Lebhaftigkeit sowie die subjektiv wahrgenommene Qualität des Ortes. Die soziale Umgebung wurde durch Angaben zu anwesenden Personen und deren Beziehung zu den Befragten beschrieben.

Die Gestaltung dieser Items orientierte sich an der Urban-Mind-Studie \parencite{bakolisUrbanMindUsing2018}, wurde jedoch in kompakter Form umgesetzt. Längere standardisierte Skalen zur Umgebungsqualität (\gls[noindex]{zb} \acrfull{peqi} \parencite{bonaiutoPerceivedResidentialEnvironment2015}, \acrfull{news} \parencite{saelensNeighborhoodEnvironmentWalkability2018}) erwiesen sich aufgrund ihrer Länge und Komplexität als ungeeignet für wiederholte Erhebungen. Die kompakte Umsetzung stellt somit einen bewussten methodischen Kompromiss dar.

Nach aktuellem Forschungsstand existiert kein standardisiertes und breit eingesetztes Instrument zur Erfassung \emph{situativen} affektiven Wohlbefindens, das für mehrfache Erhebungen pro Tag konzipiert ist. Die gängigen Skalen -- etwa \gls{panas} \parencite{yountMeasuringMoodComparison2023}, WHO-5 \parencite{toppWHO5WellBeingIndex2015} oder \gls{wemwbs} \parencite{tennantWarwickEdinburghMentalWellbeing2007} -- stammen überwiegend aus der psychologischen Gesundheitsforschung und sind auf mittlere bis längere Zeiträume (\gls[noindex]{zb} die letzten zwei Wochen) ausgelegt. Sie sind in Umfang und Formulierung nicht auf hochfrequente Erhebungen zugeschnitten und würden den zeitlichen Rahmen von wenigen Minuten pro Befragung deutlich überschreiten.

Vor diesem Hintergrund wurde ein eigener, stark reduzierter Item-Satz entwickelt, um zentrale Dimensionen des Wohlbefindens situativ abbilden zu können. Ausgewählt wurden fünf Dimensionen: generelles Wohlbefinden, Zufriedenheit, Anspannung, Energie und Zugehörigkeit. Die Antworten wurden über lineare Slider-Skalen erfasst, um eine schnelle und intuitive Bearbeitung zu ermöglichen. 

Ein zentrales Merkmal des Moduls war die Einbindung intersektionaler Perspektiven auf situativer Ebene. Ziel war es, nicht nur strukturelle Positionierungen (wie im Baseline-Modul), sondern auch deren situative Wechselwirkungen mit Raum und sozialer Wahrnehmung zu erfassen. Zu diesem Zweck wurden zwei Items entwickelt, die abfragten, ob das aktuelle Zugehörigkeits- oder Fremdheitsgefühl am Ort mit der eigenen gesellschaftlichen Positionierung zusammenhängt, sowie in welchen Merkmalen sich die Befragten im Vergleich zu Anwesenden als zugehörig oder nicht zugehörig empfanden.

Die Entwicklung dieser Items orientierte sich inhaltlich an den Überlegungen von \textcite{rodo-de-zarateIntersectionalitySpatialityEmotions2023} zur räumlichen Dimension von Emotionen und deren Rolle bei der (Re-)Produktion intersektionaler Ungleichheiten. Insbesondere die von Rodó-de-Zárate vorgeschlagene Differenzierung von (Un-)Wohlbefinden in Relation zu Machtgeometrien diente als konzeptioneller Ausgangspunkt. Mangels eines standardisierten, auf situative Mehrfacherhebungen zugeschnittenen Instruments erfolgte die konkrete Formulierung jedoch in einem pragmatischen, explorativen Prozess, mit dem Ziel, die Fragen in wenigen Sekunden beantworten zu können.

Ergänzend boten zwei offene Fragen Raum für die Benennung weiterer kontextgebundener Gründe für situatives (Un-)Wohlbefinden. Diese qualitativen Elemente ermöglichen es, affektive und kontextuelle Faktoren sichtbar zu machen, die durch geschlossene Fragen nicht erfasst werden können, und verhindern so eine Reduktion komplexer Ungleichheitsverhältnisse auf rein numerische Merkmale.

\section{Klar, verständlich, iterativ -- Der Weg zum finalen Fragebogen}

Die sprachliche Gestaltung der Fragebogen-Items stellte im Entwicklungsprozess eine zentrale methodische Herausforderung dar. Ziel war es, die Befragung möglichst zugänglich, verständlich und gleichzeitig inhaltlich präzise zu gestalten. Da die Erhebung explizit auf eine intersektionale Analyse abzielt, wurde besonderer Wert darauf gelegt, die sprachliche Zugänglichkeit möglichst breit zu gewährleisten. Der Fragebogen wurde daher mehrsprachig konzipiert und auf Deutsch, Englisch sowie Französisch umgesetzt. Weitere Sprachversionen wären aus Sicht der Zugänglichkeit sinnvoll gewesen, erforderten jedoch einen hohen Übersetzungs- und Abstimmungsaufwand, um inhaltliche Konsistenz zu sichern.

Ein bewusst gewählter Bestandteil der Konzeption war eine direkte, adressierende Sprache in der \enquote{Du}-Form. Sie sollte einen niederschwelligen Zugang fördern und hierarchische Distanz zwischen Forschenden und Teilnehmenden verringern. Gleichzeitig mussten komplexe Konzepte so operationalisiert werden, dass sie in alltagsnaher, schnell erfassbarer Form vermittelt werden konnten. So wurde das Konzept der \gls{intersektionalitaet} im Einführungsteil erläutert, in den eigentlichen Items jedoch vermieden, um unnötige Barrieren zu verhindern. Stattdessen kamen allgemeinere Formulierungen wie \enquote{persönliche Merkmale} zum Einsatz.

Besondere Aufmerksamkeit erforderte die Übersetzung und Anpassung zentraler Begriffe zwischen den Sprachversionen. Im Fall von \gls[noindex]{race} stellte sich insbesondere im deutschsprachigen Kontext die Frage nach geeigneten Begrifflichkeiten, da etablierte Termini entweder ungebräuchlich, problematisch oder unpräzise sind \parencite[\gls{vgl}][]{roigIntersectionalityEuropeDepoliticized2018}. Auch bei affektiven Zustandsbeschreibungen wurden die Formulierungen nicht wörtlich, sondern sinngemäß übertragen und kulturelle Unterschiede in der Wortverwendung berücksichtigt.

Der Übersetzungsprozess war Teil eines iterativen Entwicklungsablaufs, der auf Literaturrecherche, Rückmeldungen aus der Testphase der App (siehe \cref{sec:app_entwicklung_feldtest}) und Abstimmungen mit der betreuenden Dozentin basierte. Mehrere Überarbeitungsrunden führten zu sprachlichen und strukturellen Anpassungen, die sowohl die Verständlichkeit als auch die Anschlussfähigkeit der Items verbesserten. Ein durchgängiges Kriterium war dabei, den zeitlichen und kognitiven Aufwand für Teilnehmende gering zu halten, ohne zentrale Aspekte der Forschungsfrage zu vernachlässigen.



\section{Appentwicklung}
\subsection{Anforderungskatalog}

Dieser Abschnitt beschreibt detailliert die funktionalen und nicht-funktionalen Anforderungen an die App, die intersektionales Wohlbefinden im Stadtraum erfasst. Ziel der App ist es, mittels ortsbasierter Erhebungen in Echtzeit subjektive Wohlbefindensdaten zu sammeln – unter besonderer Berücksichtigung intersektionaler Merkmale – und dabei möglichst wenig persönliche Daten der Nutzer zu erfassen. Die Anforderungen wurden unter Anwendung etablierter Methoden (vgl. \cite{cleggCaseMethodFastTrack1994}) ermittelt und in die Kategorien Funktionalität, Sicherheit/Datenschutz, Mehrsprachigkeit sowie Erweiterbarkeit und Wartbarkeit unterteilt.

\paragraph{Funktionale Anforderungen}
\begin{itemize}
    \item \textbf{Gerätebasierte Identifikation und Datenerfassung:}  
    Die App verzichtet bewusst auf eine klassische Anmeldung oder Nutzerkonten. Stattdessen wird jedem Gerät eine eindeutige ID zugewiesen, anhand derer sämtliche Daten erfasst und verwaltet werden. Dieses Vorgehen minimiert die Erhebung personenbezogener Daten, birgt jedoch technische Herausforderungen hinsichtlich der dauerhaften Identifikation und der Datensicherheit. Sollte sich herausstellen, dass die gerätebasierte Identifikation nicht zuverlässig implementiert werden kann, wäre alternativ eine Anmeldefunktion in Erwägung zu ziehen – was allerdings im Widerspruch zum Ziel der minimal-invasiven Datenerfassung stehen würde.

    \item \textbf{Standortbasierte und intersektionale Datenerhebung:}  
    Die Erhebung erfolgt in Echtzeit, indem automatisch der aktuelle Standort des Geräts erfasst wird. Ergänzend dazu wird in einmaligen und wiederkehrenden Erhebungen das subjektive Wohlbefinden abgefragt sowie verschiedene intersektionale Merkmale (z.B. Geschlecht, Herkunft, sozioökonomischer Status) abgefragt, um einen vielschichtigen Blick auf urbane Lebenswelten zu erhalten.

    \item \textbf{Erinnerungsfunktion:}  
    Um eine möglichst repräsentative Datenerhebung zu ermöglichen, ist eine Erinnerungsfunktion implementiert, die den Nutzer dreimal täglich zu zufälligen, variierenden Zeiten (innerhalb definierter Zeitfenster) dazu auffordert, an der Befragung teilzunehmen. Die leichte Randomisierung der Erinnerungszeitpunkte soll sicherstellen, dass Daten aus unterschiedlichen Kontexten und Tageszeiten erfasst werden.

    \item \textbf{Datenselbstverwaltung und Löschfunktion:}  
    Die Nutzer sollen jederzeit in die Lage versetzt werden, ihre Daten selbstständig zu verwalten. Insbesondere ist eine Funktion zur vollständigen Löschung aller auf dem Gerät gespeicherten Daten vorgesehen, um den Prinzipien der Selbstbestimmung und des Datenschutzes gerecht zu werden.
\end{itemize}

\paragraph{Nicht-funktionale Anforderungen}
\begin{itemize}
    \item \textbf{Sicherheit und Datenschutz:}  
    Neben der Minimierung der erhobenen Personendaten (nur gerätebasierte ID) ist es essenziell, die Datensicherheit zu gewährleisten. Technisch sollen Massnahmen wie die Verschlüsselung der übertragenen Daten sowie regelmäßige Sicherheitsupdates implementiert werden. Sollte sich die gerätebasierte Identifikation als technisch unzureichend herausstellen, wäre die Einführung eines Anmeldesystems als zusätzliche Schutzmassnahme zu prüfen – dies erfolgt jedoch nur, wenn der Mehrwert hinsichtlich der Datensicherheit den potenziellen Verlust an Anonymität überwiegt.

    \item \textbf{Mehrsprachigkeit:}  
    Angesichts der intersektionalen Zielsetzung der App wird besonderer Wert auf Mehrsprachigkeit gelegt. Die Benutzeroberfläche soll in mehreren Sprachen verfügbar sein, um eine diverse Nutzergruppe anzusprechen und kulturelle Hürden beim Datenaustausch zu minimieren.

    \item \textbf{Barrierefreiheit:}  
    Obwohl der zeitliche Rahmen der Entwicklung Einschränkungen mit sich bringt, ist die Berücksichtigung von Barrierefreiheit (z.B. intuitive Navigation, Anpassbarkeit der Schriftgrössen, kontrastreiche Darstellung) ein zentrales Anliegen. Die tatsächliche Realisierung von Barrierefreiheitsstandards wird im weiteren Entwicklungsprozess konkret überprüft und angepasst.

    \item \textbf{Plattformübergreifende Kompatibilität und technische Umsetzung:}  
    Die App wird mittels React Native (Expo) entwickelt, was grundsätzlich eine plattformübergreifende Nutzung auf Android- und iOS-Geräten sicherstellt. Diese Entscheidung unterstützt die schnelle Entwicklung und die einfache Wartung der App, da die grundlegende Infrastruktur standardisiert und modular aufgebaut ist.

    \item \textbf{Erweiterbarkeit und Wartbarkeit:}  
    Eine klare Modulstruktur, umfassende Dokumentation sowie standardisierte Schnittstellen sind essenziell, um zukünftige Erweiterungen und Anpassungen zu ermöglichen. Diese Massnahmen gewährleisten, dass die App nicht nur im aktuellen Projektkontext besteht, sondern auch in zukünftigen Forschungs- und Entwicklungsprojekten weiterentwickelt werden kann.
\end{itemize}

\paragraph{Priorisierung der Anforderungen}  
Zur systematischen Priorisierung wurde die MoSCoW-Methode angewandt (vgl. \cite{cleggCaseMethodFastTrack1994}):
\begin{itemize}
    \item \textbf{Must have:}  
    \begin{itemize}
        \item Gerätebasierte Identifikation und standortbasierte Datenerhebung
        \item Intersektionale Erfassung von Daten zum Wohlbefinden
        \item Erinnerungsfunktion (dreimal täglich, variabel)
        \item Datenselbstverwaltung inklusive Löschfunktion
    \end{itemize}
    \item \textbf{Should have:}  
    \begin{itemize}
        \item Umfassende Mehrsprachigkeit der Benutzeroberfläche
        \item Erweiterte Sicherheitsmassnahmen und optionales Anmeldesystem (falls die gerätebasierte Lösung unzureichend ist)
    \end{itemize}
    \item \textbf{Could have:}  
    \begin{itemize}
        \item Zusätzliche Anpassungen und Optimierungen der Barrierefreiheit
    \end{itemize}
    \item \textbf{Won’t have:}  
    \begin{itemize}
        \item Funktionen zur Datenvisualisierung, da die App ausschliesslich zur Erfassung der Daten konzipiert ist
    \end{itemize}
\end{itemize}

Zusammenfassend adressiert der Anforderungskatalog sowohl die funktionalen Aufgaben der App als auch wesentliche Qualitätsaspekte, die zur Realisierung einer sicheren, nutzerfreundlichen und erweiterbaren Daten-Erhebungsplattform erforderlich sind. Die systematische Dokumentation dieser Anforderungen bildet die Basis für die technische Umsetzung und die spätere Evaluation der Anwendung.



% ------------------ Glossar

\clearpage
\glsaddall
\printglossary




% ----------------- Bibliographie ------------------
\newpage
\PrintBib

\newpage

\section*{Hinweis für den Einsatz von künstlicher Intelligenz (KI)}

Dieses Dokument wurde mithilfe von KI-basierten Tools überarbeitet. LanguageTool, ein KI-gestütztes Grammatik- und Stilprüfungswerkzeug, wurde verwendet, um Formulierungen zu verbessern und die Grammatik zu korrigieren. Chat-GPT von Open-AI wurde verwendet, um Feedback zur Klarheit und Strukturierung des Textes zu erhalten. Es wurde keine KI zur Erstellung von Originalinhalten verwendet.



\end{document}