\documentclass{template}
\usepackage{array}
\usepackage[ngerman]{babel}
\usepackage{graphicx}
\usepackage{float}
\usepackage{csquotes}




\SetTitle{Intersektionales Wohlbefinden im Stadtraum: \\
Konzeption und Umsetzung einer App zur räumlichen Erfassung von Wohlbefinden}
\SetAuthor{\textbf{Bachelorarbeit}\\
Lukas Batschelet, \href{mailto:lukas.batschelet@students.unibe.ch}{lukas.batschelet@students.unibe.ch} \\ 
Betreuung: Prof. Dr. Carolin Schurr und Dr. Moritz Gubler\\
Geographisches Institut Universität Bern}
\SetDate{\today}

\makeglossaries
\loadglsentries{Arbeit/glossary}

\AddBibFile{BA_Urban Mental Health.bib}


% Hinführung, wieso ist das wichtig? Klimagerechtigkeit, warum muss man das untersuchen?
% Erwarteter Beitrag -> Bernometer wohlfülmap, erste visualisierung.
% MA Daniela Kognitive Karten von Subjektives Wohlbefinden. Eine Karte anhängen, etwa so könnte eine Karte aussehen.

% Mapping Intersectional Wellbeing in the urban
% 

\begin{document}
\maketitle

\newpage

\tableofcontents


\newpage

\section*{Arbeitsdefinitionen}

\subsection*{Well-Being}

In dieser Arbeit wird unter Well-Being das subjektive psychische Erleben von Personen verstanden, welches kognitive und affektive Komponenten umfasst und durch soziale sowie räumliche Kontexte beeinflusst werden kann. Die Definition orientiert sich an einem subjektiven, multidimensionalen Verständnis von Wohlbefinden, das in der Literatur auch unter dem Begriff "subjective well-being" (SWB) diskutiert wird \parencite{diener2009, kahnemanKrueger2006}. Dabei wird zwischen langfristiger Lebenszufriedenheit (life satisfaction) und kurzfristigeren affektiven Zuständen wie Zufriedenheit, Stress oder Sicherheit unterschieden \parencite{schwanenWellBeingContextEveryday2014}.

Well-Being wird hier nicht als stabile Persönlichkeitseigenschaft aufgefasst, sondern als dynamisches, kontextabhängiges Erleben, das mit alltäglichen Aktivitäten, sozialen Kontakten und räumlichen Situationen in Verbindung steht. Die Analyse stützt sich dabei nicht auf ecological momentary assessment (EMA) oder realzeitliche digitale Erhebungstechniken, sondern auf zeitpunktbezogene Selbstauskünfte, die auf spezifische Situationen Bezug nehmen. Damit wird ein Ansatz verfolgt, der zwar kontextsensitive Daten erlaubt, aber methodisch niedrigschwellig bleibt und bewusst auf invasive oder ortsabhängige Datenerhebung verzichtet.

Die Gestaltung der Skalen und Frageformulierungen orientiert sich an etablierten Operationalisierungen von subjektivem Wohlbefinden, ohne eine standardisierte Skala vollständig zu übernehmen. Der verwendete Ansatz umfasst sowohl affektive Zustände (z. B. "angstfrei", "inkludiert", "körperlich wohl") als auch soziale und physische Kontextwahrnehmungen, die in der Forschung als Prädiktoren für situatives Wohlbefinden identifiziert wurden \parencite{bautistaWhatWellbeingScoping2023, schwanenWellBeingContextEveryday2014}.

\subsection*{Intersektionalität}

Intersektionalität bezeichnet in dieser Arbeit die gleichzeitige Wirksamkeit und gegenseitige Durchdringung mehrerer sozialer Positionierungen – etwa Geschlecht, ethnische Zugehörigkeit, Behinderung, sexuelle Orientierung, soziale Klasse – in der Erfahrung von Individuen. Der Begriff geht zurück auf die Arbeiten von Crenshaw und wurde im Kontext der Black Feminist Theory entwickelt, um strukturelle Ungleichheiten sichtbar zu machen, die aus dem Zusammenspiel mehrerer Diskriminierungsachsen entstehen \parencite{crenshawMappingMarginsIntersectionality1991, bauerIntersectionalityQuantitativeResearch2021}.

In quantitativen Designs bedeutet ein intersektionaler Zugang, dass soziale Positionen nicht als additive Einzelkategorien betrachtet werden dürfen (z. B. "Frau" plus "migrantisch" plus "behindert"), sondern dass ihre Kombination eine eigene soziale Position mit spezifischen Effekten bildet. Diese Positionen können nicht durch die bloße Addition einzelner Effekte erfasst werden, sondern müssen als ko-konstituierte Konfigurationen begriffen werden \parencite{hancock2007, bauerIntersectionalityQuantitativeResearch2021}.

Das hier entwickelte Fragebogendesign folgt diesem Verständnis, indem es zwar einzelne Dimensionen sozialer Positionierung erfasst (z. B. Geschlecht, Behinderung, Ethnizität), deren Wirkungen jedoch nicht isoliert, sondern in Kombination analysiert werden. Es wird bewusst darauf verzichtet, nach individuellen Einschätzungen des Einflusses einzelner Merkmale auf das Wohlbefinden zu fragen. Stattdessen werden im Analyseprozess durch statistische Verfahren (z. B. Interaktionstermen, Mehrebenenmodelle) differenzierte Zusammenhänge zwischen kombinatorischen Gruppenpositionen und situativem Wohlbefinden untersucht \parencite{bauerIntersectionalityQuantitativeResearch2021}.

Der Fokus liegt dabei auf struktureller Ungleichheit und der Kontextabhängigkeit sozialer Positionen. Das bedeutet auch, dass die Wirkung intersektionaler Positionen nicht in allen Kontexten gleich sein muss. Die Interaktion mit räumlichen und sozialen Bedingungen – etwa Exklusion im öffentlichen Raum – ist zentraler Bestandteil der Analyse \parencite{rodo-de-zarateDevelopingGeographiesIntersectionality2014}.

\subsection*{Zusammenführung}

Die hier vorgeschlagenen Definitionen bilden die konzeptionelle Grundlage für die Entwicklung eines Erhebungsinstruments, das sowohl subjektives Wohlbefinden als auch soziale Positionierung erfasst, ohne die Komplexität intersektionaler Erfahrungen zu reduzieren. Die Entscheidung, auf eine realzeitliche Erhebung (z. B. mittels EMA) zu verzichten, folgt dabei nicht nur aus methodischen Überlegungen, sondern auch aus dem Ziel, möglichst barrierearme, partizipative und datenschutzfreundliche Forschung zu ermöglichen.


\section{Literaturübersicht}

\subsection{Einführung und Definition des Begriffs Wohlbefinden}

Wohlbefinden bezeichnet einen Zustand, der über die Abwesenheit von Krankheit oder Beschwerden hinausgeht und sowohl psychische als auch physische Dimensionen umfasst. Während physisches Wohlbefinden oft durch objektive Indikatoren wie Gesundheitsstatus oder körperliche Funktionen erfasst wird, bezeichnet psychisches Wohlbefinden häufig subjektive Empfindungen, wie Zufriedenheit, Glück oder positive Stimmungslagen \parencite{kahnemanDevelopmentsMeasurementSubjective2006}. Im Kontext aktueller sozialwissenschaftlicher und gesundheitsbezogener Forschung gewinnt die Erfassung subjektiver Zustände zunehmend an Bedeutung, da diese die erlebte Lebensqualität und Gesundheit ganzheitlicher abbilden. Dabei unterscheidet die Forschung häufig zwischen langfristigem („Lebenszufriedenheit“) und kurzfristigem, sogenanntem momentanem Wohlbefinden, das die affektiven Zustände von Personen unmittelbar, während alltäglicher Situationen beschreibt \parencite{chenPerceivedUrbanEnvironment2025}.

Die präzise quantitative Erfassung von momentanen Wohlbefindenszuständen erfordert methodische Ansätze, die Erinnerungsverzerrungen und retrospektive Verzerrungen möglichst vermeiden. Retrospektive Bewertungen wie klassische Lebenszufriedenheits-Skalen sind stark anfällig für Verzerrungen durch kontextuelle Einflüsse oder die selektive Erinnerung besonders intensiver Momente \parencite{kahnemanDevelopmentsMeasurementSubjective2006}. Daher etablierte sich in den letzten Jahrzehnten zunehmend die Nutzung von Methoden, die Wohlbefinden unmittelbar im Moment der Erfahrung messen.

\subsection{Quantitative Methoden zur Messung von momentanen Wohlbefindenszuständen}

Die am häufigsten eingesetzten quantitativen Ansätze zur Erfassung momentanen Wohlbefindens basieren auf der \acrfull{esm}, auch bekannt als \acrfull{ema}. Beide Ansätze erfassen Wohlbefinden in Echtzeit, indem Studienteilnehmende mehrmals täglich auf kurze standardisierte Fragen über ihr aktuelles emotionales Erleben reagieren. So nutzten beispielsweise \textcite{birenboimInfluenceUrbanEnvironments2018} ein \acrshort{ema}-basiertes Smartphone-Design, bei dem Teilnehmende über acht Monate hinweg viermal täglich Fragen zu vier spezifischen affektiven Dimensionen (Sicherheit, Glück, Komfort, Ärger) beantworteten. Die quantitative Erfassung erfolgte mittels kurzer numerischer Skalen, welche die aktuelle Intensität dieser Zustände unmittelbar abfragten.

Ähnlich arbeiteten \textcite{chenPerceivedUrbanEnvironment2025} in einer Studie im japanischen Kashiwa, wo mittels einer Smartphone-App momentanes Wohlbefinden zusammen mit kontextspezifischen Umgebungsmerkmalen erhoben wurde. Dabei wurden affektive Zustände explizit im Kontext verschiedener urbaner Settings (z.B. Cafés, öffentliche Grünflächen oder belebte Plätze) erfasst und anschliessend mittels Strukturgleichungsmodellen quantitativ analysiert. Diese Methode ermöglichte es den Forschenden, sowohl kurzfristige Effekte von Umgebungen auf das Wohlbefinden als auch langfristige Zusammenhänge differenziert darzustellen.

Eine methodische Weiterentwicklung stellt das sogenannte \acrfull{gema} dar. Dabei wird die klassische \acrshort{ema} mit räumlichen Kontextinformationen kombiniert, die durch GPS-Daten automatisiert erhoben werden. \textcite{kirchnerSpatiotemporalDeterminantsMental2016} beschreiben, dass dieser Ansatz es ermöglicht, nicht nur subjektive affektive Zustände, sondern auch deren situative und räumliche Kontextfaktoren explizit quantitativ zu modellieren. So konnten \textcite{mascherekMeadowsAsphaltRoad2025} mit einer \acrshort{gema}-basierten Studie in drei deutschen Grossräumen nachweisen, dass externe Faktoren wie Sonnenschein, soziale Begleitung und Mobilität messbar positive Effekte auf die momentane Stimmung haben. In dieser Studie wurden einfache numerische Skalen eingesetzt, um die Intensität der momentanen affektiven Zustände zu quantifizieren, während gleichzeitig automatisierte räumliche Kontextinformationen über GPS erfasst wurden.

Ein innovativer methodischer Ansatz findet sich auch in der Arbeit von \textcite{hammoudSmartphonebasedEcologicalMomentary2024}, die in der „Urban Mind“-Studie über 41,000 Einzelfälle analysierten. Neben klassischen affektiven Items wurde in dieser Studie zusätzlich eine selbstberichtete Einschätzung der lokalen Biodiversität erhoben, wobei Teilnehmende angaben, wie viele unterschiedliche natürliche Elemente (z.B. Vogelarten, Baumarten) sie aktuell wahrnehmen konnten. Die Ergebnisse zeigten dabei einen signifikanten Zusammenhang zwischen wahrgenommener Biodiversität und momentaner psychischer Befindlichkeit. Methodisch wurden hier gemischte lineare Modelle eingesetzt, die es erlaubten, komplexe Wechselwirkungen zwischen individuellen, situativen und räumlichen Variablen quantitativ abzubilden.

Insgesamt zeigt sich, dass quantitative Studien zum momentanen Wohlbefinden typischerweise einfache numerische Skalen verwenden, um die affektive Dimension präzise, aber dennoch niedrigschwellig zu erfassen. Dies dient einerseits dazu, die Belastung der Teilnehmenden gering zu halten und andererseits methodisch verlässliche und validierbare Daten zu erzeugen. Dennoch betonen \textcite{cookeMeasuringWellBeingReview2016}, dass bei der Auswahl der Messinstrumente sorgfältig auf deren theoretische und psychometrische Qualität geachtet werden muss, da viele Instrumente in ihrer Aussagekraft stark variieren.

Zusammenfassend lässt sich feststellen, dass die quantitative Erfassung von momentanen Wohlbefindenszuständen durch vielfältige methodische Ansätze geprägt ist. Während einfache EMA-Methoden primär auf subjektiven Selbstauskünften basieren, erlauben erweiterte Ansätze wie GEMA und die Integration weiterer sensorischer oder umweltbezogener Variablen eine deutlich differenziertere und räumlich-kontextuelle Analyse. Die statistische Auswertung dieser Daten erfolgt zunehmend mittels komplexer, hierarchisch verschachtelter Modelle, um intra- und interindividuelle Varianzen sowie zeitliche Dynamiken adäquat abzubilden. Trotz ihrer methodischen Diversität eint alle diese Ansätze das Ziel, Wohlbefinden möglichst unmittelbar, präzise und valide zu erfassen und damit ein tieferes Verständnis für die situativen Determinanten subjektiven Wohlbefindens zu ermöglichen.

\subsection{Quantitative Methoden zur Erfassung und Analyse von Intersektionalität}

Intersektionalität beschreibt das Zusammenspiel mehrerer sozialer Kategorien, wie Geschlecht, ethnische Zugehörigkeit oder soziale Schicht, und deren Rolle bei der Entstehung und Verstärkung sozialer Ungleichheiten. Ursprünglich ein theoretisches Konzept, wurde Intersektionalität zunächst überwiegend qualitativ untersucht. Erst in jüngerer Zeit werden zunehmend quantitative Ansätze genutzt, um intersektionale Zusammenhänge systematisch und empirisch zu erfassen \parencite{bauerIntersectionalityQuantitativeResearch2021}. Die Operationalisierung intersektionaler Ansätze stellt dabei spezifische methodische Herausforderungen dar, da klassische statistische Verfahren häufig nicht in der Lage sind, die komplexe, multidimensionale Natur intersektionaler Effekte angemessen abzubilden \parencite{bowlegInvitedReflectionQuantifying2016, scottIntersectionalityQuantitativeMethods2017}.

Die systematische Übersichtsarbeit von \textcite{bauerIntersectionalityQuantitativeResearch2021} zeigt, dass quantitative Analysen intersektionaler Zusammenhänge in den letzten Jahren stark zugenommen haben, gleichzeitig aber oft nicht ausreichend theoretisch fundiert sind. Viele quantitative Studien verwenden noch immer einfache statistische Verfahren, die Intersektionalität fälschlicherweise als additive oder einfache multiplikative Effekte sozialer Kategorien modellieren. Dadurch wird das eigentliche Kernprinzip intersektionaler Theorie, nämlich die kontextabhängige, dynamische Verschränkung sozialer Positionen, häufig vernachlässigt oder sogar fehlinterpretiert \parencite{bowlegInvitedReflectionQuantifying2016}.

Ein häufig eingesetzter methodischer Ansatz in der quantitativen Intersektionalitätsforschung sind multiple Regressionsmodelle, die sogenannte Interaktionsterme zwischen verschiedenen sozialen Kategorien verwenden, um deren kombinierten Einfluss abzubilden. Wie \textcite{scottIntersectionalityQuantitativeMethods2017} zeigen, kann dieser Ansatz zwar erste Einblicke in die Wechselwirkungen sozialer Kategorien bieten, doch weisen sie gleichzeitig darauf hin, dass diese Modelle oft nur begrenzt geeignet sind, komplexere und vor allem kontextspezifische Muster intersektionaler Benachteiligung zu erfassen. Besonders problematisch ist dabei, dass Interaktionsterme in linearen Modellen typischerweise additive Interpretationen nahelegen und in nichtlinearen Modellen, wie etwa logistischen Regressionen, komplexe multiplikative Effekte entstehen, deren Interpretation deutlich anspruchsvoller ist.

Um diesen methodischen Grenzen besser zu begegnen, setzen Forschende zunehmend auf Multilevel-Modelle. Diese Modelle bieten den Vorteil, dass sie mehrere Ebenen (zum Beispiel individuelle Merkmale und strukturelle Kontextfaktoren) gleichzeitig berücksichtigen können. Ein besonders vielversprechender Multilevel-Ansatz ist der sogenannte MAIHDA-Ansatz („Multilevel Analysis of Individual Heterogeneity and Discriminatory Accuracy“), der von \textcite{grossModellingIntersectionalityQuantitative2023} ausführlich diskutiert wird. Diese Methode erlaubt es, nicht nur Unterschiede zwischen verschiedenen intersektionalen Gruppen (wie beispielsweise „schwarze Frauen“ oder „weiße Männer“) abzubilden, sondern auch Unterschiede innerhalb dieser Gruppen statistisch zu modellieren. Dadurch können Forschende präziser analysieren, wie heterogen Erfahrungen innerhalb vermeintlich homogener intersektionaler Gruppen tatsächlich sind. Dies trägt wesentlich dazu bei, dass intersektionale Analysen sowohl theoretisch konsistenter als auch empirisch differenzierter durchgeführt werden können \parencite{grossModellingIntersectionalityQuantitative2023}.

Zusätzlich zu diesen Multilevel-Ansätzen bieten sich kausale Mediationsanalysen an, die \textcite{bauerAdvancingQuantitativeIntersectionality2019} als innovative Methode hervorheben. Diese Analysen ermöglichen es, direkte und indirekte Mechanismen intersektionaler Benachteiligungen genauer zu untersuchen. Bauer und Scheim (2019) beschreiben dabei insbesondere eine dreifache kausale Zerlegung („three-way decomposition“), die verwendet werden kann, um zu unterscheiden, ob Effekte bestimmter sozialer Kategorien (z.,B. Geschlecht oder Ethnizität) direkt, indirekt oder interaktiv auf Zielvariablen wie Gesundheit oder Diskriminierung wirken. Durch diese Methode wird es möglich, komplexe Kausalpfade intersektionaler Benachteiligungen sichtbar zu machen und dadurch ein tiefergehendes Verständnis für diese Prozesse zu entwickeln.

Neben regressions- und mediationsbasierten Verfahren werden explorative statistische Verfahren wie Entscheidungsbäume oder Clusteranalysen ebenfalls zunehmend eingesetzt, um intersektionale Datenstrukturen zu untersuchen \parencite{bauerIntersectionalityQuantitativeResearch2021}. Solche explorativen Verfahren bieten den Vorteil, dass sie theoretisch weniger vorstrukturierte Analysen erlauben und es Forschenden ermöglichen, komplexe und unerwartete intersektionale Muster empirisch sichtbar zu machen. Dennoch mahnen \textcite{bauerIntersectionalityQuantitativeResearch2021} an, dass gerade diese datengetriebenen Ansätze theoretische Aspekte der Intersektionalität nicht vernachlässigen dürfen und kritisch reflektiert eingesetzt werden sollten.

Trotz des methodischen Fortschritts gibt es weiterhin zentrale Herausforderungen bei der quantitativen Analyse intersektionaler Zusammenhänge. Viele quantitative Studien riskieren durch eine mangelnde theoretische Reflexion oder eine unzureichende methodische Fundierung, zentrale Prinzipien der Intersektionalität falsch oder nur oberflächlich umzusetzen. Insbesondere werden Intersektionalitätseffekte häufig auf einfache statistische Interaktionen reduziert, wodurch die theoretische und politische Dimension des Konzepts verloren gehen könnte \parencite{bowlegInvitedReflectionQuantifying2016,bauerIntersectionalityQuantitativeResearch2021}.

Um diese Herausforderungen künftig besser zu adressieren, empfehlen \textcite{bauerIntersectionalityQuantitativeResearch2021} sowie \textcite{bauerAdvancingQuantitativeIntersectionality2019} eine systematischere Verzahnung theoretischer und methodischer Perspektiven in der quantitativen Forschung. Notwendig sei eine klarere Berichterstattung sowie eine intensivere methodische Reflexion darüber, wie und warum bestimmte methodische Ansätze ausgewählt und implementiert wurden.

Zusammenfassend lässt sich feststellen, dass die quantitative Forschung zur Intersektionalität in den letzten Jahren deutliche Fortschritte gemacht hat, insbesondere durch die Entwicklung neuerer statistischer Verfahren wie MAIHDA und kausaler Mediationsanalyse. Gleichwohl bedarf es einer kontinuierlichen methodischen Weiterentwicklung sowie einer stärkeren theoretischen Reflexion quantitativer Forschungsansätze, um die komplexe, multidimensionale Natur intersektionaler Ungleichheiten präziser und umfassender zu erfassen.


\section{Fragebogenentwicklung}
\label{sec:fragebogen}

Die Entwicklung des Fragebogens erfolgte unter Berücksichtigung verschiedener Zielsetzungen, die im Folgenden präzisiert und begründet werden. Anschließend wird der Aufbau des Fragebogens im Detail erläutert.

\subsection{Zielsetzungen}
\label{subsec:zielsetzungen}

\begin{enumerate}[label=(Z\arabic*)]
    \item \textbf{Kürze und Wiederholbarkeit:} 
    Der Fragebogen sollte für die Teilnehmenden in \emph{möglichst kurzer Zeit} beantwortbar sein (circa drei bis fünf Minuten pro Erhebung). Diese Anforderung beruht auf Erfahrungen in der Umfrageforschung, wonach die Abbruchquote bei zunehmender Befragungsdauer stark steigt \parencite{dillman_internet_2014, bradburn_asking_2004}. Gleichzeitig ist eine \emph{regelmäßige Wiederholbarkeit} intendiert, um Längsschnittdaten (bzw.\ Mehrfachmessungen) zu erzeugen. Ein kurzer Fragenkatalog erhöht die Wahrscheinlichkeit, dass Teilnehmende an mehreren Zeitpunkten bereitwillig Auskunft geben \parencite{krosnick_question_2009}.

    \item \textbf{Erfassung intersektionaler Aspekte des Wohlbefindens:}
    Soziale Kategorien wie Geschlecht, Alter, soziale Lage, sexuelle Orientierung oder ethnische Zugehörigkeit interagieren bei der Entstehung von \emph{privilegierten} und \emph{diskriminierten} Positionen \parencite{crenshaw_mapping_1991, collins_black_2002}. Das Ziel ist, verschiedene Dimensionen dieser sozialen Kategorien \emph{gleichzeitig} – im Sinne einer \emph{intersektionalen} Perspektive – abzubilden und zu untersuchen, wie stark sie das subjektive Erleben im Alltag beeinflussen \parencite{rodo-de-zarate_developing_2014}. Werden diese Aspekte konsequent in die Abfrage integriert, lassen sich feiner aufgeschlüsselte Erkenntnisse zu sozial bedingten Disparitäten im Wohlbefinden gewinnen.

    \item \textbf{Echtzeit- bzw.\ kontextnahe Erhebung:}
    Statt einer einmaligen retrospektiven Befragung soll das \emph{Ecological Momentary Assessment} (EMA) genutzt werden, um das Erleben \emph{in Echtzeit} oder zumindest zeitnah zum Geschehen zu erfassen \parencite{shiffman_ecological_2008, stone_ecological_1994}. Auf diese Weise wird das Risiko verzerrter Erinnerungsleistungen (\emph{Recall Bias}) reduziert. Zudem ermöglicht das EMA-Design, den Einfluss situativer Faktoren unmittelbar zu erkennen, was für dynamische Prozesse – etwa Tageszeiten- oder Ortswechsel – von besonderer Bedeutung ist \parencite{bakolis_urban_2018}.

    \item \textbf{Integrierte Geolokalisierung:}
    Da das Wohlbefinden häufig von räumlichen Kontexten abhängt (z.\,B.\ privat vs.\ öffentlich, urban vs.\ ländlich), sollen Standortdaten (GPS) erhoben werden. Dieser Ansatz kann aufzeigen, in welchen Orten sich Diskriminierung, Sicherheit oder (Un-)Wohlbefinden verstärken bzw.\ abschwächen \parencite{rodo-de-zarate_developing_2014}. Die Verknüpfung von sozialer und räumlicher Dimension trägt somit zu einer umfassenderen Analyse bei, welche die räumliche Verteilung von Erfahrungen erfasst \parencite{bakolis_urban_2018}.

    \item \textbf{Wissenschaftliche Güte:}
    Das Instrument sollte hinsichtlich \emph{Validität}, \emph{Reliabilität} und \emph{Objektivität} hinreichend fundiert sein \parencite{dillman_internet_2014}. Hierzu ist vorgesehen, (a) etablierte Kurzskalen (etwa die Short-Version der Warwick-Edinburgh Mental Wellbeing Scale) zu adaptieren, (b) klar formulierte Items zu entwickeln, die sich in Pretests bewähren, und (c) standardisierte Formate zur Datenerhebung (z.\,B.\ einheitliche App-Interface) zu verwenden \parencite{tennant_warwick-edinburgh_2007, krosnick_question_2009}.

    \item \textbf{Minimierung der Teilnehmerbelastung:}
    Um hohe Rücklauf- und Verbleibquoten zu erzielen, wird auf unnötige Redundanz verzichtet. Daher werden z.\,B.\ demografische Basisdaten nur einmalig beim Erststart erhoben, während die kurzen Wohlbefindens- und Intersektionalitäts-Items in \emph{wiederholten} Befragungen fokussiert abgefragt werden. Dieses Vorgehen orientiert sich an Best Practices der EMA-Forschung, die belegen, dass eine zu hohe Itemanzahl pro Erhebung die Teilnahmebereitschaft mindern kann \parencite[vgl.]{shiffman_ecological_2008}.
\end{enumerate}

\noindent
Die genannten Ziele greifen komplementär ineinander: Die \emph{Kürze} und das \emph{wiederholte} Befragungsdesign (Ziel Z1) unterstützen die \emph{Echtzeit}-Perspektive (Z3) und erzeugen hochfrequente Daten, die sich mit dem \emph{intersektionalen} Anliegen (Z2) verknüpfen lassen. Darüber hinaus erlaubt die Einbeziehung von \emph{Geolokalisierung} (Z4) eine standortbezogene Analyse, sodass sowohl soziale als auch räumliche Strukturen sichtbar werden. Schließlich soll das Instrument hohe \emph{wissenschaftliche Güte} (Z5) vorweisen und die Belastung für Teilnehmende (Z6) so gering wie möglich halten.


\subsection{Theoretische und methodische Grundlage}
Im Kern basiert die Fragebogenkonzeption auf zwei Strängen der Forschung. Erstens wird das \emph{Ecological Momentary Assessment} (EMA) zugrunde gelegt, das bereits in verschiedenen Studien – etwa bei \parencite{shiffman_ecological_2008} oder \parencite{bakolis_urban_2018} – erfolgreich eingesetzt wurde, um subjektives Wohlbefinden und kontextuelle Faktoren in Alltagssituationen kontinuierlich zu erheben. EMA ermöglicht eine zeitnahe und kontextbezogene Messung, die sogenannte \emph{Recall Biases} reduziert und Veränderungen im Befinden in nahezu \emph{Echtzeit} abbildet \parencite{stone_ecological_1994}.

Zweitens wird die Konzeption von \emph{Intersektionalität} berücksichtigt, wie sie unter anderem bei \parencite{crenshaw_mapping_1991} und \parencite{collins_black_2002} ausgeführt wird. Die Annahme ist, dass soziale Kategorien (wie Geschlecht, Ethnizität, Klasse oder sexuelle Orientierung) nicht isoliert betrachtet werden können, sondern sich gegenseitig durchdringen. Im Kontext dieser Arbeit wird darauf Bezug genommen, indem die Teilnehmenden \emph{wiederholt} einschätzen, inwieweit unterschiedliche Kategorien ihr aktuelles Wohlbefinden begünstigen oder einschränken. Zur Visualisierung solcher mehrdimensionaler Daten bieten sich Methoden wie die \emph{Relief Maps} an, um variierende Erfahrungen mit Diskriminierung oder Privilegierung in Abhängigkeit vom Ort sichtbar zu machen \parencite{rodo-de-zarate_developing_2014}.

\subsection{Aufbau und Ablauf des Fragebogens}
Der Fragebogen ist in zwei Hauptteile gegliedert:

\begin{enumerate}[label=(\Alph*)]
  \item \textbf{Einmalige Erhebung von Basisdaten:}  
  Bei der ersten Teilnahme werden demografische und kontextuelle Informationen erfragt, die sich im Regelfall nicht kurzfristig ändern. Dazu zählen:
  \begin{itemize}
    \item Alter (als numerische Angabe).
    \item Geschlechtsidentität (bspw.\ weiblich, männlich, divers / trans / inter* sowie Freitextoption).
    \item Sexuelle Orientierung (beispielsweise hetero, schwul/lesbisch, bi/pan, asexuell, keine Angabe).
    \item Ethnische Zugehörigkeit oder Migrationshintergrund.
    \item Sozioökonomische Lage (etwa Ausbildungsstand, Beruf, Selbsteinschätzung bezüglich Klasse).
  \end{itemize}
  Diese Erfassung wird einmalig beim ersten App-Start durchgeführt, um redundante Abfragen in den späteren Kurzbefragungen zu vermeiden \parencite{dillman_internet_2014}. So bleibt der Zeitaufwand bei wiederholten Erhebungen minimal.

  \item \textbf{Wiederholte Kurzbefragung:}  
  Für die eigentliche EMA-Komponente erhalten die Teilnehmenden mehrmals täglich (beispielsweise zwei- bis viermal) eine kurze Push-Benachrichtigung. Pro Erhebung (Dauer: ca.\ 3--5 Minuten) werden die folgenden inhaltlichen Module durchlaufen:
  \begin{itemize}
    \item \emph{Standort und Situation}: Erfassung des Ortes via GPS (mit Einverständnis) und ggf.\ Auswahloptionen (z.\,B.\ ,,Zuhause'', ,,Draußen'', ,,am Arbeitsplatz'').
    \item \emph{Subjektives Wohlbefinden}: Abfrage über zwei bis drei Kurzskalen-Items, angelehnt an etablierte Instrumente wie die \emph{Short Warwick-Edinburgh Mental Wellbeing Scale (WEMWBS)} \parencite{tennant_warwick-edinburgh_2007} oder ähnliche Verfahren. 
    \item \emph{Intersektionale Aspekte}: Einschätzung, inwieweit Kategorien wie Geschlecht, Alter, soziale Lage, ethnische Zugehörigkeit oder sexuelle Orientierung im aktuellen Kontext das persönliche Wohlbefinden fördern bzw.\ behindern \parencite{collins_black_2002, crenshaw_mapping_1991}. 
    \item \emph{Freitext (optional)}: Möglichkeit, Besonderheiten oder subjektive Eindrücke der Situation zu ergänzen.
  \end{itemize}
\end{enumerate}

Dieser Aufbau zielt darauf ab, das \emph{Zeitbudget} der Teilnehmenden zu schonen und zugleich aussagekräftige Daten in \emph{mehrdimensionaler} Hinsicht zu gewinnen. Die Nutzung der ortsbezogenen Informationen erlaubt zudem eine raumbezogene Datenanalyse, um etwa Unterschiede zwischen öffentlichen und privaten Räumen zu untersuchen \parencite{bakolis_urban_2018, rodo-de-zarate_developing_2014}.

\subsection{Auswahl und Ausgestaltung der Items}
\subsubsection{Skalenlänge und Antwortformate}
Zur Messung psychologischer Konstrukte wie Wohlbefinden oder Stress wird häufig eine 5-Punkt-Likert-Skala verwendet \parencite{likert_technique_1932}. Studien zeigen jedoch, dass bei gerader Itemzahl (z.\,B.\ 4 oder 6 Antwortkategorien) die Tendenz zur mittleren Kategorie verringert wird \parencite{bradburn_asking_2004}. Eine \emph{gerade} Anzahl zwingt Teilnehmende eher zu einer leichten Richtungstendenz, während eine \emph{neutrale} Mittelkategorie (also ungerade, z.\,B.\ 5 oder 7 Stufen) als legitime Antwortoption durchaus sinnvoll sein kann \parencite{krosnick_question_2009}.

In der vorliegenden Arbeit wird eine \emph{5-stufige} Likert-Skala gewählt, da sie in vielen Fragebogenstudien als praktikabel gilt und eine \emph{neutrale} Antwort ermöglicht. Zahlreiche Studien zur Wohlbefindensmessung basieren auf diesem Format, wodurch eine Vergleichbarkeit vereinfacht wird \parencite{tennant_warwick-edinburgh_2007}. Um eine ausreichende Differenzierung zu erhalten, wird zusätzlich erwogen, bei den intersektionalen Items auf eine 6- oder 7-Punkt-Skala zu setzen. Die endgültige Entscheidung wird anhand eines Pretests getroffen (siehe Abschnitt \ref{sec:gütekriterien}).

\subsubsection{Wohlbefinden}
Für das \emph{Wohlbefinden} kommen zwei bis drei Items zum Einsatz, die jeweils unterschiedliche Facetten abdecken. Als Beispiel:
\begin{enumerate}[label=\emph{(WB\arabic*)}]
  \item \emph{Stimmung}: ,,Wie ist die eigene Stimmung \textbf{im Moment}?'' (Skala: 1 = sehr schlecht bis 5 = sehr gut)
  \item \emph{Sicherheit/Respekt}: ,,Ich fühle mich an diesem Ort sicher und respektiert.'' (Skala: 1 = stimme überhaupt nicht zu bis 5 = stimme voll zu)
  \item \emph{Allgemeines Befinden}: ,,In diesem Augenblick habe ich das Gefühl, dass meine Bedürfnisse geachtet werden.'' (1--5)
\end{enumerate}
Diese Items sind angelehnt an Formulierungen aus Kurzversionen der WEMWBS oder ähnlichen validierten Wohlbefindensskalen \parencite{tennant_warwick-edinburgh_2007}.

\subsubsection{Intersektionale Einschätzung}
Angelehnt an \parencite{crenshaw_mapping_1991} und \parencite{rodo-de-zarate_developing_2014} erfolgt eine kompakte Abfrage, in welcher Teilnehmende angeben, wie stark sie sich durch spezifische soziale Kategorien in der jeweiligen Situation unterstützt oder benachteiligt fühlen. Übliche Dimensionen (Geschlecht, soziale Lage, ethnische Zugehörigkeit, sexuelle Orientierung, Alter) können in Form einer Mini-Matrix abgefragt werden. Eine mögliche Formulierung lautet:

\begin{quote}
  \emph{,,In welcher Weise beeinflussen die folgenden Aspekte Ihr momentanes Befinden?''} \\
  \textbf{A:} Geschlechtsidentität \quad
  \textbf{B:} Alter \quad
  \textbf{C:} finanzielle / soziale Lage \quad
  \textbf{D:} Ethnische Herkunft \quad
  \textbf{E:} sexuelle Orientierung \\
  \textit{(Antwortskala 1--5: 1 = gar nicht eingeschränkt / eher unterstützt, 3 = neutral, 5 = stark eingeschränkt.)}
\end{quote}

So entsteht eine mehrdimensionale Einschätzung, die im nächsten Schritt visualisiert werden kann (z.\,B.\ mithilfe der \emph{Relief Maps} nach \cite{rodo-de-zarate_developing_2014}) oder durch statistische Analysen mit dem konkreten Ort (GPS-Koordinaten) verknüpft wird.

\subsection{Wissenschaftliche Gütekriterien und Pretest}
\label{sec:gütekriterien}
Die inhaltliche und formale Gestaltung des Fragebogens orientiert sich an etablierten Qualitätsanforderungen:
\begin{itemize}
    \item \textbf{Objektivität}: Die Items und Instruktionen werden in einer einheitlichen App-Oberfläche präsentiert. Die Daten werden automatisiert erfasst, wodurch Interviewerbias minimiert wird \parencite{dillman_internet_2014}.
    \item \textbf{Reliabilität}: Durch mehrfache \emph{repeated measures} und konsistente Skalen lässt sich eine gewisse Messpräzision erreichen. Unklare oder missverständliche Fragen werden vermieden, indem sie in einem Vorabtest identifiziert und überarbeitet werden \parencite{krosnick_question_2009}.
    \item \textbf{Validität}: Die Konstrukte (z.\,B.\ Wohlbefinden) sind in der Literatur verankert. Durch die Verknüpfung mit bewährten Verfahren (z.\,B.\ \emph{WEMWBS}) wird die Inhalts- und Konstruktvalidität gestärkt \parencite{tennant_warwick-edinburgh_2007}. Zudem gewährleistet das EMA-Design eine hohe \emph{ökologische Validität}, da Befragte ihre Angaben unmittelbar in Alltagssituationen machen \parencite{stone_ecological_1994}.
\end{itemize}

Zur finalen Absicherung der Items erfolgt ein \emph{Pretest} mit einer kleineren Gruppe von Probandinnen und Probanden (ca.\ 5--10 Personen), die den Fragebogen über mehrere Tage hinweg ausprobieren. Deren Rückmeldungen werden in Bezug auf Verständlichkeit, Länge und technische Handhabung ausgewertet. Anschließend werden eventuell zu komplexe oder mehrfach missverstandene Items angepasst \parencite{krosnick_question_2009}.

\subsection{Datenschutz und ethische Aspekte}
Alle Teilnehmenden erhalten zu Studienbeginn ausführliche Informationen über die freiwillige Teilnahme, das jederzeitige Widerrufsrecht und die sichere Datenspeicherung. Da \emph{Ortsdaten} (GPS) und potenziell sensible Angaben (etwa sexuelle Orientierung, ethnische Zugehörigkeit) abgefragt werden, wird besonderes Augenmerk auf die Umsetzung der \emph{Datenschutz-Grundverordnung (DSGVO)} gelegt \parencite{noauthor_verordnung_2016}. Sämtliche personenbezogenen Daten werden pseudonymisiert gespeichert und ausschließlich zu Forschungszwecken verwendet.

\subsection{Zusammenfassung und Ausblick}
Der hier entwickelte Fragebogen stellt ein auf \emph{Ecological Momentary Assessment} basierendes Instrument dar, um intersektionales Wohlbefinden an unterschiedlichen Orten und zu verschiedenen Zeitpunkten zu erfassen. Die gewählte Kombination aus kurzen, wiederkehrenden Skalen zur subjektiven Stimmung sowie einer Mini-Matrix für intersektionale Einflussfaktoren ermöglicht eine dynamische Perspektive auf das Erleben. Durch die Einbeziehung der Geolokalisierung lassen sich ortsabhängige Muster sichtbar machen. Zukünftig besteht die Möglichkeit, mithilfe der gesammelten Daten visuelle Auswertungen vorzunehmen (z.\,B.\ Heatmaps oder Relief Maps), die in der Tradition geografisch informierter Intersektionalitätsforschung stehen \parencite{rodo-de-zarate_developing_2014}.


\section{Appentwicklung}
\subsection{Anforderungskatalog}

Dieser Abschnitt beschreibt detailliert die funktionalen und nicht-funktionalen Anforderungen an die App, die intersektionales Wohlbefinden im Stadtraum erfasst. Ziel der App ist es, mittels ortsbasierter Erhebungen in Echtzeit subjektive Wohlbefindensdaten zu sammeln – unter besonderer Berücksichtigung intersektionaler Merkmale – und dabei möglichst wenig persönliche Daten der Nutzer zu erfassen. Die Anforderungen wurden unter Anwendung etablierter Methoden (vgl. \cite{clegg_case_1994}) ermittelt und in die Kategorien Funktionalität, Sicherheit/Datenschutz, Mehrsprachigkeit sowie Erweiterbarkeit und Wartbarkeit unterteilt.

\paragraph{Funktionale Anforderungen}
\begin{itemize}
    \item \textbf{Gerätebasierte Identifikation und Datenerfassung:}  
    Die App verzichtet bewusst auf eine klassische Anmeldung oder Nutzerkonten. Stattdessen wird jedem Gerät eine eindeutige ID zugewiesen, anhand derer sämtliche Daten erfasst und verwaltet werden. Dieses Vorgehen minimiert die Erhebung personenbezogener Daten, birgt jedoch technische Herausforderungen hinsichtlich der dauerhaften Identifikation und der Datensicherheit. Sollte sich herausstellen, dass die gerätebasierte Identifikation nicht zuverlässig implementiert werden kann, wäre alternativ eine Anmeldefunktion in Erwägung zu ziehen – was allerdings im Widerspruch zum Ziel der minimal-invasiven Datenerfassung stehen würde.

    \item \textbf{Standortbasierte und intersektionale Datenerhebung:}  
    Die Erhebung erfolgt in Echtzeit, indem automatisch der aktuelle Standort des Geräts erfasst wird. Ergänzend dazu wird in einmaligen und wiederkehrenden Erhebungen das subjektive Wohlbefinden abgefragt sowie verschiedene intersektionale Merkmale (z.B. Geschlecht, Herkunft, sozioökonomischer Status) abgefragt, um einen vielschichtigen Blick auf urbane Lebenswelten zu erhalten.

    \item \textbf{Erinnerungsfunktion:}  
    Um eine möglichst repräsentative Datenerhebung zu ermöglichen, ist eine Erinnerungsfunktion implementiert, die den Nutzer dreimal täglich zu zufälligen, variierenden Zeiten (innerhalb definierter Zeitfenster) dazu auffordert, an der Befragung teilzunehmen. Die leichte Randomisierung der Erinnerungszeitpunkte soll sicherstellen, dass Daten aus unterschiedlichen Kontexten und Tageszeiten erfasst werden.

    \item \textbf{Datenselbstverwaltung und Löschfunktion:}  
    Die Nutzer sollen jederzeit in die Lage versetzt werden, ihre Daten selbstständig zu verwalten. Insbesondere ist eine Funktion zur vollständigen Löschung aller auf dem Gerät gespeicherten Daten vorgesehen, um den Prinzipien der Selbstbestimmung und des Datenschutzes gerecht zu werden.
\end{itemize}

\paragraph{Nicht-funktionale Anforderungen}
\begin{itemize}
    \item \textbf{Sicherheit und Datenschutz:}  
    Neben der Minimierung der erhobenen Personendaten (nur gerätebasierte ID) ist es essenziell, die Datensicherheit zu gewährleisten. Technisch sollen Massnahmen wie die Verschlüsselung der übertragenen Daten sowie regelmäßige Sicherheitsupdates implementiert werden. Sollte sich die gerätebasierte Identifikation als technisch unzureichend herausstellen, wäre die Einführung eines Anmeldesystems als zusätzliche Schutzmassnahme zu prüfen – dies erfolgt jedoch nur, wenn der Mehrwert hinsichtlich der Datensicherheit den potenziellen Verlust an Anonymität überwiegt.

    \item \textbf{Mehrsprachigkeit:}  
    Angesichts der intersektionalen Zielsetzung der App wird besonderer Wert auf Mehrsprachigkeit gelegt. Die Benutzeroberfläche soll in mehreren Sprachen verfügbar sein, um eine diverse Nutzergruppe anzusprechen und kulturelle Hürden beim Datenaustausch zu minimieren.

    \item \textbf{Barrierefreiheit:}  
    Obwohl der zeitliche Rahmen der Entwicklung Einschränkungen mit sich bringt, ist die Berücksichtigung von Barrierefreiheit (z.B. intuitive Navigation, Anpassbarkeit der Schriftgrössen, kontrastreiche Darstellung) ein zentrales Anliegen. Die tatsächliche Realisierung von Barrierefreiheitsstandards wird im weiteren Entwicklungsprozess konkret überprüft und angepasst.

    \item \textbf{Plattformübergreifende Kompatibilität und technische Umsetzung:}  
    Die App wird mittels React Native (Expo) entwickelt, was grundsätzlich eine plattformübergreifende Nutzung auf Android- und iOS-Geräten sicherstellt. Diese Entscheidung unterstützt die schnelle Entwicklung und die einfache Wartung der App, da die grundlegende Infrastruktur standardisiert und modular aufgebaut ist.

    \item \textbf{Erweiterbarkeit und Wartbarkeit:}  
    Eine klare Modulstruktur, umfassende Dokumentation sowie standardisierte Schnittstellen sind essenziell, um zukünftige Erweiterungen und Anpassungen zu ermöglichen. Diese Massnahmen gewährleisten, dass die App nicht nur im aktuellen Projektkontext besteht, sondern auch in zukünftigen Forschungs- und Entwicklungsprojekten weiterentwickelt werden kann.
\end{itemize}

\paragraph{Priorisierung der Anforderungen}  
Zur systematischen Priorisierung wurde die MoSCoW-Methode angewandt (vgl. \cite{clegg_case_1994}):
\begin{itemize}
    \item \textbf{Must have:}  
    \begin{itemize}
        \item Gerätebasierte Identifikation und standortbasierte Datenerhebung
        \item Intersektionale Erfassung von Daten zum Wohlbefinden
        \item Erinnerungsfunktion (dreimal täglich, variabel)
        \item Datenselbstverwaltung inklusive Löschfunktion
    \end{itemize}
    \item \textbf{Should have:}  
    \begin{itemize}
        \item Umfassende Mehrsprachigkeit der Benutzeroberfläche
        \item Erweiterte Sicherheitsmassnahmen und optionales Anmeldesystem (falls die gerätebasierte Lösung unzureichend ist)
    \end{itemize}
    \item \textbf{Could have:}  
    \begin{itemize}
        \item Zusätzliche Anpassungen und Optimierungen der Barrierefreiheit
    \end{itemize}
    \item \textbf{Won’t have:}  
    \begin{itemize}
        \item Funktionen zur Datenvisualisierung, da die App ausschliesslich zur Erfassung der Daten konzipiert ist
    \end{itemize}
\end{itemize}

Zusammenfassend adressiert der Anforderungskatalog sowohl die funktionalen Aufgaben der App als auch wesentliche Qualitätsaspekte, die zur Realisierung einer sicheren, nutzerfreundlichen und erweiterbaren Daten-Erhebungsplattform erforderlich sind. Die systematische Dokumentation dieser Anforderungen bildet die Basis für die technische Umsetzung und die spätere Evaluation der Anwendung.



% ----------------- Bibliographie ------------------
\newpage
\PrintBib

\newpage

\section*{Hinweis für den Einsatz von künstlicher Intelligenz (KI)}

Dieses Dokument wurde mithilfe von KI-basierten Tools überarbeitet. LanguageTool, ein KI-gestütztes Grammatik- und Stilprüfungswerkzeug, wurde verwendet, um Formulierungen zu verbessern und die Grammatik zu korrigieren. Chat-GPT von Open-AI wurde verwendet, um Feedback zur Klarheit und Strukturierung des Textes zu erhalten. Es wurde keine KI zur Erstellung von Originalinhalten verwendet.



\end{document}