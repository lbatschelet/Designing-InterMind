\documentclass{template}
\usepackage{array}
\usepackage{graphicx}
\usepackage{float}
\usepackage{csquotes}
\usepackage[acronym, toc]{glossaries}




\SetTitle{Intersektionales Wohlbefinden im Stadtraum: \\
Konzeption und Umsetzung einer App zur räumlichen Erfassung von Wohlbefinden}
\SetAuthor{Lukas Batschelet}
\SetDate{\today}

\makeglossaries
\newacronym{esm}{ESM}{Experience Smpling Method}
\newacronym{ema}{EMA}{Ecological Momentary Assessment}
\newacronym{gema}{GEMA}{Geographically Explicit Ecological Momentary Assessment}
\newacronym{drm}{DRM}{Day Reconstruction Method}
\newacronym{panas}{PANAS}{The Positive \& Negative Affect Schedule}
\newacronym{maihda}{MAIHDA}{Multilevel Analysis of Individual Heterogeneity and Discriminatory Accuracy}
\newacronym{cart}{CART}{Classification and Regression Trees}
\newacronym{uuid}{UUID}{Universally Unique Identifier}
\newacronym{json}{JSON}{JavaScript Object Notation}


\newglossaryentry{race}{
    name={\textit{race}},
    description={Eine im englischsprachigen Raum etablierte, gesellschaftlich konstruierte Kategorie, die rassifizierende Zugehörigkeiten beschreibt. In dieser Arbeit kursiv gesetzt, um ihre soziokulturelle Bedeutung zu betonen und sie vom biologistischen Begriff „Rasse“ abzugrenzen.}
}

\newglossaryentry{gender}{
    name={\textit{gender}},
    description={Bezeichnet die soziale Konstruktion von Geschlecht. Der Begriff verweist auf gesellschaftlich geprägte Vorstellungen und Erwartungen von Geschlechtsidentität. In dieser Arbeit kursiv gesetzt.}
}

\newglossaryentry{schwarz}{
    name={Schwarz},
    description={Politische Selbstbezeichnung von Menschen, die im Kontext rassistischer Machtverhältnisse positioniert werden. Grossgeschrieben zur Abgrenzung von farblichen Zuschreibungen.}
}

\newglossaryentry{intersektionalitaet}{
    name={Intersektionalität},
    description={Analytisches Konzept zur Untersuchung sich überschneidender Machtverhältnisse wie Rassismus, Sexismus, Klassismus etc. Ursprünglich von Kimberlé Crenshaw eingeführt.}
}
\newglossaryentry{class}{
    name={\textit{class}},
    description={Sozial konstruierte Kategorie, die ökonomische und symbolische Ungleichheiten beschreibt. In dieser Arbeit kursiv gesetzt.}
}

% Datei: glossar.tex
\newglossaryentry{reactnative}{
  name={React Native},
  description={Ein Framework zur plattformübergreifenden Entwicklung mobiler Apps. Es erlaubt die Programmierung mit \gls{javascript} oder \gls{typescript}, wobei der Code nativ auf Android- und iOS-Geräten ausgeführt wird.}
}

\newglossaryentry{expo}{
  name={Expo},
  description={Ein Toolchain und Dienst, der die Entwicklung mit \gls{reactnative} vereinfacht. Expo stellt Werkzeuge zum Testen, Debuggen und Veröffentlichen von Apps bereit – ohne dass native Programmierkenntnisse erforderlich sind.}
}

\newglossaryentry{supabase}{
  name={Supabase},
  description={Ein Open-Source-Backend, das als Alternative zu Firebase dient. Es basiert auf einer \gls{datenbank} (PostgreSQL) und bietet Funktionen wie Authentifizierung, Datei-Hosting und \gls{rowlevelsecurity}.}
}

\newglossaryentry{javascript}{
  name={JavaScript},
  description={Eine weit verbreitete Programmiersprache für Webentwicklung, die auch in mobilen Frameworks wie \gls{reactnative} verwendet wird. Sie ist dynamisch und flexibel, aber nicht typensicher.}
}

\newglossaryentry{typescript}{
  name={TypeScript},
  description={Eine von Microsoft entwickelte Programmiersprache, die auf \gls{javascript} basiert, aber zusätzliche statische Typisierung bietet. Sie erhöht die Wartbarkeit und Fehlervermeidung in grösseren Softwareprojekten.}
}

\newglossaryentry{python}{
  name={Python},
  description={Eine interpretierte, einfach lesbare Programmiersprache, die häufig in Wissenschaft, Datenanalyse und Automatisierung eingesetzt wird. Sie wurde auch zur Auswertung der App-Daten verwendet.}
}

\newglossaryentry{uuid}{
  name={UUID},
  description={Abkürzung für Universally Unique Identifier. Eine \textit{UUID} ist eine zufällig generierte Zeichenkette, die zur eindeutigen Identifikation eines Geräts oder Datensatzes dient, ohne personenbezogene Daten zu erfassen.}
}

\newglossaryentry{opensource}{
  name={Open-Source},
  description={Bezeichnet Software, deren Quellcode öffentlich einsehbar, veränderbar und frei verwendbar ist. \gls{supabase} und viele Komponenten von \gls{reactnative} und \gls{expo} sind Open-Source.}
}

\newglossaryentry{datenbank}{
  name={Datenbank},
  description={Ein digitales System zur strukturierten Speicherung, Abfrage und Verwaltung von Daten. In der App kommt eine relationale Datenbank zum Einsatz, die durch \gls{supabase} bereitgestellt wird.}
}

\newglossaryentry{rowlevelsecurity}{
  name={Row-Level Security (RLS)},
  description={Ein feingranulares Zugriffsmodell in einer \gls{datenbank}, das sicherstellt, dass Nutzer:innen nur jene Datenzeilen sehen oder ändern können, für die sie berechtigt sind. \gls{supabase} unterstützt RLS standardmässig.}
}




\AddBibFile{BA_Lukas_Batschelet.bib}


\begin{document}
\pagenumbering{gobble}

\begin{titlepage}
\sffamily
\raggedright

\vspace*{1cm}
\huge
\textbf{Intersektionales Wohlbefinden im Stadtraum}

\vspace{0.5cm}
\large
Konzeption und Umsetzung einer App zur räumlichen Erfassung von Wohlbefinden

\vspace{1.5cm}
\textbf{Lukas Batschelet}

\vspace{0.5cm}
Matrikel-Nr. 16-499-733

\vfill
\normalsize
Bachelorarbeit der Philosophisch-naturwissenschaftlichen Fakultät der Universität Bern

\vspace{0.8cm}
Betreut durch Prof. Dr. Carolin Schurr und Dr. Moritz Gubler


Geographisches Institut \\
Unit für Sozial- und Kulturgeographie\\
Bern, \today

\end{titlepage}



\newpage

\begin{abstract}
\input{Arbeit/chapters/abstract}
\end{abstract}



\newpage

\pagenumbering{roman}

\tableofcontents


\newpage

\printglossary[type=\acronymtype, title=Abkürzungsverzeichnis, toctitle=Abkürzungsverzeichnis]
\listoffigures
\listoftables


\clearpage
\pagenumbering{arabic}
\section{Einleitung}„Städte sind für alle da“ – diese Vorstellung urbaner Gleichheit wird oft in Leitbildern und Planungsstrategien bemüht. Sie knüpft an Debatten an, die auch durch den Anspruch auf das \emph{Recht auf Stadt} nach \textcite{lefebvreDroitVille1967} inspiriert sind, auch wenn Lefebvres ursprüngliches Konzept weitaus radikaler war und eine grundlegende Transformation urbanen Lebens forderte. Unabhängig von der theoretischen Tiefe dieser Forderung stellt sich jedoch die Frage: Wie erleben Menschen den urbanen Raum tatsächlich? Und wie beeinflusst ihre soziale Position – etwa hinsichtlich Alter, Geschlecht, Herkunft oder Gesundheit – ihr momentanes Wohlbefinden in bestimmten Umgebungen? Diese Fragen stehen im Zentrum der vorliegenden Bachelorarbeit, die sich der intersektionalen Analyse des unmittelbaren Wohlbefindens in alltäglichen Lebensräumen widmet.

Methoden zur Erfassung momentaner psychischer Zustände und Erfahrungen im Alltag, wie die \gls{esm} und insbesondere das \gls{ema}, wurden bereits in den 1990er Jahren konzipiert, vor allem in der Psychologie \parencite[vgl.][]{stoneEcologicalMomentaryAssessment1994, shiffmanEcologicalMomentaryAssessment2008}. Sie zielten darauf ab, kontextbezogene Daten zu erheben und Nachteile rein retrospektiver Ansätze zu überwinden \parencite{kahnemanDevelopmentsMeasurementSubjective2006}. Das volle Potenzial dieser Methoden, insbesondere für eine unmittelbare, georeferenzierte Datenerhebung in Echtzeit, entfaltete sich jedoch erst mit der Verbreitung von Smartphones. An der Schnittstelle von Stadtplanung und Psychologie wurden zudem Ansätze zur räumlich expliziten Erfassung von Alltagserfahrungen entwickelt, wie etwa das \gls{gema} \parencite[vgl.][]{kirchnerSpatiotemporalDeterminantsMental2016}. Seit etwa Mitte der 2010er Jahre ist eine deutliche Zunahme an Studien zu beobachten, welche die durch Smartphones erweiterten \gls{ema}/\gls{gema}-Möglichkeiten nutzen, um den Zusammenhang zwischen spezifischen räumlichen Umgebungen und psychischer Gesundheit bzw. Wohlbefinden detailliert zu untersuchen. Ein Beispiel hierfür ist das grossangelegte Projekt Urban Mind: Die Arbeiten von \textcite{bakolisUrbanMindUsing2018}, \textcite{bergouMentalHealthBenefits2022} und \textcite{hammoudSmartphonebasedEcologicalMomentary2024} nutzen diesen Ansatz bzw. dessen Methodik, um insbesondere den Einfluss von Grün- und Stadträumen auf die psychische Gesundheit zu analysieren. Diese Studien prägen den aktuellen Forschungstrend, situative affektive Reaktionen systematisch in Bezug auf räumliche Kontexte zu untersuchen.

Parallel dazu existiert eine umfangreiche Forschungsliteratur zur Intersektionalität und deren räumlichen Implikationen, massgeblich geprägt durch feministische und kritische Perspektiven \parencite[vgl.][]{crenshawMappingMarginsIntersectionality1991, rodo-de-zarateDevelopingGeographiesIntersectionality2014, rodo-de-zarateYoungLesbiansNegotiating2015, rodo-de-zarateIntersectionalityFeministGeographies2018}. Diese Arbeiten verdeutlichen, wie unterschiedliche soziale Kategorien wie Geschlecht, Klasse oder ethnische Zugehörigkeit in räumlichen Kontexten miteinander verwoben sind und Ungleichheiten erzeugen oder verstärken können. Insbesondere methodische Innovationen wie die Relief Maps \parencite{rodo-de-zarateDevelopingGeographiesIntersectionality2014} erlauben eine Visualisierung und Analyse dieser komplexen Wechselwirkungen.

Diese Arbeit verbindet die beiden Perspektiven: Sie nutzt die methodischen Möglichkeiten der smartphone-basierten Echtzeit-Datenerfassung, wie sie in der \gls{esm}/\gls{ema}-Forschung etabliert wurden, verknüpft diese jedoch explizit mit der intersektionalen Ungleichheitsanalyse. Der Fokus verschiebt sich dabei von einer rein 'ökologischen' Betrachtung oder einer engen Definition von 'psychischer Gesundheit' hin zu einer Untersuchung des \emph{situativen affektiven Wohlbefindens} in vielfältigen alltäglichen Umgebungen. Es wird untersucht, wie sich intersektionale Positionierungen konkret auf dieses situative Wohlbefinden auswirken. Ziel dieser Bachelorarbeit ist es, im Rahmen einer explorativen Pilotstudie das Potenzial dieser methodischen Adaption und Verknüpfung auszuloten: Es soll geprüft werden, ob und wie dieser Ansatz an der Schnittstelle von feministischer Sozial- und Kulturgeographie, Intersektionalitätsforschung und der Analyse digital erhobener Alltagsdaten erste Einblicke und Hypothesen generieren kann.

Die persönliche Motivation für diese Arbeit ergibt sich aus dem Wunsch, mit eigens entwickelten digitalen Werkzeugen neue Einblicke in Fragen sozialer Gerechtigkeit und Wohlbefinden im Alltag zu ermöglichen. Perspektivisch könnte der hier erprobte methodische Ansatz in weiterführenden Arbeiten dazu dienen, sozialräumliche Fragestellungen mit Themen wie Klimaanpassung oder -mitigation zu verbinden. Eine solche Verknüpfung könnte beispielsweise für den Berner Kontext relevant sein, etwa für die Forschung zur Stadthitze \parencite[vgl.][]{burgerModellingSpatialPattern2021} und für Projekte wie dem \emph{Bernometer}\footnote{\href{https://bernometer.unibe.ch/bernometer/}{bernometer.unibe.ch}}, die mit detaillierten raumbezogenen Daten zum Wohlbefinden weiter ausgebaut werden könnten.Im Fokus steht dabei folgende Forschungsfrage:

\begin{quote}
\emph{Wie beeinflussen räumliche Umgebungen das momentane Wohlbefinden intersektional positionierter Personen im Alltag?}
\end{quote}

Dabei geht es explizit nicht um langfristige subjektive Wohlbefindenswerte, sondern um die im Alltag erlebten situativen, affektiven Reaktionen. Ziel der Analyse ist es, aus einer intersektionalen Perspektive zu untersuchen, unter welchen Bedingungen und an welchen Orten sich Menschen zugehörig oder fremd fühlen. Es soll also ergründet werden, wie soziale Positionierungen und räumliche Kontexte zusammenwirken und das momentane Gefühl der (Nicht-)Zugehörigkeit beeinflussen. Als analytischer Ansatz zur quantitativen Untersuchung der zugrundeliegenden intersektionalen Muster dient \gls{maihda} nach \textcite{grossModellingIntersectionalityQuantitative2023}.

Zur Erhebung der für diese Arbeit notwendigen Daten wurde das digitale Werkzeug \emph{InterMind}\footnote{\href{https://intermind.ch/app}{intermind.ch/app}} entwickelt. Diese App ermöglicht es, Teilnehmende über einen festgelegten Zeitraum hinweg wiederholt zu befragen und ihre Antworten zusammen mit georeferenzierten Informationen in Echtzeit zu erfassen und anonymisiert zu speichern. Für die vorliegende Untersuchung werden im Rahmen einer Pilotstudie mit Studierenden der Universität Bern erste explorative Daten gesammelt. Die App selbst wurde bewusst Open-Source entwickelt, um eine flexible Anpassung an ähnliche Forschungskontexte zu ermöglichen und potenziell eine nachhaltige Infrastruktur für kontextualisierte Alltagsdaten zu bieten.


Der Aufbau der Arbeit gliedert sich wie folgt: Kapitel 2 erörtert den theoretischen Rahmen hinsichtlich Intersektionalität, Well-Being-Konzepten sowie \gls{ema}-Methoden. Kapitel 3 beschreibt die technische und methodische Umsetzung, insbesondere die Entwicklung der App, das Studiendesign sowie den analytischen Zugriff mittels des \gls{maihda}-Ansatzes. Kapitel 4 präsentiert zentrale empirische Ergebnisse zur Beziehung von Wohlbefinden, Raum und intersektionaler Positionierung. Abschliessend werden diese Ergebnisse in Kapitel 5 diskutiert, kontextualisiert und Implikationen für weitere Forschung aufgezeigt.

Diese Arbeit versteht sich als explorativer Beitrag, der methodische Innovationen mit gesellschaftlich relevanten Fragestellungen verbindet. Sie erhebt nicht den Anspruch auf allgemeine Repräsentativität, sondern zielt darauf ab, erste Hypothesen und methodische Potenziale für zukünftige intersektionale Analysen des momentanen Wohlbefindens in alltäglichen Lebensräumen aufzuzeigen.



\newpage

% LTeX: language=de-CH

\section{Theoretischer Rahmen} \label{sec:theoretischer_rahmen}

\subsection{Intersektionalität in der quantitativen Forschung}

\subsubsection{Definition und Ursprünge der Intersektionalität}

Der Begriff der Intersektionalität wurde ursprünglich von Kimberlé \textcite{crenshawMappingMarginsIntersectionality1991} geprägt und verweist auf die Überlagerung und wechselseitige Verstärkung unterschiedlicher Formen von Diskriminierung, insbesondere im Kontext von \emph{race} und \emph{gender} \parencite{hancockWhenMultiplicationDoesnt2007}. Ausgangspunkt dieser theoretischen Perspektive ist die Black Feminist Theory, welche unter anderen in den Arbeiten von Crenshaw sowie Patricia Hill \textcite{collinsBlackFeministThought2002}, Audre Lorde und bell hooks ihren Ausdruck findet. Black Feminist Theory formulierte eine scharfe Kritik an traditionellen feministischen Ansätzen, denen vorgeworfen wurde, primär die Erfahrungen weisser, privilegierter Frauen ins Zentrum zu stellen und somit die Lebensrealitäten Schwarzer Frauen zu marginalisieren \parencite{collinsBlackFeministThought2002}. Kimberlé Crenshaw entwickelte das Konzept der Intersektionalität explizit als Reaktion auf die Unfähigkeit bestehender theoretischer Ansätze, die spezifischen Diskriminierungserfahrungen Schwarzer Frauen adäquat zu erfassen. Dabei verdeutlichte sie, dass Diskriminierung nicht als Summe einzelner, isolierter Erfahrungen verstanden werden könne, sondern als eigenständige Form sozialer Benachteiligung, die sich an der Überschneidung sozialer Kategorien wie \emph{race}, \emph{gender} und Klasse manifestiert \parencite{crenshawMappingMarginsIntersectionality1991}.

Intersektionalität entwickelte sich somit nicht allein im akademischen Kontext, sondern ist stark verwurzelt in den politischen Kämpfen sozialer Bewegungen, insbesondere im Kontext feministischer, antirassistischer und antikapitalistischer Aktivismen der 1970er- und 1980er-Jahre \parencite{collinsBlackFeministThought2002}. Zentral für die theoretische Grundlage des intersektionalen Ansatzes ist die Anerkennung von Machtverhältnissen und sozialen Ungleichheiten als strukturell verankert und historisch bedingt. Gesellschaftliche Positionierungen wie \emph{gender}, \emph{race} oder soziale Klasse werden hierbei als sozial konstruierte Kategorien verstanden, die immer in Verbindung mit bestehenden Machtsystemen wie Sexismus, Rassismus oder Klassismus betrachtet werden müssen. Audre Lorde und bell hooks betonten insbesondere die Rolle struktureller Unterdrückung und verdeutlichten, wie sich dominante Gesellschaftsstrukturen auf individueller Ebene reproduzieren und sich somit wechselseitig verstärken \parencite{collinsBlackFeministThought2002, hancockWhenMultiplicationDoesnt2007}.

Von der ursprünglich starken Fokussierung auf \emph{race} und \emph{gender} wurde das Konzept der Intersektionalität in den folgenden Jahrzehnten zunehmend erweitert und schliesst heute eine Vielzahl sozialer Positionierungen und Identitäten ein, darunter etwa Sexualität, Alter, Behinderung, Nationalität oder Religion \parencite{bauerIntersectionalityQuantitativeResearch2021, bowlegInvitedReflectionQuantifying2016}. Diese Erweiterung verdeutlicht die breite theoretische und empirische Anwendbarkeit von Intersektionalität als Analyseinstrument zur kritischen Untersuchung gesellschaftlicher Ungleichheiten und Diskriminierungserfahrungen. Intersektionalität hat sich somit nicht nur als theoretisches Konzept, sondern auch als methodische Grundlage etabliert, welche insbesondere in feministischen, sozialwissenschaftlichen und zunehmend auch in quantitativ orientierten Diskursen verwendet wird, um die komplexen Wechselwirkungen gesellschaftlicher Macht


\subsubsection{Quantitative Ansätze und ihre Herausforderungen}

Quantitative Forschungsmethoden gewinnen in der intersektionalen Forschung zunehmend an Bedeutung, wobei unterschiedliche theoretische und methodische Ansätze verfolgt werden \parencite{bauerIntersectionalityQuantitativeResearch2021}. Quantitative Verfahren bieten das Potenzial, systematische Strukturen und Muster von Interaktionen zwischen sozialen Kategorien empirisch sichtbar und statistisch überprüfbar zu machen. Gleichzeitig ist jedoch die methodische Umsetzung intersektionaler Analysen mit erheblichen Herausforderungen verbunden.

Ein grundlegendes Spannungsfeld ergibt sich aus der Integration der theoretischen Prämissen der Intersektionalität mit den technischen Anforderungen quantitativer Analysen. So kritisieren \textcite{hancockWhenMultiplicationDoesnt2007} eindimensionale und additive statistische Modelle, welche soziale Kategorien als unabhängige Variablen betrachten und lediglich deren einzelne Haupteffekte untersuchen. Solche Modelle laufen Gefahr, die Kernannahme der Intersektionalität, wonach soziale Kategorien stets miteinander verschränkt sind, unzureichend abzubilden \parencite{bowlegInvitedReflectionQuantifying2016, bauerIntersectionalityQuantitativeResearch2021}. In der Praxis wird Intersektionalität häufig auf einfache Interaktionseffekte in Regressionsmodellen reduziert, was die Gefahr einer Fehlinterpretation oder Vereinfachung komplexer sozialer Realitäten birgt \parencite{bauerIntersectionalityQuantitativeResearch2021, scottIntersectionalityQuantitativeMethods2017}.

Um diesen Herausforderungen zu begegnen, wurden unterschiedliche methodische Ansätze entwickelt. Ein vielversprechender Ansatz ist die \acrfull{maihda}, welche es erlaubt, intersektionale Effekte differenziert abzubilden, indem sie systematisch Varianzen innerhalb und zwischen sozialen Positionen quantifiziert \parencite{grossModellingIntersectionalityQuantitative2023}. \gls{maihda} bietet insbesondere die Möglichkeit, eine grosse Anzahl sozialer Positionierungen gleichzeitig zu betrachten, ohne diese auf blosse Interaktionsterme in klassischen Regressionsmodellen zu reduzieren \parencite{bauerIntersectionalityQuantitativeResearch2021}. Eine weitere Perspektive eröffnen sogenannte \emph{Decision-Tree}-Verfahren wie Klassifikations- und Regressionsbäume (\acrshort{cart}), welche explorativ heterogene Muster innerhalb intersektionaler Gruppen offenlegen können. Diese Verfahren sind jedoch durch ihre Datenabhängigkeit und eingeschränkte Reproduzierbarkeit begrenzt \parencite{bauerIntersectionalityQuantitativeResearch2021}.

Ein zentraler methodologischer Diskussionspunkt betrifft zudem die Wahl zwischen einer sogenannten \emph{intercategorical} und einer \emph{intracategorical} Herangehensweise, ein Konzept, welches ursprünglich von \textcite{mccallComplexityIntersectionality2005} eingeführt wurde. Während intercategorical-Analysen Unterschiede zwischen verschiedenen sozialen Gruppen vergleichen, fokussieren intracategorical-Ansätze auf die Analyse von Heterogenität und Prozessen innerhalb einer spezifischen Gruppe oder eines spezifischen sozialen Schnittpunkts \parencite{bauerAdvancingQuantitativeIntersectionality2019}. Die Wahl der jeweiligen Herangehensweise beeinflusst massgeblich die Operationalisierung intersektionaler Kategorien sowie die darauf basierenden statistischen Methoden und Auswertungen.

Eine zusätzliche Herausforderung stellt die kontextsensitive Operationalisierung und Messung intersektionaler Kategorien dar. Hierfür reicht eine blosse Festlegung statischer sozialer Gruppen nicht aus, da soziale Kategorien, wie etwa \emph{race} oder \emph{gender}, stets kontextabhängig und multidimensional konstruiert werden. \textcite{rodo-de-zarateDevelopingGeographiesIntersectionality2014, rodo-de-zarateYoungLesbiansNegotiating2015} argumentieren, dass quantitative Verfahren entsprechend flexibilisiert und angepasst werden müssen, um der Dynamik und Fluidität sozialer Identitäten gerecht zu werden. Dies erfordert eine hohe theoretische Reflexivität und methodologische Sensibilität, insbesondere bezüglich der Validität verwendeter Messinstrumente sowie der Interpretation statistischer Ergebnisse \parencite{bauerIntersectionalityQuantitativeResearch2021, websterCenteringSocialtechnicalRelations2021}.

Trotz der genannten Herausforderungen bieten quantitative Verfahren jedoch bedeutende Chancen für die intersektionale Forschung. Sie ermöglichen es, sozialstrukturelle Ungleichheiten empirisch sichtbar zu machen, grössere Stichproben systematisch zu untersuchen und somit evidenzbasierte Handlungsempfehlungen abzuleiten. Um quantitative Methoden adäquat für intersektionale Analysen nutzen zu können, ist es jedoch zwingend erforderlich, methodische Innovationen aktiv weiterzuentwickeln sowie eine kritische und reflektierte Anwendung der verfügbaren statistischen Verfahren sicherzustellen \parencite{bauerIntersectionalityQuantitativeResearch2021, bauerAdvancingQuantitativeIntersectionality2019, scottIntersectionalityQuantitativeMethods2017}.


\subsection{Räumliche Umgebung und momentanes Wohlbefinden}

\subsubsection{Umweltfaktoren und deren Einfluss auf Wohlbefinden}

Die unmittelbare räumliche Umgebung beeinflusst das menschliche Wohlbefinden erheblich. Die Bedeutung der Umgebung für psychisches und emotionales Wohlbefinden wird zunehmend sowohl in der Geographie als auch in der Psychologie und Stadtplanung untersucht. Dabei spielen Faktoren wie natürliche Umgebung, urbane Infrastruktur, soziale Interaktionen und die subjektive Wahrnehmung dieser Faktoren eine zentrale Rolle \parencite{birenboimInfluenceUrbanEnvironments2018, hammoudSmartphonebasedEcologicalMomentary2024, chenPerceivedUrbanEnvironment2025}.

\textcite{birenboimInfluenceUrbanEnvironments2018} untersuchte in einer umfassenden Studie mit Hilfe einer \acrfull{esm}, wie verschiedene urbane Kontexte (z.\,B. Parkanlagen, Verkehrsinfrastrukturen und öffentliche Plätze) subjektive momentane Erfahrungen wie Sicherheit, Komfort und Freude beeinflussen. Auf Grundlage von über 5000 individuellen Befragungen zeigt sich, dass Umweltfaktoren unmittelbar und signifikant das Wohlbefinden im Alltag prägen. Vor allem Merkmale wie räumliche Charakteristika (z.\,B. öffentliche versus private Räume), die Anwesenheit anderer Menschen sowie die Art der Aktivität beeinflussen das momentane Erleben erheblich. Überraschenderweise wurden stabile Persönlichkeitsfaktoren hingegen als kaum relevant für die Variabilität momentanen Wohlbefindens identifiziert \parencite{birenboimInfluenceUrbanEnvironments2018}. Diese Ergebnisse unterstreichen die Bedeutung der spezifischen situativen und räum

Ähnliche Befunde liefert eine Studie von \textcite{mascherekMeadowsAsphaltRoad2025}, die mit einer geografischen ökologischen Momentaufnahme (\acrfull{gema}) arbeitete. In dieser Studie wurden Daten aus drei deutschen Metropolregionen analysiert, wobei sich zeigte, dass Wetterbedingungen (insbesondere Sonnenschein), soziale Begleitung und Mobilität bedeutend stärkere Effekte auf momentanes affektives Wohlbefinden hatten als die blosse Verfügbarkeit urbaner Grünflächen. Diese Ergebnisse weisen darauf hin, dass nicht nur die physischen Merkmale eines Ortes, sondern auch kontextuelle und soziale Faktoren die momentane Stimmung und das Wohlbefinden entscheidend prägen \parencite{mascherekMeadowsAsphaltRoad2025}.

Ebenso konnte eine Untersuchung von \textcite{hammoudSmartphonebasedEcologicalMomentary2024} mittels Smartphone-basierter \gls{esm} nachweisen, dass insbesondere die Vielfalt natürlicher Elemente wie Pflanzen, Bäume und Tierwelt mit höheren Werten mentalen Wohlbefindens verbunden ist. Diese Studie betont explizit die Bedeutung ökologischer Diversität für die öffentliche psychische Gesundheit und hebt die Notwendigkeit hervor, biodiversitätsreiche Orte in städtischen Umgebungen zu fördern, um das Wohlbefinden der Bevölkerung nachhaltig zu verbessern.

\textcite{chenPerceivedUrbanEnvironment2025} bestätigen diese Perspektive und ergänzen, dass bestimmte urbane Infrastrukturen wie Cafés, Kultur- und Bildungseinrichtungen sowie öffentliche Räume langfristiges Wohlbefinden fördern, wohingegen natürliche und ruhige Umgebungen vor allem mit kurzfristigem, momentanen Wohlbefinden positiv korrelieren. Umgekehrt wirkten sich belebte und bewegungsintensive urbane Kontexte negativ auf momentanes Wohlbefinden aus. Diese Studie hebt die Komplexität der räumlichen Einflussfaktoren hervor und fordert eine differenzierte Betrachtung der Wechselwirkungen zwischen räumlichen Merkmalen und psychischem Wohlbefinden.

Insgesamt zeigt sich in der Literatur, dass die räumliche Umgebung unmittelbar und deutlich Einfluss auf das momentane Wohlbefinden ausübt, wobei natürliche, soziale und infrastrukturelle Charakteristika der Umgebung besonders bedeutsam sind. Methodische Fortschritte wie \gls{esm} und \gls{gema} erlauben eine immer präzisere Erfassung dieser räumlich-temporalen Dynamiken, die für eine ganzheitliche Betrachtung von Wohlbefinden und Stadtentwicklung unverzichtbar sind \parencite{kirchnerSpatiotemporalDeterminantsMental2016, cookeMeasuringWellBeingReview2016}.

\subsubsection{Bedeutung intersektionaler Perspektiven auf räumliches Wohlbefinden}

Während die Bedeutung der räumlichen Umgebung für das Wohlbefinden inzwischen gut dokumentiert ist, besteht weiterhin Forschungsbedarf hinsichtlich intersektionaler Perspektiven, die individuelle Erfahrungen in ihren sozial differenzierten Kontexten analysieren. Bisherige Untersuchungen räumlicher Einflüsse auf Wohlbefinden berücksichtigen häufig nicht ausreichend, dass Individuen aufgrund ihrer sozialen Positionierungen wie Gender, Ethnizität, sozioökonomischem Status oder Alter sehr unterschiedliche Erfahrungen in denselben räumlichen Kontexten machen können \parencite{rodo-de-zarateDevelopingGeographiesIntersectionality2014, rodo-de-zarateIntersectionalityFeministGeographies2018}.

Intersektionale Ansätze ermöglichen es, nicht nur räumliche, sondern auch soziale Differenzierungen präzise zu erfassen und so differenzierte Aussagen darüber zu treffen, wie räumliche Umgebungen je nach sozialer Positionierung unterschiedlich wahrgenommen werden. Beispielsweise zeigen Studien zu jungen lesbischen Frauen, dass öffentliche Räume nicht generell als sicher oder unsicher empfunden werden, sondern diese Empfindungen stark von der Überschneidung individueller sozialer Positionierungen und der räumlichen Umgebung abhängen \parencite{rodo-de-zarateYoungLesbiansNegotiating2015}. Räumliche Orte können somit gleichzeitig als befreiend und bedrohlich wahrgenommen werden, je nachdem, wie sich spezifische soziale Identitäten mit räumlichen Kontexten verschränken.

Auch hinsichtlich der psychischen Gesundheit und des Wohlbefindens verdeutlicht eine intersektionale Perspektive, dass Effekte der räumlichen Umgebung nicht universell sind, sondern von sozialen Machtverhältnissen abhängen, welche sich in bestimmten Orten materialisieren. Intersektionale Forschung betont, dass soziale Ungleichheiten und Diskriminierungen sich räumlich manifestieren und dadurch spezifische Belastungen und Herausforderungen für marginalisierte Gruppen entstehen \parencite{websterCenteringSocialtechnicalRelations2021}. Ein solcher Ansatz erweitert die räumliche Wohlbefindensforschung um eine kritische Reflexion sozialer Ungleichheiten und trägt somit dazu bei, eine differenziertere und gesellschaftlich relevantere Forschungsperspektive zu entwickeln.

Zusammenfassend lässt sich festhalten, dass intersektionale Perspektiven essenzielle Ergänzungen zur Erforschung räumlicher Einflüsse auf das Wohlbefinden darstellen. Durch die Verbindung von räumlichen und sozialen Differenzierungen können spezifische Erfahrungen und Bedürfnisse unterschiedlicher Bevölkerungsgruppen sichtbar gemacht und in der Gestaltung sozial gerechter und inklusiver Städte berücksichtigt werden.



% LTeX: language=de-CH

\section{Methodisches Vorgehen} \label{sec:methodik}

Die vorliegende Arbeit verfolgt ein methodenentwickelndes Ziel: Im Zentrum steht die Konzeption, Umsetzung und Evaluation eines digitalen Erhebungsinstruments zur Analyse des affektiv situativen Wohlbefindens. Dafür wurde eine eigene Smartphone-App (\textit{InterMind}) entwickelt, welche wiederholte, kontextsensitive Befragungen des momentanen Wohlbefindens im Alltag ermöglicht. Methodologisch basiert der Ansatz auf dem Prinzip der \acrfull{esm} sowie deren räumlicher Erweiterung als \acrfull{gema}.

Die entwickelten Instrumente (App und Fragebogen) dienen der Erfassung affektiver Zustände im situativen Kontext, insbesondere mit Blick auf räumliche Umweltfaktoren und soziale Positionierungen. Die so erhobenen Daten werden anschliessend mit Hilfe von \acrfull{maihda} ausgewertet, um intersektionale Effekte auf das subjektive Wohlbefinden modellieren zu können.

Das folgende Kapitel beschreibt zunächst den methodischen Gesamtansatz und begründet die Entscheidung für \acrshort{esm}/\acrshort{gema}. Anschliessend wird die Entwicklung der App dokumentiert, die Durchführung der Datenerhebung beschrieben und die Operationalisierung des Fragebogens erläutert. Abschliessend werden Limitationen des Studiendesigns kritisch reflektiert.


\subsection{Wiederholte Befragung mit \acrshort{esm}, \acrshort{ema} und \acrshort{gema}}

Die systematische Erhebung von momentanen Wohlbefindenszuständen erfordert Methoden, die subjektive Erfahrungen möglichst unmittelbar und kontextspezifisch erfassen. Retrospektive Selbstauskünfte sind hierfür nur begrenzt geeignet, da sie Verzerrungen durch selektive Erinnerung oder nachträgliche Neubewertung unterliegen (\textit{Recall Bias}) \parencite{kahnemanDevelopmentsMeasurementSubjective2006}. Um solche Verzerrungen zu vermeiden, wurde bereits in den 1980er-Jahren die \acrfull{esm} entwickelt. Dieses Verfahren basiert auf der mehrfach wiederholten Erhebung subjektiver Zustände im Alltag – etwa durch zufällig verteilte Signale, die Teilnehmende dazu auffordern, ihre momentane Stimmung, Tätigkeit oder Umgebung zu protokollieren \parencite{csikszentmihalyiValidityReliabilityExperienceSampling1987}. Ziel ist es, das Erleben möglichst nah am Zeitpunkt der Erfahrung und im natürlichen Kontext zu erfassen.

Während \acrshort{esm} ursprünglich als psychologisches Messinstrument konzipiert war, wurde der Ansatz in den 1990er-Jahren durch das Konzept der \acrfull{ema} methodologisch erweitert. \acrshort{ema} bezeichnet nicht nur die unmittelbare Erhebung subjektiven Erlebens, sondern schliesst auch physiologische, verhaltensbezogene oder kontextuelle Daten mit ein – etwa über mobile Geräte, Sensorik oder Tagebuchsysteme \parencite{shiffmanEcologicalMomentaryAssessment2008}. Der Begriff „ökologisch“ verweist hierbei nicht auf natürliche Umwelt, sondern auf den Anspruch, Erleben und Verhalten im realweltlichen Lebenskontext zu erfassen – also dort, wo es tatsächlich stattfindet.

Mit der zunehmenden Verbreitung von GPS-fähigen Endgeräten wurde \acrshort{ema} in den 2010er-Jahren durch das Konzept der \acrfull{gema} ergänzt. \acrshort{gema} kombiniert die subjektive Momentaufnahme mit objektiven, räumlich verortbaren Kontextinformationen wie Standort, Wetter, Lärm oder Bebauungsstruktur \parencite{kirchnerSpatiotemporalDeterminantsMental2016}. Im Unterschied zu \acrshort{ema} liegt der Fokus hier auf der systematischen räumlichen Verknüpfung: Subjektive Erfahrungen werden nicht nur als situativ, sondern explizit als räumlich situiert begriffen. Entscheidend ist dabei nicht die Art der Umgebung – also ob es sich etwa um Grünflächen, urbane Plätze oder Transiträume handelt –, sondern die Möglichkeit, affektives Erleben in seiner Beziehung zum jeweils spezifischen räumlich-materiellen Kontext zu analysieren.

Die vorliegende Arbeit folgt diesem methodischen Paradigma. Ziel ist es, situativ affektive Zustände im Raum nicht nur als individuelle, sondern als kontextuell-räumlich bedingte Erfahrungen zu erfassen. Zu diesem Zweck wurde eine eigene Smartphone-Applikation (\textit{InterMind}) entwickelt, die Teilnehmende mehrmals täglich dazu auffordert, eine kurze Selbsteinschätzung ihres momentanen Wohlbefindens und ihrer Umgebungvorzunehmen. Gleichzeitig werden automatisiert Geodaten gespeichert, sodass jede Beobachtung in ihrer konkreten räumlichen Verortung analysiert werden kann. Im Unterschied zu vielen bestehenden GEMA-Studien liegt der Fokus dabei nicht auf spezifischen Umweltmerkmalen wie Vegetationsanteil oder Luftqualität, sondern auf der relationalen Analyse von Raum und subjektivem Erleben.

Die Entscheidung für ein solches Studiendesign bringt gegenüber querschnittbasierten Verfahren mehrere methodische Vorteile mit sich. Erstens reduziert die wiederholte intraindividuelle Erhebung Verzerrungen durch retrospektive Einschätzungen und erlaubt eine präzisere Erfassung situativer Schwankungen. Zweitens ermöglicht sie eine Kontrolle individueller Basisniveaus, was insbesondere für intersektionale Analysen relevant ist, die sowohl zwischen als auch innerhalb von Personen Differenzierungen vornehmen. Drittens erlaubt die Kombination von Echtzeitbefragung und Geodatenanalyse eine kontextsensitive Modellierung der Beziehungen zwischen affektivem Zustand und Umgebung – im Sinne eines relationalen, ökologisch verstandenen Raumbegriffs \parencite{mascherekMeadowsAsphaltRoad2025}.

Für die Erfassung affektiver Zustände wurden numerische Skalen (Slider) eingesetzt als auch Single- und Multiple-Choice-Fragen, die sich in bisherigen Studien als verlässlich und teilnehmendenfreundlich erwiesen haben \parencite{cookeMeasuringWellBeingReview2016}. Die räumliche Verortung erfolgte einerseits über Single- und Multiple-Choice-Fragen, die sich auf die Umgebung des Teilnehmenden beziehen, und andererseits über die Standortdaten der Smartphone-App, wodurch sich subjektive Einschätzungen und objektive Kontextdaten präzise miteinander verknüpfen lassen. Die methodische Grundlage dieser Studie lässt sich somit als eine kritische Anwendung von \acrshort{gema} verstehen, die affektives Wohlbefinden nicht als isolierte Innenwelt, sondern als kontextgebundenes Erleben in Wechselbeziehung von Raum, Situation und sozialer Positionierung begreift.

\subsection{Methodenentwickelnder Charakter und illustrativer Testdurchlauf} \label{sec:methodenentwickelnd}

Diese Arbeit ist als methodenentwickelnde Studie konzipiert. Im Zentrum steht die Entwicklung eines digitalen Erhebungsinstruments, das die situative Erfassung von affektivem Wohlbefinden mit einer intersektionalen Analyse verknüpft. Ziel ist es, einen vollständigen methodischen Workflow zu entwerfen – bestehend aus einem spezifisch konzipierten Fragebogen, einer Smartphone-Applikation zur standortbezogenen Datenerhebung sowie einer vorbereiteten Analysestruktur für für eine intersektionale Modellierung.

Im Unterschied zu klassischen empirischen Studien liegt der Fokus auf der konzeptionellen und technischen Umsetzbarkeit des Ansatzes. Die wenigen im Rahmen der Pilotstudie erhobenen Daten dienen ausschliesslich der Erprobung und exemplarischen Durchführung des methodischen Prozesses – sie erlauben aufgrund der geringen Stichprobengrösse keine  Aussagen über Zusammenhänge zwischen Umgebung, intersektionaler Positionierung und Wohlbefinden.

Die Durchführung einer \acrshort{maihda}-Analyse erfolgt demnach lediglich zu illustrativen Zwecken. Sie diente dazu, die Struktur des Modells zu testen, die Anforderungen an die Datenqualität und -quantität zu reflektieren und das methodische Zusammenspiel von Erhebungsdesign und Analyseansatz zu überprüfen. Auch andere Auswertungsschritte – etwa deskriptive Statistiken oder Visualisierungen – verfolgen keine analytische Zielsetzung im engeren Sinn, sondern dienen der Überprüfung der Funktionsfähigkeit des entwickelten Instruments.

Die methodische Reflexion dieser exemplarischen Anwendung bildet einen zentralen Teil der Arbeit. Sie erlaubt erste Einschätzungen dazu, welche praktischen, technischen oder konzeptionellen Herausforderungen bei der Umsetzung auftreten und wo Anpassungen für künftige Studien notwendig wären. Der wissenschaftliche Mehrwert der Arbeit liegt entsprechend nicht in empirischen Erkenntnissen, sondern in der Bereitstellung und kritischen Diskussion eines erprobten methodischen Zugangs, der für zukünftige Forschungsvorhaben adaptiert und weiterentwickelt werden kann.


\subsection{Vergleich mit bestehenden Erhebungsinstrumenten}

Die im Rahmen dieser Arbeit entwickelte App bewegt sich im Spannungsfeld zweier methodischer Herangehensweisen: der Echtzeiterhebung räumlich kontextualisierter affektiver Zustände (wie bei \textit{Urban Mind}) und der explizit intersektionalen Analyse subjektiver Raumwahrnehmungen (wie bei \textit{Relief Maps+}). Beide bestehenden Instrumente bilden wichtige Referenzpunkte, da sie jeweils zentrale Teilaspekte des hier verfolgten Ansatzes adressieren, jedoch keine vollständige Integration beider Perspektiven vornehmen. Der folgende Vergleich dient dazu, methodische Gemeinsamkeiten und Unterschiede herauszuarbeiten.

\subsubsection{Urban Mind: \acrshort{gema} ohne intersektionale Perspektive}

Das \textit{\gls{urbanmind}}-Projekt\footnote{Siehe \href{https://www.urbanmind.info/}{urbanmind.info}} stellt ein beispielhaftes Werkzeug dar, um subjektives momentanes Wohlbefinden in städtischen Kontexten mittels Echtzeiterhebungen systematisch zu erfassen und zu analysieren \parencite{bakolisUrbanMindUsing2018}. Es basiert auf einer mobilen Smartphone-App, die mithilfe von \acrshort{gema} detaillierte Einblicke in den Zusammenhang zwischen unmittelbaren Umweltfaktoren und psychischer Gesundheit ermöglicht.

Zentrales Anliegen des \textit{Urban Mind}-Tools ist es, die Effekte spezifischer natürlicherElemente der Umgebung, wie beispielsweise Bäume, Himmel, Wasser oder Vogelgesang, auf die psychische Gesundheit in Echtzeit zu untersuchen. Hierfür werden Proband\genderstern innen mehrmals täglich über einen Zeitraum von zwei Wochen aufgefordert, kurze standardisierte Fragen zu ihrer aktuellen Umgebung und ihrem momentanen Wohlbefinden zu beantworten \parencite{bakolisUrbanMindUsing2018}. Die Datenerhebung erfolgt sowohl mittels Selbsteinschätzungen der räumlichen und sozialen Umgebung als auch über Geodaten, welche automatisiert die exakte räumliche Verortung der Teilnehmer\genderstern innen ermöglichen.

\begin{figure}[h]
    \centering
    \includegraphics[width=0.3\textwidth]{Arbeit/images/urban_mind01.jpeg}
    \caption{Screenshot einer typischen Frageseite aus der Urban Mind-App}
    \label{fig:urban_mind_screenshot_1}
\end{figure}

Im Gegensatz zu traditionellen querschnittlichen Designs erlaubt das Urban Mind-Tool explizit die Analyse unmittelbarer und zeitverzögerter Effekte (Lag-Effekte). So konnten beispielsweise signifikant positive Effekte von natürlichen Elementen wie Vogelgesang oder dem Vorhandensein von Bäumen auf das momentane Wohlbefinden nachgewiesen werden, welche auch mehrere Stunden nach dem eigentlichen Kontakt noch messbar waren \parencite{bakolisUrbanMindUsing2018}. Darüber hinaus betont das Tool die Bedeutung individueller Differenzen und psychologischer Charakteristika, wie beispielsweise Impulsivität, die sich als moderierende Variable herausstellte: Personen mit höherer Impulsivität, welche typischerweise ein erhöhtes Risiko für psychische Erkrankungen aufweisen, profitieren stärker von unmittelbaren Naturerfahrungen.

Hinsichtlich des Designs und der Bedienbarkeit überzeugt die Urban Mind-App durch eine intuitive grafische Gestaltung (siehe \Cref{fig:urban_mind_screenshot_1}) sowie durch motivierende Elemente wie eine visuelle Übersicht über ausgefüllte und verpasste Fragebögen. Zudem ermöglicht sie Nutzer\genderstern innen, ihre eigenen Daten retrospektiv aufzubereiten, was zu einer angeleiteten Reflexion des eigenen Wohlbefindens beiträgt. Dieses Feature unterstützt insbesondere eine nachhaltige und motivierte Teilnahme über den gesamten Erhebungszeitraum hinweg.

Obwohl Urban Mind zahlreiche methodische und technische Stärken aufweist, berücksichtigt es intersektionale Perspektiven bisher nicht explizit. So sind beispielsweise soziale Kategorien wie Geschlecht, Ethnizität oder sozioökonomischer Status zwar als demografische Variablen erfasst, werden jedoch nicht systematisch in einer intersektionalen Analyse miteinander in Beziehung gesetzt. Theoretisch wäre es möglich, intersektionale Analysen retrospektiv auf Grundlage der erhobenen Daten durchzuführen, eine solche methodische Perspektive wurde jedoch bislang nicht verfolgt.

\subsubsection{Relief Maps+: Reflexive und intersektionale Kartierung retrospektiver Erfahrungen}

Im Unterschied zu Echtzeit-Tools wie „Urban Mind“, die affektives Wohlbefinden situativ-erlebensnah quantifizieren, verfolgt \textit{\gls{reliefmaps}}\footnote{Siehe \href{https://reliefmaps.upf.edu/}{reliefmaps.upf.edu}} einen qualitativ-reflexiven Ansatz, der retrospektiv subjektive Erfahrungen intersektional positioniert sichtbar macht \parencite{rodo-de-zarateDevelopingGeographiesIntersectionality2014}. Aufbauend auf der ursprünglichen Version der „Relief Maps“ integriert die digitale Anwendung drei miteinander verschränkte Dimensionen – geografische Orte, soziale Identitäten und emotionale Bewertungen – und legt dabei besonderen Wert auf die Förderung individueller Selbstreflexion und kollektiver Sichtbarmachung diskriminierender Raumstrukturen.

Zu Beginn des Erhebungsprozesses erstellen Nutzer\genderstern innen einen Avatar auf Basis intersektional relevanter Merkmale wie Geschlecht, Sexualität, Klasse, Herkunft, Körperbild oder (Dis-)Ability. Darauf aufbauend reflektieren sie in mehreren Schritten über emotionale Erfahrungen in verschiedenen Raumkategorien wie „öffentliche Räume“, „Gesundheitseinrichtungen“ oder „virtuelle Räume“ (siehe \cref{fig:relief_maps_plus_screenshot_1}). Für jede Achse sozialer Positionierung können in einem nächsten Schritt Orte je nach erfahrenem (Un-)Wohlsein als unterdrückend, kontrovers, neutral oder entlastend klassifiziert werden. Ergänzend können Orte direkt auf einer Karte verortet und mit freien Kommentaren sowie Emotionslabels wie „Angst“, „Sicherheit“ oder „Empowerment“ versehen werden. Diese Funktion fördert eine dichte, kontextualisierte Beschreibung subjektiver Erlebnisse, die sich nicht auf standardisierte Itemskalen reduzieren lässt.

\begin{figure}[htbp]
    \centering
    \includegraphics[width=\textwidth]{Arbeit/images/reliefmap.png}
    \caption{Beispielhafte Ausgabe aus dem Relief Maps+ Tool}
    \label{fig:relief_maps_plus_screenshot_1}
\end{figure}

Ein zentrales methodisches Merkmal von Relief Maps+ ist der Versuch, die emotionale Wirkung sozialer Machtverhältnisse räumlich darstellbar zu machen – ohne diese in eindimensionale Kausalbeziehungen zu überführen. Die Nutzer\genderstern innen bewerten ihre Erfahrungen explizit entlang einzelner \glspl{identitaetsachse}. Gleichzeitig zeigt sich hier eine zentrale methodologische Spannung: Die isolierte Betrachtung einzelner Diskriminierungsachsen widerspricht dem Grundgedanken intersektionaler Analyse, der gerade auf die Verwobenheit und Gleichzeitigkeit verschiedener Machtverhältnisse verweist. Eine konsequente intersektionale Operationalisierung bleibt damit methodisch herausfordernd.

Einige technische Merkmale von Relief Maps+ sind auch im Hinblick auf die Entwicklung eigener Tools relevant. Die browserbasierte Anwendung erlaubt es Forschenden, eigenständig Projekte zu erstellen und auszuwerten. Allerdings ist der Zugang derzeit stark auf den katalanischen Kontext zugeschnitten: Verfügbare Sprachen sind Katalanisch, Spanisch und Englisch; Optionen zur Erweiterung oder Lokalisierung sind nicht dokumentiert. Da der Quellcode nicht öffentlich zugänglich ist, bleiben Fragen zur Anpassbarkeit, Wiederverwendbarkeit und langfristigen Wartbarkeit offen. Aus methodischer Sicht stellt sich somit die Frage, inwiefern die Software übertragbar ist auf andere sprachliche, kulturelle und geografische Kontexte.

Trotz dieser Einschränkung eröffnet Relief Maps+ wichtige Potenziale: Die bewusste Integration von Reflexivität, die aktive Beteiligung der Nutzer\genderstern innen an der Interpretation ihrer eigenen Erfahrungen sowie die Sichtbarmachung räumlich kontextualisierter Ungleichheiten markieren einen innovativen Zugang für intersektionale, subjektzentrierte Geographien. Die methodische Fundierung des Tools beruht auf einem iterativen Validierungsprozess unter Einbezug feministischer, queerer und dekolonialer Perspektiven \parencite{luizdesouzaSpiralValidationProcess2025}.



\subsubsection{Einordnung des eigenen Ansatzes}

Die im Rahmen dieser Arbeit entwickelte App \textit{InterMind} versteht sich als offen zugängliches und flexibel einsetzbares Werkzeug für Studien im Rahmen der \acrshort{gema}. Ziel war es, eine technisch eigenständige, quelloffene Infrastruktur bereitzustellen, die eine situative, geolokalisierte Erhebung affektiven Wohlbefindens ermöglicht – ein Instrument, das in dieser Form bislang nicht allgemein verfügbar war. Die Entwicklung orientierte sich in Teilen an bestehenden Tools wie \textit{Urban Mind}, insbesondere was das Interface-Design und die Nutzerführung betrifft, basiert jedoch auf einer unabhängig konzipierten Codebasis und wurde vollständig neu implementiert.

Die App selbst ist methodisch nicht innovativ im engeren Sinne, sondern stellt eine robuste, anpassbare Plattform dar, die für verschiedenste \acrshort{gema}-Studien konfiguriert werden kann. Ihr modularer Aufbau erlaubt die Integration beliebiger Fragebögen und Fragetypen – einschliesslich Freitextfeldern, Schiebereglern oder Mehrfachantworten. Damit kann das System flexibel an unterschiedliche Forschungskontexte angepasst und in zukünftigen Studien weiterverwendet werden.

Im Zentrum der vorliegenden Arbeit steht ein spezifisch entwickelter Fragebogen, der auf der App zum Einsatz kommt. Dieser kombiniert klassische \gls{ema}-Items zur situativen Erfassung von Kontext und Wohlbefinden mit explizit intersektional angelegten Fragen. Dabei werden Dimensionen wie Geschlecht, Herkunft oder sozioökonomischer Status getrennt erfasst – ein Ansatz, der zwar theoretisch nicht vollständig der Idee intersektionaler Verwobenheit entspricht, aber eine quantitative Auswertbarkeit ermöglicht. Gleichzeitig bleibt durch offene Antwortformate Raum für reflexive Auseinandersetzung mit der eigenen Erfahrung in konkreten räumlichen Situationen.

\subsection{Entwicklung der App \textit{InterMind}}

Im Zuge dieser Arbeit wurde die App \textit{InterMind} entwickelt, die als technische Grundlage für wiederholte, geolokalisierte und pseudonymisierte Befragungen dient. Sie bildet die Infrastruktur für anschliessend durchgeführte Pilot-Studie. Die App und der in dieser Arbeit eingesetzte Fragenkatalog wurden parallel und iterativ konzipiert. Während dieser Abschnitt die technische Entwicklung der App dokumentiert, wird die inhaltliche Gestaltung des Fragebogens im \cref{sec:fragebogen} erläutert. Der vollständige Quellcode der App ist auf \gls{github}\footnote{\href{https://github.com/lbatschelet/intermind}{https://github.com/lbatschelet/intermind}} veröffentlicht.


\subsubsection{Ziele und Rahmenbedingungen}

Klar machen wieso eine App entwickelt wurde
Klarmachen dass App und Fragebogen nicht dasselbe sind
Weniger labern, weniger wiederholen

sozusagen argumentativ dahin kommen wieso es die app braucht, kann man das analog erheben? nein, kann man das mit vorhandenen tools machen? ja, aber nicht so flexibel wie gewünscht.

stringender formulieren, einleitungstext üb erhaupt zur app entwicklung, dann erst der ganze rest

Die zentrale Erhebungslogik der vorliegenden Arbeit basiert auf wiederholten, geolokalisierten Erhebungen zum situativ-affektiven Wohlbefinden der Teilnehmenden. Daraus resultieren spezifische Anforderungen an das Instrument, mit dem diese Daten erfasst werden sollen. Ein geeignetes Erhebungstool muss insbesondere folgende Kriterien erfüllen: Es soll mobil und einfach nutzbar sein, situative Antworten unmittelbar im Alltag der Teilnehmenden ermöglichen, dabei Standortdaten automatisch erfassen und gleichzeitig datenschutzrechtliche sowie technische Hürden für die Nutzer\genderstern innen minimieren. Darüber hinaus war es von Beginn an wichtig, dass das System flexibel und nachhaltig konzipiert ist, um auch für zukünftige Arbeiten eingesetzt werden zu können. Konkret bedeutet dies, dass die Fragenkataloge sowie die Inhalte der App einfach austauschbar und an neue Forschungsfragen oder Zielgruppen anpassbar sein sollten.

Bereits verfügbare Lösungen erfüllten diese Anforderungen nur teilweise oder gar nicht. Kommerzielle Angebote, wie beispielsweise die Marktforschungsplattform Avicenna\footnote{\href{https://avicennaresearch.com/}{https://avicennaresearch.com/}}, sind aufgrund hoher Lizenzkosten für eine studentische Abschlussarbeit nicht praktikabel. Zudem erlauben viele solcher Dienste in keine vollständige Kontrolle über die verarbeiteten Daten und bieten nur begrenzte Anpassungsmöglichkeiten hinsichtlich Fragenstruktur und Datenerfassung. Auf der anderen Seite stehen Apps wie Urban Mind\footnote{\href{https://urbanmind.info/}{https://urbanmind.info/}}, die zwar grundsätzlich für \gls{gema}-Erhebungen im Forschungskontext entwickelt wurden, jedoch nicht quelloffen und entsprechend auch nicht eigenständig erweiterbar sind. Zudem ist mir persönlich in dieser Arbeit bei der sensible Daten zu Wohlbefinden und sozialen Zugehörigkeiten erhoben werden, eine transparente und sichere Datenverarbeitung von besonderer Bedeutung.

Vor diesem Hintergrund wurde das Ziel formuliert, eine eigene digitale Anwendung zu entwickeln, die bewusst quelloffen und modular gestaltet ist. Diese \gls{opensource}-Architektur sollte es ermöglichen, die gesamte Datenverarbeitung transparent und nachvollziehbar zu gestalten sowie künftige Anpassungen unkompliziert vorzunehmen. Aufgrund der limitierten zeitlichen Ressourcen innerhalb der Bachelorarbeit wurde darüber hinaus darauf geachtet, weit verbreitete Technologien und Frameworks zu wählen, um die Entwicklung möglichst effizient, wartungsarm und für Dritte nachvollziehbar zu halten.

\subsubsection{Konzeptionsphase}
Auf Basis der beschriebenen Anforderungen wurde zunächst ein detaillierter Anforderungskatalog entwickelt, der als zentraler Leitfaden für die weiteren Schritte der Entwicklung diente. Dieser Katalog wurde iterativ ergänzt, konkretisiert und während des gesamten Entwicklungsprozesses kontinuierlich an methodische und technische Erkenntnisse angepasst. In Anlehnung an etablierte Konzepte aus der Softwareentwicklung wurde dabei zwischen funktionalen und nicht-funktionalen Anforderungen unterschieden.

Funktionale Anforderungen definieren dabei konkret, \textit{was} die App im praktischen Einsatz leisten muss, und legen somit die notwendigen Funktionen und Abläufe der Anwendung fest. Für diese Studie bedeutete dies insbesondere, dass die App den Teilnehmenden täglich drei zufällig über den Tag verteilte Beantwortungszeiträume von einer Stunde ermittelt und jeweils zum Start dieser Zeiträume \glspl{pushnotification} sendet. Diese Anforderung schloss bereits früh eine Browser-basierte Erhebung aus, und führte zum Entscheid eine App-basierte Erhebung zu wählen. Weiter wurde festgelegt, dass bei jeder erfolgten Befragung der aktuelle Standort automatisiert mit erfasst werden soll, sofern die Teilnehmenden dies technisch erlauben. Um die Erhebung flexibel und bedarfsgerecht zu gestalten, wurden zudem verschiedene Fragetypen vorgesehen, darunter Single-Choice, Multiple-Choice, Skalen-basierte Fragen (Slider) sowie Freitextfelder. Schliesslich wurde es als zwingende funktionale Anforderung definiert, dass Teilnehmende jederzeit eigenständig sämtliche gespeicherten Daten löschen können. Die Teilnahme erfolgt dabei vollständig anonym, über eine gerätegebundene, automatisch generierte pseudonyme \gls{uuid}, ohne jegliche Form der Registrierung oder der Eingabe personenbezogener Daten.
 
Nicht-funktionale Anforderungen legen hingegen fest, \textit{wie} diese Funktionen umgesetzt werden sollen, und beschreiben qualitative Merkmale wie Sicherheit, Benutzerfreundlichkeit oder technische Kompatibilität. In diesem Projekt wurden insbesondere Datenschutz und Datensicherheit als zentrale nicht-funktionale Anforderungen definiert. Sämtliche Datenverarbeitungsprozesse müssen entsprechend den Vorgaben des \acrfull{dsg} und der Europäischen Datenschutzgrundverordnung \acrshort{dsgvo} erfolgen. Weiterhin wurde Mehrsprachigkeit (Deutsch, Englisch und Französisch) als Voraussetzung formuliert, ebenso wie die Möglichkeit einer späteren Erweiterung auf weitere Sprachen. Darüber hinaus sollte die Anwendung ursprünglich grundsätzlich offlinefähig sein. Im laufe der Entwicklung wurde diese Anforderung jedoch aufgegeben, da das dazu geführt hätte, dass jede Änderung im Fragenkatalog ein Update der App und anschliessend je nachdem nicht kompatible Versionen der App entstünden. Um Teilnehmenden mit unterschiedlichen Mobilgeräten die Teilnahme möglichst einfach zu machen, war zudem eine plattformübergreifende Kompatibilität für  und Android erforderlich. Schliesslich war eine offene, modulare und nachvollziehbare Codebasis wichtig, sodass Anpassungen und Erweiterungen des Systems durch andere Forschende mit minimalem Aufwand möglich bleiben. Dies wurde dadurch erreicht, dass die App als \gls{opensource}-Projekt auf \gls{github}\footnote{\href{https://github.com/lbatschelet/intermind}{https://github.com/lbatschelet/intermind}} veröffentlicht wurde.

Die Priorisierung und Auswahl dieser Anforderungen erfolgte unter Berücksichtigung der konkreten Forschungsziele, der vorhandenen Literatur zu mobilen Anwendungen im Bereich \gls{esm}/\gls{gema} \parencite[u.a.][]{chenPerceivedUrbanEnvironment2025, bakolisUrbanMindUsing2018, randallDevelopmentTrialMobile2013}, datenschutzrechtlicher Vorgaben sowie praktischer Erfahrungen aus dem eigenen Studium. Aufgrund des iterativen Vorgehens während der Entwicklung kam es dabei auch später immer wieder zu Anpassungen und Nachjustierungen einzelner Anforderungen.

\subsubsection{Datenschutz als Gestaltungsprinzip}

Datenschutz spielte von Beginn an eine zentrale Rolle im Entwicklungsprozess und beeinflusste sowohl die technische Architektur als auch methodische Entscheidungen. Es wird konsequent dem Prinzip \textit{Privacy by Design} gefolgt, das vorsieht, Datenschutzanforderungen bereits bei der Konzeption einer Anwendung mitzudenken und nicht nachträglich zu ergänzen \parencite{cavoukianPrivacyDesign72009}. Ziel ist es, ein hohes Mass an Privatsphäre zu gewährleisten und gleichzeitig volle Transparenz über die Erhebung und Verarbeitung der Daten sicherzustellen.

In der Umsetzung wurde das Prinzip \textit{Privacy by Design} konkret durch technische Massnahmen wie \textit{privacy by architecture} realisiert \parencite{spiekermannEngineeringPrivacy2009}. Die App erfasst keine personenbezogenen Angaben wie Namen, Telefonnummern oder E-Mail-Adressen. Stattdessen wird beim ersten Start automatisch eine gerätegebundene \glsfirst{uuid} generiert, über die alle Daten pseudonymisiert zugeordnet werden. Eine kontinuierliche Ortung findet nicht statt; Standortdaten werden ausschliesslich zum Zeitpunkt einer beantworteten Befragung erhoben.


Die Speicherung der Daten erfolgt auf einem Server in der Schweiz unter Verwendung der Plattform \gls{supabase} und einer \gls{postgresql}-\gls{datenbank}. Eine Zugriffskontrolle auf Zeilenebene (\gls{rls}) stellt sicher, dass jedes Endgerät nur auf die eigenen Daten zugreifen kann. Alle Datenübertragungen zwischen App und Server sind verschlüsselt und erfolgen über authentifizierte Schnittstellen.

Teilnehmende können ihre Datensätze jederzeit direkt über die App löschen. Damit werden sämtliche Einträge, die mit ihrer \gls{uuid} verknüpft sind, dauerhaft entfernt. Die Kontrolle über die eigenen Daten bleibt somit vollständig bei den Nutzer\genderstern innen.

Alle datenschutzrelevanten Aspekte sind in einer eigenen Datenschutzrichtlinie dokumentiert, die über die App sowie auf der Projektwebseite\footnote{\href{https://intermind.ch/privacy-policy.html}{https://intermind.ch/privacy-policy.html}} öffentlich zugänglich ist. Die Richtlinie erläutert zudem die Rechte der Teilnehmenden nach Schweizer Datenschutzgesetz (\gls{dsg}) und der Europäischen Datenschutzgrundverordnung (\gls{dsgvo}).


\subsubsection{Implementierung}

Die technische Umsetzung orientierte sich an etablierten Prinzipien des Software Engineerings \parencite{sommervilleSoftwareEngineering2016} sowie an den zentralen Gestaltungsprinzipien von \gls{solid} \parencite{martinAgileSoftwareDevelopment2002}. Im Zentrum standen dabei eine saubere Trennung zwischen Anwendungslogik, Datenhaltung und Benutzeroberfläche, eine modulare Struktur der Komponenten sowie eine klare Zuordnung von Verantwortlichkeiten.

Für die Umsetzung wurde das \gls{framework} \gls{reactnative} in Kombination mit der Entwicklungsplattform \gls{expo} gewählt. Diese Entscheidung ermöglichte es, mit einer einheitlichen Codebasis sowohl \gls{ios}- als auch \gls{android}-Geräte zu unterstützen. Dadurch reduzierte sich der Entwicklungsaufwand, während gleichzeitig eine konsistente Benutzererfahrung auf beiden Plattformen sichergestellt werden konnte. Als serverseitige Infrastruktur kam \gls{supabase} zum Einsatz – ein \gls{opensource} \gls{backend}-as-a-Service auf Basis einer relationalen \gls{postgresql}-\gls{datenbank}, das \gls{authentifizierung}, \gls{autorisierung}, Datenspeicherung und sichere Datenübertragung integriert bereitstellt.

Der Fragenkatalog ist nicht im Quellcode verankert, sondern wird dynamisch über eine \gls{json}-Konfigurationsdatei aus der \gls{datenbank} geladen. Dadurch können Inhalte ohne App-Update angepasst werden, was weitestgehend verhindert, dass inkompatible Versionen der App entstehen. Diese Flexibilität erfordert jedoch eine aktive Internetverbindung für das Laden der Befragungsinhalte.

Nach dem erstmaligen Ausfüllen eines Fragebogens berechnet die App automatisch drei individuelle und zufällige Befragungszeitpunkte pro Tag. Diese werden lokal auf dem Gerät gespeichert. Die Zeitpunkte werden täglich zufällig innerhalb von drei Tagesabschnitten (Morgen, Mittag/Nachmittag, Abend) gewählt. Ein Mindestabstand zwischen den einzelnen Befragungen garantiert eine gleichmässige zeitliche Verteilung. Sobald ein Zeitpunkt erreicht ist, erhalten die Teilnehmenden eine Benachrichtigung und haben ab diesem Moment exakt eine Stunde Zeit, um den Fragebogen auszufüllen. Wird der Fragebogen nicht innerhalb dieser Zeitspanne ausgefüllt, verfällt der Slot und die App schickt eine weitere Benachrichtigung beim nächsten geplanten Zeitpunkt.

Das \gls{frontend} wurde minimalistisch und funktional gestaltet, um eine intuitive Nutzung zu ermöglichen und eine möglichst neutrale Darstellung der Fragen sicherzustellen \parencite{rogersInteractionDesignHumancomputer2023}. Die App gliedert sich in drei Hauptbereiche: einen Startbildschirm mit dem nächsten Befragungszeitfenster, den Fragebogenbereich und einen Informations- und Einstellungsbildschirm mit Hinweisen zum Datenschutz. Zur visuellen Unterstützung wurden generische \gls{opensource}-Vektorgrafiken von Katerina Limpitsouni\footnote{\href{https://undraw.co/}{undraw.co/}} verwendet.


\subsubsection{Test und Iteration}
Zur Sicherstellung der technischen Zuverlässigkeit sowie der Gebrauchstauglichkeit der App wurde im Rahmen dieser Arbeit ein zweistufiger Testprozess durchgeführt. Aufgrund der begrenzten Ressourcen und des Projektumfangs kam dabei keine strukturierte automatisierte Testsuite zum Einsatz. Stattdessen wurde ein pragmatischer Ansatz gewählt, bei dem auftretende Probleme laufend manuell identifiziert und unmittelbar behoben wurden. Obwohl diese Vorgehensweise zu einer insgesamt soliden Funktionalität der Anwendung führte, erwies sich die fehlende automatisierte Qualitätssicherung im Entwicklungsprozess wiederholt als Herausforderung und führte zu einem insgesamt höheren zeitlichen Aufwand bei der Fehlersuche.

In einer ersten Phase fanden fortlaufende technische Funktionstests während der Implementierung statt. Im Fokus dieser Tests standen insbesondere die korrekte Verarbeitung des dynamisch geladenen Fragenkatalogs (\gls{json}-Datei), die fehlerfreie Übertragung von Daten an das \gls{supabase}-\gls{backend}, das Verhalten der App bei unterbrochener oder unzuverlässiger Internetverbindung sowie die plattformübergreifende Zuverlässigkeit der lokalen Planung und Ausführung von \glspl{pushnotification}. Da sich iOS und Android in ihrer Handhabung von Hintergrundprozessen, Benachrichtigungssystemen und Berechtigungsanfragen teilweise stark unterscheiden, wurde der Funktionsumfang regelmässig auf verschiedenen Geräten mit unterschiedlichen Betriebssystemversionen manuell geprüft.

In der zweiten Phase wurde ein interner Pretest mit vier Personen durchgeführt, welche die App über einen Zeitraum von zwei Wochen nutzten. Die Teilnehmenden erhielten hierzu über die offiziellen Beta-Testplattformen (TestFlight für iOS und Google Play Console für Android) Zugang zur Anwendung. Dabei wurden gezielte Aufgaben gestellt, um spezifische technische und inhaltliche Aspekte zu überprüfen sowie allgemeines Feedback zur Benutzerfreundlichkeit und Verständlichkeit der Fragen zu sammeln. Die Rückmeldungen der Testpersonen bezogen sich vor allem auf die Klarheit und Lesbarkeit einzelner Fragen und Texte, auf kleinere Probleme bei der erstmaligen Vergabe von App-Berechtigungen (z. B. Standortzugriff), sowie auf Darstellungsprobleme bestimmter \gls{ui}-Elemente auf verschiedenen Geräten.

Die Ergebnisse aus dieser Testphase führten sowohl zu technischen Anpassungen als auch zu inhaltlichen Änderungen im Fragebogen. Mehrere Fragen wurden sprachlich präzisiert oder gekürzt, einzelne Items wurden ganz gestrichen. Weiterhin wurden Anpassungen an der Benutzeroberfläche vorgenommen, beispielsweise bei Abständen und der Platzierung von Slider-Beschriftungen. Aufgrund der fehlenden automatisierten Tests erforderte jede Änderung erneut eine manuelle Überprüfung der gesamten Funktionalität, was insgesamt erheblichen zusätzlichen Aufwand bedeutete.

Insgesamt ermöglichte der iterative Testprozess jedoch, wesentliche Schwachstellen vor dem Beginn der eigentlichen Datenerhebung zu identifizieren und zu beheben. Obwohl auf eine formalisierte Usability-Evaluation verzichtet wurde, konnten durch diese Vorgehensweise zentrale technische sowie benutzerorientierte Probleme effektiv adressiert werden.

\subsubsection{Veröffentlichung und Distribution}

Um die entwickelte App für die eigentliche Datenerhebung nutzen zu können, wurde eine Veröffentlichung über die offiziellen App-Stores von Apple (iOS) und Google (Android) angestrebt. Beide Plattformen stellen dabei unterschiedliche technische, administrative und finanzielle Anforderungen, die den Veröffentlichungsprozess massgeblich beeinflussten.

Die Veröffentlichung im Apple App Store setzte zunächst den Erwerb einer kostenpflichtigen Entwicklerlizenz voraus, für die eine jährliche Gebühr von CHF 100 zu entrichten war. Nach erfolgreicher Einrichtung dieses Entwicklerkontos wurde die App zur Veröffentlichung eingereicht, allerdings von Apple zunächst nicht für eine finale Veröffentlichung im regulären App Store zugelassen. Als Begründung wurde angegeben, die App weise zu wenig inhaltlichen Mehrwert auf – eine Entscheidung, die aus Sicht der Entwicklung nur schwer nachvollziehbar war. Der Prüfprozess bei Apple ist zum Zeitpunkt des Abschlusses dieser Arbeit noch nicht vollständig abgeschlossen. Dennoch konnte die App über Apples eigene Plattform für öffentliche Beta-Tests („TestFlight“) bereitgestellt werden, sodass Teilnehmende der Studie über einen offiziellen TestFlight-Link Zugang zur App erhielten.

Im Gegensatz dazu verlangte Google für eine Veröffentlichung im Android Play Store keine laufenden Lizenzkosten. Allerdings stellte Google die Bedingung, dass vor einer offenen Betaversion zunächst ein geschlossener Test mit mindestens 20 Personen über einen Zeitraum von zwei Wochen durchgeführt werden musste. Da es innerhalb des zeitlichen Rahmens dieser Bachelorarbeit nicht möglich war, eine ausreichende Anzahl Testpersonen mit Android-Geräten zu rekrutieren, wurde hierfür ein externer Dienstleister in Anspruch genommen, welcher diesen erforderlichen Test für eine Gebühr von CHF 30 durchführte. Nach erfolgreichem Abschluss dieses Tests wurde die App im Play Store als offene Beta veröffentlicht und war somit öffentlich verfügbar.

Darüber hinaus verlangten beide Plattformen, dass eine öffentlich zugängliche Datenschutzrichtlinie zur Verfügung steht. Zu diesem Zweck wurde die Website intermind.ch eingerichtet, auf der neben der vollständigen Datenschutzerklärung auch ergänzende Informationen zum Forschungshintergrund abrufbar sind. Die Kosten hierfür beliefen sich auf einmalig CHF 10 für die Domainregistrierung; Hosting-Kosten entstanden keine zusätzlichen, da auf bereits bestehende Infrastruktur zurückgegriffen wurde.

Durch die Veröffentlichung über offizielle App-Plattformen konnten technische Hürden für die Teilnehmenden minimiert und zugleich plattformspezifische Anforderungen (beispielsweise hinsichtlich Datenschutzrichtlinien und Update-Management) zuverlässig erfüllt werden. Trotz vereinzelter Schwierigkeiten im Freigabeprozess ermöglichte dieses Vorgehen letztlich eine unkomplizierte Distribution und Nutzung der Anwendung im Rahmen dieser Arbeit.

\subsection{Fragebogenentwicklung}\label{sec:fragebogen}

Der Fragebogen besteht aus zwei zentralen Elementen:

\paragraph{Baseline-Befragung (einmalig):}
Hier wurden grundlegende demografische Variablen (Alter, Geschlecht, sexuelle Orientierung, Behinderung, sozioökonomischer Hintergrund) erhoben, um soziale Positionierungen für spätere intersektionale Analysen verfügbar zu machen. Fragen wurden bewusst offen oder mit freier Spezifikation gestaltet, um Normativität in den Antwortoptionen zu vermeiden und soziale Positionierungen differenziert erfassen zu können.

\paragraph{Situative Befragung (wiederholt):}
Der wiederholt eingesetzte Fragebogen besteht aus kurzen situativen Erhebungen, die mittels EMA durchgeführt wurden. Teilnehmer:innen wurden regelmässig aufgefordert, Fragen zu ihrem momentanen Wohlbefinden (z.\,B. empfundene Sicherheit, Zugehörigkeit, Entspannung) sowie zur aktuellen räumlichen und sozialen Umgebung zu beantworten. Hierbei kamen insbesondere Slider-Fragen zum Einsatz, um eine kontinuierliche Bewertung zu ermöglichen. Ergänzend wurde ein optionales Freitextfeld angeboten, das qualitative Kontextinformationen und subjektive Reflexionen zuliess.

Der Begriff \emph{Operationalisierung} bezeichnet hierbei die konkrete Umsetzung theoretischer Konstrukte in messbare Indikatoren. Wohlbefinden wurde operationalisiert über sieben zentrale Dimensionen (u.\,a. Entspannung, Sicherheit, Zugehörigkeit), die in der Literatur als relevant identifiziert wurden. Räumliche und soziale Kontexte wurden mit Items operationalisiert, die beispielsweise aus der Forschung von Bakolis et al. (2018) übernommen und für die vorliegende Studie adaptiert wurden.

\subsubsection{Ablauf und Durchführung der Datenerhebung}

Die Datenerhebung fand im Rahmen der einführenden Exkursion „Recht auf Stadt“ im ersten Studienjahr des Bachelorstudiengangs Geographie an der Universität Bern im Mai 2025 statt. Die teilnehmenden Studierenden wurden zu Beginn der Exkursion über Zielsetzung und Ablauf informiert und konnten anschliessend freiwillig an der Befragung teilnehmen. Die Nutzung der App wurde über den gesamten Exkursionszeitraum von drei Tagen durchgeführt, wobei die Teilnehmenden via Push-Benachrichtigungen mehrfach täglich aufgefordert wurden, die kurzen situativen Befragungen auszufüllen. Die Baseline-Befragung erfolgte einmalig zu Beginn.

Die Durchführung im Exkursionssetting ermöglichte eine kontrollierte Testung der technischen Funktionalität und eine hohe Compliance bei den Teilnehmenden. Gleichzeitig erlaubte dieses Setting, reale räumliche Kontexte, wie unterschiedliche urbane Umgebungen in Zürich, Basel und Bern, unmittelbar in die Datenerhebung einzubeziehen. Insgesamt zeichneten sich die Daten durch eine hohe räumliche und kontextuelle Varianz aus, die zentrale Grundlage für die späteren intersektionalen Analysen bildete.

Die vollständigen Fragebögen (Baseline- und situative Befragungen) sind als ergänzende Dokumentation digital im GitHub-Repository der App hinterlegt. Dies folgt dem Prinzip offener und transparenter Forschung. Eine detaillierte Fragebogenübersicht kann zusätzlich im Anhang dieser Arbeit eingesehen werden, um die inhaltliche Struktur und die Operationalisierung der theoretischen Konstrukte nachvollziehbar zu machen.

Zusammenfassend stellt die entwickelte App somit ein methodisch differenziertes, technisch flexibles Instrument dar, das sowohl situative Dynamiken als auch intersektionale soziale Strukturen systematisch erfassbar macht. Die enge Verknüpfung mit einem explizit intersektional ausgerichteten Fragebogen ermöglicht es, bestehende methodische Ansätze (Urban Mind, Relief Maps+) gezielt zu erweitern und dabei neue Erkenntnisse zur Beziehung von Raum, Wohlbefinden und sozialer Positionierung zu generieren.


\subsection{Limitationen und Herausforderungen der Datenerhebung}

\subsubsection{Geringe Rücklaufquote und mögliche Ursachen}

\subsubsection{Auswirkungen auf die Datenqualität und Analyse}


% LTeX: language=de-CH

\section{Ergebnisse} \label{sec:ergebnisse}

\subsection{Beschreibung des Datensatzes}

\subsubsection{Soziodemografische Merkmale der Stichprobe}

Die Stichprobe des Probelaufs umfasst insgesamt 24 Personen. Die Mehrheit gehört der Altersgruppe \emph{16–25 Jahre} an (n = 20; 80\,\%). Nur wenige Personen entfallen auf die Gruppen \emph{26–35 Jahre} (n = 3; 12\,\%) und \emph{56–65 Jahre} (n = 1; 4\,\%); eine Person machte keine Altersangabe.

Bezüglich des \emph{sozialen Geschlechts} gaben 15 Personen (60\,\%) an, \emph{Mann} zu sein, 9 Personen (36\,\%) identifizierten sich als \emph{Frau}, und eine Person (4\,\%) als \emph{trans Mann}. Als biologisches Geschlecht gaben 16 Personen (64\,\%) an, \emph{männlich} zu sein, 8 (32\,\%) an, \emph{weiblich} zu sein, eine Person machte keine Angabe.

Die \cref{tab:kreuztabelle_abs} zeigt die Verteilung von sozialem Geschlecht und Altersgruppe (absolute Häufigkeiten).

\begin{table}[H]
    \centering
    \caption{Kreuztabelle: Soziales Geschlecht und Altersgruppe (absolute Häufigkeiten)}
    \label{tab:kreuztabelle_abs}
    \begin{tabular}{lccccc}
    \toprule
    \textbf{Geschlecht} & 16--25 & 26--35 & 56--65 & Keine Angabe & Gesamt \\
    \midrule
    Mann       & 12 & 2 & 0 & 1 & 15 \\
    Trans Mann &  0 & 0 & 1 & 0 & 1  \\
    Frau       &  8 & 1 & 0 & 0 & 9  \\
    \midrule
    \textbf{Gesamt} & 20 & 3 & 1 & 1 & 25 \\
    \bottomrule
    \end{tabular}
\end{table}

Weitere soziodemografische Merkmale der Teilnehmenden umfassen u.\,a. sexuelle Orientierung, Bildungsstand, Erwerbsstatus, Haushaltseinkommen, Haushaltsstruktur und -finanzierung, sowie Erfahrungen mit Diskriminierung. Eine vollständige Übersicht über die Verteilung dieser Merkmale findet sich in \cref{sec:appendix_demographics}.


\subsubsection{Beschreibung der erfassten Momentaufnahmen}

Im Rahmen der Pilotstudie wurden insgesamt 106 Momentaufnahmen erhoben, verteilt auf die 25 Teilnehmenden. Die Anzahl abgeschlossener Befragungen pro Person variierte dabei erheblich (\textit{M} = 4{,}2; \textit{SD} = 2{,}9; \textit{Min} = 1; \textit{Max} = 12), was auf eine ungleichmässige Nutzung der App innerhalb der Teilnehmendengruppe hinweist (siehe \cref{fig:survey_counts}).

Die Tätigkeiten, die während der Beantwortung der Umfrage durchgeführt wurden, decken ein breites Spektrum ab. Am häufigsten gaben Teilnehmende an, zu arbeiten oder zu studieren (n = 48, 43\,\%), gefolgt von Freizeit- und Entspannungsaktivitäten (n = 19, 22\,\%) und Reisen bzw. Pendeln (n = 9, 10\,\%). Weitere Angaben umfassten etwa Kochen, Medienkonsum, soziale Aktivitäten oder Kombinationen mehrerer Aktivitäten.

Bezüglich des Aufenthaltsortes befanden sich die meisten Personen zum Zeitpunkt der Umfrage entweder an einer Bildungsinstitution (n = 38, 35\,\%) oder zu Hause (n = 28, 27\,\%). Weitere häufig genannte Orte waren unterwegs zu Fuss, per Fahrrad oder im Auto (n = 11, 11\,\%), öffentliche Verkehrsmittel (n = 8, 7\,\%) sowie die Wohnung anderer Personen oder Parks und Grünflächen (vgl. \cref{app:location_table}). Die Aufenthaltsorte verteilten sich dabei nahezu gleichmässig auf Innenräume (n = 54, 51\,\%) und Aussenräume (n = 52, 49\,\%).

Auch die soziale Situation während der Umfrage war sehr unterschiedlich: Ein Drittel der Momentaufnahmen wurde allein durchgeführt (n = 37, 35\,\%), ein weiteres Drittel in Gegenwart von Freund*innen (n = 28, 26\,\%). Weitere häufige Angaben betrafen die Anwesenheit von Fremden (n = 10, 9\,\%), Kolleg*innen (n = 8, 8\,\%) oder verschiedenen Kombinationen dieser Gruppen (vgl. \cref{app:people_table}).

Diese Vielfalt an Tätigkeiten, Kontexten und sozialen Situationen zeigt das Potenzial des Erhebungsinstruments, subjektives Wohlbefinden in unterschiedlichen Alltagssituationen ökologisch valide zu erfassen.



\begin{figure}[htbp]
    \centering
    \includegraphics[width=8cm]{analysis/plots/survey_counts.pdf}
    \caption{Aufteilung nach Anzahl abgeschlossener Umfragen pro Person}
    \label{fig:survey_counts}
\end{figure}




\subsubsection{Umfang und demografische Merkmale}

\subsubsection{Qualitative Rückmeldungen der Teilnehmenden}

\subsection{Intersektionale Analysen am erhobenen Material}

\subsubsection{Exemplarische Analysen mittels MAIHDA}
\label{sec:pilot_maihda}

In einem Probelauf wurde geprüft, ob die vorliegenden Daten eine intersektionale Multilevel-Analyse nach dem \gls{maihda}-Ansatz zulassen. Konkret sollte untersucht werden, ob (a) genügend Beobachtungen je intersektionalem Stratum vorhanden sind und (b) die zwischenstratale Varianz gross genug ist, um stabile Random-Effects-Schätzungen zu erhalten.

\paragraph{Iterative Spezifikation der Strata}
Ausgangspunkt war ein Set über alle erhobenen Achsen: Biologisches Geschlecht, Soziales Geschlecht, Sexuelle Orientierung, Ausbildungsstufe, Gruppiertes Äquivalenz-Einkommen, Anstellungsverhältnis, Geburtsland, Vorhandene Behinderungen.

Die Kombination dieser Merkmale ergab 20 unterschiedliche Strata. Die Zellgrössen (Anzahl Personen pro Stratum) sind allerdings sehr klein (siehe \cref{tab:zellgroessen_alle_achsen}).

\begin{table}[h]
    \centering
    \begin{tabular}{rl}
        count & 20 \\
        mean & 1.25 \\
        std & 0.55 \\
        min & 1 \\
        max & 3 \\
    \end{tabular}
    \caption{Zellgrössen pro Stratum mit allen Achsen}
    \label{tab:zellgroessen_alle_achsen}
\end{table}

Mit diesem Set von Strata ist die Modellierung nicht möglich, da die Zellgrössen zu klein sind und eine gute Schätzung der Varianzanteile nicht möglich ist.

Um die Modellierbarkeit zu erhöhen, wurde das Stratum anschliessend auf zwei theoretisch zentrale Achsen reduziert: Biologisches Geschlecht und Alter.

Dies führte zu insgesamt $6$ Strata, mit folgenden Zellgrössen:

\begin{table}[h]
    \centering
    \begin{tabular}{rl}
        count & 6 \\
        mean & 4.17 \\
        std & 4.7 \\
        min & 1 \\
        max & 12 \\
    \end{tabular}
    \caption{Zellgrössen pro Stratum mit reduzierten Achsen}
    \label{tab:zellgroessen_reduzierte_achsen}
\end{table}

Auch hier gibt es noch einzelne Strata mit weniger als 3 Beobachtungen.

\begin{table}[h]
    \centering
    \begin{tabular}{lll}
        Gender & Altersgruppe & Anzahl \\
        \hline
        man & 16 – 25 & 2 \\
        trans man & 56 – 65 & 1 \\
        man & missing & 1 \\
        woman & 26 – 35 & 1 \\
    \end{tabular}
    \caption{Strata mit weniger als 3 Beobachtungen}
    \label{tab:zellgroessen_reduzierte_achsen}
\end{table}

Trotzdem wurde versucht, mit diesem Set von Strata eine MAIHDA-Analyse durchzuführen.

Für jedes kontinuierliche Outcome (\texttt{sense\_of\_belonging}, \texttt{environmen\_pleasure}, \texttt{environment\_lively}, \texttt{environment\_nature}, \texttt{environment\_noise}) wurde ein zweistufiges MAIHDA-Setting geschätzt:
\begin{description}
    \item[Modell~1A (Nullmodell):] Zufallsinterzept auf Stratum-Ebene, keine festen Effekte der Achsen.
    \item[Modell~1B (Additives Modell):] Zusätzlich feste Haupteffekte von \texttt{age\_group} und \texttt{gender}; der verbleibende Stratum-Random-Effect wird als Interaktionsanteil interpretiert.
\end{description}
Aufgrund der geringen Zellgrössen wurde kein zusätzlicher Random-Intercept auf Personenebene modelliert. Die Schätzung erfolgte mittels \texttt{statsmodels.mixedlm} (MLE, Optimierer \texttt{lbfgs}).


\subsection{Varianzzerlegung (VPC, PCV)}
Die geschätzten Varianzanteile zwischen Strata (Variance Partition Coefficient, VPC) waren durchgängig extrem klein. Beispielhaft:

\begin{center}
\begin{tabular}{lrrrr}
\toprule
Outcome & VPC$_{\text{Null}}$ & VPC$_{\text{add}}$ & PCV & $n$ (Zeilen) \\
\midrule
\texttt{sense\_of\_belonging}      & $1.24\times 10^{-5}$ & $1.15\times 10^{-6}$ & 90.7\% & 106 \\
\texttt{environmen\_pleasure}      & $\approx 0$          & $4.40\times 10^{-7}$ & --      & 106 \\
\texttt{environment\_lively}       & $2.95\times 10^{-4}$ & $3.43\times 10^{-5}$ & 88.2\%  & 106 \\
\texttt{environment\_nature}       & $1.03\times 10^{-2}$ & $1.12\times 10^{-6}$ & 99.99\% & 106 \\
\texttt{environment\_noise}        & $1.95\times 10^{-2}$ & $2.98\times 10^{-5}$ & 99.85\% & 106 \\
\bottomrule
\end{tabular}
\end{center}

Die nahezu Null liegenden VPCs belegen, dass (a) die Outcomes sich zwischen den Strata kaum unterscheiden und (b) die Stratum-Varianz im Modell auf Null \emph{geschrumpft} wird. PCV-Werte sind bei einem praktisch Null-VPC im Nullmodell numerisch instabil (z.\,B.\ negative oder extrem grosse Werte) und daher nicht interpretierbar.

\subsection{Schlussfolgerung}
Die Pilotanalyse zeigt, dass mit den vorliegenden Daten keine sinnvolle MAIHDA-Varianzzerlegung durchführbar ist. Gründe:
\begin{enumerate}
    \item \textbf{Zu kleine Strata-Zellgrössen}: Die meisten Strata enthalten nur eine Person bzw.\ sehr wenige Beobachtungen.
    \item \textbf{Geringe zwischenstratale Varianz}: Die betrachteten Outcomes variieren kaum zwischen den (reduzierten) Strata.
    \item \textbf{Numerische Instabilität}: Die Random-Effects-Kovarianzmatrix wird singular; die Schätzung kollabiert auf Randlösungen.
\end{enumerate}

\paragraph{Implikation für die weitere Analyse.}
Für die Beantwortung der Forschungsfrage (Einfluss situativer Umweltfaktoren auf affektives Wohlbefinden) bietet es sich an, die Umweltvariablen als Level-1-Prädiktoren in einem vereinfachten Modell (z.\,B.\ lineares Modell mit cluster-robusten Standardfehlern nach Person oder ein Mixed Model nur mit Personen-Random-Intercept) zu analysieren. Intersectionale Unterschiede können vorerst über feste Effekte (z.\,B.\ \texttt{C(age\_group)}, \texttt{C(gender)}) kontrolliert werden. Eine vollwertige MAIHDA-Anwendung ist erst mit grösserer Stichprobe und ausreichenden Zellgrössen pro Stratum sinnvoll.

\subsubsection{Illustration möglicher Zusammenhänge zwischen Umwelt und Wohlbefinden}

\subsection{Interpretation der explorativen Befunde}




% LTeX: language=de-CH

\chapter{Diskussion} \label{sec:diskussion}

\section{Potential und Grenzen des entwickelten Erhebungsinstruments}

Die Entwicklung von \textit{InterMind} war im Rahmen dieser Arbeit nicht nur ein technisches, sondern auch ein methodisches Experiment. Ziel war es, mit begrenzten Ressourcen ein Werkzeug zu schaffen, das situative, geolokalisierte Erhebungen zuverlässig durchführen kann -- und dabei die Grundprinzipien von Transparenz, Datenschutz und Anpassungsfähigkeit wahrt. Die im Kapitel skizzierten technischen Entscheidungen waren dabei stets auch methodische Abwägungen: Sie bestimmten nicht nur, wie die App funktioniert, sondern auch, welche Formen der Datenerhebung und -auswertung überhaupt möglich waren.

Besonders prägend war die Wahl eines bewusst reduzierten, clientseitig gesteuerten Systemdesigns. Diese Architektur minimierte Abhängigkeiten von externer Infrastruktur, reduzierte potenzielle Datenschutzrisiken und erlaubte eine transparente, vollständig nachvollziehbare Funktionsweise. Gleichzeitig bedeutete sie den Verzicht auf Funktionen, wie sie in komplexeren GEMA-Implementierungen üblich sind -- etwa geofence-basierte Trigger oder serverseitige Kontextlogiken. Dadurch blieb die App methodisch auf feste, vordefinierte Erhebungszeitpunkte beschränkt und konnte nicht adaptiv auf räumliche oder kontextuelle Veränderungen reagieren. Für explorative Pilotstudien wie die vorliegende war dies ausreichend, in längerfristigen oder gross angelegten Projekten wäre jedoch eine dynamischere, kontextsensitivere Architektur wünschenswert.

Die Entscheidung zur Open-Source-Veröffentlichung stellt einen zentralen Bestandteil des Projekts dar. Sie ermöglicht anderen Forschenden nicht nur die Nachnutzung des Codes, sondern schafft auch die Grundlage für kollaborative Weiterentwicklungen. Gleichzeitig machte die Erfahrung mit den App-Store-Gatekeeping-Prozessen deutlich, dass Offenheit allein keine Garantie für breite Zugänglichkeit ist: Die Distribution über zentrale Plattformen bleibt an kommerzielle und intransparente Strukturen gebunden, die auch nicht-kommerzielle, wissenschaftliche Projekte einschränken können. Hier zeigt sich ein strukturelles Spannungsfeld zwischen der offenen, gemeinschaftsorientierten Logik von Open-Source-Software und den geschlossenen, marktkontrollierten Ökosystemen der grossen Plattformanbieter.

Im Rückblick wird deutlich, dass die App-Entwicklung in dieser Form einerseits ein funktionierendes, forschungsnahes Werkzeug hervorgebracht hat, andererseits aber auch klare Grenzen aufweist. Diese liegen weniger in der Stabilität oder Bedienbarkeit, sondern vielmehr in der eingeschränkten Kontextanpassung, der fehlenden Echtzeitauswertung und der aufwändigen Anpassbarkeit für andere Forschungssettings. Zukünftige Iterationen könnten hier ansetzen -- etwa durch die Ergänzung serverseitiger Module, die Entwicklung eines webbasierten Dashboards für Monitoring und Feedback, oder die modularisierte Integration zusätzlicher Erhebungsmethoden.

Damit verdeutlicht \textit{InterMind} sowohl die Chancen als auch die Grenzen einer eigenständigen Entwicklung im Rahmen einer Abschlussarbeit: Sie eröffnet Handlungsspielräume, schafft technologische Unabhängigkeit im Entwicklungsprozess und macht Forschungsinfrastruktur transparent -- bleibt aber eingebettet in grössere, teils restriktive Strukturen, die den Handlungsspielraum letztlich mitbestimmen.

\section{Reflexion und Weiterentwicklungspotenzial des Fragebogens}

Der entwickelte Fragebogen erwies sich im Feld als grundsätzlich funktional und gut in den Ablauf der Studie integrierbar. Er erfüllte die Anforderung, situative Erhebungen in kurzer Zeit und mit geringer Belastung für die Teilnehmenden durchführen zu können. Gleichzeitig zeigte sich jedoch, dass diese Stärken teilweise mit methodischen Einbussen erkauft wurden, die den wissenschaftlichen Anspruch der Erhebung begrenzen.

Besonders deutlich wird dies bei der Auswahl der Items zur Erfassung situativen affektiven Wohlbefindens. Die gewählten Dimensionen -- darunter „generelles Wohlbefinden“, Zufriedenheit, Anspannung, Energie und Zugehörigkeit -- erlaubten zwar eine kompakte Erfassung, entstanden jedoch nicht aus einer stringenten theoretischen Modellierung heraus. Diese pragmatische Herangehensweise erleichterte zwar die Umsetzung im Rahmen einer Mehrfacherhebung, führte aber zu einer geringeren konzeptuellen Schärfe und erschwerte den direkten Vergleich mit bestehenden Studien.

Auch der Verzicht auf etablierte standardisierte Skalen hatte ambivalente Folgen. Er trug dazu bei, den Fragebogen schlank zu halten und die Akzeptanz bei den Teilnehmenden zu erhöhen, schränkte jedoch die Vergleichbarkeit der Daten und ihre Anschlussfähigkeit an bestehende Forschungsinstrumente ein. Eine gekürzte, modulare Integration validierter Skalen hätte hier einen Ausgleich zwischen Praktikabilität und methodischer Robustheit schaffen können.

Die mehrsprachige Umsetzung des Instruments war ein wichtiger Schritt in Richtung Zugänglichkeit, blieb jedoch ohne formalisierte Validierung durch muttersprachliche Expert:innen. Dadurch ist nicht auszuschliessen, dass inhaltliche Nuancen, insbesondere bei affektiven Zustandsbeschreibungen, zwischen den Sprachversionen leicht variierten. Diese Unsicherheiten verstärkten sich bei sensiblen Konzepten wie \gls[noindex]{race}, für das im deutschsprachigen Kontext keine etablierten, diskriminierungssensiblen Kategorien verfügbar sind. Die gewählte Operationalisierung über Geburts- und Aufenthaltsland senkte zwar die Erhebungsbarrieren, konnte die Komplexität rassifizierter Erfahrungen jedoch nur unvollständig erfassen.

Schliesslich war der Entwicklungsprozess des Fragebogens zwar iterativ angelegt und von kontinuierlichem Feedback begleitet, basierte jedoch nicht auf einem formalen Pretest mit einer breiten und divers zusammengesetzten Testgruppe. Dadurch wurden potenzielle Verständnisschwierigkeiten oder kulturelle Unschärfen nur in begrenztem Umfang sichtbar.

Insgesamt bleibt festzuhalten, dass der Fragebogen in seiner vorliegenden Form eine praktikable, aber methodisch eingeschränkte Lösung darstellt. Für zukünftige Studien bieten sich mehrere Ansatzpunkte zur Weiterentwicklung: eine engere theoretische Anbindung der Items, die gezielte Integration gekürzter validierter Skalen, ein systematischeres Übersetzungs- und Validierungsverfahren sowie umfassendere Pretests. Auf diese Weise liesse sich die inhaltliche Aussagekraft der Erhebung stärken, ohne die für hochfrequente Befragungen notwendige Niedrigschwelligkeit aufzugeben.

\section{Empfehlungen für weiterführende Forschung}

\subsection{Verbesserungsvorschläge zur Erhöhung der Teilnahmequote}

\subsection{Optimierung der intersektionalen Datenerhebung und Analyse}

\subsection{Integration qualitativer Verfahren}


% ohne technische details mal hier rüberkopiert:

% \section{Eigenständig, aber nicht unabhängig -- Entwicklung im Plattformzeitalter}

% Die Entwicklung von \gls{intermind} war mein erstes grösseres Projekt in \gls{typescript} und mit \gls{reactnative}. Die Umstellung vom strikt objektorientierten Denken in \gls{java} auf den dynamischeren, komponentenbasierten Ansatz war anspruchsvoll, aber enorm lehrreich. Insbesondere das konsequente Anwenden der \gls{solid}-Prinzipien half dabei, die Struktur der Anwendung nachvollziehbar zu halten -- gerade in einem neuen Ökosystem. Die App funktioniert stabil, sieht gut aus, und hat ihren Zweck erfüllt.

% Trotz einer bewussten Orientierung an Prinzipien wie \gls{solid} und einem grundlegenden Architekturkonzept zeigte sich im Verlauf der Entwicklung, dass eine noch systematischere Auseinandersetzung mit der Softwarearchitektur hilfreich gewesen wäre. Zwar wurde auf eine modulare Struktur geachtet, viele Designentscheidungen wurden jedoch eher situativ getroffen und nicht im Sinne eines übergeordneten Gesamtdesigns immer wieder überprüft. Gerade im weiteren Projektverlauf wäre es sinnvoll gewesen, gezielt zu früheren architektonischen Überlegungen zurückzukehren und diese zu reflektieren oder anzupassen.

% Methoden wie \emph{Test-Driven Development} hätten diesen Prozess zusätzlich stützen können, indem sie klare Schnittstellen und Verantwortlichkeiten frühzeitig erzwingen. Auch der Aufbau automatisierter Tests und eine kontinuierlich integrierte Codeanalyse hätten dazu beigetragen, Fehlerquellen frühzeitig zu identifizieren und die langfristige Wartbarkeit der Anwendung zu verbessern. Viele kleinere Schwächen im Code wurden zwar pragmatisch behoben, ein strukturierteres Qualitätsmanagement hätte jedoch die Notwendigkeit späterer Refactoring-Prozesse deutlich reduziert.

% In diesem Sinne reiht sich die App auch in eine typische Dynamik vieler \gls{opensource}-Projekte ein: Sie wurde aus einem konkreten Forschungsbedarf heraus entwickelt, funktioniert zuverlässig, ist öffentlich dokumentiert -- aber nicht in jedem Teilbereich optimal strukturiert. Durch die Offenlegung des Quellcodes besteht jedoch die Möglichkeit, dass andere Entwickler\genderstern innen auf dieser Grundlage aufbauen, Verbesserungsvorschläge einbringen oder eigene Erweiterungen umsetzen.

% \vspace{1em}

% Trotz stabiler Funktionalität und durchdachter Grundstruktur weist das entwickelte System klare Begrenzungen auf -- insbesondere im Hinblick auf die situative Reaktionsfähigkeit und Kontextanpassung. So verzichtet \gls{intermind} bewusst auf kontinuierliches Geotracking, automatisierte Trigger oder serverseitige Kontextlogiken, wie sie in anderen \acrshort{gema}-Systemen Anwendung finden.

% Ein Beispiel dafür bietet das im Rahmen einer kanadischen Studie zu Nationalparks entwickelte \acrshort{health}-Plattform \parencite{wrayHealthyEnvironmentsActive2025}. Die Dokumentation zu diesem Tool ist erst während der Entstehung dieser Arbeit als Preprint veröffentlicht worden. Die App wird derzeit exklusiv im Rahmen des \textit{ParkSeek}-Projekts\footnote{\href{https://parkseek.ca/}{parkseek.ca}} eingesetzt und ist nicht öffentlich zugänglich. Ihre zugrundeliegende Systemarchitektur erlaubt eine kontinuierliche Standorterfassung und serverseitige Kontextverarbeitung, wodurch komplexe Logiken wie geofence-basierte Trigger umgesetzt werden können. So lassen sich etwa Benachrichtigungen auslösen, wenn sich Teilnehmende über längere Zeit in spezifischen Umwelten aufhalten. Diese technisch anspruchsvolle Lösung erlaubt eine besonders enge Verzahnung zwischen räumlichem Verhalten und situativer Befragung, geht jedoch mit einem hohen Aufwand sowie erheblichen Anforderungen an Datenschutz, Datenmanagement und Infrastruktur einher.

% Im Rahmen eines Bachelorprojekts wäre die Implementierung eines derart umfassenden Systems weder zeitlich noch organisatorisch realistisch gewesen. Stattdessen wurde ein datensparsamer, clientseitig gesteuerter Ansatz gewählt, der mit begrenzten Mitteln eine funktionale, transparente und reflektierte Umsetzung ermöglicht. Die Entscheidung für ein reduziertes Systemdesign war damit nicht nur eine Frage des Aufwands, sondern auch ein bewusster Kompromiss zugunsten von Kontrollierbarkeit und Datenschutz.

% Eine weitere Limitation des aktuellen Systemdesigns liegt im Fehlen eines serverseitigen Dashboards oder einer integrierten Auswertungsoberfläche. Es besteht keine Möglichkeit, Rückmeldungen in Echtzeit zu visualisieren, aggregierte Antworten einzusehen oder Monitoring-Funktionen während der Erhebung zu nutzen. Solche Features wären insbesondere für die Steuerung längerer Erhebungsphasen, die Qualitätssicherung oder für Feedbackschleifen mit den Teilnehmenden von Vorteil gewesen. Ihre Umsetzung hätte jedoch zusätzliche Entwicklungsressourcen sowie eine komplexere Backend-Architektur vorausgesetzt. Gleichwohl bleibt die Möglichkeit bestehen, entsprechende Funktionen in zukünftigen Iterationen oder auf Basis der veröffentlichten Codebasis nachzurüsten.

% \vspace{1em}

% Die Erfahrungen rund um die Veröffentlichung in den App Stores hat zentrale Spannungsfelder digitaler Infrastruktur deutlich gemacht. Obwohl die App funktional einsatzbereit war und über TestFlight bzw. den Play Store zugänglich gemacht werden konnte, blieb die reguläre Veröffentlichung im Apple App Store aufgrund einer intransparenten Ablehnung verwehrt.

% Solche Prozesse offenbaren strukturelle Abhängigkeiten, die weit über dieses Projekt hinausgehen: Zwei gigantische multinationale Tech-Konzerne kontrollieren in weiten Teilen den Zugang zu digitaler Infrastruktur. Dabei wirken sie zugleich als Regelsetzer, Infrastrukturbetreiber und ökonomische Gatekeeper. Diese doppelte Rolle ist nicht demokratisch legitimiert, aber mit erheblicher Lenkungsmacht verbunden. Gerade nicht-kommerzielle, experimentelle oder aktivistische Projekte sind von diesen Kontrollmechanismen besonders betroffen, da sie sich nicht ohne Weiteres den geforderten Verwertungslogiken oder Standardprozessen unterwerfen.

% Auch wenn Open-Source-Prinzipien allein diese strukturellen Hürden nicht auflösen können, war die Entscheidung zur Veröffentlichung des Quellcodes dennoch zentral: Sie schafft Transparenz, ermöglicht Weiterentwicklung und signalisiert ein bewusstes Gegenmodell zu proprietären, intransparenten Systemen. Gleichzeitig offenbart sich hier ein grundlegendes Spannungsfeld: Die offene, zugängliche und gemeinschaftsorientierte Logik von \gls{opensource}-Software steht in einem scharfen Kontrast zu den geschlossenen, marktkontrollierten Strukturen kommerzieller Distributionsplattformen. Wer eine App entwickeln und öffentlich zugänglich machen will, ist faktisch gezwungen, sich diesen Plattformen zu unterwerfen.

% Die Arbeit an der App war dabei nicht nur funktional motiviert, sondern auch von der Erfahrung getragen, ein eigenes digitales Werkzeug gestalten zu können -- mit all seinen Herausforderungen, aber auch mit dem unmittelbaren Lernerfolg und der Freude am konkreten Entstehungsprozess. Gerade vor dem Hintergrund einer zunehmend von privatwirtschaftlichen Plattformen dominierten digitalen Infrastruktur bleibt die Fähigkeit, eigene Werkzeuge zu entwickeln, ein wichtiger Akt technischer Aneignung.

% auch hier nicht nochmal entwickelnd erklären, je nachdem acuh ganz sein lassen

% \section{Klar, verständlich, iterativ -- Der Weg zum finalen Fragebogen}

% Die sprachliche Gestaltung der Fragebogen-Items stellte im Entwicklungsprozess eine zentrale methodische Herausforderung dar. Ziel war es, die Befragung möglichst zugänglich, verständlich und gleichzeitig inhaltlich präzise zu gestalten. Da die Befragung explizit auf eine intersektionale Analyse abzielt, wurde besonderer Wert darauf gelegt, die sprachliche Zugänglichkeit möglichst breit zu gewährleisten. Folglich wurde der Fragebogen bewusst mehrsprachig konzipiert und auf Deutsch, Englisch sowie Französisch umgesetzt. Weitere Sprachversionen wären zwar aus Sicht der intersektionalen Zugänglichkeit wünschenswert gewesen, scheiterten jedoch am hohen Aufwand für qualitativ hochwertige und inhaltlich konsistente Übersetzungen.

% Ein grundsätzliches Anliegen war eine möglichst direkte, adressierende Sprache in der \enquote{Du}-Form, um einen niederschwelligen Zugang zur Befragung zu fördern und hierarchische Distanz zwischen Forschenden und Teilnehmenden zu reduzieren. Gleichzeitig mussten die Formulierungen prägnant, alltagsnah und schnell erfassbar sein, da insbesondere die situativen Erhebungen kurz gehalten werden sollten. Hier ergab sich ein methodischer Balanceakt: Einerseits sollte die Befragung leicht verständlich bleiben, andererseits mussten komplexe Konzepte in zugänglicher Sprache operationalisiert werden. So wurde \gls{bspw} das Konzept der \gls{intersektionalitaet} im Einführungsteil des Fragebogens erläutert, danach jedoch bewusst vermieden, um unnötige Barrieren zu reduzieren. Stattdessen wurden alternative Formulierungen wie \enquote{persönliche Merkmale} verwendet, die jedoch teilweise inhaltliche Unschärfen mit sich brachten. 

% Besonders deutlich wurde diese Herausforderung im Umgang mit dem Konzept \gls[noindex]{race}. Gerade im deutschsprachigen Kontext existieren hier nur schwer geeignete Begrifflichkeiten: Formulierungen wie \enquote{Rasse} oder \enquote{Ethnizität} sind entweder sprachlich ungebräuchlich, problematisch oder stark mit kolonialen und biologistischen Zuschreibungen assoziiert \parencite[\gls{vgl}][]{roigIntersectionalityEuropeDepoliticized2018}. Alternativ verwendete Begriffe wie \enquote{Herkunft} oder \enquote{Aussehen} sind wiederum unpräzise und greifen die Dimension rassifizierter Diskriminierung nur unvollständig auf.

% Die Übersetzung der Items erfolgte nicht wörtlich, sondern sinngemäss, wobei insbesondere bei affektiven Zustandsbeschreibungen semantische Abstimmungen zwischen den Sprachversionen vorgenommen wurden. Dabei wurden auch kulturelle Unterschiede in der Alltagsverwendung bestimmter Begriffe berücksichtigt. Dieser Kompromiss ermöglichte trotz beschränkter Ressourcen eine hinreichend konsistente Mehrsprachigkeit, brachte jedoch gewisse methodische Limitierungen hinsichtlich der Vergleichbarkeit der Sprachversionen mit sich.

% Der beschriebene Sprach- und Übersetzungsprozess war eingebettet in einen breiteren, iterativen Entwicklungsprozess, der sowohl auf der Analyse bestehender Literatur als auch auf kontinuierlichem Feedback basierte. Ausgangspunkt bildeten Studien wie die Urban-Mind-Studie \parencite{bakolisUrbanMindUsing2018}, deren methodische Ansätze zur Erhebung situativen Wohlbefindens und räumlicher Wahrnehmung als Orientierung dienten. Diese Ansätze wurden jedoch um eigene Überlegungen zur intersektionalen Erhebung sozialer Positionierung ergänzt und in mehreren Durchläufen kritisch reflektiert.

% Während der Testphase der App Entwicklung (Siehe \cref{sec:app_entwicklung_feldtest}) sind ebenfalls zahlreiche kleinere Rückmeldungen zu Formulierungen und sprachlichen Feinheiten eingegangen. Diese wurden laufend eingearbeitet. Ebenfalls wurde der Fragebogen mit der betreuenden Dozentin durchgegangen und danach ebenfalls nochmals entsprechend überarbeitet.
% Ein wiederkehrendes methodisches Kriterium bei diesen Diskussionen war stets, die Belastung für Teilnehmende so gering wie möglich zu halten, ohne zentrale Aspekte der Forschungsfrage zu vernachlässigen. Durch den iterativen Ansatz konnte die Perspektive potenzieller Befragter frühzeitig einbezogen werden, was zu einer praxisnahen Optimierung der Items und des Fragebogenaufbaus führte.

% Im Rückblick lassen sich einige Punkte identifizieren, die bei einer erneuten Durchführung anders gestaltet werden könnten. Die Auswahl der Wohlbefindensdimensionen erfolgte nicht vollständig theoriegeleitet; insbesondere das Item „generelles Wohlbefinden“ wirkt im Nachhinein wenig trennscharf. Eine stärkere konzeptuelle Fundierung der Items wäre sinnvoll, um die Aussagekraft einzelner Skalen zu erhöhen.

% Der Verzicht auf standardisierte Skalen ermöglichte eine kompakte Erhebung, schränkt jedoch Vergleichbarkeit und Validität ein. Eine modularisierte Integration validierter Instrumente -- etwa in gekürzter Form -- könnte eine tragfähige Alternative darstellen.

% Die mehrsprachige Umsetzung war aus methodischer Sicht wichtig, konnte jedoch mangels Ressourcen nicht vollständig abgesichert werden. Eine zusätzliche Validierung durch muttersprachliche Expert:innen wäre wünschenswert gewesen, um semantische Konsistenz über Sprachversionen hinweg besser zu gewährleisten.

% Insgesamt zeigen sich an mehreren Stellen Stellschrauben für eine künftige Weiterentwicklung -- etwa durch eine engere theoretische Anbindung, gezielte Pretests oder eine systematischere Überprüfung von Übersetzungen und Antwortformaten. Gleichzeitig hat sich der gewählte Zugang als praktikabel und kontextsensibel erwiesen, insbesondere im Hinblick auf Zugänglichkeit und situative Anschlussfähigkeit.

% Besonders herausfordernd war die Erfassung von \gls[noindex]{race}. Im europäischen Kontext existieren kaum etablierte Kategorien, die rassifizierte Zugehörigkeiten erfassen, ohne problematische koloniale oder biologistische Zuschreibungen zu reproduzieren \parencite[\gls{vgl}][]{roigIntersectionalityEuropeDepoliticized2018}. Anders als in der US-amerikanischen Tradition, in der standardisierte Selbstkategorisierungen weit verbreitet sind, fehlen im hiesigen Kontext praktikable, breit akzeptierte Formate für quantitative Erhebungen. 

% Vor diesem Hintergrund wurde aus pragmatischen Gründen lediglich erfasst, ob Teilnehmende aktuell in einem anderen Land leben als in jenem, in dem sie geboren wurden. Diese Lösung reduzierte die Erhebungsbarrieren, blieb jedoch analytisch begrenzt: Sie kann nur indirekt auf rassifizierte Erfahrungen hinweisen und wird der Komplexität intersektionaler Ungleichheiten nicht gerecht. Rückblickend wäre eine offene, selbstbezeichnungsbasierte Erhebung vorzuziehen gewesen, um dieser sozialen Dimension angemessen Sichtbarkeit zu verleihen.

% Vor diesem Hintergrund wurde ein eigener, stark reduzierter Item-Satz entwickelt, um zentrale Dimensionen des Wohlbefindens situativ abbilden zu können. Ausgewählt wurden fünf Dimensionen: generelles Wohlbefinden, Zufriedenheit, Anspannung, Energie und Zugehörigkeit. Die Antworten wurden über lineare Slider-Skalen erfasst, um eine schnelle und intuitive Bearbeitung zu ermöglichen. 

% Diese Lösung stellt jedoch einen methodischen Kompromiss dar. Die Auswahl der Dimensionen erfolgte nicht auf Basis eines validierten theoretischen Modells, sondern primär pragmatisch und unter der Prämisse minimaler Befragungsdauer. Entsprechend ist die Validität und Vergleichbarkeit der erhobenen Werte eingeschränkt. Rückblickend erweist sich dies als Schwachpunkt der Studie: Die Messung bleibt inhaltlich und konzeptuell weniger trennscharf als wünschenswert, und es fehlt an einer etablierten Referenz, um die Ergebnisse eindeutig einzuordnen. Eine künftige Weiterentwicklung sollte daher auf einer systematischeren Konzeptualisierung beruhen und -- sofern möglich -- gekürzte, validierte Instrumente integrieren, die auf die Anforderungen situativer Mehrfacherhebungen angepasst sind.




% LTeX: language=de-CH

\section{Fazit und Ausblick} \label{sec:fazit}

\subsection{Zusammenfassung der zentralen Erkenntnisse}

\subsection{Ausblick auf mögliche Folgeprojekte}



% ------------------ Glossar

\newpage
\printglossary


% ----------------- Bibliographie ------------------
\newpage
\phantomsection
\printbibliography[heading=bibintoc]

\newpage

\phantomsection
\section*{Hinweis für den Einsatz von künstlicher Intelligenz (KI)}

Dieses Dokument wurde mithilfe von KI-basierten Tools überarbeitet. LanguageTool, ein KI-gestütztes Grammatik- und Stilprüfungswerkzeug, wurde verwendet, um Formulierungen zu verbessern und die Grammatik zu korrigieren. Chat-GPT von Open-AI wurde verwendet, um Feedback zur Klarheit und Strukturierung des Textes zu erhalten. Es wurde keine KI zur Erstellung von Originalinhalten verwendet.


\newpage

\phantomsection
\section*{Selbstständigkeitserklärung}
\addcontentsline{toc}{section}{Selbstständigkeitserklärung}

Ich erkläre hiermit, dass ich diese Arbeit selbstständig verfasst und keine anderen als die angegebenen Quellen benutzt habe. Alle Stellen, die wörtlich oder sinngemäss aus Quellen entnommen wurden, habe ich als solche gekennzeichnet. Mir ist bekannt, dass andernfalls der Senat gemäss Artikel 36 Absatz 1 Buchstabe r des Gesetzes vom 5. September 1996 über die Universität zum Entzug des aufgrund dieser Arbeit verliehenen Titels berechtigt ist.

Für die Zwecke der Begutachtung und der Überprüfung der Einhaltung der Selbständigkeitserklärung bzw. der Reglemente betreffend Plagiate erteile ich der Universität Bern das Recht, die dazu erforderlichen Personendaten zu bearbeiten und Nutzungshandlungen vorzunehmen, insbesondere die schriftliche Arbeit zu vervielfältigen und dauerhaft in einer Datenbank zu speichern sowie diese zur Überprüfung von Arbeiten Dritter zu verwenden oder hierzu zur Verfügung zu stellen.

\vspace{8cm}

\noindent Bern, \today

\vspace{2cm}

\noindent\rule{6cm}{0.4pt} \\
\noindent Lukas Batschelet




\clearpage
\pagenumbering{alph}
   
   \appendix

\begin{appendices}


\section{Soziodemografische Merkmale der Stichprobe}
\label{sec:appendix_demographics}

\begin{longtable}{p{5.5cm}p{5.5cm}rr}
    \caption{Übersicht über die Verteilung zentraler soziodemografischer Merkmale und Erfahrungen}
    \label{tab:soziodemografie_gesamt}\\
    \toprule
    Frage & Kategorie & Anzahl & Prozent \\
    \midrule
    \endfirsthead

    \multicolumn{4}{c}{{\bfseries Tabelle \thetable{} -- Fortsetzung}} \\
    \toprule
    Frage & Kategorie & Anzahl & Prozent \\
    \midrule
    \endhead
    
    \midrule
    \multicolumn{4}{r}{Fortsetzung auf der nächsten Seite}\\
    \endfoot
    
    \bottomrule
    \endlastfoot

    In welcher Altersgruppe befindest Du dich? & 16 – 25 & 20 & 80.0 \\*
     & 26 – 35 & 3 & 12.0 \\*
     & 56 – 65 & 1 & 4.0 \\*
     & Keine Angabe & 1 & 4.0 \\
    \midrule
    \addlinespace
    Welches Geschlecht wurde Dir bei der Geburt zugewiesen? & Männlich & 16 & 64.0 \\*
     & Weiblich & 8 & 32.0 \\*
     & Keine Angabe & 1 & 4.0 \\
    \midrule
    \addlinespace
    Mit welcher Geschlechtsidentität identifizierst Du dich? & Mann & 15 & 60.0 \\*
     & Frau & 9 & 36.0 \\*
     & Trans Mann & 1 & 4.0 \\
    \midrule
    \addlinespace
    Mit welchen Begriffen würdest du Deine sexuelle Orientierung beschreiben? & Heterosexuell & 17 & 68.0 \\*
     & Bisexuell & 3 & 12.0 \\*
     & Homosexual & 3 & 12.0 \\*
     & Queer & 1 & 4.0 \\*
     & Asexuell & 1 & 4.0 \\
    \midrule
    \addlinespace
    Was ist Dein höchster Bildungsabschluss? & Matura / Äquivalent & 23 & 92.0 \\*
     & Universitätsabschluss & 2 & 8.0 \\
    \midrule
    \addlinespace
    Wie ist Deine derzeitige berufliche oder schulische Situation? & Student\genderstern in / Schüler\genderstern in & 22 & 88.0 \\*
    & Angestellt & 3 & 12.0 \\
    \midrule
    \addlinespace
    Wie hoch ist ungefähr Euer gemeinsames monatliches Haushaltseinkommen (nach Abzug von Steuern)? & < CHF 1 500 & 7 & 28.0 \\*
     & CHF 1 500 – 3 000 & 2 & 8.0 \\*
     & CHF 3 000 – 4 500 & 2 & 8.0 \\*
     & CHF 6 000 – 7 500 & 2 & 8.0 \\*
     & CHF 7 500 – 10 000 & 1 & 4.0 \\*
     & > CHF 10 000 & 5 & 20.0 \\*
     & Nicht bekannt / bevorzugt nicht anzugeben & 6 & 24.0 \\
    \midrule
    \addlinespace
    Wie viele Personen leben in Deinem Haushalt (einschliesslich Dir selbst)? & 1 & 2 & 8.0 \\*
     & 2 & 3 & 12.0 \\*
     & 3 & 10 & 40.0 \\*
     & 4 & 6 & 24.0 \\*
     & 5 & 1 & 4.0 \\*
     & 6 & 2 & 8.0 \\*
     & 9 & 1 & 4.0 \\
    \midrule
    \addlinespace
    Wie viele Personen in Deinem Haushalt tragen (einschliesslich dir selbst) zum gemeinsamen Einkommen bei? & 1 & 6 & 24.0 \\*
     & 2 & 13 & 52.0 \\*
     & 3 & 4 & 16.0 \\*
     & 5 & 1 & 4.0 \\*
     & 6 & 1 & 4.0 \\
     \midrule
    \addlinespace
    Berechnetes Äquivalenz-Einkommen \parencite[nach][]{bundesamtfuerstatistikVerteilungVerfuegbarenAequivalenzeinkommens2025} & Armutsgefährdet & 8 & 32.0 \\*
     & Tief & 4 & 16.0 \\*
     & Mittel & 5 & 20.0 \\*
     & Hoch & 2 & 8.0 \\*
     & Unbekannt & 6 & 24.0 \\
     \midrule
    \addlinespace
    Hast Du eine körperliche oder psychische Beeinträchtigung, chronische Erkrankung oder andere gesundheitliche Einschränkung, die Deinen Alltag beeinflusst? & Nein & 25 & 100.0 \\
    \midrule
    \addlinespace
    Lebst Du in einem anderen Land, als in welchem du geboren wurdest? & Nein & 17 & 68.0 \\*
     & Ja & 7 & 28.0 \\*
     & Keine Angabe & 1 & 4.0 \\
     \midrule
    \addlinespace
    Hast Du im Alltag schon Diskriminierung aufgrund persönlicher Merkmale erlebt? & Ja, wegen meines Geschlechts & 4 & 16.0 \\*
     & Ja, wegen meiner Sprache oder meines Akzents & 4 & 16.0 \\*
     & Ja, wegen meiner Herkunft & 4 & 16.0 \\*
     & Ja, wegen meiner sexuellen Orientierung & 3 & 12.0 \\*
     & Ja, wegen meiner Kleidung oder meines Stils & 2 & 8.0 \\*
     & Ja, wegen meiner sozialen oder finanziellen Situation & 1 & 4.0 \\*
     & Ja, wegen meiner Hautfarbe oder meines Aussehens & 1 & 4.0 \\*
     & Ja, wegen meines Alters & 0 & 0.0 \\*
     & Ja, wegen meines Gesundheitszustands oder einer Behinderung & 0 & 0.0 \\*
     & Ja, aus einem anderen Grund & 0 & 0.0 \\*
     & Nein & 12 & 48.0 \\*
     & Keine Angabe & 0 & 0.0 \\
     \midrule
    \addlinespace
    Anzahl unterschiedlicher erlebter Diskriminierungsarten pro Person & 0 & 12 & 48.0 \\*
     & 1 & 8 & 32.0 \\*
     & 2 & 4 & 16.0 \\*
     & 3 & 1 & 4.0 \\
     \bottomrule
\end{longtable}
    
\section{Beschreibung der erfassten Momentaufnahmen}
\label{sec:appendix_moments}


\begin{longtable}{p{4cm}p{5cm}rr}
    \toprule
    Frage & Kategorie & Anzahl & Prozent \\
    \midrule
    \endfirsthead
    \toprule
    Frage & Kategorie & Anzahl & Prozent \\
    \midrule
    \endhead
    \midrule
    \multicolumn{4}{r}{Fortsetzung auf der nächsten Seite} \\
    \midrule
    \endfoot
    \bottomrule
    \endlastfoot
    Was machst Du gerade hauptsächlich? & Arbeiten oder studieren & 53 & 50.0 \\*
     & Freizeit oder Entspannung & 27 & 25.5 \\*
     & Unterwegs sein oder pendeln & 12 & 11.3 \\*
     & Kochen oder Essen & 8 & 7.5 \\*
     & Mediennutzung & 8 & 7.5 \\*
     & Soziale Aktivitäten & 7 & 6.6 \\*
     & Haushalt oder Aufräumen & 2 & 1.9 \\*
     & Ruhen / Schlafen & 2 & 1.9 \\*
     & Einkaufen oder Besorgungen & 2 & 1.9 \\*
     & Betreuungspflichten & 0 & 0.0 \\*
     & Sonstiges & 1 & 0.9 \\
    \midrule
    \addlinespace
    Bist Du drinnen oder draussen? & Drinnen & 54 & 50.9 \\*
     & Draussen & 52 & 49.1 \\
    \midrule
    \addlinespace
    Wo genau befindest Du dich? & Schule oder Universität & 38 & 35.8 \\*
     & Zuhause & 29 & 27.4 \\*
     & Unterwegs (zu Fuss, Fahrrad, Auto) & 12 & 11.3 \\*
     & Öffentlicher Verkehr & 8 & 7.5 \\*
     & Bei jemand anderem zuhause & 6 & 5.7 \\*
     & Arbeitsplatz & 5 & 4.7 \\*
     & Park oder Grünfläche & 5 & 4.7 \\*
     & Einkaufen oder Dienstleistungen & 2 & 1.9 \\*
     & Freizeit- oder Sporteinrichtung & 1 & 0.9 \\*
     & Café / Restaurant / Bar & 0 & 0.0 \\*
     & Kultureller oder religiöser Ort & 0 & 0.0 \\*
     & Gesundheitseinrichtung / Therapie & 0 & 0.0 \\*
     & Anderer Ort & 2 & 1.9 \\
     \midrule
    \addlinespace
    Mit wem bist Du gerade zusammen? & Freund\genderstern innen & 40 & 37.7 \\*
     & Allein & 38 & 35.8 \\*
     & Arbeitskolleg\genderstern innen & 17 & 16.0 \\*
     & Fremde & 17 & 16.0 \\*
     & Familie & 4 & 3.8 \\*
     & Bekannte & 3 & 2.8 \\*
     & Partner\genderstern in & 2 & 1.9 \\*
     & Tiere und Haustiere & 0 & 0.0 \\*
     & Kinder & 0 & 0.0 \\*
     & Andere & 2 & 1.9 \\
\end{longtable}


\begin{longtable}{p{7cm}*{7}{S[table-format=1.2]}}
    \caption{Deskriptive Kennwerte der Skalenitems (0–1)}
    \label{tab:scales_desc}\\
    \toprule
    Item & {Mean} & {SD} & {Min} & {25\%} & {50\%} & {75\%} & {Max} \\
    \midrule
    \endfirsthead
                
    \multicolumn{8}{c}{Tabelle \thetable{} -- Fortsetzung} \\
    \toprule
    Item & {Mean} & {SD} & {Min} & {25\%} & {50\%} & {75\%} & {Max} \\
    \midrule
    \endhead

    \bottomrule
    \endfoot
    
    Wie fühlst Du dich gerade insgesamt? & 0.67 & 0.19 & 0.00 & 0.57 & 0.69 & 0.79 & 1.00 \\
    Ganz allgemein – wie zufrieden fühlst Du dich im Moment? & 0.65 & 0.22 & 0.00 & 0.56 & 0.67 & 0.80 & 1.00 \\
    Wie angespannt oder entspannt fühlst Du dich? & 0.58 & 0.24 & 0.00 & 0.38 & 0.62 & 0.76 & 1.00 \\
    Wie wach fühlst Du dich im Moment? & 0.46 & 0.25 & 0.00 & 0.29 & 0.39 & 0.68 & 1.00 \\
    Wie nimmst Du die Geräuschkulisse an diesem Ort wahr? & 0.61 & 0.29 & 0.00 & 0.35 & 0.68 & 0.87 & 1.00 \\
    Wie viel Natur ist an diesem Ort sichtbar? & 0.40 & 0.31 & 0.00 & 0.14 & 0.37 & 0.63 & 1.00 \\
    Wie lebhaft oder ruhig wirkt der Ort? & 0.54 & 0.32 & 0.00 & 0.26 & 0.60 & 0.82 & 1.00 \\
    Wie angenehm empfindest Du den Ort insgesamt? & 0.71 & 0.24 & 0.01 & 0.59 & 0.77 & 0.90 & 1.00 \\
    Wie zugehörig oder fremd fühlst Du dich an diesem Ort? & 0.75 & 0.26 & 0.01 & 0.63 & 0.81 & 0.98 & 1.00 \\

\end{longtable}


\end{appendices}

\end{document}